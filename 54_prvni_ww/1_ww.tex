\documentclass{article}
\usepackage{fullpage}
\usepackage[czech]{babel}
\usepackage{amsfonts}

\title{\vspace{-2cm}První světová válka\vspace{-1.7cm}}
\date{}
\author{}

\begin{document}
\maketitle

-- doplnit od Hanky --

\begin{itemize}
    \vspace{-0.5em}
    \setlength\itemsep{0.15em}
    \item[28.7.1914] \textsc{Rakousko-Uhersko vyhlásilo válku Srbsku}, považováno za počátek první světové války
    \item[1.8.] Německo válku Rusku, Německo ultimátum Belgii, číjž porušili její integritu $\rightarrow$ Britové vyhlásili Němcům válku
    \item[6.8.] Rakousko-Uhersko vyhlašuje válku Německu
    \item[$-$] Itálie zůstává neutrální
\end{itemize}

\subsection*{Charakteristika}
\begin{itemize}
    \vspace{-0.5em}
    \setlength\itemsep{0.15em}
    \item[$-$] hlavním bojištěm je Evropa, dála ale i Afrika, Asie a oceány
    \item[$-$] zemřelo asi 10 milionů lidí, dvojnásobek zraněn
    \item[$-$] z počátku se označovala jako \textit{velká válka}, až později jako první světová
    \item[$-$] ženy mnohdy nahrazovaly mužské profese, protože byli ve válce
    \item[$-$] nejrůznější vynálezy, motorisace armády, roste význam zbrojního, těžkého průmyslu
    \item[$-$] zájmy civilního obyvatelstva jsou podružné zakázkám jako výroba zbraní, zásobování fronty atd. $\rightarrow$ přeměna v centrální řízenou ekonomiku
\end{itemize}

\subsection*{Západní fronta}
\begin{itemize}
    \vspace{-0.5em}
    \setlength\itemsep{0.15em}
    \item[$-$] německý \textit{Schlieffenův plán} vytvořen už na začátku století, měla to být \uv{blesková válka}, mělo dojít k isolaci Velké Británie, chtěli docílit k dobytí Evropy
    \item[20.8.] Němci dobývají Brusel
    \item[září] Němci na francouzské hranici, postupují rychle na francouzské území, téměř až k Paříži
    \item[5.9.-15.9.] \textsc{boje na řece Marně} (zázraky na Merně), Francouzům se podařilo Němce zatlačit zpátky, zvrat ve válce, padá naděje na bleskovou válku, problém pro Němce
    \item[$-$] po této události bylo nesmírně složité, aby jakákoliv strana kamkoliv postoupila
    \item[říjen až listopad] běh k moři, Francouzi i Němci chtějí obklíčit toho druhého a dobýt území směrem k moři, Němci dobyli ještě Antverpy, Flanderské vánoční příměří (o Vánocích se boje zastavily)
\end{itemize}

\subsection*{Balkán, srbská fronta}
\begin{itemize}
    \vspace{-0.5em}
    \setlength\itemsep{0.15em}
    \item[28.7.1914] Rakousko-Uhersko vyhlásilo válku Srbsku, na jejich stranu se přidá Černá Hora, Rakušané mají špatnou armádu, Rakušané jsou zatlačeni až za Dunaj
    \item[září 1915] vstup Bulharů na balkánskou frontu s cílem získat další území na stranu Rakouska-Uherska, během krátké doby jsou Srbové poraženi, jejich jednotky jsou převezeny na ostrov Corfu a odsud poté převezeni do Řecka, kde bojovali na straně   Řecka, které bojovalo na straně dohody
    \item[$-$] později poražena i Černá Hora;
    \item[$-$] důsledky: Rakousko-Uhersko  spojeno s Bulhary a s osmanskými Turky, spojení centrálních mocností, mohly si pomáhat
    \item[$-$] \textit{Opreace Gallipolio}: pokus Dohody vylodit se v Gallipoli, jejich cílem bylo získat Dardanely a dostat se do ČErného moře a pomoct zásobovat Rusy, neúspěšné, politické zemětřesení u Britů
    \item[srpen 1916] vstup Rumunska do války na straně Dohody, během roka byli poraženi
    \item[1917] vstup Řeků taky na straně Dohody, ti však poraženi nebyli
    \item[$-$] \textit{manifest Mým národům}, tím císař oznamoval, že jsme vstoupili do války 
\end{itemize}




\end{document}
