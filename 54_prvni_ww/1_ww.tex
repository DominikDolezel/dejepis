\documentclass{article}
\usepackage{fullpage}
\usepackage[czech]{babel}
\usepackage{amsfonts}

\title{\vspace{-2cm}První světová válka\vspace{-1.7cm}}
\date{}
\author{}

\begin{document}
\maketitle

-- doplnit od Hanky --

\begin{itemize}
    \vspace{-0.5em}
    \setlength\itemsep{0.15em}
    \item[28.7.1914] \textsc{Rakousko-Uhersko vyhlásilo válku Srbsku}, považováno za počátek první světové války
    \item[1.8.] Německo válku Rusku, Německo ultimátum Belgii, číjž porušili její integritu $\rightarrow$ Britové vyhlásili Němcům válku
    \item[6.8.] Rakousko-Uhersko vyhlašuje válku Německu
    \item[$-$] Itálie zůstává neutrální
\end{itemize}

\subsection*{Charakteristika}
\begin{itemize}
    \vspace{-0.5em}
    \setlength\itemsep{0.15em}
    \item[$-$] hlavním bojištěm je Evropa, dála ale i Afrika, Asie a oceány
    \item[$-$] zemřelo asi 10 milionů lidí, dvojnásobek zraněn
    \item[$-$] z počátku se označovala jako \textit{velká válka}, až později jako první světová
    \item[$-$] ženy mnohdy nahrazovaly mužské profese, protože byli ve válce
    \item[$-$] nejrůznější vynálezy, motorisace armády, roste význam zbrojního, těžkého průmyslu
    \item[$-$] zájmy civilního obyvatelstva jsou podružné zakázkám jako výroba zbraní, zásobování fronty atd. $\rightarrow$ přeměna v centrální řízenou ekonomiku
\end{itemize}

\subsection*{Západní fronta}
\begin{itemize}
    \vspace{-0.5em}
    \setlength\itemsep{0.15em}
    \item[$-$] německý \textit{Schlieffenův plán} vytvořen už na začátku století, měla to být \uv{blesková válka}, mělo dojít k isolaci Velké Británie, chtěli docílit k dobytí Evropy
    \item[20.8.] Němci dobývají Brusel
    \item[září] Němci na francouzské hranici, postupují rychle na francouzské území, téměř až k Paříži
    \item[5.9.-15.9.] \textsc{boje na řece Marně} (zázraky na Merně), Francouzům se podařilo Němce zatlačit zpátky, zvrat ve válce, padá naděje na bleskovou válku, problém pro Němce
    \item[$-$] po této události bylo nesmírně složité, aby jakákoliv strana kamkoliv postoupila
    \item[říjen až listopad] běh k moři, Francouzi i Němci chtějí obklíčit toho druhého a dobýt území směrem k moři, Němci dobyli ještě Antverpy, Flanderské vánoční příměří (o Vánocích se boje zastavily)
\end{itemize}

\subsection*{Balkán, srbská fronta}
\begin{itemize}
    \vspace{-0.5em}
    \setlength\itemsep{0.15em}
    \item[28.7.1914] Rakousko-Uhersko vyhlásilo válku Srbsku, na jejich stranu se přidá Černá Hora, Rakušané mají špatnou armádu, Rakušané jsou zatlačeni až za Dunaj
    \item[září 1915] vstup Bulharů na balkánskou frontu s cílem získat další území na stranu Rakouska-Uherska, během krátké doby jsou Srbové poraženi, jejich jednotky jsou převezeny na ostrov Corfu a odsud poté převezeni do Řecka, kde bojovali na straně   Řecka, které bojovalo na straně dohody
    \item[$-$] později poražena i Černá Hora;
    \item[$-$] důsledky: Rakousko-Uhersko  spojeno s Bulhary a s osmanskými Turky, spojení centrálních mocností, mohly si pomáhat
    \item[$-$] \textit{Opreace Gallipoli}: pokus Dohody vylodit se v Gallipoli, jejich cílem bylo získat Dardanely a dostat se do Černého moře a pomoct zásobovat Rusy, neúspěšné, politické zemětřesení u Britů
    \item[srpen 1916] vstup Rumunska do války na straně Dohody, během roka byli poraženi
    \item[1917] vstup Řeků taky na straně Dohody, ti však poraženi nebyli
    \item[$-$] \textit{manifest Mým národům}, tím císař oznamoval, že jsme vstoupili do války
\end{itemize}

\subsection*{Východní fronta 1914-1915}
\begin{itemize}
    \vspace{-0.5em}
    \setlength\itemsep{0.15em}
    \item[1.8.1914] Německo vyhlásilo válku Rusku
    \item[6.8.1914] Rakousko-Uhersko vyhlásilo válku Rusku
    \item[$-$] německý útok (Paul von Hindenburg, Erich Ludendorff), Rusové několikrát poraženi, pro Rusy katastrofa
    \item[$-$] Rakušanům se nedaří: v čele Rusů generál Brusilov, na úkor Rakouska pronikl až k hranicím dnešního Slovenska ke Krakovu
    \item[1915] proto naplánovaly Centrální mocnosti tzv. \textit{Gorlický průlom} $\rightarrow$ posouvají hranici dál na východ
    \item[$-$] na západní frontě obrovské bitvy, tím, že by Rusové kapitulovali, mohli by armády přesunout na západní frontu
    \item[1916] Rusové se pokusily v rámci \textit{Brusilovovy ofensivy} posunout linii, podařilo se jim to trochu
    \item[$-$] dohodovým zemím měli pomoct Rumuni, ale ti byli během pár měsíců úplně zlikvidováni
    \item[1917] vstup Řecka do války, otevření Soluňské fronty, kam byli dováženi Srbové, kteří byli už dřív poraženi
\end{itemize}

\subsection*{Jižní (italská) fronta}
\begin{itemize}
    \vspace{-0.5em}
    \setlength\itemsep{0.15em}
    \item[$-$] mezi Italy a Rakušáky a Italy a Němci
    \item[23.5.1915] Itálie vyhlašuje válku Rakousku-Uhersku, o rok později Německu
    \item[duben 1915] dohodové země slibují Italům území, která chtějí v \textit{Londýnské smlouvě}
    \item[$-$] tak si to tam nějak šolichali
    \item[podzim 1917] Italové poraženi \textsc{u Caporetta}, linie se posunula na řeku Piavu
\end{itemize}

\subsection*{Turecké fronty}
\begin{itemize}
    \vspace{-0.5em}
    \setlength\itemsep{0.15em}
    \item[$-$] Kavkazská fronta (Osmani x Rusové), Mezopotámská fronta (Britové x Osmani), syrsko-palestinská fronta (Osmani x Briti)
    \item[$-$] do roku 1915 Rusové vítězí
    \item[$-$] genocida Arménů, Turci jich zlikvidovali asi milion a půl
    \item[$-$] mezopotámská fronta se posouvá na úkor Osmanů
    \item[$-$] na syrsko-palestinské frontě jsou Osmané zpočátku úspěšní, jde jim o Suezský průplav, krátce se jim to povedlo, pak jsou zatlačeni zpět Brity
\end{itemize}

\subsection*{Západní fronta}
\begin{itemize}
    \vspace{-0.5em}
    \setlength\itemsep{0.15em}
    \item[březen 1915] francouzská ofenzíva u Champagni
    \item[duben 1915] Ypry, chlor (první použití bojového otravného plynu)
    \item[ún. - pro. 1916]  Verdun, 600 000 padlých, Němci neúspěšní $\rightarrow$ střídání v německém velení
    \item[čvc. - lis. 1916]  na řece Sommě, 1 300 000 padlých, neúspěch na straně dohody, 1917 Philipe Petaine (nastrčený stařík?)
\end{itemize}


\subsection*{Boje na moři}
\begin{itemize}
    \vspace{-0.5em}
    \setlength\itemsep{0.15em}
    \item[$-$] proti němcům především Velká Británie (a  Japonsko)
    \item[$-$] dohodové země úspěšné, získají přístup k německým koloniím v Africe
    (bitvy: u Helgolandu, Dogger Bank, Skaggeraku, Jutska) prostě v Severním moři..
    \item[1917]  německá ponorková válka - jsou úspěšní, potápějí (potopili i loď s Američany (7. 5. 1915 Lusitania), to byl jeden z důvodů proč se pak USA zapojily do války)
\end{itemize}



\subsection*{Politické změny a situace}
\begin{itemize}
    \vspace{-0.5em}
    \setlength\itemsep{0.15em}
    \item[1916]  \uv{válečný kabinet} David Lloyd George
    \item[1916]  nóta W. Wilsona válčícím státům (stanovit cíle války)
    \item[21. 11. 1916]  (R-U) Karel I. + Zita Parmská, s karlem přišel refresh, amnestie, jednal s dohodou (bratr Zity Sixtus Bourbonský jim chce pomoct z války, ale zjistilo se to = Sixtova aféra) $\rightarrow$  nová smlouva s německem (obnovení spojenectví)
    změny ve francouzské vládě
    \item[1917] Německo vede neomezenou ponorkovou válku
    \item[6. 4. 1917] vstup USA do války
    \item[1918] V fronta, 3. 3. Brest-litevský mír
    \item[$-$] Německo vede ofenzívu na západní frontě,
    červenec - srpen: druhá bitva na Marně
    \item[$-$] prolomení Siegfriedovy linie
    \item[5. 10. 1918]  německá žádost o příměří
\end{itemize}

\subsection*{Jižní fronta}
\begin{itemize}
    \vspace{-0.5em}
    \setlength\itemsep{0.15em}
    \item[25. 10.]  zhroucení italské fronty
    \item[26. 10.]  Karel I. požádá o separátní mír a okamžité příměří
    \item[27. 10.]  Gyula Andrássy, R-U přijímá mírové podmínky
    \item[3. 11.]  příměří
    \item[6. 11.]  demobilizace armády
    \item[11. 11.]  Karel I. opouští Schönbrünn
\end{itemize}

\subsection*{Balkán}
\begin{itemize}
    \vspace{-0.5em}
    \setlength\itemsep{0.15em}
    \item[září 1918]  ofenziva dohody
    \item[29. 9.]  bulharsko mír
    \item[30. 9.]  (?) příměří
\end{itemize}


\subsection*{Západní fronta 1918}
\begin{itemize}
    \vspace{-0.5em}
    \setlength\itemsep{0.15em}
    \item[8. 11.]  podmínky příměří
    \item[9. 11.]  Vilém II. abdikuje
    \item[11. 11.]  Compiegne, N x Dohoda, konec první světové války (dnes slavíme den veteránů)
\end{itemize}

\subsection*{Výsledky první světové války}
\begin{itemize}
    \vspace{-0.5em}
    \setlength\itemsep{0.15em}
    \item[$-$] vyhrála Dohoda $\rightarrow$ zánik Rakouska-Uherska (rozpad na 5 nástupnických států: ČSR, Rakousko, Maďarsko, Polsko, království Srbů, Chorvatů a Slovinců)
    \item[$-$] z Německa vznikne Výmarská republika
    \item[$-$] Rusko ztratí území: Finsko, Poblatí, Litva, Bělorusko, Ukrajina připadnou Polsku, Besarabie připadne Rumunům
    \item[$-$] Osmané: zrušen sultanát, stávají se republikou Turecko
    \item[$-$] Belgie obnovena v původním rozsahu
    \item[$-$] ekonomickým vítězem jsou Spojené státy, Evropa už není na vrcholu
    \item[$-$] poražené státy: Německo, Rakousko, Maďarsko, Turecko, Bulharsko
\end{itemize}



\end{document}
