\documentclass{article}
\usepackage{fullpage}
\usepackage[czech]{babel}
\usepackage{amsfonts}

\title{\vspace{-2cm}Sámova říše, VM, Přemyslovci, Uhry, Polsko, Byzanc\vspace{-1.7cm}}
\date{}
\author{}

\begin{document}
\maketitle

\section*{Sámova říše (623/29 -- 658/59), zkrátka 7. st.}
\begin{itemize}
    \vspace{-0.5em}
    \setlength\itemsep{0.15em}
    \item[$-$] Slované
    \begin{itemize}
        \vspace{-0.5em}
        \setlength\itemsep{0.15em}
        \item[$-$] původní sídla mezi Vislou a Dněprem
        \item[$-$] jdou na území, kde dříve byli Germáni
        \item[$-$] přichází ve vlnách v 6. st.
    \end{itemize}
    \item[$-$] \textbf{Kosmas} (1045 -- 1125): literární fikce
    \item[$-$] byzantští historici: \textbf{Prokopios}, \textbf{Jordanes}, Slované $\sim$ Venedové, Autové
    \item[$-$] nejvýznamější zdroj informací: \textit{Historia Francorum}, autor \textbf{Fredegar}
    \item[$-$] polyteismus (Perun, Dažbog, Veles, ...), kmenoví náčelníci
    \item[$-$] obživa: pěstování olivovin, luštěnin, keramika pražského typu -- na hrnčířském kruhu, rybolov, \textit{brtnictví} = sběr včelího medu
    \item[$-$] \textit{polozemnice} -- příbytek zapuštěný do země
    \item[576] Avaři přichází do Maďarska = Panonie
    \item[(620)] protiavarské povstání, svrhnutí nadvlády Avarů
    \item[$-$] \textbf{Sámo} (623 -- 658) několikrát porazil Avary, proto si ho zvolili za \uv{krále}
    \item[631] \textsc{bitva u Wogastisburgu}, poražení \textbf{Dagoberta z Meroveovců} (Franská ř.)
    \item[$-$] \textbf{Dagobert I.} si myslel, že na kupcích, kteří prochází přes S. ř. je pácháno zlo, proto záminka k bivě
    \item[$-$] problém s Avary končí na sklonku 8. st. díky tažení \textbf{Karla Velikého}, 3 tažení, zbavení Evropy nebezpečí Avarů
\end{itemize}

\section*{Velká Morava}

\subsection*{Vznik}
\begin{itemize}
    \vspace{-0.5em}
    \setlength\itemsep{0.15em}
    \item[$-$] Moravské knížectví, \textbf{Mojmír}; Nitranské knížectví, \textbf{Pribina}
    \item[$-$] křesťanství
    \item[$-$] Mojmír pokřtěn v Pasově biskupem \textbf{Reginhardem}
    \item[833] Mojmír ovládl území Pribiny, t. j. Nitranské knížectví, vyhnal ho $\rightarrow$ základy VM
    \item[$-$] informace máme od: \textbf{Konstantina Porfyrogennéta} -- byzantský císař, vládl v 1. pol. 10. st.
    \item[$-$] \textit{Letopisy (anály) Fuldské}
\end{itemize}

\subsection*{Mojmír I. (833 -- 846)}
\begin{itemize}
    \vspace{-0.5em}
    \setlength\itemsep{0.15em}
    \item[843] \textsc{Verdunská smlouva}, rozpad Franské říše, Frankové velkým nebezpečím
    \item[$-$] uznává svrchovanost východofanského panovníka, platil tribut, aby neútočili
\end{itemize}

\subsection*{Rastislav I. (846 -- 870)}
\begin{itemize}
    \vspace{-0.5em}
    \setlength\itemsep{0.15em}
    \item[$-$] nechce platit tribut, chce VM vymanit z franského vlivu
    \item[$-$] vyhnal franské a bavorské duchovenstvo (protože šíří v latině)
    \item[(861)] obrací se na papeže \textbf{Mikuláše I.}, jestli by mu neposlal nějaké duchovenstvo, odmítá
    \item[(863)] obrací se na \textbf{Michala III.}, ten mu pošle \textbf{Konstantina} (vystudoval teologii, filozofii, literaturu) a \textbf{Metoděje} (právnické vzdělání), pochází ze Soluně
\end{itemize}

\subsection*{Bratři ze Soluně}
\begin{itemize}
    \vspace{-0.5em}
    \setlength\itemsep{0.15em}
    \item[$-$] vytvořili \textit{hlaholici}, přeložení částí bible, mluví staroslověnsky
    \item[$-$] \textit{Proglas} = předzpěv k Evangeliu, Kyjevské listy
    \item[$-$] \textit{Zákon sudnyj ljudem}
    \item[$-$] zakládají církevní školy
    \item[$-$] přinesení ostatků sv. Klimenta
    \item[$-$] jejich působení trnem v oku tzv. \textit{trojjazyčníkům} (lidé vedoucí bohoslužby v latině, řečtině a hebrejštině), kteří na ně útočí $\rightarrow$ vydají se do Říma
    \item[867] Řím, staroslověnština potvrzena jako plnoprávný bohoslužebný jazyk (Hadrián II.)
    \item[(869)] Konstantin -- Cyril umírá
    \item[$-$] na Moravu se vrací jen Metoděj, který byl potvrzen Moravsko-panonským arcibiskupem
    \item[880] bula \textit{Industria tuae} opět potvrzuje platnost staroslověnštiny
\end{itemize}

\subsection*{Svatopluk (871 -- 894)}
\begin{itemize}
    \vspace{-0.5em}
    \setlength\itemsep{0.15em}
    \item[$-$] Rastislavův synovec
    \item[$-$] skončil ve vězení, na jeho místo dosazeni Vilém a Englšalk
    \item[$-$] oni požádali Svatopluka, aby jim pomohl v bojích, ten však přeběhl k Moravanům, porazili a vyhnali Franky
    \item[(874)] \textsc{Forcheimský mír}, Svatopluk slibuje, že bude odvádět mírový tribut Frankům, za odměnu získá nezávislost
    \item[$-$] v této době se vrací Metoděj, překlad bible do staroslověnštiny, později zničeno
    \item[(883)] křest Bořivoje a Ludmily
    \item[$-$] nitranský biskup \textbf{Wiching} chce na VM latinské bohoslužby
    \item[$-$] maximální územní rozsah: Čechy, Morava, Lužice, Slezsko, Krakovsko, Panonie
    \item[(885)] smrt Metoděje, jejich žáci vyhnáni do Bulharska, na jeho místo latinští kazatelé
\end{itemize}

\subsection*{Mojmír II. (894 -- 906)}
\begin{itemize}
    \vspace{-0.5em}
    \setlength\itemsep{0.15em}
    \item[$-$] období vnitřních rozkladů, útoky z Východofranské říše, odtrhnutí Čech od VM
    \item[906/7] vpád Maďarů $\rightarrow$ konec VM
\end{itemize}

\subsection*{Kultura}
\begin{itemize}
    \vspace{-0.5em}
    \setlength\itemsep{0.15em}
    \item[$-$] \textit{hradiště} -- centra řemesla, obchodu (Staré město u Uherského Hradiště, Valy u Mikulčic, Děvín, \dots)
    \item[$-$] řemeslo, šperkařství, křesťané $\rightarrow$ kostrové hroby
    \item[$-$] architekrura: dlouhé kostely (Kopčany -- kostel sv. Markéty), baziliky, rotundy
    \item[$-$] \textit{gombíky} = knoflíky
    \item[$-$] hrnčířský kruh
    \item[$-$] směnný obchod, později první platidla \textit{platýnky}, kůže, hřivny železa
\end{itemize}


\section*{Český stát}

\subsection*{Přírodní podmínky}
\begin{itemize}
    \vspace{-0.5em}
    \setlength\itemsep{0.15em}
    \item[$-$] nížiny, husté a nepropustné lesy $\rightarrow$ přirozená hranice
    \item[$-$] \textit{trojpolní systém} -- lada, ozim, jař
    \item[$-$] keramika, textilie, směnný obchod
    \item[$-$] půdu vlastní šlechta, lidé si ji pronajímají za \textit{rentu}
    \item[$-$] \textit{rustikál} = půda, kterou si vesničané pronajímají od šlechty
    \item[$-$] \textit{dominikál} = půda, kterou vlastní šlechta (zámky, statky krále)
    \item[$-$] kolem vesnic pastviny ve společném majetku obce
    \item[$-$] obchodníci v podhradí
\end{itemize}

\subsection*{Osídlení}
\begin{itemize}
    \vspace{-0.5em}
    \setlength\itemsep{0.15em}
    \item[$-$] Charváti (V Čechy) -- Libice nad Cidlinou, vládnou Slavníkovci
    \item[$-$] Čechové (střední Čechy) -- Budeč, Levý Hradec, vládnou Přemyslovci
    \item[$-$] Přemyslovci název od Přemysla Oráče, postupně sjednotí ostatní kmeny
\end{itemize}

\begin{itemize}
    \vspace{-0.5em}
    \setlength\itemsep{0.15em}
    \item[poč. 9. st.] \textbf{Karel Veliký} vtrhne do Čech, vynutí si \textit{tribut} = poplatek za to, že na ně nebude útočit
    \item[845] bylo v Řezně pokřtěno 14 českých kmenových knížat
    \item[2. pol. 9. st.] sjednocovací proces: prostor Čech ovládl kmen Čechů, vládnou Přemyslovci
    \item[$-$] legendární knížata: Přemysl, Nezamysl, Mnata, Vojen, Vnislav, Křesomysl, Neklan, Hostivít, \dots
\end{itemize}

\subsection*{Bořivoj (867 -- 894)}
\begin{itemize}
    \vspace{-0.5em}
    \setlength\itemsep{0.15em}
    \item[$-$] manželka Ludmila
    \item[$-$] zástupce Svatupluka -- knížete VM v Čechách
    \item[883] křest
    \item[$-$] stavba rotundy sv. Klimenta -- první křesťanský kostel v Čechách, začátek stavby Pražského hradu, kostelík Panny Marie
    \item[$-$] synové: Spytihněv a Vratislav
\end{itemize}

\subsection*{Spytihněv (894 -- 915)}
\begin{itemize}
    \vspace{-0.5em}
    \setlength\itemsep{0.15em}
    \item[$-$] dostává se na trůn, když Svatopluk umírá
    \item[$-$] odtrhne Čechy od Velké Moravy
    \item[$-$] rotunda na Budči sv. Petra a Pavla, budování Pražského hradu
    \item[$-$] základy státní správy
\end{itemize}

\subsection*{Vratislav (915 -- 921)}
\begin{itemize}
    \vspace{-0.5em}
    \setlength\itemsep{0.15em}
    \item[$-$] manželka Drahomíra (nechala zavraždit Ludmilu, prohlášena za svatou)
    \item[$-$] dva synové: Václav a Boleslav
    \item[$-$] úspěšné boje s Maďary
\end{itemize}

\subsection*{Svatý Václav (921 -- 935)}
\begin{itemize}
    \vspace{-0.5em}
    \setlength\itemsep{0.15em}
    \item[$-$] nastupuje mladý
    \item[$-$] o vliv na Václava bojují matka Drahomíra s babičkou Ludmilou
    \item[$-$] Drahomíra nechala Ludmilu uškrtit šálou
    \item[$-$] současník Jindřich I. Ptáčník, pokračování v placení mírového tributu
    \item[28.9.935] Boleslav ho poslal do Mladé Boleslavi, u kostela na něj zaútočil mečem, zabili ho jeho družiníci, popsáno v Gumpoldově legendě
    \item[$-$] prohlášen za svatého, patron českého národa
\end{itemize}

\subsection*{Boleslav I. (935 -- 972) Ukrutný}
\begin{itemize}
    \vspace{-0.5em}
    \setlength\itemsep{0.15em}
    \item[$-$] stabilizace země, hospodářství, vybírání daní
    \item[$-$] současník Otty I.
    \item[$-$] nezávislost na SŘŘ, nechce odvádět tribut
    \item[955] \textsc{bitva u Lešských polí}, porážka Maďarů
    \item[$-$] expanze do Moravy, Slezska, Krakovska, povodí řeky Váhu
    \item[$-$] budování hradské správy, v rukou kastelánů
    \item[$-$] \textit{stříbrný denár} = první přemyslovská mince
    \item[$-$] děti: \textbf{Doubravka} (provdala se za Měška z rodu Piastovců) -- zakladatelé polského státu; \textbf{Mlada} (abatyše benediktinek při sv. Jiří); \textbf{Boleslav}
\end{itemize}

\subsection*{Boleslav II. (972 -- 999) Pobožný}
\begin{itemize}
    \vspace{-0.5em}
    \setlength\itemsep{0.15em}
    \item[973] vznik biskupství v Praze, první biskup \textbf{Dětmar}, druhý \textbf{sv. Vojtěch} ze Slavníkovců (pohřben v Hnězdně), později zřízeno arcibiskupství: arcibiskup \textbf{Radim}
    \item[$-$] expanze na Východ až ke Lvovu
    \item[$-$] podpora příchodu benediktínů, klášter v Břevnově
    \item[28.9.995] \textsc{vyvraždění Slavníkovců v Libici} při oslavách sv. Václava
\end{itemize}

\subsection*{Krize Českého státu}
\begin{itemize}
    \vspace{-0.5em}
    \setlength\itemsep{0.15em}
    \item[$-$] boj o moc: \textbf{Boleslav III. Ryšavý} je krutý, neschopný vs. \textbf{Jaromír} vs. \textbf{Oldřich}
    \item[$-$] této situace využívá \textbf{Boleslav Chrabrý} (syn Doubravky a Měška, polský král), Vladivoj posílá vládnout do Čech, umírá
    \item[$-$] poté Jaromír, následně Oldřich, který vyvede Čechy z krize
\end{itemize}

\subsection*{Oldřich (1012 -- 1033)}
\begin{itemize}
    \vspace{-0.5em}
    \setlength\itemsep{0.15em}
    \item[$-$] potká Boženu, s ní má syna Břetislava
    \item[$-$] vyhnal Poláky z našeho území
    \item[(1019)] znovu definitivně připojil Moravu
    \item[$-$] Sázavský klášter, benediktíni, slovanská liturgie
\end{itemize}

\subsection*{Břetislav I. (1035 -- 1055), český Achilles}
\begin{itemize}
    \vspace{-0.5em}
    \setlength\itemsep{0.15em}
    \item[$-$] stabilizace Českého státu
    \item[$-$] manželka \textbf{Jitka ze Svinibrodu}, unesl si ji z kláštera, pět synů
    \item[1039] \textsc{polské tažení}, dobyl Slezsko a Krakovsko; velká kořist
    \item[$-$] nad hrobem Vojtěcha v Hnězdně pronesl tzv. \textit{Břetislavovy dekrety} = nejstarší český právní dokument (křesťanský způsob života)
    \item[1054] \textit{stařešinský řád} = nejstarší člen rodu má právo na vládu; pro zbylé zřídil na Moravě tzv. \textit{údělná knížectví}: Brněnské, Olomoucké, Znojemské
    \item[$-$] Rajhradský klášter
    \item[$-$] několik bitev proti císaři SŘŘ Jindřichu III., neúspěšné
\end{itemize}

\subsection*{Vratislav II. (1061 -- 1092)}
\begin{itemize}
    \vspace{-0.5em}
    \setlength\itemsep{0.15em}
    \item[$-$] sídlo přesunul na Vyšehrad
    \item[$-$] bazilika sv. Petra a Pavla na Vyšehradě
    \item[$-$] nejstarší pražská rotunda -- rotunda sv. Martina
    \item[$-$] zřízení olomouckého biskupství
    \item[1085] královský titul od Jindřicha IV. (za pomoc v boji o investituru), jako první český král
    \item[$-$] \textit{Kodex vyšehradský}
    \item[$-$] po jeho smrti probíhají boje o moc mezi jeho dětmi
\end{itemize}

\subsection*{Soběslav I.}
\begin{itemize}
    \vspace{-0.5em}
    \setlength\itemsep{0.15em}
    \item[$-$] Vratislavův nejmladší syn
    \item[1126] v \textsc{bitvě u Chlumce u Nakléřovského průsmyku} zastavil vpád vojáků SŘŘ pod vedením Lothara, budoucího císaře
    \item[$-$] buduje opevnění Pražského hradu, rotunda sv. Jiří na Řípu
\end{itemize}

\subsection*{Vladislav II. (1140 -- 72)}
\begin{itemize}
    \vspace{-0.5em}
    \setlength\itemsep{0.15em}
    \item[$-$] vnuk Vratislava II., druhý český král
    \item[1147] \textsc{druhá křížová výprava}
    \item[1158] zisk královského titulu od Fridricha Barbarossy
    \item[$-$] manželka \textbf{Judita}, iniciovala stavbu Juditinu mostu
    \item[$-$] rotunda sv. Kateřiny ve Znojmě, vznik řady klášterů
    \item[$-$] reformní řády: premonstráti, cisterciáci, johanité -- vychází z benediktínů
    \item[1172] se vzdává svého královského titulu ve prospěch syna \textbf{Bedřicha}
\end{itemize}

\begin{itemize}
    \vspace{-0.5em}
    \setlength\itemsep{0.15em}
    \item[1182] Fridrich Barbarossa využívá situace, odtrhl Moravu -- markrabství moravské (Konrád II. Ota), je přimo podřízená jemu, povýšil biskupa pražského na říšské kníže, též podléhá jemu
    \item[1189] \textit{Statuta Conradi} = kodifikace zvykového práva Konrádem II. Otou, přiznal šlechtě dědictví půdy
\end{itemize}

\subsection*{Přemysl Otakar I. (1197 -- 1230)}
\begin{itemize}
    \vspace{-0.5em}
    \setlength\itemsep{0.15em}
    \item[$-$] stabilizace země
    \item[1212] \textsc{zlatá bula sicilská} = definitivně Přemyslovcům přiznán dědičný titul krále, anulováno odtržení Moravy, český král si může vybírat biskupy
    \item[$-$] manželka \textbf{Konstancie Uherská}, spojována se vznika kláštera Porta Coeli
\end{itemize}


\section*{Uherské království}
\subsection*{Obecně.}
\begin{itemize}
    \vspace{-0.5em}
    \setlength\itemsep{0.15em}
    \item[896] příchod ugrofinských kočovníků pod vedením \textbf{Arpáda}
    \item[$-$] 7 staromaďarských kmenů
    \item[$-$] stáli u zániku Velké Moravy
    \item[955] \textsc{bitva na řece Lechu} zastavení Maďarů Čechy a Otou I., stáhnou se do Panonie
    \item[965] \textbf{Gejza z Arpádovců} zakládá první Uherské knížectví, opírá se o křesťanství, šíří ho Vojtěch
\end{itemize}

\subsection*{Vajko = Štěpán I. Svatý (997 -- 1038)}
\begin{itemize}
    \vspace{-0.5em}
    \setlength\itemsep{0.15em}
    \item[$-$] syn Gejzy
    \item[1000] první Uherský král
    \item[1000] v Ostřihomi zřízeno arcibiskupství
    \item[$-$] vyhnal Boleslava Chrabrého ze Slovenska; připojuje území Slovenska, Sedmihradska
    \item[$-$] \textit{komitáty} = \textit{župy} = základní správní jednotka, \textit{župani} = královští úředníci
    \item[$-$] prohlášen za svatého, svatoštěpánské korunovační klenoty
\end{itemize}

\subsection*{11. -- 12. st.}
\begin{itemize}
    \vspace{-0.5em}
    \setlength\itemsep{0.15em}
    \item[$-$] připojení Slovenska Slovinska, Slavonie, Chorvatska, Dalmácie
\end{itemize}

\subsection*{13. st.}
\begin{itemize}
    \vspace{-0.5em}
    \setlength\itemsep{0.15em}
    \item[1222] \textsc{Zlatá bula} \textbf{Ondřeje II.} = osvobození církve a šlechty od daní, dědičně dědí majetky
    \item[$-$] \textbf{Béla IV.}: boje proti Tatarům
    \item[$-$] stát se stabilizoval díky německé kolonizaci
    \item[1260] \textsc{bitva u Kressenbrunnu} Béla proti Přemyslu Otakaru II., Béla prohrál
    \item[$-$] poslední Arpádovec: \textbf{Ondřej III.}
\end{itemize}

\section*{Polské království}
\subsection*{Obecně.}
\begin{itemize}
    \vspace{-0.5em}
    \setlength\itemsep{0.15em}
    \item[$-$] Slezané, Vislané (Krakov), Pomořané, Polané (SZ, Hnězdno)
    \item[$-$] tyto kmeny sjednotil \textbf{Měšek} z Piastovců (963 -- 992), manželka Doubravka (dcera Boleslava I.) -- šiřitelka křesťanství
    \item[$-$] sídelní město: Hnězdno
    \item[$-$] oráč Piast
\end{itemize}

\subsection*{Boleslav Chrabrý (992 -- 1025)}
\begin{itemize}
    \vspace{-0.5em}
    \setlength\itemsep{0.15em}
    \item[$-$] zahájil expanzi:
    \begin{itemize}
        \vspace{-0.5em}
        \setlength\itemsep{0.15em}
        \item[$-$] na východ, až do Kijeva
        \item[$-$] na západ, Lužice
    \end{itemize}
    \item[(1018)] \textsc{Budyšínský mír} s Jindřichem II.
    \item[$-$] nadstandartní vztahy se Silvestrem II.
    \item[(1000)] v Hnězdně zřízeno arcibiskupství při hrobu sv. Vojtěcha, arcibiskup \textbf{Radim} (nevlastní bratr Vojtěcha)
    \item[(1025)] na konci života královský titul $\rightarrow$ první polský král
    \item[$-$] soustava hradů
\end{itemize}

\subsection*{11. st.}
\begin{itemize}
    \vspace{-0.5em}
    \setlength\itemsep{0.15em}
    \item[1038] sídelním místem Krakov
    \item[1039] zisk Krakovska, Hnězdna Břetislavem I.
\end{itemize}

\subsection*{12. st.}
\begin{itemize}
    \vspace{-0.5em}
    \setlength\itemsep{0.15em}
    \item[(1076)] \textbf{Boleslav II. Smělý}: titul krále polského
    \item[$-$] \textbf{Boleslav III. Křivoústý}: zánik jednotného Polského státu
    \item[(1138)] nástupnický řád, kdo bude kde vládnout
\end{itemize}

\subsection*{13. st.}

\subsubsection*{Konrád Mazovský}
\begin{itemize}
    \vspace{-0.5em}
    \setlength\itemsep{0.15em}
    \item[1226] pozval do země řád německých rytířů, aby šířili křesťanství $\rightarrow$ ovládli Prusko
    \item[$-$] porážka až v \textsc{bitvě u Grunwaldu}
\end{itemize}

\subsubsection*{Jindřich II. Pobožný}
\begin{itemize}
    \vspace{-0.5em}
    \setlength\itemsep{0.15em}
    \item[1241] porážka Mongoly v \textsc{bitvě u Legnice}
\end{itemize}


\section*{Byzantská říše}

\subsection*{Obecně.}
\begin{itemize}
    \vspace{-0.5em}
    \setlength\itemsep{0.15em}
    \item[$-$] Východořímská říše (vznik 395), od 12. st. Byzantská
    \item[$-$] Konstantin I. založil Byzantion
    \item[$-$] považují se za následníky Římanů
    \item[$-$] pojítkem křesťanství, liturgickým jazykem \textit{řečtina}
    \item[$-$] města: \textbf{Konstantinopol} (správní centrum), Antiochea, Níkaáia
    \item[$-$] textilnictví, zlatnictví, hedvábí, mince
    \item[$-$] absolutismus
\end{itemize}

\subsection*{Justinián I. (527 -- 565)}
\begin{itemize}
    \vspace{-0.5em}
    \setlength\itemsep{0.15em}
    \item[$-$] nejvýznamnější panovník
    \item[$-$] chce obnovit území celé Římské říše $\rightarrow$ expanzivní politika
    \item[$-$] maželka \textbf{Theodora}
    \item[$-$] vykupoval mír tributem Novoperské říši
    \item[532] \textsc{povstání Niká}, ustál
    \item[$-$] expanze: Apeninský pol. Středomořské ostrovy, S Afriky; obnovil území Římské říše
    \item[$-$] \textit{Corpus iuris civilis} (svod občanského práva) -- soupis římského práva
    \item[$-$] theokratický absolutismus, považoval se za \uv{prodlouženou ruku boha na zemi}
    \item[529] zákaz pohanských škol (i té v Athénách)
    \item[$-$] bazilika San Vitale v Ravenně s krásnými mozajkami
    \item[$-$] \textit{césaropapismus} = způsob vlády, není jen v čele moci světské, ale i církevní
\end{itemize}

\subsection*{Velké schizma}
\begin{itemize}
    \vspace{-0.5em}
    \setlength\itemsep{0.15em}
    \item[$-$] \textbf{Mikuláš I.} vs. \textbf{Fótios}, navzájem se sesadaili
    \item[1054] \textsc{Velké schizma}: křesťanská církev se roztrhla na dvě části (Z -- papež, latina; V -- ortodoxní, Kristus je lidské podstaty)
    \item[$-$] misie: Velká Morava 863, Bulharsko, Kyjevská Rus 988
\end{itemize}

\subsection*{7. -- 11. století, krize}
\begin{itemize}
    \vspace{-0.5em}
    \setlength\itemsep{0.15em}
    \item[$-$] ze severu příchod Bulharů pod vedením chána \textbf{Asparucha}, z jihu příchod Arabů
    \item[$-$] \textit{ikonoklasmus} = obrazoborecké hnutí, odmítá zobrazování svatých v kostelech atd.
    \item[1071] \textsc{bitva u Mantzikertu}: seldžúkští Turci porazili byzantskou armádu
    \item[1204] \textsc{čtvrtá křížová výprava}, vtrhli do Byzance, zlikvidovali Cařihrad, na místě Byzance vzniká latinské císařství, Byzantinci se ztáhli do Nikajského císařství
    \item[1261] obnovení Byzantské říše
    \item[1453] vyvrácení Byzantské říše Turky
\end{itemize}

\subsection*{Kultura}
\begin{itemize}
    \vspace{-0.5em}
    \setlength\itemsep{0.15em}
    \item[$-$] architektura: půdorys tvar kříže, mozaiky, hodně zlata
    \item[$-$] výborné školství, vysoká úroveň práva, medicíny, přírodních věd
    \item[$-$] literatura: Jordanes, Prokopios, Jan Zlatoústý
    \item[$-$] byzantský sloh: sv. Sofie v Kyjevě, San Vitale v Ravenně, \dots
\end{itemize}





\end{document}
