\documentclass{article}
\usepackage{fullpage}
\usepackage[czech]{babel}
\usepackage{amsfonts}

\title{\vspace{-2cm}Jiří z Poděbrad, Jagellonci, kultura a vzdělanost vrcholného středověku, počátek novověku, předkolumbovské civilizace, renesance, humanismus\vspace{-1.7cm}}
\date{}
\author{}

\begin{document}
\maketitle

\section*{Jiří z Poděbrad}

\begin{itemize}
    \vspace{-0.5em}
    \setlength\itemsep{0.15em}
    \item[$=$] \uv{král dvojího lidu}, t.j. vládne kališníkům i katolíkům, akceptuje jejich existenci
\end{itemize}

\subsection*{Situace před Jiřím z Poděbrad}
\begin{itemize}
    \vspace{-0.5em}
    \setlength\itemsep{0.15em}
    \item[1437 -- 1439] \textbf{Albrecht Habsburský}, manželka \textbf{Alžběta}, syn \textbf{Ladislav Pohrobek}, který je malý a nemůže vládnout $\rightarrow$
    \item[1439 -- 1453] \textit{husitské interregnum} = bezvládí
    \item[1440] \textit{Mírný list} -- dohoda o neválčení mezi husity a křižáky
    \item[$-$] české království rozděleno na \textit{landfrídy} (kraj)
    \item[$-$] nejvýznamnější rody: páni z Kunštátu a Poděbrad, Rožmberkové
    \item[1448] \textbf{Jiří z Poděbrad} (z rodu kališníků) \textsc{dobývá Prahu} a stává se zemským správcem (1452)
    \item[1453 -- 1457] Ladislav Pohrobek, jako třináctiletý, o čtyři roky později těsně před svatbou (s francouzskou princeznou) umírá
    \item[1458] zemský sněm zvolil za krále \textbf{Jiřího z Poděbrad}
    \item[$-$] ve stejném roce zvolen králem uherským \textbf{Matyáš Korvín}
\end{itemize}

\subsection*{Vláda}
\begin{itemize}
    \vspace{-0.5em}
    \setlength\itemsep{0.15em}
    \item[$-$] ekonomická obnova Čech, podpora obchodu, likvidace lupičů, zavedení velké daně, právo na držení pozemků (???)
    \item[$-$] akceptuje kališníky i katolíky, sám patří k umírněným kališníkům
    \item[$-$] pronásleduje členy Jednoty bratrské
    \item[1462] papež prohlásil, že ruší platnost \textit{Basilejských kompaktát} a uvrhl Jiřího do klatby (1466), protože je podporoval i po jejich zrušení
    \item[1466] křížová výprava po celé Evropě, cíl: najití spojenců pro boj s \textit{papežskou stolicí}, vede jeho švagr \textbf{Zdeněk Lev z Rožmberka}, účastnil se jí i Václav Šašek z Bířkova
    \item[$-$] této situace využívá \textbf{Matyáš Korvín}, král uherský
    \item[1471] umírá, chce, aby nastoupil \textbf{Vladislav Jagellonský}
\end{itemize}


\subsection*{Matyáš Hunyadi (Hyundai) Korvín}
\begin{itemize}
    \vspace{-0.5em}
    \setlength\itemsep{0.15em}
    \item[$-$] Jednota zelenohorská = spolek poněmčených českých měst
    \item[1468] napadá Moravu
    \item[1469] rozhodující \textsc{bitva u Vilémova}, vítězí Jiří z Poděbrad
    \item[$-$] téhož roku Matyáš prohlášen v Olomouci za českého krále, Jiří ztrácí Moravu, Srbsko a Lužici, tedy zbývají mu jen Čechy

\end{itemize}

\section*{Jagellonci}

\begin{itemize}
    \vspace{-0.5em}
    \setlength\itemsep{0.15em}
    \item[1471] zemský sněm zvolil 1471 \textbf{Vladislava Jagellonského} (na doporučení Jiřího z Poděbrad)
\end{itemize}

\subsection*{Vladislav II. Jagellonský (1471 – 1516)}
\begin{itemize}
    \vspace{-0.5em}
    \setlength\itemsep{0.15em}
    \item[$-$] Jiří se dostal do sporu s papežskou stolicí: chtěl aby byla dodržována bazilejská kompaktáta, papež je zrušil, dal Jiřího do klatby a vyhlásil na něj křížovou výpravu $\rightarrow$
    \item[1469] \textsc{bitva u Vilémova}, Jiří vyhrál, Matyáš Korvín se ovšem přesto nechá korunovat králem, drží Moravu a Slezsko
    \item[$\Rightarrow$] do 1471 drží Vladislav II. jen Čechy
    \item[$-$] bojoval s Matyášem, k ničemu to nevedlo, jen se potvrdil status quo
    \item[1479] \textit{Olomoucké dohody}: oba budou mít titul krále českého, až jeden z nich zemře, přejde titul na druhého z nich
    \item[1490] Matyáš Korvín umírá; uherští stavové souhlasili, že králem Uher bude Vladislav II. Jagellonský $\rightarrow$ českouherská personální unie (jen do roku 1526)
    \end{itemize}

\noindent Pozn.: v Uhrách měli silný vliv magnáti, zvolili si za krále Matyho Korvína, aby ochránil zemi před Osmany, ten je ale silný až moc, \uv{zkrotil} je; v Uhrách byl tehdy na dvoře spisovný jazyk čeština, silný vliv bratříčků

\begin{itemize}
    \vspace{-0.5em}
    \setlength\itemsep{0.15em}
    \item[$-$] vláda: slabá, sídlil v Budíně (v Uhrách), silní stavové, stavovská monarchie
    \item[$-$] dochází ke změnám v hospodářství: šlechta začala podnikat; vznikají velkostatky, zakládají sklárny, hutě, rybníky (významní zakldatelé rybníků: Josef Štěpánek Netolický, Jakub Krčín z Jelčan -- založil Rožmberk), pěstuje vinné révy
    \item[$-$] rozvíjí se města co založila šlechta -- poddanská, konkurenti královská města, šlechta chce zbavit královská města místa na zemském sněmu (královská města posílila během husitských válek)
    \item[$-$] začali jsme znovu obchodovat se zahraničím, obnova po husitských válkách, vystupujeme z izolace
    \item[$-$] roste počet poddaných $\rightarrow$ řada povstání:
    \begin{enumerate}
        \vspace{-0.5em}
        \setlength\itemsep{0.15em}
        \item \textbf{Dalibor z Kozojed}: zeman města na Litoměřicku, ujal se nevolníků, kteří utíkali od svého pána ze sousedního města, poté je odmítal vydat; byl odsouzen, umístěn do vězení na věž, které se po něm říká Daliborka
        \item \textbf{Dóžovo povstání}: velké povstání v Uhrách, chystala se křížová výprava proti Osmanům, nachystaná armáda se změnila ve velké protifeudální povstání, v čele \textbf{Jiří Dóža}, krutě pobito
    \end{enumerate}
    \item[$-$] Vladislav II. je katolík, podporuje katolíky, narůstá napětí, především v Praze $\Rightarrow$
    \item[1483] \textsc{2. pražská defenestrace}, lidé různě po Praze zaútočili na katolické konšely i na katolické kostely opravené po husitských válkách
    \item[1485] náboženský smír v Kutné Hoře; mezi katolíky a kališníky, cílem klid v zemi
    \item[1500] \textit{Vladislavské zřízení}; listina, kterou vydala šlechta, chtěla zbavit královská města práv a míst na zemském sněmu, chtěla hospodářská privilegia pro poddanská města
    \item[1517] \textit{Svatováclavská smlouva}: \uv{kompromis}, ale výhodný pro šlechtu, dostala vlastně vše co chtěla, ale stavové z královských měst zůstanou na zemském sněmu

\end{itemize}


\subsection*{Ludvík Jagellonský (1516 – 1526)}
\begin{itemize}
    \vspace{-0.5em}
    \setlength\itemsep{0.15em}
    \item[$-$] nastupuje 10letý, anarchie, slabá vláda
    \item[$-$] sestra Anna Jagellonská
    \item[1526] zahynul v \textsc{bitvě u Moháče}, boj proti Osmanům, které vedl Sulejman Nádherný


\end{itemize}

\subsection*{Kultura}
\begin{itemize}
    \vspace{-0.5em}
    \setlength\itemsep{0.15em}
    \item[$-$] Klaudiánova mapa: neodpovídá realitě, sever je dole
    \item[$-$] Matěj Rejsek: Prašná brána, chrám Sv. Barbory v Kutné Hoře
    \item[$-$] Benedikt Ried (Rejt): Vladislavský sál na Pražském hradě
    \item[$-$] Antonín Pilgram: kostel Sv. Jakuba, portál staré brněnské radnice
\end{itemize}



\section*{Kultura a vzdělanost vrcholného středověku}
\begin{itemize}
    \vspace{-0.5em}
    \setlength\itemsep{0.15em}
    \item[$-$] \textit{scholastika} = filozofické učení, obhajuje teologické teze
    \item[$-$] za začátek považován dvůr Karla Velikého ve vrcholném středověku (8. st.)
    \item[$-$] ze začátku ovlivněno Platónem, poté je dominantní Aristoteles
    \item[$-$] 2 proudy: racionalistický, podle \textbf{Tomáše Akvinského} \textit{tomismus}, \uv{k pravdě lze dojít dvěma způsoby, které se nevylučují: vírou a rozumem}, oproti tomu iracionalistický směr, podle \textbf{Jan Duns Scotus} \textit{scotismus}, \uv{víra je nadřazená rozumu}
    \item[$-$] nositeli jsou mnišské řády: \textit{dominikání} (školství, inkvizice, podle sv. Dominika) vs. \textit{františkáni} (bohatí, vlivní, pokora a láska k rozumu, podle Františka z Assisi)
    \item[$-$] reformní řády (z benediktínů, podle Benedikta z Nursie): \textit{cisterciáci} (kolonizovali a kultivovali půdu, kult Panny Marie, podle města Citeaux ve Francii) a \textit{premonstráti} (kazatelé, podle města Prémonstré ve Francii)
    \item[$-$] školy: při farách a klášterech, ve městech tzv. \textit{partikulární} školy (učí se základy jako číst, psát, počítat), univerzity (studium sedmera svobodných umění: právnická, lékařská, teologická fakulta)
    \item[$-$] na učitele a profesory se vztahuje kanonické právo (sice netuším, co to znamená, ale)
    \item[$-$] nejstarší univerzity: Bologna, Padova, Sorbonna, Salamanca, Cambridge, Oxford, Praha, Krakov, Vídeň, Lipsko
\end{itemize}

\subsection*{Rytířská kultura}
\begin{itemize}
    \vspace{-0.5em}
    \setlength\itemsep{0.15em}
    \item[$-$] vzniká na jihu Francie někdy ve 12. století, rytířské turnaje
    \item[$-$] ušlechtilé dvorní chování, věrný svému pánovi, chrání slabé, kult ženy
    \item[$-$] \textit{trubadúři} (J Francie), \textit{truvéři} (Francie), \textit{minnesängři} (Německo)
    \item[$-$] Zbraslavská, Dalimilova kronika, cestopis Milion, ...
\end{itemize}


\subsection*{Gotika}
\begin{itemize}
    \vspace{-0.5em}
    \setlength\itemsep{0.15em}
    \item[$-$] periodizace: \textit{raná} (přechodné období), \textit{vrcholná} (13. -- 14. st.), \textit{pozdní} (15. st.)
    \item[$-$] vzniká ve Francii v okolí Paříže, první stavba Saint Denis, název pejorativní podle Gótů (barbarské), vznikl až za renesance
    \item[$-$] u nás: \textit{přemyslovská} (13. st.), \textit{lucemburská} (14. st.), \textit{jagellonská} (15. st.)
    \item[$-$] stavby \textit{sakrální} a \textit{světské}
\end{itemize}

\subsubsection*{Architektura}
\begin{itemize}
    \vspace{-0.5em}
    \setlength\itemsep{0.15em}
    \item[$-$] základní znaky: lomený oblouk, žebrová klenba, opěrný systém, vertikalita, bohaté zdobení, členitost, stavěno z kamene, vysoké cihly, tzv, \textit{buchty}
    \item[$-$] \textit{fiála} = špička, jehlan, na kterém ledacos může být, \textit{krakorec} = nosný článek vystupující ze zdi, slouží k podpěře
    \item[$-$] druhy staveb: katedrály, kláštery, kostely, hrady, městská opevnění, městské stavby (radnice, měšťanské domy, mosty)
    \item[$-$] \textit{katedrála} = biskupský nebo arcibiskupský kostel, honosnější, např. Chrám Matky Boží v Paříži
    \item[$-$] Katedrála Sv. Víta. Vojtěcha a Václava: architekt Matyáš z Arrasu, Petr Parléř, Josef Mocker, ale dokončil až ve 20. st. Kamil Hilbert
\end{itemize}

\subsubsection*{Sochařství}
\begin{itemize}
    \vspace{-0.5em}
    \setlength\itemsep{0.15em}
    \item[$-$] sochy stojí ve volném prostoru, buď ze dřeva nebo z kamene, \textit{polychromie}  = vícebarevné, mají výraz
    \item[$-$] náměty: \textit{madona} = Panna Marie v náručí s Ježíšem, \textit{pieta} =
Panna Marie drží v náručích Ježíše sňatého z kříže
    \item[$-$] socha Svatého Jiří, jak bojuje s drakem

\end{itemize}


\subsubsection*{Malba}
\begin{itemize}
    \vspace{-0.5em}
    \setlength\itemsep{0.15em}
    \item[$-$] Giotto di Bondone: typicky modrá barva
    \item[$-$] lidé mají individuální rysy, nejsou všichni stejní
    \item[$-$] Přebohaté hodinky vévody z Berry: překrásně zdobená kniha modliteb, vrchol gotické knižní malby
    \item[$-$] u nás: Mistr vyšebrodského, třeboňského oltáře, Mistr Theodorik
    \item[$-$] knižní malba: Velislavova bible, Bible Václava IV.
\end{itemize}


\section*{Počátek novověku}
\begin{itemize}
    \vspace{-0.5em}
    \setlength\itemsep{0.15em}
    \item[1492] mezník mezi středověkem a novověkem
\end{itemize}

\subsection*{Příčiny}
\begin{itemize}
    \vspace{-0.5em}
    \setlength\itemsep{0.15em}
    \item[$-$] Osmané brání v cestě na Orient
    \item[$-$] do Evropy se dováží koření, bavlna, porcelán, čaj, hedvábí, které najednou nemají
    \item[$-$] v Evropě došly zdroje zlata a stříbra
    \item[$-$] touha po zisku území za účelem zlepšení pozice, odbytiště pro výrobky
\end{itemize}

\subsection*{Předpoklady}
\begin{itemize}
    \vspace{-0.5em}
    \setlength\itemsep{0.15em}
    \item[$-$] posloužila mapa Eratosthena z Kyrémy z doby helénistické
    \item[$-$] mapa \textbf{Toscanelliho}, též oživil představu toho, že Země je kulatá
    \item[$-$] znalost kompasu od Arabů, \textit{astroláb} = určuje polohu hvězd podle polohy Slunce
    \item[$-$] nové typy lodí, \textit{karaky} či \textit{karabely}: mají hlubší ponor $\rightarrow$ stabilnější, hlubší podpalubí
    \item[$-$] \textbf{Martin Boheim}, autor prvního globu
    \item[$-$] na Pyrenejském poloostrově skončila reconquista
\end{itemize}

\subsection*{Koncepce}
\begin{itemize}
    \vspace{-0.5em}
    \setlength\itemsep{0.15em}
    \item[$-$] jak se dostat do Indie?
    \item[a.] plout kolem Afriky
    \item[b.] plout pořád na západ a prostě na ni \uv{narazit}
    \item[$-$] vítězí první možnost obeplouvání Afriky
    \item[1487] \textbf{Bartolomeo Díaz} (Portugalec) dorazí k Mysu Dobré naděje
    \item[1497] \textbf{Vasco da Gama} (Portugalec) se dostane až na západní pobřeží Indie, opěrný bod Kalikat
    \item[$-$] podél trasy zakládány obchodní stanice, dováží koření
    \item[1500] \textbf{Pablo Álvárez Cabral} (Portugalec) doplouvá do Brazílie, považován za objevitele
    \item[$-$] od 16. století pronikají na území dnešní Číny, Japonska
\end{itemize}

\subsection*{Španělsko}
\begin{itemize}
    \vspace{-0.5em}
    \setlength\itemsep{0.15em}
    \item[$-$] strategie: plují na západ a doufají, že do Indie dorazí
    \item[$-$] Isabela Kastilská a ferdinand Aragonský podporují Kryštofa Kolumba
    \item[$-$] záchytný bod Kanárské ostrovy
    \item[1492] Kryštof Kolumbus vyplouvá z přístavu Palos se třemi loďmi, po nějaké době doplouvají ke Karibiku, pak k Hispaniole
    \item[$-$] Kolumbus umírá v chudobě zapomenutý
    \item[$-$] \textbf{Amerigo Vespuci} začal popisovat, co na výpravě s Kolumbem vidí, začíná se mluvit o zemi Amerigově $\rightarrow$ Amerika, termín už od roku 1507
    \item[$-$] Mercatorova mapa
    \item[1500] \textbf{Vicente Pinzón} objevuje ústí Amazonky
    \item[1513] \textbf{Vasco de Balboa} překračuje Panamskou šíji -- potvrzení objevu nového kontinentu
    \item[1519] \textbf{Fernao Magalhaes} obeplul celou zeměkouli, Portugalec působící ve španělských službách, čímž doložil, že je kulatá
\end{itemize}


\subsection*{Koloniální válka}
\begin{itemize}
    \vspace{-0.5em}
    \setlength\itemsep{0.15em}
    \item[$-$] o nově objevená území mezi Španěli a Portugalci
    \item[$-$] 2000 km západně od Kapverdských ostrovů linie, na západ od ní Španělské a na východ Portugalské
    \item[$-$] od 16. století začínají pronikat mimo Evropu nová koloniální vlastníci: Nizozemci, Britové
\end{itemize}

\subsection*{Důsledky}
\begin{itemize}
    \vspace{-0.5em}
    \setlength\itemsep{0.15em}
    \item[$-$] do Evropy příliv zlata a stříbra
    \item[$-$] vytvoření světového obchodu, zánik tradičních obchodních center na Ap. pol. a přesouvají se do Evropy
    \item[$-$] přivezení nových nemocí, třeba Syfilis
\end{itemize}


\section*{Předkolumbovské civilizace}
\subsection*{Inkové}
\begin{itemize}
    \vspace{-0.5em}
    \setlength\itemsep{0.15em}
    \item[$-$] Latinská Amerika, Andy, údolí Cuzco; nejstarší osídlení v této oblasti: 1. tis. př. n. l.
    \item[$-$] v 15. st. mají minimálně 10 000 000 obyvatel
    \item[$-$] nejznámější místo: Machu Picchu
    \item[$-$] Titicaca (jezero), goeglyfy na náhorní plošině Nazca
    \item[$-$] skvělí stavitelé -- bez pojiva, jen kvádry -- chrámy, silnice, terasovitá pole
    \item[$-$] kalendář 365,25 dne, uzlíkové písmo -- Kipu, kovy, ale ne kolo, chovají lamy a taky alpaky, kult boha slunce -- zlato
    \item[1531 -- 1535] zánik: Francisco Pizarro, Diego Almagro, založili Limu

\end{itemize}

\subsection*{Mayové}
\begin{itemize}
    \vspace{-0.5em}
    \setlength\itemsep{0.15em}
    \item[$-$] poloostrov Yucatán, Guatemala, Salvádor
    \item[$-$] počátek letopočtu -- městské státy
    \item[$-$] pyramidy Chichén Itza, Palanqe, kalendář $18 \times 20 + 5$, znali 0, Mayské číslice, neznali kovy, lidské oběti, hieroglyfy

\end{itemize}

\subsection*{Aztékové}
\begin{itemize}
    \vspace{-0.5em}
    \setlength\itemsep{0.15em}
    \item[$-$] Mexiko, hl. m. Tenochtitlan, až 10 000 000 ob.
    \item[$-$] hieroglyfy, akao i jako platidlo
    \item[1519 -- 1521] zánik: Hernán Cortés

\end{itemize}



\section*{Renesance}
\begin{itemize}
    \vspace{-0.5em}
    \setlength\itemsep{0.15em}
    \item[$-$] 14. – 16. st.
    \item[$\approx$] humanismus; znovuzrození, návrat k antické kultuře
    \item[$-$] rozkvět severoitalských měst (Florencie, Benátky, Janov, Řím)
    \item[$-$] \textbf{antropocentrismus} = obrat k člověku a jeho schopnostem
    \item[$-$] nárůst světské moci, rozvoj poznání, obchodu
    \item[$-$] italové se seskupuji do politicko-vojenských skupin = \textbf{guelfové} (za SSŘ) a \textbf{ghibelini} (ZA papeže)
\end{itemize}

\begin{enumerate}
    \vspace{-0.5em}
    \setlength\itemsep{0.15em}
    \item \textit{Torecento} (14. st.)
        \begin{itemize}
            \vspace{-0.5em}
            \setlength\itemsep{0.15em}
            \item[$-$] na pomezí mezi středověkem a renesancí
            \item[$-$] cílem je vytvořit silnou společnost a sjednotit Itálii
            \item[$-$] Dante Alighieri (spisovatel, politik); Cola di Rienzo, Francesco Petrarca (chtěli sjednotit Apeninský polostrov); Giovanni Boccaccio: Dekameron; Giotto di Bondone
        \end{itemize}
    \item \textit{Quattrocento} (15. st.)
    \begin{itemize}
        \vspace{-0.5em}
        \setlength\itemsep{0.15em}
        \item[$-$] pochopili, že sjednotit je nemožné, nejvýznamějším městem je Florencie:
        \item[$-$] Cosimo de Medici (republikánské zřízení = \textbf{signoire})
        \item[$-$] Lorenzo de Medici (největší rozkvět Florencie)
        \item[$-$] Filippo Brunelleschi (dóm ve Florencii)
        \item[$-$] Donatello: bronzový David, Sandro Botticelli: Zrození Ven(o)uše, Primavera
        \item[$-$] architektura: naprosto horizontální, prvky antiky, geometričnost, symetrie, \textbf{arkády} = podloubí, \textbf{sala terrena} = přízemní místnost otevřená do zahrady, \textbf{rustika} = zdivo z neopracovaných kvádrů
        \item[] přelom 15. a 16. st.:
        \item[$-$] Leonardo da Vinci: vynálezce, malíř, architekt, sochař, působil zejména ve FLorencii a pak u rodu Sforzů v Miláně, zemřel ve Francii; Madonna v jeskyni, dáma s hranostajem, Svatá Anna, Vitruvius = dokonale proporční člověk, Mona Lisa, Poslední večeře Páně (v Miláně)
        \item[$-$] Raffael Santi: Sixtinská Madonna (ideál tělesné a duševní krásy), Athénská škola, výzdoba Vatikánu
        \item[$-$] Michelangelo: sochař, malíř, architekt, ale psal i básně, spojen s Florencií a Římem; stavba Svatopetrské baziliky, Sixtinská kaple (soukromá kaple jednoho z papežů), palác Farnese v Římě, palác Senátorů v Římě, socha Mojžíše, Pieta, Davida
    \end{itemize}
    \item \textit{Cinquecento} (16. st.)
    \begin{itemize}
        \vspace{-0.5em}
        \setlength\itemsep{0.15em}
        \item[$-$] spojeno s Benátkami
        \item[$-$] malíři znají anatomii lidského těla a nebojí se přiznat lidskou nahotu, realistické obrazy
        \item[$-$] Masaccio (patří do kvatročenta wtf)
        \item[$-$] Tizian: Venuše urbinská, Apollo a Marsyas (v arcibiskupském zámku v Kroměříži), autoportrét
        \item[$-$] Tintoretto
    \end{itemize}
\end{enumerate}


\section*{Humanismus}
\begin{itemize}
    \vspace{-0.5em}
    \setlength\itemsep{0.15em}
    \item[$-$] myšlénkový proud, \textit{humanus} = lidský
    \item[$-$] za prvního představitele považován Francesco Petrarca
    \item[$-$] lidé už se nezabývají jen studiem teologie, ale i studiem odvětví ryze lidských = studia \textit{divina}
    \item[$-$] \textit{ad fontes} = k pramenům (ne jen poslouchat komentáře, ale umět řecky a přečíst si díla sám)
    \item[$-$] základní knihou je bible, všechny knihy vytištěny do roku 1500 = \textit{inkunábule}, jsou nesmírně drahé
    \item[$-$] Kronika trojanská
\end{itemize}

\subsection*{Filosofie}
\begin{itemize}
    \vspace{-0.5em}
    \setlength\itemsep{0.15em}
    \item[$-$] Lorenzo Valla: mezi prvními říkal, že Konstantinova donace (díky které církev odvozuje svoje právo na Řím) je faleš, že to nepsal Konstantin, spis O slasti (nejlepší dobro pro člověka je slast)
    \item[$-$] Pietro Pomponazzi, Niccolló Machiavelli (vladař)
    \item[$-$] \textit{platonismus} = ve Florencii platonská akademie
\end{itemize}

\subsection*{Věda}
\begin{itemize}
    \vspace{-0.5em}
    \setlength\itemsep{0.15em}
    \item[$-$] svět není konečný a vše se netočí kolem Země
    \item[$-$] Mikuláš Koperník: heliocentrická teorie, rotace Země kolem své osy
    \item[$-$] Johannes Kepler: Keplerovy zákony, planety neobíhají po kružnici, ale po elipse, zkonstruoval dalekohled
    \item[$-$] Tycho de Brahe: astronom na dvoře Rudolfa II., pohřben v kostele Panny Marie před Týnem
    \item[$-$] Galileo Galilei: dalekohled, Měsíc není pravidelná koule, nakonec pronásledován církví, dožil v domácím vězení, \uv{a přece se točí}, což nejspíš neřekl
    \item[$-$] Giordano Bruno: dominikán, vzdělanec, vesmír je nekonečný, upálen
\end{itemize}

\subsection*{Anatomie}
\begin{itemize}
    \vspace{-0.5em}
    \setlength\itemsep{0.15em}
    \item[$-$] poskočila znalost lidského těla v rámci prvních pitev
\end{itemize}

\subsection*{Společenské vědy}
\begin{itemize}
    \vspace{-0.5em}
    \setlength\itemsep{0.15em}
    \item[$-$] Erasmus Rotterdamský: kritizuje nežádoucí jevy ve společnosti, sepsal v díle Chvála bláznivosti
    \item[$-$] Jan Amos Komenský: filosof, pedagog, Labyrint světa a ráj srdce
    \item[$-$] \textit{utopie} = něco neuskutečnitelného v praxi, nereálného
    \item[$-$] Thomas More: Utopia, líčí ideální společnost
    \item[$-$] Thomaso Campanela: utopie sepsána v díle Sluneční stát
\end{itemize}
\subsection*{Reformace v 16. st. v Německu}
\begin{itemize}
    \vspace{-0.5em}
    \setlength\itemsep{0.15em}
    \item[$=$] snaha o nápravu, obnovení církve
    \item[$-$] žít v chudobě a čistotě, vyřešit papežské schizma, ne svatokupectví, ne korupci v církvi, ne světské moci v církvi
    \item[$-$] \textbf{konciliarismus} = svolávání koncilů
    \item[$-$] \textbf{papalismus} = svolávání kocilů papežem
    \item[$-$] šlechta si dělá zálusk na majetek církve
    \item[$-$] tzv. \textit{první reformace} \textbf{John Wyclif} 14. st.
    \item[$-$] tzv \textit{česká reformace} 15. st. $\rightarrow$ rozdělení církve
    \item[$-$] \textbf{Martin Luther} kazatel, tolog, reformátor
    \begin{itemize}
        \vspace{-0.5em}
        \setlength\itemsep{0.15em}
        \item[1517] přibil na dveře kláštera v Wittenberku 95 tezí proti církvi, kvůli knihtisku se rychle šíří
        \item[$-$] proti odpustkům, kritizuje církev "mezi člověkem a bohem nemusí být nic", přeložil bibli do němčiny, chtěl pod obojí, proti celibátu
        \item[1520] papež \textbf{Lev X.} vydává bulu s hrozbou do kladby \implies Luther veřejně spálil, 1521 papež ho exkomunikoval
        \item[1521] \textbf{Karel V.} vydává \textsc{Wormský edikt}
        \item[$-$] \textbf{Fridrich III. Moudrý}, saský kurfiřt umožnil bydlet na Wartbugu Lutherovi
        \item[1522] \textsc{zakládá lutheránství}
    \end{itemize}
    \item[$-$] Filip Melanchon -- přítel a poradce Luthera
    \item[1524 -- 1526] \textsc{německá selská válka}, cílená proti vrchnosti, chtějí snížit daně a odstranit robotu, lovit ryby v panských rybnících -- sociální, nikoliv náboženské požadavky, spojeno s \textit{chiliasmem}; aktivity nevzdělaných jedinců, významný vůdce: \textbf{Tomáš Müntzer} (kazatel)
    \item[1525] povstání potlačeno v Durynsku, o rok později i v ostatních částech
    \item[1533 -- 34] \textit{Münsterská komuna} -- nejradikálnější program reformačního hnutí, poražen po 16 měsících
    \item[1529] \textsc{říšský sněm ve Špýru} - katolíci se snaží potlačit luterýny $\rightarrow$ bránili se, že hlasovat o něčí víře se nedá $\rightarrow$ označení \textit{protestanti}
    \item[1531] \textit{Šmalkaldský svaz} = vojenské sjednocení protestantů
    \item[1546 -- 47] \textsc{šmalkadská válka}, protestanti proti katolíkům, \textsc{bitva u Mühlbergu}, porážka protestantů
    \item[$-$] lutheránství se ale v SŘŘ šíří dál
    \item[1555] \textsc{Augspurský mír}: čí vláda, toho víra (\textit{cuius regio, eius religio}), Karel V. s tím není ztotožněný $\rightarrow$
    \item[$-$] důsledek: šlechta se může rozhodnout, poddaní mají víru svého pána
    \item[1556] Karel V. odestupuje $\rightarrow$ faktické vítězství protestantů
    \item[$-$] důsledek: SŘŘ je rozdrobená

\end{itemize}

\subsection*{Šíření luteránství}
\begin{itemize}
    \vspace{-0.5em}
    \setlength\itemsep{0.15em}
    \item[$-$] v těchto státech: Skandinávie (Švédsko, Dánsko), Podabltí (polská města jako Toruň, Poznaň, hanzovní města Talin a Riga), Uhersko (Slovensko, Sedmihradsko), České země (německy mluvící města ve Slezsku a v Čechách)
\end{itemize}


\subsection*{Švýcarská reformace (Měšťanská reformace)}
\begin{itemize}
  \item[$-$] v této době probíhal švýcarský boj o nezávislost na SŘŘ v 20. letech 16. st., válka mezi jednotlivými švýcarskými kantony (katolické x švýcarské)
  \item[$-$] \textbf{Ulrich Zwingli} (1484-1531) -- vůdce švýcarských kantonů, radikálnější než Luther, důraz na morálku, poddanstvo má právo se vzbouřit, když se šlechta nechová správně, odmítají katolické tradice (přijímání, ikony), pak padl v boji
  \item[$-$] \textbf{Jan Kalvín} (1509-1564) -- teolog francouzského původu, útočiště v Ženevě, vlivy Luthera, Zwingliho, učení předdestinace -- je už o člověku od začátku určeno, zda bude spasen nebo zatracen, ale člověk to neví, takže by se měl chovat správně, v průběhu života se dovídá zda je ten bohem vybraný
  \item[$-$] \textit{kalvinismus} -- důraz na pracovitost, skromnost, individualitu, podnikavost - jakýsi prvopočátek (dejme tomu) kapitalistického, podnikavého mindsetu, pevné morální zásady - prostě se v životě nebav, nebo nebudeš spasen
  \item[$-$] kalvinismus rozšířen v Anglii (významné pro osidlování S Ameriky), Uhersku, ...
\end{itemize}

\subsection*{Protireformace}
\begin{itemize}
  \item[$-$] Katolíci chtějí zpátky získat ztracené pozice, zabránit šíření nekatolických církví, snaží se o přestavbu katolické církve
  \item[$-$] se souhlasem papeže byl r. 1540 založen nový řád -- \textbf{Jezuité} (Societas Jesu), zakladatel Ignác z Loyoly, přísný řád, odpovídají jen papežovi, účely rekatolizace, misionářství, zřizovali a vedli většinu škol
  \item[$-$] r. 1542 založení \textit{Svaté inkvizice} -- no, inkvizice, dále zakazují některé knihy, píší jejich seznamy (Index librum prohibitorum)
  \item[$-$] \textsc{Tridentský koncil} (1545-1563) -- přelom ve fungovaní katolické církve, odmítnuta radikální reformace, jediná autorita je papež, ale musí se omezit hromadění církevních úřadů v rukou jednotlivce a že duchovní musí být vzdělaní -- biskupské semináře pro výchovu nových kněží
\end{itemize}




\end{document}
