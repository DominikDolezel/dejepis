\documentclass{article}
\usepackage{fullpage}
\usepackage[czech]{babel}
\usepackage{amsfonts}

\title{\vspace{-2cm}Anglická revoluce (1640-1688)\vspace{-1.7cm}}
\date{}
\author{}

\begin{document}
\maketitle

\section*{Jakub I. (1603-1625)}
\begin{itemize}
    \vspace{-0.5em}
    \setlength\itemsep{0.15em}
    \item[$-$] nastupuje po Alžbětě I., personální unie Anglie a Skotska
    \item[$-$] podporuje podnikání, rozvoj obchodu a rybolovu na úkor Nizozemí
    \item[$-$] jeho dcera Alžběta byla manželkou Fridricha Falckého
    \item[$-$] za něj vypukne třicetiletá válka, působí Shakespeare, Francis Bacon (filozof, vědec, spisovatel)
\end{itemize}

\section*{Karel I. (1625-1649)}
\begin{itemize}
    \vspace{-0.5em}
    \setlength\itemsep{0.15em}
    \item[$-$] syn Jakuba I., snaží se vládnout bez parlamentu, prosazuje absolutismus
    \item[1629] rozpuštěn parlament
    \item[$-$] mimořádné daně nebo dávky bez souhlasu parlamentu, půjčky na měšťanech
    \item[$-$] uplatňuje panovnický monopol na prodej mýdla, vína, uhlí
    \item[$-$] tyto aktivity iniciovaly vytvoření opozice, která je rekrutována z řad kalvinismu:
    \begin{itemize}
        \vspace{-0.5em}
        \setlength\itemsep{0.15em}
        \item[$-$] \textit{puritáni} se snaží očistit církev v duchu kalvinismu, hlavní jádro opozice, významní vlastníci půdy, díky svým názorům pronásledováni
        \item[$-$] \textit{presbyteriáni} připouští zachování konstituční monarchie, třeba bohatí bankéři, obchodníci, chtějí změnit hierarchii stávající církve, chtejí, aby v jejím čele byli volení \textit{presbyteři}
        \item[$-$] \textit{independenti} vůdcem je Oliver Cromwell, chtějí republiku, třeba majitelé manufaktur, menší obchodníci, chtějí volný výklad bible nezávislý na presbyterech
        \item[$-$] \textit{levelleři} chtějí rovnost mužů, vypracovali ústavu (ta nevešla v platnost), chtějí republiku, prosazují náboženskou svobodu
    \end{itemize}
    \item[$-$] ve Skotsku je kalvinismus, Irsko je katolické, Stuartovci tu násilím zaváději anglikánskou církev
    \item[1639] \textsc{Skotské povstání}, skotská armáda vpadla ze severu do Anglie, Karel svolá parlament a chce, aby odsouhlasil daně, aby mohl vytvořit armádu, která porazí Skoty, parlament však odmítá poslušnost a chce potrestat královy rádce, během 14 dnů Karel parlament opět rozpustil = \textit{krátký parlament}
    \item[1640] Karel opět svolává parlament, postupně ho ovládnou presbyteři, parlament nebyl rozpuštěn až do roku 1653 = \textit{dlouhý parlament}
    \item[1642] pokus Karla I. zatknout vůdce opozice v parlamentu, neúspěšné, utíká z Londýna do Osxfordu, začíná otevřená občanská válka
    \item[$-$] stoupenci krále: vyznavači anglikánské církve, stoupenci parlamentu: nová šlechta, kalvinisté, Londýn, Cromwellova armáda (budována na základě dobrovolnosti)
    \item[1644] \textsc{válka u Marston Moor} obrat ve válce, vyhrává armáda parlamentu, velí jí Cromwell
    \item[1645] \textsc{bitva u Naseby} vítězí parlament, král prchá do Skotska
    \item[1647] SKoti krále vydávají parlamentu
    \item[$-$] byl stvořen nový parlament, do kterého se dostanou radikálové (independenti, levelleři) vs. umírnění (presbyteři), radikálové vyhráli $\Rightarrow$
    \item[30.1.1649] král popraven
\end{itemize}

\section*{Oliver Cromwell (1649-1658)}
\begin{itemize}
    \vspace{-0.5em}
    \setlength\itemsep{0.15em}
    \item[1649] Anglie republikou, stanul v čele republiky po smrti Karla I.
    \item[1653-58]  rozehnal parlament, nastolil vojenskkou diktaturu $\Rightarrow$ \textit{lord protektor}
    \item[$-$] posílení moci, oporou armáda a podnikatelé
    \item[1651] \textit{navigační akta} = do Anglie může dovážet zboží buď anglická loď, nebo ta, ze které to zboží je $\Rightarrow$ Nizozemci nemohou, protože zboží jen přeprodávají
    \item[$-$] po jeho smrti otevřen prostor k obnovení monarchie
\end{itemize}

\section*{Karel II. (1660-1685)}
\begin{itemize}
    \vspace{-0.5em}
    \setlength\itemsep{0.15em}
    \item[1660] stoupenci monarchie v čele s generálem Monckem obnovují monarchii, nastupuje syn Karla I. ze Stuartovců
    \item[1679] \textit{Habeas Corpus Act} = nelze někoho věznit bez doloženého obvinění
    \item[1665-1666] mor (100 000 obětí) a požár v Londýně , díky požáru se Londýn zbavil morové epidemie
\end{itemize}

\section*{Jakub II. (1685-1688)}
\begin{itemize}
    \vspace{-0.5em}
    \setlength\itemsep{0.15em}
    \item[$-$] bratr Karla II., prosazuje absolutismus, sesazen parlamentem
    \item[$-$] parlament se obrací na \textbf{Viléma III. Oranžského} (zeť Jakuba II.), dochází k dočasnému spojení Nizozemí a Anglie, vylodí se v Anglii $\Rightarrow$
    \item[1688] \textsc{Slavná revoluce}, nastupuje Vilém Oranžský, chce se odvděčit parlamentu $\Rightarrow$
    \item[1689] \textit{Bill of Rights} = jednoznačně se dělí o moc s parlamentem, nemůže rozhodovat bez jeho souhlasu
    \item[$-$] v dolní sněmovně \textit{torryové} (konzervativní, velkostatkáři), \textit{whigové} (liberální, podnikatelé, buržoazie)
    \item[$-$] zrušení cenzury, prostor pro veřejné mínění
    \item[1701] Vilém nemá následníka trůnu $\Rightarrow$ \textit{Dekret o nástupnictví}, nastoupí Hannoverská dynastie, před nimi však ještě
    \item[$-$] \textbf{Anna Stuartovna} (1702-1714), za její vlády 1707 z personální unie vzniká jednotný stát SPoijené království Velké Británie, 1713 zisk Gibraltaru
\end{itemize}

\section*{Jiří I. Hannoverský}
\begin{itemize}
    \vspace{-0.5em}
    \setlength\itemsep{0.15em}
    \item[$-$] Hannoversko dočasně spojeno s Anglií, personální unie
    \item[$-$] z německého prostředí $\Rightarrow$ problém komunikace krále a parlamentu $\Rightarrow$ krále zastupuje premiér, král tedy moc nevládne, roste vliv parlamentu
\end{itemize}


\end{document}
