\documentclass{article}
\usepackage{fullpage}
\usepackage[czech]{babel}
\usepackage{amsfonts}

\title{\vspace{-2cm}Lucemburkové\vspace{-1.7cm}}
\date{}
\author{}

\begin{document}
\maketitle

\begin{itemize}
    \vspace{-0.5em}
    \setlength\itemsep{0.15em}
    \item[1310] počátek, nastupuje \textbf{Jan Lucemburský}
    \item[$-$] \textbf{Karel IV.}, \textbf{Václav IV.}
    \item[1437] konec za \textbf{Zmikunda Lucemburského}
\end{itemize}

\subsection*{Jindřich VII. Lucemburský}
\begin{itemize}
    \vspace{-0.5em}
    \setlength\itemsep{0.15em}
    \item[$-$] otec Jana Lucemburského
    \item[$-$] zvolen: nastupuje do čela SŘŘ po smrti syna Rudolfa Habsburského, Albrechta
    \item[$-$] podporovatel \textbf{Petr z Aspeltu}, hlasoval pro něj i při volbě, později jeden z poradců
    \item[$-$] tehdejší český vládce: \textbf{Jindřich Korutanský}, který má slabou vládu, období nestability $\rightarrow$ česká šlechta se vydala za Jindřichem Lucemburským, aby v Česku vládl jeho syn po sňatku s \textbf{Eliškou Přemyslovnou}
\end{itemize}

\subsection*{Jan Lucemburský}
\begin{itemize}
    \vspace{-0.5em}
    \setlength\itemsep{0.15em}
    \item[1310] \textsc{dobytí Prahy}, vyhnání Jindřicha z Korutan
    \item[$-$] skvělý válečník, diplomat, \textit{král cizinec}
    \item[$-$] \textbf{Jindřich z Lipé}, vůdce české člechty, žena \textbf{Eliška Rejčka}, žili v Brně
    \item[1310] šlechta si vynutila ústupky $\rightarrow$ \textit{Inaugurační diplom}: šlechta nemusí podnikat zahraniční výboje, nemusí platit daně, do významných úřadů nebudou jmenování cizinci, není plněno
    \item[1318] hrozba občanské války $\rightarrow$ \textsc{vtrhnul do Brna}
    \item[1318] vmísí se do toho císař SŘŘ \textbf{Ludvík Bavorský}, který zprostředkuje domluvu mezi šlechtou a králem: \textit{Úmluvy domažlické}, král rezignuje na správu Čech, zaměřuje se jen na říšskou politiku
    \item[(1319)] krize v manželství, sebral jí děti a byla vykázána na Mělník, následník trůnu Václav do Francie, tam právě vládne \textbf{Karel IV.}, což je jeho strýc
    \item[$-$] územní zisky: Chebsko, Horní Lužice (J od Lužice), část Slezska, Lucca (S Itálie)
    \item[$-$] současníkem \textbf{Kazimír III. Veliký}, společná dohoda: Kazimír se vzdal části území a Jan si na oplátku nenárokoval polskou korunu
    \item[$-$] \textit{dukáty}, mincovna přesunuta do Prahy
    \item[1333] Jan při biřmování přijímá jméno Karel
    \item[1334] kralevic Karel markrabětem moravským
    \item[1337] Karel spoluvladařem
    \item[$-$] úpoloženy základy Staroměstské radnice
    \item[$-$] stavba katedrály Sv. Víta: Matyáš z Arrasu, Petr Parléř
    \item[$-$] vzal si další ženušku \textbf{Beatrix Bourbonskou}
    \item[(1340)] uvědomuje si že je starý a slepý, sepíše \textit{Janovu závěť}: Karel dědí Čechy, Lužici a Slezsko, jeho bratr \textbf{Jan Jindřich} je markrabě moravský a \textbf{Václav Český} (dítě s Beatrix) získává Lucembursko
    \item[(1341)] Karel je \textit{rex junior} = mladší král, počítá se s ním jako s budoucím králem
    \item[(1344)] společná výprava s Karlem za avignonským papežem \textbf{Klementem VI.}, původním jménem Pierre Roger, což byl Karlův bývalý vychovatel ve Francii a díky jejich nadstandardním vztahům je Pražské biskupství povýšeno na arcibiskupství, první arcibiskup \textbf{Arnošt z Pardubic}; též zřízeno nové biskupství v Litomyšli $\rightarrow$ vymanění české církve z nadřízenosti Mohučského arcibiskupství
    \item[1346] v SŘŘ vládne Ludvík Bavor, který má špatné vztahy s Klimentem a kritizuje papeže, takže avignonský papež iniciuje zvolení protikandidáta: 5 hlasů pro něj $\rightarrow$ Karel \textbf{král římský}
    \item[(26.8.) 1346] \textsc{bitva u Kresčaku}, Jan Lucemburský zahynul $\rightarrow$ Karel králem českým, první maželka Blanka z Valois, nechal vytvořit Svatováclavskou korunu
\end{itemize}

\subsection*{Karel IV., otec vlasti (1346 -- 1378)}
\begin{itemize}
    \vspace{-0.5em}
    \setlength\itemsep{0.15em}
    \item[1346] král římský, český
    \item[1355] lombardská koruna
    \item[1355] manželka \textbf{Anna Svídnická}, císař SŘŘ
    \item[$-$] Montecarlo = pevnost v Toskánsku
    \item[$-$] Praha sídelním městěm císaře SŘŘ, Čechy centerm SŘŘ
    \item[$-$] kult Sv. Václava: Svatováclavská koruny, Svatováclavská kaple, Sv. Václav na pečetidle KU
    \item[$-$] Země koruny české: České království a vedlejší země: markrabství moravské, Slezsko, Lužice (Horní i Dolní), Horní Falc, Lucembursko, Braniborsko
    \item[$-$] vysocí úředníci: Arnošt z Pardubic, Jan ze Středy, Francesco Petrarca; všeobecně se opírá o šlechtu
    \item[$-$] Jan Očko z Vlašimi, to jest arcibiskup
    \item[1355] \textit{Maiestas Carolina} = neúspěšný návrh zemského zákoníku, ale pro šlechtu nepřijatelný, údajně shořel
    \item[1356]  \textit{Zlatá bula} = říšský zákoník, zvýhodnil postavení českého krále mezi kurfiřty, pro český trůn platila ženská posloupnost, při volbě krále nemusela platit absolutní shoda, ale jen větší polovina hlasů, platila až do 1806
    \item[$-$] druhé, neoficiální sídlo Karla je Norimberk
    \item[$-$] hospodářství: víno, ovocnářství, PIVO, rybníkářství, bohaté stříbrné doly
    \item[(1348)] Moravské zemské desky = Moravské cúdy (soudy) = dvakrát do roka se konaly zemské soudy a jejich výsledky se píší do těchto desk
    \item[$-$] zakladatelská činnost: Karlova univerzita (první univerzita francouzského typu u nás: 4 fakulty -- artistická, právnická, lékařská, teologická), Nové město pražské, úprava Pražského hradu, Svatovítská katedrála, kamenný most, Staroměstská mostecká věž, Karlštejn, klášter Emauzy, Hladová zeď (součást městského opevnění), chrám Panny Marie Sněžné, Staroměstská radnice
    \item[$-$] manželky: Blanka z Valois, Anna Falcká, Anna Svídnická (syn Václav IV.), Alžběta Pomořanská (synové Zikmund, Jan Zhořelecký a dcera Anna Česká)
    \item[$-$] celkem 11 dětí
    \item[$-$] umírá na zápal plic
\end{itemize}

\subsection*{Kultura}
\begin{itemize}
    \vspace{-0.5em}
    \setlength\itemsep{0.15em}
    \item[$-$] Mistr Theodorik -- malíř, autor nástěnných obrazů v kapli Sv. Kříže
    \item[$-$] Mistr Třeboňského, Vyšebrodského, Rajhradského oltáře
    \item[$-$] \textit{iluminace} = ilustrace knih, např. ve Velislavovi bibli
    \item[$-$] socha Sv. Jiří na nádvoří Pražského hradu
    \item[$-$] rozvoj školství: univerzita, klášterní, farní, partikulární ve městech
    \item[$-$] legendy o sv. Kateřině, o sv. Prokopu
    \item[$-$] \textit{postily}, Trojánská kronika, Závišova píseň, \textit{Vita caroli}, Klaretův slovník
    \item[$-$] první překlad bible do češtiny
\end{itemize}

\subsection*{Václav IV.}
\begin{itemize}
    \vspace{-0.5em}
    \setlength\itemsep{0.15em}
    \item[(1363)] už jako dvouletý korunován Českým králem
    \item[(1376)] králem SŘŘ
    \item[(1378)] smrt otce, dostává se k moci
\end{itemize}


\subsection*{Zikmund Lucemburský}
\begin{itemize}
    \vspace{-0.5em}
    \setlength\itemsep{0.15em}
    \item[$-$] král Uherský, spolu s bratrem Janem Zhořeleckým drží Braniborsko
\end{itemize}

\subsection*{Jan Zhořelecký}
\begin{itemize}
    \vspace{-0.5em}
    \setlength\itemsep{0.15em}
    \item[$-$] Zhořelecko = Horní lužice, s bratrem se děli o Braniborsko
\end{itemize}


\subsection*{Jan Jindřich}
\begin{itemize}
    \vspace{-0.5em}
    \setlength\itemsep{0.15em}
    \item[$-$] bratr Karla IV.
    \item[$-$] synové: Jošt, Prokop, Jan Soběslav, markrabata moravská, ale nakonec je to Jošt
\end{itemize}

\subsection*{Václav IV. (1378 -- 1419)}
\begin{itemize}
    \vspace{-0.5em}
    \setlength\itemsep{0.15em}
    \item[$-$] matka \textbf{Anna Svídnická}, narozen v Norimberku, vzdělaný
    \item[$-$] král SŘŘ díky titulu jeho otce
    \item[$-$] manželky: \textbf{Johana Bavorská}, \textbf{Žofie Bavorská}, žádné děti
    \item[$-$] vládne v době stoleté války
    \item[1378--1417] \textit{papežské schizma} = dvojpapežství, papeži v Římě a Avignonu
    \item[1409] svolal \textsc{Pisánský koncil}, aby zvolili nového papeže, ale oba dosavadní papežové se odmítají vzdát své funkce $\rightarrow$ tři papežové
    \item[$-$] bojí se vysoké šlechty, opírá se o měšťany a drobnou šlechtu, z nich si vybírá dvořany a rádce, to se nelíbí vysoké šlechtě, která se postupně spojuje do tzv. \textit{panské jednoty}
    \item[$-$] arcibiskup pražský: \textbf{Jan z Jenštejna} chce, aby církev podléhala jemu a aby byla nezávislá králi, to se nelíbí Václavovi $\rightarrow$ nepřátelé
    \item[1393] vygradování konfliktu, václavovi lidé vtrhli do arcibiskupova sídla a unesli \textbf{Jana z Pomuku}, podle legendy z Karlova mostu shozen do Vltavy, následně prohlášen za svatého
    \item[$-$] jan z Jenštejna si stěžuje u arcibiskupa, ale ten potřebuje mít nekloněné České země kvůli možné válce, takže mu nevyhoví
    \item[$-$] do čela šlechty se postavil \textbf{Jošt Lucemburský}
    \item[1394] \textsc{šlechta v čele s Joštem Lucemburským zajala Václava IV.}, protože byli nespokojení s jeho činností, jeho propuštění dojedná Joštův bratr Jan Zhořelecký
    \item[$-$] všichni arcibiskupové kritizují Václava IV., protože neřeší papežské schizma $\rightarrow$
    \item[1400] \textsc{zbaven říšské koruny}, nahradí ho \textbf{Ruprecht III. Falcký}
    \item[$-$] nespokojený s jeho politikou je i Zikmund, král uherský $\rightarrow$
    \item[1402] \textsc{vpád Zikmunda do Čech}, chce na sebe strhnout moc, ale neúspěšně, všichni se za něj postavili
    \item[1409] \textit{Pisánský koncil}, kde byl zvolen další papež, ti předchozí nechtějí odstoupit, nový papež opět zvolil Václava říšským králem
    \item[1410] \textbf{Ruprecht} (král SŘŘ) umírá, noví kandidáti: Zikmund Lucemburský, Jošt Moravský, Václav IV., vítězí Jošt
    \item[1411] \textsc{Jošt umírá} za záhadných okolností $\rightarrow$ ostatní Lucemburkové se dohodli, že si Václav nechá titul, ale fakticky vládne Zikmund
    \item[1417] \textsc{konec trojpapežství} díky \textsc{Koncilu v Kostnici}, který svolal Zikmund, dále řešeny otázky nápravy církve: měl se prot nim vymezit, \textbf{Mistr Jan Hus} se snažil vysvětlit své učení, avšak 1415 \textsc{upálen}, přijel ho obhajovat kamarád \textbf{Jeroným Pražský}, 1416 \textsc{upálen}
    \item[$\rightarrow$] polarizace společnosti, kritika Václava IV. a Zikmunda
    \item[$-$] \textit{Stížný list české šlechty} -- šlechta si stěžuje na šlechtu a Kostnický koncil
    \item[30.7.1419] vyvrcholení: \textsc{První pražská defenestrace}, smrt několika konšelů
    \item[16.8.1419] \textsc{umírá} Václav IV. (mrtvice nebo epilepsie)
\end{itemize}

\subsection*{Zikmund Lucemburský \#2}
\begin{itemize}
    \vspace{-0.5em}
    \setlength\itemsep{0.15em}
    \item[$\rightarrow$] problém nástupnictví: \textbf{Zikmund Lucemburský}, král uherský a SŘŘ (Češi ho nechtějí) $\rightarrow$ využil křížových výprav (celkem 5)
    \item[1420] první: mezi boji se nechal korunovat na českého krále, nikdo ho neuznává
    \item[1421] \textsc{Čáslavský sněm Zikmunda sesadil}, on si ale za titulem stojí
    \item[1431] druhá: \textsc{husité opět vítězí} $\rightarrow$ svolán \textsc{koncil v Bazileji}, kde se řeší, jak se jich zbavit, husité je však sami zbví trápení
    \item[1434] \textsc{bitva u Lipan} radikální vs. umírnění, vyhrávají umírnění $\rightarrow$ otevření cesty k vyjednávání s katolíky, papežem, uzavřena \textit{Bazilejská kompaktáta} -- Češi mohou přijímat pod obojí
    \item[$\rightarrow$ 1436] Zikmund českým králem
    \item[1437] Zikmund ve Znojmě umírá
    \item[1433] císař SŘŘ -- nejvýznamnější sourozenec Václava IV.
\end{itemize}


\subsection*{Problém nástupnictví}
\begin{itemize}
    \vspace{-0.5em}
    \setlength\itemsep{0.15em}
    \item[$-$] po Zikmundově smrti opět problém -- Václav IV. nemá potomky, Zikmund pouze dceru \textbf{Alžbětu Lucemburskou}, hledají ženicha: \textbf{Albrecht Habsburský}, sňatek
    \item[1437] Albrecht českým králem, ale o dva roky později umírá na úplavici
    \item[$-$] Alžběta je těhotná -- syn \textbf{Ladislav Pohrobek}, vládne 1453 -- 1457, ponechává si \textbf{Jiřího z Kunštátu} a \textbf{z Poděbrad} jaho zemského správce
    \item[$-$] dříve, než se stihne oženit, zemře $\rightarrow$ králem \textbf{Jiří z Poděbrad}



\end{itemize}


\end{document}
