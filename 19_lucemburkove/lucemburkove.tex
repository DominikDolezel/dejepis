\documentclass{article}
\usepackage{fullpage}
\usepackage[czech]{babel}
\usepackage{amsfonts}

\title{\vspace{-2cm}Lucemburkové\vspace{-1.7cm}}
\date{}
\author{}

\begin{document}
\maketitle

\begin{itemize}
    \vspace{-0.5em}
    \setlength\itemsep{0.15em}
    \item[1310] počátek, nastupuje \textbf{Jan Lucemburský}
    \item[$-$] \textbf{Karel IV.}, \textbf{Václav IV.}
    \item[1437] konec za \textbf{Zmikunda Lucemburského}
\end{itemize}

\subsection*{Jindřich VII. Lucemburský}
\begin{itemize}
    \vspace{-0.5em}
    \setlength\itemsep{0.15em}
    \item[$-$] otec Jana Lucemburského
    \item[$-$] zvolen: nastupuje do čela SŘŘ po smrti syna Rudolfa Habsburského, Albrechta
    \item[$-$] podporovatel \textbf{Petr z Aspeltu}, hlasoval pro něj i při volbě, později jeden z poradců
    \item[$-$] tehdejší český vládce: \textbf{Jindřich Korutanský}, který má slabou vládu, období nestability $\rightarrow$ česká šlechta se vydala za JIndřichem Lucemburským, aby v Česku vládl jeho syn po sňatku s \textbf{Eliškou Přemyslovnou}
\end{itemize}

\subsection*{Jan Lucemburský}
\begin{itemize}
    \vspace{-0.5em}
    \setlength\itemsep{0.15em}
    \item[1310] \textsc{dobytí Prahy}, vyhnání Jindřicha z Korutan
    \item[$-$] skvělý válečník, diplomat, \textit{král cizinec}
    \item[$-$] \textbf{Jindřich z Lipé}, vůdce české člechty, žena \textbf{Eliška Rejčka}, žili v Brně
    \item[1310] šlechta si vynutila ústupky $\rightarrow$ \textit{Inaugurační diplom}: šlechta nemusí podnikat zahraniční výboje, nemusí platit daně, do významných úřadů nebudou jmenování cizinci, není plněno
    \item[1318] hrozba občanské války $\rightarrow$ \textsc{vtrhnul do Brna}
    \item[1318] vmísí se do toho císař SŘŘ \textbf{Ludvík Bavorský}, který zprostředkuje domluvu mezi šlechtou a králem: \textit{Úmluvy domažlické}, král rezignuje na správu Čech, zaměřuje se jen na říšskou politiku
    \item[(1319)] krize v manželství, sebral jí děti a byla vykázána na Mělník, následník trůnu Václav do Francie, tam právě vládne \textbf{Karel IV.}, což je jeho strýc
    \item[$-$] územní zisky: Chebsko, Horní Lužice (J od Lužice), část Slezska, Lucca (S Itálie)
    \item[$-$] současníkem \textbf{Kazimír III. Veliký}, společná dohoda: Kazimír se vzdal části území a Jan si na oplátku nenárokoval polskou korunu
    \item[$-$] \textit{dukáty}, mincovna přesunuta do Prahy
    \item[1333] Jan při biřmování přijímá jméno Karel
    \item[1334] kralevic Karel markrabětem moravským
    \item[1337] Karel spoluvladařem
    \item[$-$] úpoloženy základy Staroměstské radnice
    \item[$-$] společná výprava s Karlem za avignonským papežem \textbf{Klementem VI.}, původním jménem Perre Roger, což byl Karlův bývalý vychovatel ve Francii a díky jejich nadstandardním vztahům je pražské biskupství povýšeno na arcibiskupství, první arcibiskup \textbf{Arnošt z Pardubic}; též zřízeno nové biskupství v Litomyšli $\rightarrow$ vymanění české církve z nadřízenosti mohučského arcibiskupství
    \item[$-$] stavba katedrály Sv. Víta: Matyáš z Arrasu, Petr Parléř
    \item[$-$] vzal si další ženušku \textbf{Beatrix Bourbonskou}
    \item[(1340)] uvědomuje si že je starý a slepý, sepíše \textit{Janovu závěť}: Karel dědí Čechy, Lužici a Slezsko, jeho bratr \textbf{Jan Jindřich} je markrabě moravský a \textbf{Václav Český} (dítě s Beatrix) získává Lucembursko  
\end{itemize}


\end{document}
