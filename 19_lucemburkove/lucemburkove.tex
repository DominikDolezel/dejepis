\documentclass{article}
\usepackage{fullpage}
\usepackage[czech]{babel}
\usepackage{amsfonts}

\title{\vspace{-2cm}Lucemburkové\vspace{-1.7cm}}
\date{}
\author{}

\begin{document}
\maketitle

\begin{itemize}
    \vspace{-0.5em}
    \setlength\itemsep{0.15em}
    \item[1310] počátek, nastupuje \textbf{Jan Lucemburský}
    \item[$-$] \textbf{Karel IV.}, \textbf{Václav IV.}
    \item[1437] konec za \textbf{Zmikunda Lucemburského}
\end{itemize}

\subsection*{Jindřich VII. Lucemburský}
\begin{itemize}
    \vspace{-0.5em}
    \setlength\itemsep{0.15em}
    \item[$-$] otec Jana Lucemburského
    \item[$-$] zvolen: nastupuje do čela SŘŘ po smrti syna Rudolfa Habsburského, Albrechta
    \item[$-$] podporovatel \textbf{Petr z Aspeltu}, hlasoval pro něj i při volbě, později jeden z poradců
    \item[$-$] tehdejší český vládce: \textbf{Jindřich Korutanský}, který má slabou vládu, období nestability $\rightarrow$ česká šlechta se vydala za Jindřichem Lucemburským, aby v Česku vládl jeho syn po sňatku s \textbf{Eliškou Přemyslovnou}
\end{itemize}

\subsection*{Jan Lucemburský}
\begin{itemize}
    \vspace{-0.5em}
    \setlength\itemsep{0.15em}
    \item[1310] \textsc{dobytí Prahy}, vyhnání Jindřicha z Korutan
    \item[$-$] skvělý válečník, diplomat, \textit{král cizinec}
    \item[$-$] \textbf{Jindřich z Lipé}, vůdce české člechty, žena \textbf{Eliška Rejčka}, žili v Brně
    \item[1310] šlechta si vynutila ústupky $\rightarrow$ \textit{Inaugurační diplom}: šlechta nemusí podnikat zahraniční výboje, nemusí platit daně, do významných úřadů nebudou jmenování cizinci, není plněno
    \item[1318] hrozba občanské války $\rightarrow$ \textsc{vtrhnul do Brna}
    \item[1318] vmísí se do toho císař SŘŘ \textbf{Ludvík Bavorský}, který zprostředkuje domluvu mezi šlechtou a králem: \textit{Úmluvy domažlické}, král rezignuje na správu Čech, zaměřuje se jen na říšskou politiku
    \item[(1319)] krize v manželství, sebral jí děti a byla vykázána na Mělník, následník trůnu Václav do Francie, tam právě vládne \textbf{Karel IV.}, což je jeho strýc
    \item[$-$] územní zisky: Chebsko, Horní Lužice (J od Lužice), část Slezska, Lucca (S Itálie)
    \item[$-$] současníkem \textbf{Kazimír III. Veliký}, společná dohoda: Kazimír se vzdal části území a Jan si na oplátku nenárokoval polskou korunu
    \item[$-$] \textit{dukáty}, mincovna přesunuta do Prahy
    \item[1333] Jan při biřmování přijímá jméno Karel
    \item[1334] kralevic Karel markrabětem moravským
    \item[1337] Karel spoluvladařem
    \item[$-$] úpoloženy základy Staroměstské radnice
    \item[$-$] stavba katedrály Sv. Víta: Matyáš z Arrasu, Petr Parléř
    \item[$-$] vzal si další ženušku \textbf{Beatrix Bourbonskou}
    \item[(1340)] uvědomuje si že je starý a slepý, sepíše \textit{Janovu závěť}: Karel dědí Čechy, Lužici a Slezsko, jeho bratr \textbf{Jan Jindřich} je markrabě moravský a \textbf{Václav Český} (dítě s Beatrix) získává Lucembursko
    \item[(1341)] Karel je \textit{rex junior} = mladší král, počítá se s ním jako s budoucím králem
    \item[(1344)] společná výprava s Karlem za avignonským papežem \textbf{Klementem VI.}, původním jménem Pierre Roger, což byl Karlův bývalý vychovatel ve Francii a díky jejich nadstandardním vztahům je Pražské biskupství povýšeno na arcibiskupství, první arcibiskup \textbf{Arnošt z Pardubic}; též zřízeno nové biskupství v Litomyšli $\rightarrow$ vymanění české církve z nadřízenosti Mohučského arcibiskupství
    \item[1346] v SŘŘ vládne Ludvík Bavor, který má špatné vztahy s Klimentem a kritizuje papeže, takže avignonský papež iniciuje zvolení protikandidáta: 5 hlasů pro něj $\rightarrow$ Karel \textbf{král římský}
    \item[(26.8.) 1346] \textsc{bitva u Kresčaku}, Jan Lucemburský zahynul $\rightarrow$ Karel králem českým, první maželka Blanka z Valois, nechal vytvořit Svatováclavskou korunu
\end{itemize}

\subsection*{Karel IV., otec vlasti (1346 -- 1378)}
\begin{itemize}
    \vspace{-0.5em}
    \setlength\itemsep{0.15em}
    \item[1346] král římský, český
    \item[1355] lombardská koruna
    \item[1355] manželka \textbf{Anna Svídnická}, císař SŘŘ
    \item[$-$] Montecarlo = pevnost v Toskánsku
    \item[$-$] Praha sídelním městěm císaře SŘŘ, Čechy centerm SŘŘ
    \item[$-$] kult Sv. Václava: Svatováclavská koruny, Svatováclavská kaple, Sv. Václav na pečetidle KU
    \item[$-$] Země koruny české: České království a vedlejší země: markrabství moravské, Slezsko, Lužice (Horní i Dolní), Horní Falc, Lucembursko, Braniborsko
    \item[$-$] vysocí úředníci: Arnošt z Pardubic, Jan ze Středy, Francesco Petrarca; všeobecně se opírá o šlechtu
    \item[$-$] Jan Očko z Vlašimi, to jest arcibiskup
    \item[1355] \textit{Maiestas Carolina} = neúspěšný návrh zemského zákoníku, ale pro šlechtu nepřijatelný, údajně shořel
    \item[1356]  \textit{Zlatá bula} = říšský zákoník, zvýhodnil postavení českého krále mezi kurfiřty, pro český trůn platila ženská posloupnost, při volbě krále nemusela platit absolutní shoda, ale jen větší polovina hlasů, platila až do 1806
    \item[$-$] druhé, neoficiální sídlo Karla je Norimberk
    \item[$-$] hospodářství: víno, ovocnářství, PIVO, rybníkářství, bohaté stříbrné doly
    \item[(1348)] Moravské zemské desky = Moravské cúdy (soudy) = dvakrát do roka se konaly zemské soudy a jejich výsledky se píší do těchto desk
    \item[$-$] zakladatelská činnost:
\end{itemize}






\end{document}
