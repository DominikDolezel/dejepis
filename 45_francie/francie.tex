\documentclass{article}
\usepackage{fullpage}
\usepackage[czech]{babel}
\usepackage{amsfonts}

\title{\vspace{-2cm}Francie ve druhé poloivině 19. st.\vspace{-1.7cm}}
\date{}
\author{}

\begin{document}
\maketitle

\begin{itemize}
    \vspace{-0.5em}
    \setlength\itemsep{0.15em}
    \item[$-$] spojeno s Ludvbíkem Bonapartem
    \item[20.12.1848] Ludvík Bonaparte se stává presidentem
    \item[1851] na deset let má plnou moc, tato posice je potvrzena plebiscitem, diktature
    \item[2.12.1852] Napoleon III. císařem
    \item[$-$] vypracována nová ústava, děcka, potvrzovala existenci parlamentu, ale ze začátku neměl žádné pravomoce, všechno dělal Napoleon
    \item[$-$] parlamentní monarchie vytvářena postupně, ze začátku vládne absolutisticky, omezeny občanské svobody (shromažďovací právo)
    \item[$-$] oporou Napoleona jsou bohaté vrstvy, armáda, policie, církev, zemědělci
\end{itemize}

\subsection*{Domácí politika}
\begin{itemize}
    \vspace{-0.5em}
    \setlength\itemsep{0.15em}
    \item[$-$] ralisace průmyslové revoluce, ve velkém se staví továrny, železnice, silnice, velkolepá výstavba v Paříži
    \item[$-$] zmírnění cla na dovážené britské produkty
    \item[$-$] obava z toho, že by se dělníci mohli bouřit, zákaz stávek, chce vyřešit nezaměstnanost $\rightarrow$ podpora výstavby, urbanisace
    \item[$-$] přestavbu vede architekt Haussmann, typická zástavba Paříže pochází právě z této doby
\end{itemize}

\subsection*{Druhé francouzské císařství}
\begin{itemize}
    \vspace{-0.5em}
    \setlength\itemsep{0.15em}
    \item[$-$] koloniální výboje, i když sliboval mír, finančné náročné
    \item[1853-1856] \textsc{krymská válka}, Francouzi na straně Osmanů (Sardinského království) a Velké Británie proti Rusku
    \item[$-$] výprava do Sýrie a Egypta, stavba Suezského průplavu
    \item[1859] podpora Sardinského království v bojích s Habsburky, za odměnu získává Savojsko a Nice
    \item[$-$] proniká do Číny a Indočíny (Vietnam, Laos, Kambodža)
    \item[$-$] v Mexiku vyhlašuje císařství, kam v roce 1861 poslal expedici, protože mexičané nespláceli své dluhy, do čela Mexika prosazuje Maxmiliána Habsburského, nakonec armádu stahuje, Maxmilián zatčen a popraven
    \item[1870-1871] \textsc{válka s Pruskem}, záminka: kdo nastoupí na španělský trůn, Bismarck záměrně upravil Enžskou depeši, vyprovokoval ho k vyhlášení války; pro Napoleona končí tragicky, Francouzi poraženi u Méty, v \textsc{bitvě u Sedanu} kapituluje většina francouzské armády a Napoleon je zajat
    \item[4.9.1870] Francie vyhlášena republikou
\end{itemize}


\subsection*{Pařížská komuna}
\begin{itemize}
    \vspace{-0.5em}
    \setlength\itemsep{0.15em}
    \item[$-$] Francouzi se nechtějí vzdát Alsaska a Lotrinska
    \item[18.1.1871] Němci si vyhlásí ve Versailles Německé císařství v čele s Vilémem I.
    \item[$-$] ještě v lednu Paříž kapituluje, následně obsazují s Němci příměří a předběžnou mírovou smlouvu, ve které Francouzi uznávají ztrátu Alsaska a Lotrinska a francouzská armáda je odzbrojena
    \item[$-$] v Paříži zůstávají ozbrojené jednotky, tzv. \textit{národní gardy}, která chránila Paříž, vláda je chtěla odzbrojit, zbraně si však koupili za vlastní prostředky a to se jim nelíbilo $\rightarrow$ vyhlásili \textit{Pařížskou komunu}
    \item[$-$] levicové síly ovládly Paříž: dělníci, anarchisté
    \item[$-$] Pařížská komuna zavádí pevné ceny potravin, bezplatnou výuku či lékařskou péči, rovnoprávnost žen, povinný podíl dělníků na řízení podniků
    \item[květen 1871] francouzské vládě se podařilo s Němci uzavřít mír, ztratila Alsasko a Lotrinsko, ohradili se proti Komuně
    \item[$-$] poslední boje probíhaly u zdi Pére-Laichaise, kde byla Komuna definitivně zlikvidována, její členové deportováni či popraveni
\end{itemize}


\subsection*{Třetí republika}
\begin{itemize}
    \vspace{-0.5em}
    \setlength\itemsep{0.15em}
    \item[1875] vypracování nové ústavy, všeobecné hlasovací právo, bezplatné vzdělávání a. j.
    \item[$-$] proti republice se ze začítku neúspěšně ohrazují monarchisté, postupně se stabilizuje
    \item[$-$] dvě politické strany:
    \begin{itemize}
        \vspace{-0.5em}
        \setlength\itemsep{0.15em}
        \item[$-$] \textit{radikáové}: na konci století vůdčí politicko ustranou, chtějí demokratisaci, odpojení církve od státu, oslabení kompetencí presidenta
        \item[$-$] \textit{socialisté}: na začátku nejednotní, postupně se spojí do jedné strany
    \end{itemize}
    \item[$-$] \textit{Dreifusova aféra}: odsouzen z proněmecké špionáže, odpor intelektuálů, inspirováno antisemitismem
    \item[$-$] na problém antisemitismus poukazuje Émile Zola
    \item[$-$] Eiffelova věž, basilika Sacré-Coeur
    \item[$-$] Louis Blériot přeletěl La Mancheský průliv
\end{itemize}

\subsection*{Kultura}
\begin{itemize}
    \vspace{-0.5em}
    \setlength\itemsep{0.15em}
    \item[$-$] impresionismus: Claude Monet (považován za zakladatele), Piére-Auguste Renoir, Auguste Rodin (sochař)
    \item[$-$] kubismus: Georges Braque (jeden ze zakladatelů kubismu)
    \item[$-$] secese: Alfons Mucha
    \item[$-$] realismus: Gustav Flaubert, Guy da Maupassant
    \item[$-$] naturalismus: Émile Zola
    \item[$-$] symbolismus: Paul Verlain, Arthur Rimbaud
    \item[$-$] postimpresionismus: Paul Cézanne (otec moderního umění), Vincenc van Gogh, Paul Gauguin
    \item[$-$] pointilismus: Georges Seurat
    \item[$-$] expresinismus: Vasilij Kandinskij, Edvard Munch
    \item[$-$] surrealismus: Salvador Dalí, Joan Miró
\end{itemize}





\end{document}
