\documentclass{article}
\usepackage{fullpage}
\usepackage[czech]{babel}
\usepackage{amsfonts}

\title{\vspace{-2cm}Francie ve druhé poloivině 19. st.\vspace{-1.7cm}}
\date{}
\author{}

\begin{document}
\maketitle

\begin{itemize}
    \vspace{-0.5em}
    \setlength\itemsep{0.15em}
    \item[$-$] spojeno s Ludvbíkem Bonapartem
    \item[20.12.1848] Ludvík Bonaparte se stává presidentem
    \item[1851] na deset let má plnou moc, tato posice je potvrzena plebiscitem, diktature
    \item[2.12.1852] Napoleon III. císařem
    \item[$-$] vypracována nová ústava, děcka, potvrzovala existenci parlamentu, ale ze začátku neměl žádné pravomoce, všechno dělal Napoleon
    \item[$-$] parlamentní monarchie vytvářena postupně, ze začátku vládne absolutisticky, omezeny občanské svobody (shromažďovací právo)
    \item[$-$] oporou Napoleona jsou bohaté vrstvy, armáda, policie, církev, zemědělci
\end{itemize}

\subsection*{Domácí politika}
\begin{itemize}
    \vspace{-0.5em}
    \setlength\itemsep{0.15em}
    \item[$-$] ralisace průmyslové revoluce, ve velkém se staví továrny, železnice, silnice, velkolepá výstavba v Paříži
    \item[$-$] zmírnění cla na dovážené britské produkty
    \item[$-$] obava z toho, že by se dělníci mohli bouřit, zákaz stávek, chce vyřešit nezaměstnanost $\rightarrow$ podpora výstavby, urbanisace
    \item[$-$] přestavbu vede architekt Haussmann, typická zástavba Paříže pochází právě z této doby
\end{itemize}

\subsection*{Druhé francouzské císařství}
\begin{itemize}
    \vspace{-0.5em}
    \setlength\itemsep{0.15em}
    \item[$-$] koloniální výboje, i když sliboval mír, finančné náročné
    \item[1853-1856] \textsc{krymská válka}, Francouzi na straně Osmanů (Sardinského království) a Velké Británie proti Rusku
    \item[$-$] výprava do Sýrie a Egypta, stavba Suezského průplavu
    \item[1859] podpora Sardinského království v bojích s Habsburky, za odměnu získává Savojsko a Nice
    \item[$-$] proniká do Číny a Indočíny (Vietnam, Laos, Kambodža)
    \item[$-$] v Mexiku vyhlašuje císařství, kam v roce 1861 poslal expedici, protože mexičané nespláceli své dluhy, do čela Mexika prosazuje Maxmiliána Habsburského, nakonec armádu stahuje, Maxmilián zatčen a popraven
\end{itemize}



\end{document}
