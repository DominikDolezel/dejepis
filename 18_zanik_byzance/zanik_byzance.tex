\documentclass{article}
\usepackage{fullpage}
\usepackage[czech]{babel}
\usepackage{amsfonts}

\title{\vspace{-2cm}Zánik Byzantské říše\vspace{-1.7cm}}
\date{}
\author{}

\begin{document}
\maketitle

\begin{itemize}
    \vspace{-0.5em}
    \setlength\itemsep{0.15em}
    \item[$-$] první podlomení za 4. křížové výpravy
    \item[$-$] Latinské císařství, poté Nikájské císařství
    \item[(1261)] \textsc{dobytí Konstantinopole} $\rightarrow$ obnovení Byzance, vládnou \textbf{Palailogovci}
\end{itemize}

\section*{Palaiologovci}
\begin{itemize}
    \vspace{-0.5em}
    \setlength\itemsep{0.15em}
    \item[$-$] nepřátelé: italští obchodníci (konkurence), sousedi
    \item[(1274)] \textit{Lyonská církevní unie} = spojení křesťanství V a Z církve, hlavou má být papež
    \item[$-$] sociální nepokoje
\end{itemize}


\section*{Turecká expanze}
\begin{itemize}
    \vspace{-0.5em}
    \setlength\itemsep{0.15em}
    \item[$-$] příchod Seldžúků
    \item[$-$] Mantzikert (1071), Jeruzalém (1076)
    \item[13. / 14. st.] příchod osmanských Turků $\rightarrow$ krize Seldžuků
\end{itemize}


\section*{Osmanská říše 1299 -- 1922}
\begin{itemize}
    \vspace{-0.5em}
    \setlength\itemsep{0.15em}
    \item[$-$] v SZ dnešního Turecka
    \item[1299] \textbf{Osman I. Gází} se prohlásil za vládce
    \item[$-$] jeho syn \textbf{Orchan} dobyl Nikáiu, Nikomédii
    \item[$-$] \textit{janičáři} = elitní pěchota turecké armády, rekrutováni z řad křesťanských kluků
    \item[$-$] \textit{siphaiové} = jízda
    \item[$-$] teokratický stát, opírá se o armádu
\end{itemize}

\subsection*{Murad I. (1359 -- 1389)}
\begin{itemize}
    \vspace{-0.5em}
    \setlength\itemsep{0.15em}
    \item[$-$] ovládl Thrákii, dobyl Adrianopol $\rightarrow$ hlavní město
    \item[1389] \textsc{bitva na Kosově poli}, podrobení Srbů
\end{itemize}

\subsection*{Bajezid I.}
\begin{itemize}
    \vspace{-0.5em}
    \setlength\itemsep{0.15em}
    \item[$-$] dobytí Serdiky (dnešní Sofie)
    \item[1393] vyvrácení Bulharska
    \item[1396] \textsc{bitva u Nikopole} (vedl Zikmund Lucemburský), Osmané vyhráli, Evropané neúspěšní
    \item[$-$] vznikla \textit{Florentská unie}, aby se mohla uskutečnit křížová výprava proti Osmanům
\end{itemize}

\subsection*{Murad II.}
\begin{itemize}
    \vspace{-0.5em}
    \setlength\itemsep{0.15em}
    \item[1444] \textsc{bitva u Varny} proti křižákům, Osmané opět vítězí
\end{itemize}

\begin{itemize}
    \vspace{-0.5em}
    \setlength\itemsep{0.15em}
    \item[1453] dobytí Cařihradu, smrt Konstantina XI.
    \item[$\Rightarrow$] definitivní konec Byzance, Konstantinopol = Istanbul jako hlavní město Osmanů
    \item[$-$] Evropané se nafrní a nechtějí s nimi obchodovat, hledají nové obchodní cesty na Orient
\end{itemize}

\subsection*{Sulejman I. Nádherný}
\begin{itemize}
    \vspace{-0.5em}
    \setlength\itemsep{0.15em}
    \item[1526] \textsc{bitva u Moháče}, kde bylo poraženo uherské vojsko, Osmani získávají drtivou většinu Uher
    \item[(1429)] Osmané obléhají Vídeň
    \item[(1541)] dobytí Budína
    \item[1683] konečně vítězství Evropanů \textsc{u Vídně}
\end{itemize}


\end{document}
