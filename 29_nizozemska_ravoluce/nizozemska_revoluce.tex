\documentclass{article}
\usepackage{fullpage}
\usepackage[czech]{babel}
\usepackage{amsfonts}

\title{\vspace{-2cm}Nizozemská revoluce (1572-1579)\vspace{-1.7cm}}
\date{}
\author{}

\begin{document}
\maketitle

\section*{Španělsko (15.-16. století)}

\begin{itemize}
    \vspace{-0.5em}
    \setlength\itemsep{0.15em}
    \item[$-$] u vzniku stojí Ferdinand Aragonský a Isabela Kastilská, personální unie, každý si vládne na svém písečku
    \item[1492] konec \textit{reconquisty}, působení inkvizice, plavba Kryštofa Kolumba
    \item[$-$] proti zakladatelům Španělska opozice velkých feudálů = \textit{grandové}, ti získali sílu z bojů s Araby
    \item[$-$] Ferdinand se tedy přiklonil na stranu měst, která se spojila do tzv. \textit{Sv. hermanandy}, s jeho pomocí města porazila šlechtice, \textit{grandy}
    \item[$-$] \textbf{Johana Šílená} (Jana Kastilská), dcera Ferdinanda a Isabely se provdala sa Filipa Habsburského
    \item[$-$] právě Filip Habsburský sňatkem přinesl Nizozemí
\end{itemize}

\subsection*{Karel I. Habsburský (1516-1556)}
\begin{itemize}
    \vspace{-0.5em}
    \setlength\itemsep{0.15em}
    \item[$-$] od roku 1516 králem Španělským, 1519 císař SŘŘ, za něj probíhá reformace
    \item[$-$] Augsburský mír, on sám rezignuje na obojí (jak krále Španělska tak císaře SŘŘ), protože neudržel katolické náboženství
    \item[$-$] opíral se o vysokou šlechtu $\Rightarrow$
    \item[1520] \textsc{povstání komunerů} = povstání měst v čele s Toledem, Karel tvrdě potlačil, města byla nejen poražena, ale i zatížena daněmi $\Rightarrow$ ekonomické zatížení, postupně se do čela Evropy začíná dostávat spíše Anglie a Francie
    \item[$-$] Španělsko má sice hodně bohatství z kolonií, ale celé je to díky nařízení daní promrháno
\end{itemize}


\subsection*{Filip II. Habsburský (1556-1598)}
\begin{itemize}
    \vspace{-0.5em}
    \setlength\itemsep{0.15em}
    \item[$-$] syn Karla I., manželka Marie Tudorovna
    \item[$-$] tvrdě prosazuje katolicismus, oporou jeho politiky je tedy církev a střední šlechta
    \item[$-$] hospodářská politika nemá žádnou koncepci $\Rightarrow$ státní bankrot, který se snaží vyřešit vysokými daněmi
    \item[1588] \textsc{porážka španělské Armady} (loďstvo) Alžbětou Tudorovnou, i když se prezentovala jako neporazitelná
    \item[$-$] výstavba královského paláce Escorialu
    \item[$-$] dočasně připojil Portugalsko
\end{itemize}


\section*{Nizozemská revoluce}
\subsection*{Nizozemské provincie}

\begin{itemize}
    \vspace{-0.5em}
    \setlength\itemsep{0.15em}
    \item[$-$] nejbohatší španělská provincie, tvořeno 17 provinciemi, téměř nezávislá země na Španělsku
    \item[$-$] provincie mají vlastní sněm, prodlouženou rukou FIlipa II. byl \textit{generální místodržitel}, tehdy \textbf{Markéta Parmská}, jeho nevlastní dcera
    \item[$-$] pestré náboženské složení: katolíci, ale především stoupenci Martina Luthera = lutherání, stoupenci Jana Kalvína = kalvinisté, novokřtěnci
    \item[$-$] po státním bankrotu FIlip zvyšuje daně, ... kde jinde než (v ČEZ) v Nizozemí, snaží se tam prosadit absolutismus
    \item[$\Rightarrow 1566$] \textit{obrazoborecké hnutí} = hnutí zaměřené proti Filipu II., především nižší vrstvy, které ničily katolické kostely
    \item[$-$] šlechta se snaží spíš o diplomatické vyjednávání s Filipemjsi říkal že máš
    \item[$\Rightarrow$] záminka pro Filipa, aby do Nizozemí poslal armádu, v čele je \textbf{vévoda z Alby} (Fernand Álvarez de Toledo), který nastolil krutovládu, teror, zničil Nizozemskou opozici
    \item[$-$] někteří z Nizozemí utíkají, třeba \textbf{Vilém Oranžský}, který později stanul v čele povstání proti Španělům
    \item[$-$] \uv{řádění Španělů} vyvolalo celonárodní revoluci, v jejím čele je \textbf{Vilém Oranžský}, je to první buržoazní revoluce
    \item[$-$] \textit{gézové} = žebráci, i oni bojovali po boku Nizozemských stavů proti Španělům, dělí se na jižní (říkají si lesní) a severní (mořští)
    \item[1572] mořští gézové napadli jeden z přístavů ovládaných Španělskem, počátek revoluce
    \item[$-$] po odchodu vévody z Alby nastupuje \textbf{Juan d'Austria}
    \item[$-$] \textit{gentská pacifikace} = po vyplenění Antverp Španěli se stavové domluvili na společném postupu
    \item[1579] nizozemské provincie se rozdělují:
    \begin{itemize}
        \vspace{-0.5em}
        \setlength\itemsep{0.15em}
        \item[$-$] na jihu se utvoří tzv. \textbf{Arraská unie}, tyto provincie už nechtějí válku a chtějí se se Španěli domluvit o stažení vojsk a oni jim za odměnu zůstanou věrní, součástí Španělska
        \item[$-$] na severu se utvoří tzv. \textbf{Utrechtská unie}, tyto provincie chtějí bojovat do té doby, než Španěle nevyženou, což se nakonec podařilo
    \end{itemize}
    \item[1581] severní provincie vyhlašují \textit{Spojené nizozemské provincie}, tedy nezávislost na Španělsku, v jejím čele stanul \textbf{Vilém I. Oranžský}, moc dlouho nevydrží, po něm nastupuje jeho syn \textbf{Mořic Oranžský}
    \item[1609] příměří se Španěli, mají uznávát Nizozemsko, platí však jen do 1621 (30letá válka), museli se tedy přidat na stranu Habsburků, po jejím konci v roce 1648 (Vestfalský mír) bylo Nizozemsko uznáno \textit{de iure} uznáno za svobodný stát
    \item[$-$] jižní Nizozemí, které zůstalo věrné Španělsku, je budoucí Belgie
\end{itemize}





\end{document}
