\documentclass{article}
\usepackage{fullpage}
\usepackage[czech]{babel}
\usepackage{amsfonts}

\title{\vspace{-2cm}Nizozemská revoluce (1572-1579)\vspace{-1.7cm}}
\date{}
\author{}

\begin{document}
\maketitle

\section*{Španělsko (15.-16. století)}

\begin{itemize}
    \vspace{-0.5em}
    \setlength\itemsep{0.15em}
    \item[$-$] u vzniku stojí Ferdinand Aragonský a Isabela Kastilská, personální unie, každý si vládne na svém písečku
    \item[1492] konec \textit{reconquisty}, působení inkvizice, plavba Kryštofa Kolumba
    \item[$-$] proti zakladatelům Španělska opozice velkých feudálů = \textit{grandové}, ti získali sílu z bojů s Araby
    \item[$-$] Ferdinand se tedy přiklonil na stranu měst, která se spojila do tzv. \textit{Sv. hermanandy}, s jeho pomocí města porazila šlechtice, \textit{grandy}
    \item[$-$] \textbf{Johana Šílená}, dcera Ferdinanda a Isabely se provdala sa Filipa Habsburského
    \item[$-$] právě Filip Habsburský sňatkem přinesl Nizozemí
\end{itemize}

\section*{Karel I. Habsburský (1516-1556)}
\begin{itemize}
    \vspace{-0.5em}
    \setlength\itemsep{0.15em}
    \item[$-$] od roku 1516 králem Španělským, 1519 císař SŘŘ, za něj probíhá reformace
    \item[$-$] Augsburský mír, on sám rezignuje na obojí, protože neudržel katolické náboženství
    \item[$-$] opíral se o vysokou šlechtu $\Rightarrow$
    \item[1520] \textsc{povstání komunerů} = povstání měst v čele s Toledem, Karel tvrdě potlačil, města byla nejen poražena, ale i zatížena daněmi $\Rightarrow$ ekonomické zatížení, postupně se do čela Evropy začíná dostávat spíše Anglie a Francie 
\end{itemize}


\end{document}
