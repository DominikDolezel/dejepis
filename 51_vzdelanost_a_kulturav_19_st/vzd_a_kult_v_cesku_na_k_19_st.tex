\documentclass{article}
\usepackage{fullpage}
\usepackage[czech]{babel}
\usepackage{amsfonts}

\title{\vspace{-2cm}Vzdělanost a kultura v českých zemí na konci 19. st.\vspace{-1.7cm}}
\date{}
\author{}

\begin{document}
\maketitle

\begin{itemize}
    \vspace{-0.5em}
    \setlength\itemsep{0.15em}
    \item[$-$] ČEské země -- gramotnost až 97 \%, v rámci monarchie velmi vzdělané
    \item[$-$] hustá síť základních škol, 5 let, pak 3 roky měšťanka, pak se budovala síť gymnasií, těch jsou dva typy:
    \begin{itemize}
        \vspace{-0.5em}
        \setlength\itemsep{0.15em}
        \item[$-$] )\textit{klisická}:  řečtina, latina, humanitní předměty
        \item[$-$] \textit{reálná}: matematika, moderní jazyky, přírodní vědy, rýsování, kreslení
    \end{itemize}
    \item[$-$] obchodní akademie, uměleckoprůmyslové školy, vysoké školy (rozštěpení KFU 1882 na českou a německou část), polytechnický ústav $\rightarrow$  česká a německá technika
    \item[$-$] 1. dívčí gymnasium, o jeho vznik se zasloužila Eliška Krásnohorská
    \item[$-$] Jan Otto (největší české nakladatelství), Mánes (umělecký spolek)
    \item[1891] Jubilejní zemská výstava: 1. elektrická dráha, výstaviště -- Letná, Petřínská rozhledna, Stromovka, František Křižík (oblouková lampa, tramvaje, fontána)
\end{itemize}

\subsection*{Národní divadlo}
\begin{itemize}
    \vspace{-0.5em}
    \setlength\itemsep{0.15em}
    \item[$-$] heslo \uv{národ sobě} (nad jevištěm)
    \item[1868] položení základního kamene
    \item[1881] dokončení, korunní princ Rudolf, opera Libuše, v tomtéž roce požár
    \item[1883] znovuotevření, Libuše
    \item[$-$] \textit{triga} = trojspřeží
    \item[$-$] \textit{generace Národního divadla} (umělci spjati s ND): Aleš, Ženíšek (původní opona), Mařák, Hynais (dnešní opona), první shořela, Schnirch (socha trojspřežení = \textit{triga})
\end{itemize}

\subsection*{Další tvůrci (mimo generaci ND)}
\begin{itemize}
    \vspace{-0.5em}
    \setlength\itemsep{0.15em}
    \item[$-$] Ladislav Šaloun -- socha Jana Husa na Straoměstském náměstí
    \item[$-$] Antonín Wiehl -- hrobka Slavín na Vyšehradě
    \item[$-$] Josef Václav Myslbek -- socha svatý Václav na koni, Václavské náměstí
    \item[$-$] Ema Destinnová -- úspěchy Berlín, New York
    \item[$-$] historismus: Hluboká, Bouzov
    \item[$-$] secese: Benešova ulice v Brně, Tivoli, Obecní dům v Praze
    \item[$-$] lidová tvorba v architektuře: Jurkovičova vila, Libušín
    \item[$-$] České země speciální, objevuje se u nás kubismus v architektuře -- kubistické stavby: Dům U Černé Datky Boží, Josef Chochol, Otto Gutfreund
    \item[$-$] impresionismus: Antonín Slavíček,
    \item[$-$] Josef Uprka: krojované ženy, slovácký, lidové
    \item[$-$] secese: Mucha
    \item[$-$] česká kultura, vzdělanost dosahuje světových parametrů, úrovně
\end{itemize}

\end{document}
