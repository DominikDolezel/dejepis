\documentclass{article}
\usepackage{fullpage}
\usepackage[czech]{babel}
\usepackage{amsfonts}

\title{\vspace{-2cm}Klasicismus\vspace{-1.7cm}}
\date{}
\author{}

\begin{document}
\maketitle

\begin{itemize}
    \vspace{-0.5em}
    \setlength\itemsep{0.15em}
    \item[$-$] název pochází z \textit{classicus} = vynikající, vzorový
    \item[$-$] vzniká ve Francii už za doby Ludvíka XIV. (počátek 18. st.), šíří se do Ruska a do USA
    \item[$-$] návrat do antiky
    \item[$-$] \textit{empír} = vrchol klasicismu, počátek 19. století, postupně se šíří do Evropy z Francie, spojován s Napoleonovým císařstvím
    \item[$-$] ve srovnání s barokem je patrná symetrie, strohost, snaha o střídmost, kázeň, preciznost
    \item[$-$] v této době vznikají muse, divadla a podobně
    \item[$-$] stavby: Pantheon v Paříži, Vítězný oblouk, Britské museum v Londýně, Braniborská brána v Berlíně, Walhalla (památník s bustami výrazných Němců), Petrohrad (Ermitáž), Washington (Kapitol, Bílý dům)
\end{itemize}

\subsection*{Malířství}
\begin{itemize}
    \vspace{-0.5em}
    \setlength\itemsep{0.15em}
    \item[$-$] nejrůznější antické prvky
    Jacques-Louis David: Přísaha Horatiů, Napoleonova korunovace
    \item[$-$] Jean-Auguste-Comonique Ingres: typické hnědé mandlové oči žen
\end{itemize}

\subsection*{Kladicismus u nás}
\begin{itemize}
    \vspace{-0.5em}
    \setlength\itemsep{0.15em}
    \item[$-$] Stavovské divadlo v Praze, Kynžvart, Kačina, částečně Lednice, Františkovy Lázně, Mariánské Lázně
    \item[$-$] sochařství bych ignorovala
\end{itemize}

\subsection*{Empír}
\begin{itemize}
    \vspace{-0.5em}
    \setlength\itemsep{0.15em}
    \item[$-$] monument Vittoria Emmanuela II. v Římě, divadlo u Hybernů
    \item[$-$] tak jsme to nějak přeskotačili
\end{itemize}




\end{document}
