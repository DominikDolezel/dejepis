\documentclass{article}
\usepackage{fullpage}
\usepackage[czech]{babel}
\usepackage{amsfonts}

\title{\vspace{-2cm}Poválečné Rusko\vspace{-1.7cm}}
\date{}
\author{}

\begin{document}
\maketitle

\begin{itemize}
    \vspace{-0.5em}
    \setlength\itemsep{0.15em}
    \item[$-$] zástupci Ruska nebyli pozváni na Pařížskou smlouvu
    \item[1917/18-1920/21] \textsc{občanská válka}: jeden z vůdců bolševiků Lev Davidovič Trocký proti neruským národnostem, bělogvardějcům (bílí -- monarchisté, vysocí církevní hodnostáři, bohatí) a intervenčním armádám (vojáci z dohodových zemí, kteřá se snaží pomoci bolševické oposici, aby se znovuotevřela, zahynulo minimálně 10 milionů lidí východní fronta)
    \item[(8. 3. 1917)] vytvoření komunistické strany z bolševiků
    \item[(12. 3. 1917)] Moskva hlavním městěm místo Ptrohradu
    \item[(17. 7. 1918)] poprava cara v Jekatěrinburgu
    \item[1918-1920] \textsc{válka s Polskem}, kteří ztratili území na úkor Ruska, chtějí obnovit svoje území tak, jak bylo před trojím dělením Polska -- Rusko iniciuje válku
    \item[1919] \textit{Kominterna}: komunistická internacionála, odsud se měly šířit socialistické myšlenky
    \item[30. 12. 1922] přejmenování na SSSR: Bělorusko, Ukrajina, Kavkazské republiky a postupné rozšiřování území
\end{itemize}

\subsection*{Občanská válka}
\begin{itemize}
    \vspace{-0.5em}
    \setlength\itemsep{0.15em}
    \item[$-$] hladomor, kvůli němu taky zemřelo hodně lidí
    \item[$-$] bolševiky úspěšně vede gen. Trockij, postupně ovládli centra
    \item[$-$] docházelo k bojům s legionáři
    \item[$-$] bělogvardějci
    \item[$-$] celé Rusko nastaveno na válečný komunismus -- vše bylo podřízeno vítězství bolševiků
    \item[2.-3.1921] nespokojení s tím, co dělali vojáci: vzpoura námořníků v Kronštasdtu (nelíbilo se jím počínání bolševiků), tvrdě potlačeno, symbolický konec občanské války
    \item[$-$] Michael Tuchačevskij -- jeden z velitelů rudé armády, který byl zlikvidován Stalinem ve třicátých letech, taky provedl reorganizaci rudé armády
\end{itemize}

\subsection*{Rusko-polská válka (1918-1920)}
\begin{itemize}
    \vspace{-0.5em}
    \setlength\itemsep{0.15em}
    \item[1918] obnovení Polska, mělo za cíl se obnovit pokud možno jako před trojím dělením Polska (vzali si jej Rusko, Prusko a Habsburkové)
    \item[$-$] Poláci postupovali směrem na Rusko, obsadili Kyjev
    \item[leden 1918] \textsc{útok rudé armády} na Polsko (generál Michail Tuchačevskij)
    \item[srpen 1920] zázrak \textsc{na Visle}, Poláci porazili Rusy (polský generál Józef Pilsudski)
    \item[září 1920] \textsc{bitva na řece Němenu}, Poláci získali část Litvy
\end{itemize}

\subsection*{Ekonomika}
\begin{itemize}
    \vspace{-0.5em}
    \setlength\itemsep{0.15em}
    \item[$-$] Józef Pilsudski: táhne polsko doprava, opírání o armádu, jeden z jeho ministrů Beck (zahraničí, ostře vystupoval proti Čechoslovákům v otázce Těšínska, snažil se o spolupráci s Němci)
    \item[$-$] \textit{válečný komunismus} během války, všechna výroba se soustředí na vítězství bolševiků, povolžský hladomor (smrt 5 milionů lidí)
    \item[(1921)] NEP: nová ekonomická politika, strůjcem je Nikolaj Bucharin, později popraven, v Rusku vznikají soukromé podniky, příliv zahraničního kapitálu, základ ekonomiky má být zemědělství, industrializace
    \item[1929] direktivní ekonomika, první pětiletá plán
    \item[$-$] \textit{kolektivizace}: združstevnění, vznik zemědělských družstev, likvidace velkých majitelů půdy, zakládání \textit{kolchozů} (venkovská zemědělská družstva)
    \item[1919] \textit{gulag}: (glavnoje upravlenije lagerej), pracovní tábory, končí tam opozice (liberálové, umělci)
    \item[30. léta] hladomor na Ukrajině
    \item[$-$] bělomořsko-baltský kanál: pokus spojit Bílé a Baltské moře, nefunkční, byl málo hluboký
\end{itemize}

\subsection*{Politika}
\begin{itemize}
    \vspace{-0.5em}
    \setlength\itemsep{0.15em}
    \item[1922] V. I. Lenin vyřazen z politiky, místo něj nastupuje \textbf{Josif Vissarionovič Stalin}
    \item[24.1.1924] Lenin umírá jako lidská troska
    \item[$-$] boj o moc kolem Stalina se pohybují Zinověv a Kameněv, na druhé straně Trockij, včichni chtějí být generálním tajemníkem, moc na sebe strhl Stalin a postupně se ostatních zbavil
    \item[(1928)] vyloužení Zimověva a Kameněva z komunistické strany
    \item[1936] Zimověv a Kameněv popraveni, Trockij vypovězen, skončil v Mexiku
    \item[$-$] Frida Kahlo, jedna z nejznámějších mexických umělkyň, s Trockým měla jakýsi románek
    \item[(1940)] zavražděn Trockij
    \item[30. léta] čistky, stály stát obrovské síly, protože třeba za druhé světové války potom neměli odborníky
    \item[$-$] ČEKA, NKVD, KGB: tajné policie, stalinův noschled \textbf{Berija} spravoval gulagy nebo Katyňský masakr (1940)
    \item[$-$] zahraniční politika: sovětské Rusko v mezinárodní izolaci, vytáhli se z ní díky:
    \item[4.-5.1922] \textit{Konference v Janově}, měla se tu řešit otázka Německých reparací, byli tam i zástupci Němců a Rusů a ti se pak sešli v blízkém Rapaldu, kde vznikla
    \item[16.4.1922] \textit{Rapallská smlouva}: navzájem si odpustili reparace, tím se dostali z politické izolace
    \item[1934] vstup Ruska do Společenství národů, protože se evropské mocnosti bály Hitlera, jenže po útoku na Finsko jsou zase vyhozeni   
\end{itemize}




\end{document}
