\documentclass{article}
\usepackage{fullpage}
\usepackage[czech]{babel}
\usepackage{amsfonts}

\title{\vspace{-2cm}Poválečné Rusko\vspace{-1.7cm}}
\date{}
\author{}

\begin{document}
\maketitle

\begin{itemize}
    \vspace{-0.5em}
    \setlength\itemsep{0.15em}
    \item[$-$] zástupci Ruska nebyli pozváni na Pařížskou smlouvu
    \item[1917/18-1920/21] \textsc{občanská válka}: jeden z vůdců bolševiků Lev Davidovič Trocký proti neruským národnostem, bělogvardějcům (bílí -- monarchisté, vysocí církevní hodnostáři, bohatí) a intervenčním armádám (vojáci z dohodových zemí, kteřá se snaží pomoci bolševické oposici, aby se znovuotevřela, zahynulo minimálně 10 milionů lidí východní fronta)
    \item[(8. 3. 1917)] vytvoření komunistické strany z bolševiků
    \item[(12. 3. 1917)] Moskva hlavním městěm místo Ptrohradu
    \item[(17. 7. 1918)] poprava cara v Jekatěrinburgu
    \item[1918-1920] \textsc{válka s Polskem}, kteří ztratili území na úkor Ruska, chtějí obnovit svoje území tak, jak bylo před trojím dělením Polska -- Rusko iniciuje válku
    \item[1919] \textit{Kominterna}: komunistická internacionála, odsud se měly šířit socialistické myšlenky
    \item[30. 12. 1922] přejmenování na SSSR: Bělorusko, Ukrajina, Kavkazské republiky a postupné rozšiřování území
\end{itemize}

\subsection*{Občanská válka}
\begin{itemize}
    \vspace{-0.5em}
    \setlength\itemsep{0.15em}
    \item[$-$] hladomor, kvůli němu taky zemřelo hodně lidí
    \item[$-$] bolševiky úspěšně vede gen. Trockij, postupně ovládli centra
    \item[$-$] docházelo k bojům s legionáři
    \item[$-$] bělogvardějci
    \item[$-$] celé Rusko nastaveno na válečný komunismus -- vše bylo podřízeno vítězství bolševiků
    \item[2./3. 1921] nespokojení s tím, co dělali vojáci: vzpoura námořníků v Kronštasdtu (nelíbilo se jím počínání bolševiků), tvrdě potlačeno, symbolický konec občanské války
    \item[$-$] Michael Tuchačevskij -- jedn z velitelů rudé armády, který byl zlikvidován Stalinem ve třicátých letech, taky provedl reorganizaci rudé armády
\end{itemize}


\end{document}
