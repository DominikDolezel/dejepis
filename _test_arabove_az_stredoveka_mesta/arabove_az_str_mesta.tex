\documentclass{article}
\usepackage{fullpage}
\usepackage[czech]{babel}
\usepackage{amsfonts}
\usepackage{multicol}

\title{\vspace{-2cm}Arabové, křížové výpravy, středověká města, Francie a Anglie, stoletá válka, válka růží, středověká města\vspace{-1.7cm}}
\date{}
\author{}

\begin{document}
\maketitle

\section*{Arabové}

\begin{itemize}
    \vspace{-0.5em}
    \setlength\itemsep{0.15em}
    \item[$-$] Semité, na Arabském poloostrově
    \item[$-$] \textit{beduín} = člen kmenu, \textit{šajcha} = vůdce kmenu
    \item[$-$] Velká karavanní obchodní cesta (mezi Orientem a Čínou)
    \item[$-$] postupně vznik aristokracie a šlechty
    \item[$-$] \textbf{Mekka} = náboženská i obchodní křižovatka
    \item[6. st.] začínají ohrožovat Etiopané a Íránci $\rightarrow$ sjednotitel \textbf{Mohamed} z rodiny Kurajšovců
\end{itemize}

\subsection*{Mohamed (570 -- 632)}
\begin{itemize}
    \vspace{-0.5em}
    \setlength\itemsep{0.15em}
    \item[$-$] oženil se s vdovou Chadidže
    \item[(613)] začíná hlásat islám, předává učení \textbf{Alláha}, považoval se za posledního proroka
    \item[$-$] zdroje islámu: judaismus, křesťanství, jednotlivá kmenová polyteistická náboženství
    \item[$-$] \textit{súry} = Mohamedova zjevení, seskládány do posvátného textu = \textit{Korán}
    \item[$-$] \textit{Sunna} = posvátný text o Mohamedových činech
    \item[$-$] 5 pilířů islámu: pouť do Mekky, v měsíci ramadánu půst, modlit se pětkrát denně, víra v jediného boha, dávat almužnu chudým
    \item[$-$] \textit{džihád} = povinnost šířit islám mečem a ohněm
    \item[622] \textit{hidžra} = \uv{vystěhování}, odešel od Mediny, kde založil první muslimskou obec, počátek muslimského kalendáře
    \item[630] \textsc{dobytí Mekky} Mohamedem
    \item[632] smrt
\end{itemize}

\subsection*{Chalífát}
\begin{itemize}
    \vspace{-0.5em}
    \setlength\itemsep{0.15em}
    \item[$-$] po Mohamedově smrti jsou jeho (\textit{chalífové}) nástupci volení
    \item[$-$] bujná expanze
    \item[$-$] teokratický stát: spojuje je náboženství, vláda je propojena s vírou
    \item[$-$] otázka nástupnictví: rozdělení na \textit{šíity} (uznávají jako panovníky jen Mohamedovy přímé potomky) a \textit{sunnity} (stačí většinová shoda, dnes víc)
\end{itemize}

\subsection*{Umajjovci (661 -- 749)}
\begin{itemize}
    \vspace{-0.5em}
    \setlength\itemsep{0.15em}
    \item[$-$] nové hlavní město \textbf{Damašek}
    \item[$-$] expanze na sever do Arménie, Byzanc, Kypr, do Z Indie
    \item[711] na Pyrenejském poloostrově, kde vybudovali pevnost Tárikova skála
    \item[$-$] nejvýznamější centrum na Pyr. pol.: \textbf{Cordoba}
    \item[732] \textsc{v bitvě u Poitiers} a \textsc{u Tours} Karel Martell zastavil postup Arabů
\end{itemize}


\subsection*{Abbásovci (749 -- 1258)}
\begin{itemize}
    \vspace{-0.5em}
    \setlength\itemsep{0.15em}
    \item[$-$] \textit{vezír} = zástupce chalífy
    \item[$-$] území se postupně rozpadá na jednotlivé \textit{emiráty}
    \item[$-$] nové hlavní město \textbf{Bagdád}
    \item[$-$] \textbf{Hárún ar-Rašíd}, jeden z nejvýznamnějších chalífů, současník Karla Velikého, podpra vědy a kultury
    \item[$-$] \textit{iktá} = obdoba léna, ale nikdy dědičné
    \item[1071] \textsc{bitva u Mantzikertu, vpád Seldžuků} $\rightarrow$ zánik Arabské říše
    \item[$-$] \textit{reconquista} = \uv{znovudobytí}: od 11. století evropští křesťané vytlačují Araby
    \item[1492] v Evropě definitivně poraženi v \textsc{bitvě u Granady}
\end{itemize}


\subsection*{Kultura}
\begin{itemize}
    \vspace{-0.5em}
    \setlength\itemsep{0.15em}
    \item[$-$] centra Bagdád, Damašek, Alexandrie, Cordoba
    \item[$-$] přinesli nám poznatky Aristotela
    \item[$-$] geografie: Ibn Fadlán, \textbf{Ibrahím ibn Jákúb} (10. st., zastavil se v Praze, kterou pozitivně hodnotí), Ibn Ibrisi
    \item[$-$] astronomie, nástroje na pozorování hvězd
    \item[$-$] matematika, arabské číslice
    \item[$-$] fyzika, optické přístroje, léky
    \item[$-$] medicína, oční lékařství, brýle
    \item[$-$] broskve, růže, lilie
    \item[$-$] filosofie, Ibn Síná (Avicenna), Ibn Rušd (Averroes)
    \item[$-$] Pohádky tisíce a jedné noci
    \item[$-$] řemeslo: textilnictví, tolecké dýky, damastencké meče, sklo, porcelán, voňavky, papír
    \item[$-$] mešita, minaret, muezzín (kdo svolává na mši), oslí hřbet = typ oblouku
    \item[$-$] paláce v Cordobě a Granadě
    \item[$-$] mozaiky, arabesky = nepřítomnost figur na kresbách, afigurálnost, kaligrafie = krasopis
\end{itemize}

\section*{Vrcholný středověk -- obecný úvod}
\begin{itemize}
    \vspace{-0.5em}
    \setlength\itemsep{0.15em}
    \item[$-$] osidlování dříve neosidlených území
    \item[$-$] rozšiřování orné půdy $\rightarrow$ větší výnosy, rozvoj řemesla a obchodu
    \item[$-$] peněžní hospodářství,  ve městech univerzity
    \item[$-$] poddanská povstání, morové epidemie (umřela $1 / 3$ obyvatelstva)
    \item[13. st.] církev na vrcholu
    \item[$-$] gotická kultura
    \item[$-$] \textit{stavové} = šlechta, církev, měšťané
    \item[$-$] vpád Mongolů do Evropy
\end{itemize}


\section*{Křížové výpravy = kruciáty (1095 -- 1291)}
\begin{itemize}
    \vspace{-0.5em}
    \setlength\itemsep{0.15em}
    \item[$-$] výpravy křižáků do Levanty, která je ovládána Araby
    \item[$-$] příčiny: dobytí svaté země; \textit{majorát} = dědičné právo, první syn dědí vše, proto se ostatní účastní výprav, aby získali majetek; italská města -- jsou tu Seldžukové, nemohou provozovat obchod
    \item[$-$] vyhlašuje papež
    \item[1071] \textsc{bitva u Mantzikertu}, Seldžukové porážejí Byzantince
    \item[1076] Jeruzalém, Boží hrob
\end{itemize}

\subsection*{První křížová výprava 1095 -- 1099}
\begin{itemize}
    \vspace{-0.5em}
    \setlength\itemsep{0.15em}
    \item[$-$] vyhlásil \textbf{Urban II.}, má problematický vztah s Jindřichem IV.
    \item[1095] \textsc{koncil v Piazenze}, byzantský císař \textbf{Alexios I.} posílá posly, aby papeže požádali o pomoc v boji se Seldžuky
    \item[1095] \textsc{koncil v Klermontu}, vyhlášení první křížové výpravy
    \item[(1096)] lidová křížová výprava, nebyla úspěšná
    \item[$-$] poté i rytíři, probojují se do oblasti Levanty, získají svatá místa
    \item[1099] dobytí Jeruzaléma $\rightarrow$ svaté místo
    \item[$-$] křižácké státy podřízené Jeruzalémskému království: Antiochejské knížectví, Tripoliské hrabství, Edesské hrabství
\end{itemize}

\subsection*{Druhá křížová výprava (1147 -- 1149)}
\begin{itemize}
    \vspace{-0.5em}
    \setlength\itemsep{0.15em}
    \item[$-$] francouzský král Ludvík VII., Přemyslovec Vladislav II., SŘŘ KOnrád III.
    \item[$-$] též neúspěšná, úzeí nerozšířeno, sály vyrovnána
    \item[1187] \textsc{bitva u Hattínu} sultán \textbf{Saladin} poráží křižáky, dobyl Jeruzalém a další, konec rovnováhy
\end{itemize}


\subsection*{Třetí křížová výprava (1189 -- 1192)}
\begin{itemize}
    \vspace{-0.5em}
    \setlength\itemsep{0.15em}
    \item[$-$] císař SŘŘ Fridrich Barbarossa, francouzský král Filip II. August, anglický král Richard Lví Srdce
    \item[$-$] Fridrich během tažení utonul
    \item[$-$] též neúspěch, jenom povolení návštěv Jeruzaléma
    \item[$-$] \textsc{znovudobytí strategických bodů Akkonu a Jaffy}
\end{itemize}


\subsection*{Čtvrtá křížová výprava (1202 -- 1204)}
\begin{itemize}
    \vspace{-0.5em}
    \setlength\itemsep{0.15em}
    \item[$-$] původně měla směřovat do Egypta, cíl: oslabit muslimy v tomto prostředí
    \item[$-$] jenže žoldáci zdrženi v Benátkách dóžetem \textbf{Enricem Dandolem}, takže vtrhli a zlikvidovali Byzanc
\end{itemize}


\subsection*{Konec křížových výprav}
\begin{itemize}
    \vspace{-0.5em}
    \setlength\itemsep{0.15em}
    \item[1291] \textsc{pád Akkonu}, konec křížových výprav
    \item[$-$] křižáci se stahují zpět
\end{itemize}


\subsection*{Důsledky}
\begin{itemize}
    \vspace{-0.5em}
    \setlength\itemsep{0.15em}
    \item[$-$] církev na vrcholu, bohatství a moc
    \item[$-$] centralizace, posílené královské moci
    \item[$-$] rozvoj měst na Apeninském poloostrově: Benátky, Janov
    \item[$-$] peněžní hospodářství, nové plodiny, technologie, hrady, gotický sloh
    \item[$-$] rytířská kultura, tři rytířské řády: johanité, templáři, němečtí rytíři
    \item[$-$] prolínání různých kultur
\end{itemize}


\section*{Mocenské soupeření mezi Anglií a Francií, 12. -- 15. st.}
\subsection*{Anglie: Normandská dynastie}
\begin{itemize}
    \vspace{-0.5em}
    \setlength\itemsep{0.15em}
    \item[1066] zakladatel \textbf{Vilém I. Dobyvatel} poražením \textbf{Harolda II. Godwindsona} v \textsc{bitvě u Hastings}
    \item[$-$] korunován ve Westminsteru
    \item[$-$] Normandie součástí Anglie, protože jej v roce 911 Vikingové dostali od Francouzů
    \item[$-$] Vilém přiděluje přivržencům drobná léna, aby neměli moc velkou moc
    \item[$-$] dělení území na \textit{hrabství}, v jejich čele \textit{šerifové} = královští úředníci, dbají na odvádění daní, soudní pravomoce
    \item[(1086)] \textit{Domesday Book} = soupis veškerého majetku, kde se co nachází $\rightarrow$ později k zavedení daní
\end{itemize}


\subsection*{Anglie: Dynastie Plantagenetů}
\subsubsection*{Jindřich II. Plantagenet}
\begin{itemize}
    \vspace{-0.5em}
    \setlength\itemsep{0.15em}
    \item[$-$] \textbf{Thomas Becket}, Jindřichův přítel, arcibiskup, chce vytvořit církev nezávislost na králi, později zabit
    \item[$-$] \textit{putující soudci} = kontroloři šerifů
    \item[$-$] \textbf{Eleonora Akvitánská}, manželka francouzského krále, po rozvodu se provdala za Jindřicha $\rightarrow$ územní zisky, Francouzi ztácí přístup k moři na západě
    \item[$-$] francouzský král \textbf{Filip II. August} chce území dobýt zpět
\end{itemize}
\subsubsection*{Richard Lví srdce}
\begin{itemize}
    \vspace{-0.5em}
    \setlength\itemsep{0.15em}
    \item[$-$] třetí křížová výprava, při cestě zpět zajat v Rakousku (1192), vykoupen zpět
    \item[$-$] \textsc{válka s Filipem II.}, neúspěch (nic nedobyli), Richard poraněn šípem, umírá
\end{itemize}
\subsubsection*{Jan Bezzemek, bratr Richardův}
\begin{itemize}
    \vspace{-0.5em}
    \setlength\itemsep{0.15em}
    \item[$-$] \textsc{bitva u Bouvignes} proti Filipovi, ztrácí většinu území ve Francii
    \item[1215] \textit{Magna charta libertatum} (velká listina svobod) = oslabil pozici anglického krále $\rightarrow$ zárodky budoucího parlamentu, Velká královská rada $\rightarrow$ budoucí horní sněmovna lordů
\end{itemize}
\textbf{Eduard I.} definitivně přiojil Wales\\
\textbf{Eduard II.} jeho manželka ho donutila se vzdát královského titulu ve prospěch Eduarda III., jeho syna


\subsection*{Francie: Kapetovci}
\begin{itemize}
    \vspace{-0.5em}
    \setlength\itemsep{0.15em}
    \item[$-$] zakladatel \textbf{Hugo Kapet}, zvolen 987
    \item[$-$] \textbf{Filip II. August} získává území Angličanů vš. Normandie
    \item[$-$] Ludvík IX. Svatý
\end{itemize}
\subsubsection*{Filip IV. Sličný}
\begin{itemize}
    \vspace{-0.5em}
    \setlength\itemsep{0.15em}
    \item[$-$] vrchol centralizace
    \item[$-$] chce zdanit církev $\rightarrow$ konflikt s papežem \textbf{Bonifácem VIII.} $\rightarrow$ nechal ho zatknout, Bonifác ve vězení umírá
    \item[$-$] na místo papeže dosazuje \textbf{Klimenta V.}, donutí ho, aby církev přesídlila z Říma do Avignonu = \textit{avignonské zajetí}
    \item[$-$] templáři mají na jeho vkus moc majetku $\rightarrow$ obviní je z kacířství, zatýká \textbf{Jacquesa de Molay} (poslední velmistr templářů, později upálen), tento řád zrušil, špičky upáleny
    \item[$-$] \uv{černý pátek} $\dots$ zajetí onoho velmistra probehlo v pátek 13.
    \item[$-$] chce získat Flandry, neúspěch
    \item[1302] Filip potřebuje pomoc s dobýváním, začíná svolávat tzv. \textit{generální stavy} = zástupci šlechty, církve a měst, např. rozhodují o daních $\rightarrow$ získá si je na svou stranu
\end{itemize}


\subsection*{Příčiny stoleté války}
\begin{itemize}
    \vspace{-0.5em}
    \setlength\itemsep{0.15em}
    \item[$-$] Anglie chce území ve Francii: Flandry, protože tam vyváží vlnu
    \item[1328] vymírají Kapetovci (Karel IV.), provdá svou dceru za \textbf{Filipa z Valois}, proti tomu se vymezili Angličané (Eduard III.), vytváří si nároky na Francouzskou korunu
    \item[$-$] Filip VI. z Valois zakládá novou dynastii, chce získat Flandry pod svou nadvládu $\rightarrow$ chce být králem celé Francie $\rightarrow$ válka
\end{itemize}


\section*{Stoletá válka 1337 -- 1453}
\begin{itemize}
    \vspace{-0.5em}
    \setlength\itemsep{0.15em}
    \item[1337] \textsc{začátek války}, Francie napadla Gaskoňsko, Anglie zničila francouzskou flotilu u Flander
    \item[1346] \textsc{bitva u Kresčaku}, Angličané vítězí i když Francouzi mají až dvakrát větší armádu díky \textit{lučišníkům}, účastnil se i český král \textbf{Jan Lucemburský}
    \item[$\rightarrow$] \textbf{Eduard III.} dobyl přístav Calais, několik dalších invazí $\rightarrow$ \textsc{výhodný mír} (1360)
    \item[(1348)] \textsc{morová epidemie} = \textit{Černá smrt}  -- původně z Kazachstánu, šířeno pomocí blech, zemřela třetina evropské populace $\rightarrow$ málo jídla, daní, lidí, pracovní síly
\end{itemize}

\subsection*{Jindřich V.}
\begin{itemize}
    \vspace{-0.5em}
    \setlength\itemsep{0.15em}
    \item[$-$] anglický král
    \item[$-$] zahájení nové invaze, spojenectví s Burgundskem, kde zuří občanská válka
    \item[1420] poražení Francouzů (Karel VI.) v \textsc{bitvě u Azincourtu}, Anglie získává rozsáhlá území ve Francii
    \item[$-$] francouzský král mu dal za ženu svou dceru, Jindřichovy děti budou francouzští králové
    \item[(1422)] náhle umírá, jeho děti jsou malé, Anglii nemá kdo vést, postupně prohrává
\end{itemize}

\subsection*{Johanka z Arku}
\begin{itemize}
    \vspace{-0.5em}
    \setlength\itemsep{0.15em}
    \item[$-$] \textit{panna orleánská} $\dots$ osvobodila Orleans
    \item[$-$] z V Francie, obyčejná dívka
    \item[$-$] dle legendy má vidění, mluvili s ní svatí, vnímá se jako vyvolená $\rightarrow$ jede za francouzský králem \textbf{Karlem VII.}, dodá mu odvahu, že Francie vyhraje $\rightarrow$ svěří jí malou vojenskou jednotku
    \item[$-$] odjíždí s touto vojenskou jednotkou do Orleans, Francouzi \textsc{osvobodí Orleans a Remeš} $\rightarrow$ důkaz výroku, že Johanka je vyvolená Bohem, Remeš se vzdává bez boje
    \item[(1430)] zajata Burgunďany a prodána Angličanům, obviněna z čarodějnictví (vede boj, nosí mužské oblečení, komunikuje s ďáblem) $\rightarrow$ odsouzena a o rok později upálena
\end{itemize}

\subsection*{Karel VII.}
\begin{itemize}
    \vspace{-0.5em}
    \setlength\itemsep{0.15em}
    \item[$-$] francouzský král, nepřestává ve vítězstvích proti Anglii
    \item[$-$] nový systém vybírání daní, důsledné dodržování, \textit{daň ze soli}
    \item[1453] \textsc{bitva u Castillonu} -- \textsc{konec války}: Francie \textsc{dobyje zpět Gaskoňsko}, Anglie na severu odražena, Francie má své území celé pod kontrolou až na přístav Calais, kterému vládne Anglie
\end{itemize}

\subsection*{Důsledky}
\begin{itemize}
    \vspace{-0.5em}
    \setlength\itemsep{0.15em}
    \item[$-$] sjednocení Francie, ovládá celé území vs. zdevastovaná a vyrabovaná, po moru
    \item[$-$] centralizace panovnické moci francouzských králů, snížen počet šlechticů
    \item[$-$] Anglie po prohře upadá do občanské války = \textsc{válka růží}, jsou na pokraji úpadku
    \item[$-$] počátek nepřátelství mezi Anglií a Francií
\end{itemize}



\section*{Francie po stoleté válce}
\begin{itemize}
    \vspace{-0.5em}
    \setlength\itemsep{0.15em}
    \item[$-$] vládne \textbf{Ludvík XI. z Valois}, posílí pozici panovníka, centralizace, daň \textit{taille}
    \item[$-$] počátek novodobé Francie, prosazení francouzštiny
    \item[$-$]rozvoj hospodářství a obchodu, chrám Notre Dame
\end{itemize}


\section*{Anglie po stoleté válce}
\begin{itemize}
    \vspace{-0.5em}
    \setlength\itemsep{0.15em}
    \item[$-$] dosavadní dynastii \textbf{Lancaster} vystřídá dynastie \textbf{York}
\end{itemize}

\subsection*{Válka růží 1455 -- 1485}
\begin{itemize}
    \vspace{-0.5em}
    \setlength\itemsep{0.15em}
    \item[$-$] de facto občanská válka
    \item[$-$] sever: Lancaster, červená růže vs. jih: York, bílá růže
    \item[$-$] vyvrcholení za \textbf{Richarda III.} z Yorku, krutá vláda, proti němu \textbf{Jindřich Tudor} z Lancasteru
    \item[1485] \textsc{bitva u Bosworthu}, Jindřich poráží Richarda, který umírá $\rightarrow$ konec dynastie Plantagenetů, nastupují Tudorovci, první \textbf{Jindřich VII. Tudor}
\end{itemize}


\section*{Brno}
\begin{itemize}
    \vspace{-0.5em}
    \setlength\itemsep{0.15em}
    \item[$-$] spojování čtyř osadních celků: český, německý, flanderský, židovský
    \item[1243] městská privilegia udělil \textbf{Václav I.} ve Velkém a Malém privilegium
    \item[$-$] u Mendlova náměstí existovalo už v 11. st. slovanské hradiště, po vybudování Špilberku ztrácí na významu
\end{itemize}


\section*{Středověká města}
\begin{itemize}
    \vspace{-0.5em}
    \setlength\itemsep{0.15em}
    \item[$-$] příčiny: změny v zemědlěství $\rightarrow$ více lidí $\rightarrow$ specializace povolání, obchodní styky $\rightarrow$ peníze $\rightarrow$ potřeba ochrany $\rightarrow$ opevněná města
    \item[$-$] \textit{kolonizace} = zakládání nových osad, měst, od 11. st., vrchol v 13. st. za Přemysla Otakara II.
    \item[$-$] v našem prostředí vliv německých kolonizátorů, zvlášť v pohraničí $\rightarrow$ dvojjazyčnost
    \item[$-$] vznik měst:
    \begin{enumerate}
        \vspace{-0.5em}
        \setlength\itemsep{0.15em}
        \item navazují na starší osídlení, postupně se rozrůstá (podhradí, řemeslné osady, poblíž dolů)
        \item zakládání na zeleném drnu -- \textit{lokátoři} = \uv{developer} $\dots$ vyměří obvod města, parcely, náměstí, sežene budoucí obyvatele, zaplatí výstavbu, za odměnu získá ve městě půdu, dům a někdy se stane rychtářem
    \end{enumerate}
    \item[$-$] ti, co moc zlobili, se trestají na \textit{pranýři}
    \item[$-$] druhy měst:
    \begin{enumerate}
        \vspace{-0.5em}
        \setlength\itemsep{0.15em}
        \item \textit{královská} $\dots$ podřízena králi, největší a nejvýznamnější (Kadaň, Brno, Znojmo)
        \begin{itemize}
            \vspace{-0.5em}
            \setlength\itemsep{0.15em}
            \item[$-$] horní -- naleziště drahých kovů (Jihlava, Kutná Hora)
        \end{itemize}
        \item \textit{věnná} $\dots$ král daroval své choti, aby na něm nebyla závislá (Dvůr Králové)
        \item \textit{poddanská} $\dots$ patřila šlechtě nebo církvi (Český Krumlov)
    \end{enumerate}
    \item[$-$] městská práva:
    \begin{multicols}{2}
        \begin{itemize}
            \vspace{-0.5em}
            \setlength\itemsep{0.15em}
            \item[$-$] \textit{hrdelní} = popravovat lidi
            \item[$-$] \textit{trhové} = provozovat trhy
            \item[$-$] \textit{hradební} = mít hradby
            \item[$-$] \textit{mílové} = právo monopolu v okruhu 10 km
            \item[$-$] \textit{soudní} = právo soudit
            \item[$-$] \textit{várečné} = právo vařit pivo
        \end{itemize}
    \end{multicols}
    \item[$-$] \textit{rychtář} = správce, vybírá daně, policejní funkce; postupem času je více bohatých, rychtář ztrácí funkci $\rightarrow$ městská rada
    \item[$-$] \textit{městská rada} = radnice, 12 konšelů (nejbohatší měšťané), v čele \textit{purkmistr} = jeden z konšelů, po měsíci se střídají
    \item[$-$] obyvatelé = \textit{měšťané} :
    \begin{itemize}
        \vspace{-0.5em}
        \setlength\itemsep{0.15em}
        \item[$-$] \textit{patriciát} = bohatí obchodníci
        \item[$-$] mistři řemesla = vlastní dům s dílnou
        \item[$-$] \textit{tovaryši} = staří pracovníci v dílnách, \textit{učedníci} = mladí pracovníci v dílnách, \textit{sluhové}, \textit{nádeníci} = pracovníci najímáni na jeden den
        \item[$-$] \textit{žebráci} -- pevnou součástí lidské společnosti
        \item[$-$] Židé často žijí v oddělených oblastech = \textit{ghetto}
    \end{itemize}
    \item[$-$] \textit{cechy} = spolky různých řemeslníků, kteří se sdružují proti konkurenci
    \begin{itemize}
        \vspace{-0.5em}
        \setlength\itemsep{0.15em}
        \item[$-$] v čele \textit{cehmistr}
        \item[$-$] např. koláři, hrnčíři, kováři, ranhojiči, mlynáři, krejčí, $\dots$
        \item[$-$] určovali si ceny a množství výrobků
        \item[$-$] účastnili se náboženských procesí
        \item[$-$] od 13. do 19. století, po industriální revoluci brzdy pokroku $\rightarrow$ zrušeny
    \end{itemize}
    \item[$-$] \textit{hanzovní města} = města na severu u moře
    \item[$-$] \textit{hanza} = seskupení hanzovních měst, spojí se proti nepříteli, obchodují)
    \item[14. st.] Dánsko konkurencí $\rightarrow$ hanzovní města se spojují do Kolínské federace
    \item[1370] Dánsko poraženo $\rightarrow$ spojenectví Kalmarská unie se Švédskem a Norskem

\end{itemize}}


\end{document}
