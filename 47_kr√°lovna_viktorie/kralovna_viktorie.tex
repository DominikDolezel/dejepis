\documentclass{article}
\usepackage{fullpage}
\usepackage[czech]{babel}
\usepackage{amsfonts}

\title{\vspace{-2cm}Anglie za královny Viktorie (1819-1901)\vspace{-1.7cm}}
\date{}
\author{}

\begin{document}
\maketitle

\begin{itemize}
    \vspace{-0.5em}
    \setlength\itemsep{0.15em}
    \item[$-$] královnou se stává až v roce 1837
    \item[1876] císařovnou indie
    \item[2. pol. 19. st.] dovršení průmyslové revoluce, v této době nejvyspělejší země světa, v téže době se do popředí dostává i Německo a USA
    \item[$-$] zejména textilní průmysl, koloniální velmoc
    \item[$-$] \uv{královna moří a oceánů}, britské impérium tvoří pětinu světa
    \item[$-$] nejsilnější vojenské i obchodní loďstvo
    \item[$-$] demokratický parlamentarismus, čím dál víc větší roli hraje vláda, která je odpovědná parlamentu, síla královny postupně klesá
    \item[1865] volební reforma, volebná právo majhí všichni vlastníci nemovitostí
    \item[1851] \textit{Londýnská světová výstava}, první světová výstava, využívá dlouhého období míru, v londýnském křišťálovém paláci se presentovaly různé státy, nejvelkolepější měli právě Angličané
\end{itemize}

\subsection*{Rodinný život}
\begin{itemize}
    \vspace{-0.5em}
    \setlength\itemsep{0.15em}
    \item[$-$] manželem její bratranec, princ Albert
    \item[$-$] poslední panovnice z Hannoverské dynastie
    \item[$-$] 9 dětí, 42 vnoučat, manželství po celé Evropě, nazývána \uv{evropskou babičkou}
    \item[$-$] její syn Eduard VII. nastoupí už jako dynastie Windsor
\end{itemize}

\subsection*{Vláda}
\begin{itemize}
    \vspace{-0.5em}
    \setlength\itemsep{0.15em}
    \item[$-$] sídlila v nově vybudovaném Buckinghanském paláci
    \item[$-$] rádcem její strýc, tehdejší belgický král Leopold
    \item[$-$] během její vlády na ni bylo neúspěšně spácháno několik atentátů
\end{itemize}

\subsection*{Irsko}
\begin{itemize}
    \vspace{-0.5em}
    \setlength\itemsep{0.15em}
    \item[1845] postiženo neúrodou brambor, více než milion životů a emigrace dalšího milionu lidí, tehdy turecký sultán nabídl větší peněžní pomoc než Anglie
\end{itemize}

\begin{itemize}
    \vspace{-0.5em}
    \setlength\itemsep{0.15em}
    \item[1861] smrt prince Alberta, po celý zbytek života nosí černý šat jako symbol
    \item[1901] umírá
\end{itemize}


\subsection*{Politická situace}
\begin{itemize}
    \vspace{-0.5em}
    \setlength\itemsep{0.15em}
    \item[$-$] ve vládě se střídají konzervativci (\textit{thoriové}) a liberálové (\textit{Whigové}), vzniká také dělnická strana: \textit{Labour party}
    \item[$-$] na konci století se konstituuje \textit{hnutí sufrežetek} -- ženy, jež usilují o volební právo
    \item[$-$] těsně před první světovou válkou je premiérem    David George
    \item[$-$] zrušení obilních zákonů, jež chránily anglický trh před dovozem levného obilí ze zahraničí, to bylo nevýhodné pro městské obyvatelstvo, proti těmto zákpnům vystupují jak whigové, tak jeden z toryovských premiérů
    \item[$-$] \uv{královna moří a oceánů}, nejvýznamnější kolonií je Indie, která byla pod Anglickou správou do roku 1704, ta byla zrušena kvůli povstání Sikhů?, poté to není Východoindická společnost, ale  něco , aerlfjvbasfbvasdlkh viktorie je císařství, Viktorie císařovna
    \item[$-$] Britové z Indie pronikají i do okolních zemí jak Afghánistánu a Persie, nad Afghánistánem vyhlásili protektorát a Persini třetina nezávislá, třetina Rusové a třetina Britové
    \item[$-$] ještě za existence Východoindické společnosti problém vyvážení opia do Číny, Číňani jsou závislí na opiu, vypuknutí \textsc{opiových válek}, britové vyhráli, Čína má otevřít pět svých přístavů, Hong Kong přišel pod britskou správu
    \item[$-$] v dalších sdvou opiových válkách se k Británii připojila i Francie, opět úspěšné
    \item[1865] radikální volební reforma, mohou volit všichni, kteří mají nějakou nemovitost
    \item[$-$] liberálové prosazují školní docházku, omezují privilegia anglikánské církve, zatahovali liberalismus do ekonomiky, konzervativci omezují pracovní dobu na maximálně 56 hodin týdně, snaží se zlepšit životní péči, díky nim taky Britové drží Kypr, protože podporovali expanzní politiku
    \item[$-$] \textsc{búrské války} s Holanďany v Jižní Africe, vyhrávají Britové
    \item[$-$] na zákoladě snah kanady a Austrálie o nezávislost jim Británie udělila status dominia, to jest status větší nezávislosti a samosprávy
    \item[$-$] problematická situace s Irskem, kteří dlouho usilovali o odtržení, až po první světové válce se v roce 1921 se Severní a Jižní Irsko oddělilo
\end{itemize}




\end{document}
