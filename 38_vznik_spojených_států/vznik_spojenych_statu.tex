\documentclass{article}
\usepackage{fullpage}
\usepackage[czech]{babel}
\usepackage{amsfonts}

\title{\vspace{-2cm}Vznik Spojených států\vspace{-1.7cm}}
\date{}
\author{}

\begin{document}
\maketitle

\subsection*{Anglické kolonie}
\begin{itemize}
    \vspace{-0.5em}
    \setlength\itemsep{0.15em}
    \item[$-$] kolonizují především Angličané, Francouzi a Nizozemci
    \item[1624-1752] 13 anglických kolonií
    \item[1584-1587] Vitginie, anglická osada na ostrově Roanoke (mořeplavci Francis Drake, Walter Ralleigh) $\rightarrow$ akce neúspěšná, osada se neudržela
    \item[1607] až za Startovce Jakuba I. se zakládá \textit{Virginia} jako první anglická osada, \textbf{Jamestown}
    \item[$-$] \textbf{Mayflower} -- otcové poutníci, zakládají osady: 1620 z Plymouthu, 1630 založení Bostonu
    \item[$-$] Den díkuvzdání: děkuji indiánům, že je naučili pěstovat kukuřici a poskytli jim potravu pro přežití
    \item[$-$] jsou tu i Nizozemci $\rightarrow$ \textsc{anglo-nizozemská válka}, nizozemské město \textbf{Nový Amsterdam}, vyhrávají Angličané $\rightarrow$ \textbf{New York}
    \item[$-$] anglo-francouzská válka = \textsc{sedmiletá válka}, Angličané získávají Kanadu
    \item[$-$] každá kolonie má vlastní samosprávu, krále zastupuje guvernér, nemají zastoupení v anglickém parlamentu
    \item[$-$] kolonie postupně zabírají další území, Angličané však tato území považují za svá
    \item[$-$] utlačování a likvidace indiánů
\end{itemize}

\subsection*{Jiří III. z Hannoverovců (1760-1820)}
\begin{itemize}
    \vspace{-0.5em}
    \setlength\itemsep{0.15em}
    \item[$-$] po sedmileté válce je i navzdory vítězství Anglie ekonomicky zdecimována
    \item[$-$] konec vstřícné politiky vůči koloniím
    \item[pol. 18. st.] osady se chtějí dál rozvíjet (průmysl, podnikání, obchod), to se Britům v duchu merkantilismu nelíbí
    \item[$-$]  na severu spíše drobní farmáři, \textit{puritáni}, ze severu se vyváží železná ruda, dřevo, kožešiny, vznikají nejstarší a nejznámější americká města; na jihu velkostatkáři a plantáže (bavlna, rýže), využívání práce černochů dovážených z Afriky
    \item[1765] zavádí \textit{kolkovné}: kolek je jakási známka, která se lepí na oficiální tiskoviny, dokumenty, $\rightarrow$ Anglie z toho má peníze, protože jim je prodává, osady odporují
    \item[1767] dovozní cla na papír, sklo, \dots
    \item[1770] \textsc{Bostonský masakr}, osadníci obklíčili anglické vojáky v mostu a vojáci začali střílet
    \item[1773] \textsc{Bostonské pití čaje}: osadníci z lodí vyházeli krabice s čajem (protest proti dovozním clům) $\rightarrow$ Britové uzavírají bostonský přístav a posílají tam další vojsko
    \item[1774] první Kontinentální kongres ve Filadelfii, píší Jiřímu petice a stížnosti, nulová reakce, připravují se na boj s Angličany
    \item[1775] \textsc{incident v Lexingtonu a Concordu}: Anglické vojsko chce odzbrojit armádu osadníků
    \item[1775] vytváření armády osadníků, ta je však vytížená např. sklizní, postupně se však na jejich stranu začínají připojovat i ostatní země, které mají konflikty s Anglií (Francie, Nizozemsko), vede George Washington
    \item[1776] Britové vyhnáni z Bostonu
    \item[1775-1781] druhý Kontonentální kongres, přibyl k nim Thomas Paine, požaduje \uv{zdravý rozum} -- nezávislost osad
    \item[4.7.1776] sepsání \textit{Prohlášení o nezávislosti}, jako hlavní písař považován \textbf{Thomas Jefferson}, vychází z osvícenských myšlenek
    \item[$\Rightarrow$] zahájení \textsc{války o nezávislost} mezi osadníky a Brity (1776-1783)
    \item[1777] přijetí první ústavy = \textit{Články Konfederace}, každý stát stejný hlas v jednokomorovém parlamentu, už od začátku nefungovalo
    \item[1777] \textsc{bitva u Saratogy} mezi osadníky a Angličany, vítězství osadníků $\Rightarrow$  Francie slibuje pomoc osadníkům, což se opravdu stalo, v čele amerických vojáků
    \item[1781] \textsc{bitva u Yorktownu}, američní osadníci definitivně vítězí
    \item[1783] mírovou smlouvou v Paříži Anglie uznává ztrátu svých kolonií v Americe
    \item[$-$] nový stát má hranice podél řeky Mississippi, Florida ještě španělská
    \item[1787] nová ústava, nově federace, dodnes, flexibilní (dá se jednoduše měnit)
    \item[1791] Bill of Rights: 10 dodatků do ústavy ohledně základních lidských práv
    \item[$-$] moc výkonná v rukou prezidenta, též hlavou kabinetu (vlády), prezidentská republika
    \item[$-$] moc zákonodárná tvořena dvoukomorovým kongresem (Kapitol): sněmovna reprezentantů (podle lidnatosti), senát (dva zástupci z každého státu), komory jsou rovnocenné
    \item[$-$] moc soudní tvoří nejvyšší soud a další nezávislé soudy
    \item[$-$] první prezident \textbf{George Washington}, velel armádě osadníků, po válce dostal obrovské pozemky, prý největším statkářem v USA
    \item[$-$] po něm \textbf{John Adams}, třetí \textbf{Thomas Jefferson}
    \item[$-$] koncepce vývoje státu:
    \begin{itemize}
        \vspace{-0.5em}
        \setlength\itemsep{0.15em}
        \item[$-$] Thomas Jefferson chce pořád stát zemědělský, přerušení vztahů s Británií, orientace na Francii, slabá ústřední moc
        \item[$-$] Alexander Hamilton chce silnou ústřední moc, orientaci na Británii, průmyslovou zemi
    \end{itemize}

\end{itemize}
\end{document}
