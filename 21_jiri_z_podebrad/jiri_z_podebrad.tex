\documentclass{article}
\usepackage{fullpage}
\usepackage[czech]{babel}
\usepackage{amsfonts}

\title{\vspace{-2cm}Jiří z Poděbrad\vspace{-1.7cm}}
\date{}
\author{}

\begin{document}
\maketitle

\begin{itemize}
    \vspace{-0.5em}
    \setlength\itemsep{0.15em}
    \item[$=$] \uv{král dvojího lidu}, t.j. vládne kališníkům i katolíkům, akceptuje jejich existenci
\end{itemize}

\section*{Situace před Jiřím z Poděbrad}
\begin{itemize}
    \vspace{-0.5em}
    \setlength\itemsep{0.15em}
    \item[1437 -- 1439] \textbf{Albrecht Habsburský}, manželka \textbf{Alžběta}, syn \textbf{Ladislav Pohrobek}, který je malý a nemůže vládnout $\rightarrow$
    \item[1439 -- 1453] \textit{husitské interregnum} = bezvládí
    \item[1440] \textit{Mírný list} -- dohoda o neválčení mezi husity a křižáky
    \item[$-$] české království rozděleno na \textit{landfrídy} (kraj)
    \item[$-$] nejvýznamnější rody: páni z Kunštátu a Poděbrad, Rožmberkové
    \item[1448] \textbf{Jiří z Poděbrad} (z rodu kališníků) \textsc{dobývá Prahu} a stává se zemským správcem (1452)
    \item[1453 -- 1457] Ladislav Pohrobek, jako třináctiletý, o čtyři roky později těsně před svatbou (s francouzskou princeznou) umírá
    \item[1458] zemský sněm zvolil za krále \textbf{Jiřího z Poděbrad}
    \item[$-$] ve stejném roce zvolen králem uherským \textbf{Matyáš Korvín}
\end{itemize}

\section*{Vláda}
\begin{itemize}
    \vspace{-0.5em}
    \setlength\itemsep{0.15em}
    \item[$-$] ekonomická obnova Čech, podpora obchodu, likvidace lupičů, zavedení velké daně, právo na držení pozemků (???)
    \item[$-$] akceptuje kališníky i katolíky, sám patří k umírněným kališníkům
    \item[$-$] pronásleduje členy Jednoty bratrské
    \item[1462] papež prohlásil, že ruší platnost \textit{Basilejských kompaktát} a uvrhl Jiřího do klatby (1466), protože je podporoval i po jejich zrušení
    \item[1466] křížová výprava po celé Evropě, cíl: najití spojenců pro boj s \textit{papežskou stolicí}, vede jeho švagr \textbf{Zdeněk Lev z Rožmberka}, účastnil se jí i Václav Šašek z Bířkova
    \item[$-$] této situace využívá \textbf{Matyáš Korvín}, král uherský
    \item[1471] umírá, chce, aby nastoupil \textbf{Vladislav Jagellonský} 
\end{itemize}


\section*{Matyáš Hunyadi (Hyundai) Korvín}
\begin{itemize}
    \vspace{-0.5em}
    \setlength\itemsep{0.15em}
    \item[$-$] Jednota zelenohorská = spolek poněmčených českých měst
    \item[1468] napadá Moravu
    \item[1469] rozhodující \textsc{bitva u Vilémova}, vítězí Jiří z Poděbrad
    \item[$-$] téhož roku Matyáš prohlášen v Olomouci za českého krále, Jiří ztrácí Moravu, Srbsko a Lužici, tedy zbývají mu jen Čechy

\end{itemize}


\end{document}
