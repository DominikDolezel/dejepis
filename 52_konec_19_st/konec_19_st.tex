\documentclass{article}
\usepackage{fullpage}
\usepackage[czech]{babel}
\usepackage{amsfonts}

\title{\vspace{-2cm}Konec devatenáctého století\vspace{-1.7cm}}
\date{}
\author{}

\begin{document}
\maketitle

\subsection*{Koloniální expanse}
\begin{itemize}
    \vspace{-0.5em}
    \setlength\itemsep{0.15em}
    \item[$-$] \textit{kolonie}: strategický, vojenský, ekonomický význam
    \item[1891] všeněmecký svaz: Němci žijící všude, Mitteleuropa -- představa ovládnutí prostoru střední Evropy Němci
    \item[$-$] sinusoidy antisemitismu -- vzrůstá, další vlna $\rightarrow$  \textit{sionismus}: snaha vytvořit vlastní stát (Theodor Herzel)
    \item[1800-1898] Afriku si rozdělí Italové (Libye, Somálsko, kus Erithrei), Německo (Togo, Cameron, Německá východní  západní Afrika), Francie (sever), Británie (jih)
    \item[$-$] střetnutí v Súdánu (Fašodě): francouzský sbor se vrátil, později symbol porážky
\end{itemize}

\subsection*{Blízký a Střední východ}
\begin{itemize}
    \vspace{-0.5em}
    \setlength\itemsep{0.15em}
    \item[$-$] střet různých koloniálních velmocí, především kvůli ropnému bohatství, je tu osmanské císařství
    \item[$-$] německý císař nabízí Osmanům vystavění Bagdádské dráhy, Osmané souhlasí, nelíbí se to Britům, jižněji vyhlásili protektorát a zastavili tak pronikání Němců směrem k Perskému zálivu
    \item[1907] Rusko-Britská smlouva, později povede k vytvoření bloku válčících v první světové válce
    \item[$-$] Němci začínají stavět námořní flotilu, Britové se cítí ohrožení
    \item[$-$] konfilkty na Korejském poloostrově
    \item[1894-1895] \textsc{čánsko-japonský válka}, vítězí Japonci, získávají vliv a Taiwan
    \item[1904-1905] \textsc{rusko-japonská válka}, vítězí opět Japonci, poté krvavá neděle v Rusku, Říjnový diplom Mikuláše II.
    \item[$-$] Japonci získávají třeba přístav Port Arthur, Jižní Sachalin, kus Mandžuska (SV provincie Číny)
    \item[$-$] Velká Británie: Indie, Kašmír, Nepál, Barma, protektorát nad Afganistánem, kus Persie, Šalamounovy ostrovy, Papua
    \item[$-$] Siam (dnešní Thajsko) si udrželo nezávislost
    \item[$-$] Francie: drží Indočínu (Vietnam, Laos, Kambodža), postupně proniká do Číny
    \item[$-$] Nizozemí drží Indonesii
    \item[$-$] Němci: část Nové Guinee a Bismarckovy ostrovy
    \item[$-$] USA: 1898 \textsc{americko-španělská válka}, získávají Filipíny, havajské ostrovy, ostrov Guam
    \item[$-$] Čína: zatím císařstvím, ale kolabule
    \item[$-$] \textit{Povstání boxerů} v Číně proti pronikání Evropanů do Číny
    \item[1912] v Číně vyhlášena republika, dvě strany (\textit{Kuomintang} vedené Čankajšekem proti komunistické straně vedené Mao-Ce-tungem), po nějaké době začíná občanská válka po vytvoření vlády, nechtějí komunisty
    \item[1937] konec občanské války vpádem Japonska do Číny, opět se spojili proti Japonsku během druhé světové války
\end{itemize}

\subsection*{Evropa}
\begin{itemize}
    \vspace{-0.5em}
    \setlength\itemsep{0.15em}
    \item[$1873$] \textbf{Spolek tří císařů} X tzv. \textit{východní otázka}
\end{itemize}

\subsection*{Balkán}
\begin{itemize}
  \item[$-$]nezávislé státy: Řecko; Srbské, Valašské a Moldavská knížectví
  \item[$1778$] \textbf{San Stefano} Rusko donutilo Osmany uznat Rumunsko, Srbsko, Černou Horu
  %Císaři se snaží pomáhat Osmanům -> blokují rusko; Rusko podporuje knizectvi -> proti osmanum
  \item[$1878$]\textbf{Berlínský kongres} %nemam tuseni, neda se to stihat,
  \item[$-$] po něm nastupuje Fridirchův szn jako Vilém II.
  \item[$-$] někdy se o tomto roce mluvá jako o roce tří císařů
  \item[$-$] Vilém má jiné smýšlení než Bismarck
  \item[$-$] ještě za Viléma I.: do Afriky, do Asie, vznik německého koloniálního spolku , budování námořní flotily
  \item[$-$]  \textit{Spolek tří císařů}, \textit{Všeněmecký spolek}: snaha o německou světovládu i v zemích, v nichž se nachází Němci, dobytá Evropy
  \item[$-$] stavba Bagdádské dráhy
  \item[$-$] militarismus, velmocenská politika, zbrojařská horečka
  \item[$-$] nově chemický, elektrotechnický, dopravní průmysl (AEG, Siemens)
  \item[$-$] francouzské reparace splaceny v roce 1872 v podobě množství akciových společností
  \item[1873] krize  na Vídeňské burse, jež zlikvidovala spoustu společností, zejména železářské společnosti, hlavní problém: nevrátily se peníze na nějakou technickou výstavu, po které zůstal velký dluh
\end{itemize}

\subsection*{?}
\begin{itemize}
    \vspace{-0.5em}
    \setlength\itemsep{0.15em}
    \item[1879] vznik dvojspolku Němci--Rakousko-Uhersko
    \item[1883] s podmínkou, že získají území na úkor Rakouska-Uherska se připojuje i Itálie $\rightarrow$ trojspolek; Rakušané se území vzdát nechtějí
    \item[1914] propuknutí války, Itálie se prohlásila za neutrální, později do války vstoupili na základě dohody (Londýnská smlouva), na základě které jí tato území byla slíbena
    \item[$-$] na straně Německa bojují centrální mocnosti: Německo, Rakousko-Uhersko, Osmani, Bulharsko
    \item[$-$] druhý blok -- Dohoda: Britové, Francouzi a Rusové
\end{itemize}

\subsection*{Francie}
\begin{itemize}
    \vspace{-0.5em}
    \setlength\itemsep{0.15em}
    \item[$-$] 3. republika, 1875 nová ústava
    \item[$-$] po Britech druhá největší koloniální velmoc
    \item[$-$] ekonomika: reparace, ztráta Alsaska a Lotrinska, spíše střední průmysl, investice do domácí výroby
    \item[$-$] vnímáni jako bankéři světa
    \item[$-$] vkládají kapitál do cenných papírů, vyvážejí kapitál do méně vyspělých zemí
    \item[$-$] Dreyfusova aféra, Blériot, který přeletěl la-Manche
    \item[1893] francouzsko-ruská smlouva
    \item[1898] Fašoda
    \item[1904] Srdečná dohoda po napravení vztahů po fašodě v Africe
\end{itemize}

\subsection*{Velká Británie}
\begin{itemize}
    \vspace{-0.5em}
    \setlength\itemsep{0.15em}
    \item[$-$] Viktorie I., pojejí smrti končí Hannoverská dynastie
    \item[$-$] syn Eduarda VII., Jiří V. (dědeček Alžbety II.) zakládá dynastii Windsoru, to je ta současná
    \item[$-$] Jiří V. má dva syny: Edvard VIII. se zamiloval do americké dámy a díky tomu neustál posici britského krále, v témže roce, kdy nastoupil, abdikuje, po něm nastupuje druhý syn, Jiří VI., v roce 1952 umírá a nastupuje Alžběta II. (dcera Jiřího VI.)
    \item[$-$] Britové ztrácí posici nejvyspělejší země na úkor Ameriky a Němců
    \item[1876] Viktorie císařovna indická
    \item[$-$] korunní kolonie, v této době už pouze Indie a Irsko, všechny ostatní jsou už dominia (mají vlastní samosprávu)
    \item[$-$] snaží se vytvořit silné loďstvo, proto jim vadila válečná flotila budovaná Němci
    \item[1904] srdečná smlouva s Francií, narovnání vztahů pošramocených fašodou
    \item[1902] smlouva s Japonskem, porušují dosavadní politiku isolace, smlouva proti Rusku
    \item[1907] britsko-ruská smlouva, vymezení sfér vlivu
\end{itemize}

\subsection*{USA}
\begin{itemize}
    \vspace{-0.5em}
    \setlength\itemsep{0.15em}
    \item[$-$]
\end{itemize}



\end{document}
