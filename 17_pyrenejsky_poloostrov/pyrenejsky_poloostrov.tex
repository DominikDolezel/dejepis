\documentclass{article}
\usepackage{fullpage}
\usepackage[czech]{babel}
\usepackage{amsfonts}

\title{\vspace{-2cm}Španělsko\vspace{-1.7cm}}
\date{}
\author{}

\begin{document}
\maketitle

\section*{Vznik Španělska}
\begin{itemize}
    \vspace{-0.5em}
    \setlength\itemsep{0.15em}
    \item[$-$] výsledek \textit{reconquisty} -- křesťané chtějí znovu dobýt území
\end{itemize}

\subsection*{Situace před vznikem}
\begin{itemize}
    \vspace{-0.5em}
    \setlength\itemsep{0.15em}
    \item[$-$] \textbf{Vizigóti} Tolosánská říše, později Vizigótská říše, centrum v Toledu
    \item[711] území si podrobili Arabové, centrum \textbf{Córdoba}
    \item[10. st.] vznik arabského chalífátu
    \item[$-$] tolerance vůči původním obyvatelům, vzkvétá kultura a vzdělanost
    \item[$-$] významná centra: Granada, Sevilla, Toledo
    \item[$-$] \textit{Al-Andalus}: arabský název Pyrenejského poloostrova v době, kdy tu byli muslimové
\end{itemize}

\subsection*{Reconquista}
\begin{itemize}
    \vspace{-0.5em}
    \setlength\itemsep{0.15em}
    \item[8. st.] (Pelayo), dobývá Asturii a Galicii
    \item[$-$] \textbf{Karel Veliký} vybuduje \textit{španělskou marku} = oblast kolem Pyrenejí, kterou ovládl, Píseň o Rolandovi
\end{itemize}

\subsection*{11. -- 12. století}
\begin{itemize}
    \vspace{-0.5em}
    \setlength\itemsep{0.15em}
    \item[$-$] vznik Kastilie a Leonu, Aragonu a Katalánska a nakonec Portugalska
    \item[$-$] křesťané zatlačili muslimy na jih
    \item[$-$] \textbf{Ferdinand I.}, kastilský král a jeho syn \textbf{Alfonso VI.} -- dobývá Toledo, Cordobu, Sevillu a Zaragozu
    \item[$-$] \textit{hidalgové} = vojáci ze středních a vyšších vrstev, např. \textbf{Rodrigo Díaz de Vívar} = Cid, poté králem Valencie
    \item[$-$] \textit{berbeři} = obyvatelé severní Afriky
    \item[$-$] \textit{maurové} = obyvatelé severní Afriky a Arabové
    \item[1212] \textsc{bitva u Las Navas de Tolosa}, zásadní porážka Arabů, od té doby jsou vytlačováni
    \item[1492] \textsc{dobytí Granady}
    \item[(1469)] Kastilské království dědí \textbf{Isabela Kastilská}, dědicem Aragonu \textbf{Ferdinand II. Aragonský} $\rightarrow$ sňatek
    \item[1479] personální unie $\rightarrow$ základ Španělského království
    \item[$-$] \textit{inkvizice}: cíl vyhnat obyvatele jiného než katolického náboženství $\rightarrow$ hospodářský úpadek
    \item[$-$] \textit{autodafé} = veřejné vyhlášení někoho za kacíře a provedení rozsudku, většinou upálení
    \item[$-$] Isabela podporuje \textbf{Kryštofa Kolumba}
    \item[$-$] jejich dcera \textbf{Johana Šílená}, žena Filipa Habsburského
    \item[$-$] jejich syn \textbf{Karel V.} = \textbf{Carlos I.}, později císař SŘŘ
    \item[$-$] jeho bratr \textbf{Ferdinand I. Habsburský}, po vymření Jagellonců položil základy Habsburské monarchie, jejíž jsme též byli součástí, později císař SŘŘ
\end{itemize}





\end{document}
