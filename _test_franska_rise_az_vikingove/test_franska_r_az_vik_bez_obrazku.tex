\documentclass{article}
\usepackage{fullpage}
\usepackage[czech]{babel}
\usepackage{amsfonts}
\usepackage{hyperref}
\usepackage{graphicx}
\usepackage{fontawesome5}

\title{\vspace{-2cm}Franská říše, Francie, SŘŘ, Vikingové\vspace{-1.7cm}}
\date{}
\author{}

\begin{document}
\maketitle

\section*{Periodizace středověku}
\begin{description}
    \vspace{-0.5em}
    \setlength\itemsep{0.15em}
    \item[raný] \textbf{476 -- 11. st.}
    \item[vrcholný] \textbf{11. st. -- 15./16. st.} (1492 -- objevení Ameriky nebo 1453 -- dobytí Cařihradu)
    \item[pozdní] \textbf{16. -- pol. 17. st.} (raný novověk)
\end{description}

\section*{Franská říše (481 -- 843)}
\subsection*{Meroveovci 481 -- 751}
\begin{itemize}
    \setlength\itemsep{0.15em}
    \item[$-$] založena Franky, kteří sjednotili Germány
    \item[$-$] zakladatel \textbf{Chlodvík z Meroveovců} (481 -- 511)
        \begin{description}
            \vspace{-0.5em}
            \setlength\itemsep{0.15em}
            \item[497/98] nechal se pokřtít v Remeši\\
            $\rightarrow$ podporoval katolickou církev
            \item[$-$] \textit{Lex Salica} -- kodifikování práva
        \end{description}
    \vspace{-0.5em}
    \item[$-$] \textbf{Řehoř z Tours} -- mnich, napsal kroniku Franků (\textit{Historia Francorum})
    \item[$-$] \textbf{Dagobert I.} (629 -- 639)
        \vspace{-0.5em}
        \begin{description}
            \setlength\itemsep{0.15em}
            \item[631] \textsc{bitva u Wogastisburgu} vs. Sámova říše, (S. ř. vyhrává)
        \end{description}
    \vspace{-0.5em}
    \item[$-$] \uv{\textit{líní králové}} $\rightarrow$ \textit{majordomové} (správci paláců) na sebe strhávají moc $\rightarrow$  z nich dynastie Karlovců
    \item[$-$] \textbf{Pipin II. Prostřední} (679 -- 714)
        \begin{description}
            \setlength\itemsep{0.15em}
            \vspace{-0.5em}
            \item[(687)] \textsc{bitva u Tertry}
        \end{description}
        \vspace{-0.5em}
    \item[$-$] \textbf{Karel Martell} (715 -- 741)
    \begin{description}
        \vspace{-0.5em}
        \setlength\itemsep{0.15em}
        \item[732] \textsc{bitva u Poitiers} a \textsc{bitva u Tours} -- zastavení postupu Arabů
    \end{description}
\end{itemize}

\subsection*{Karlovci 751 -- 843}
\begin{itemize}
    \setlength\itemsep{0.15em}
    \item[$-$] \textbf{Pipin III. Krátký}
    \begin{description}
        \vspace{-0.5em}
        \setlength\itemsep{0.15em}
        \item[751] prohlásil se za krále = \textit{rex francorum}, korunován papežem v Saint Denis
        \item[(755/6)] porážka Langobardů
    \end{description}
    \item[$-$] \textbf{Karel Veliký} (768 -- 814)
        \begin{description}
            \vspace{-0.5em}
            \setlength\itemsep{0.15em}
            \item[800] první císař (papež Lev III.)
            \item[$-$] definitivně poráží Langobardy v S Itálii
            \item[$-$] rozvoj kultury
            \item[Správa státu:] rozděleno na hrabství, v čele \textit{hrabata}, biskupství, v čele \textit{biskup}, v okrajových oblastech \textit{marka}, v čele \textit{markrabě}
        \end{description}
    \item[$-$] \textit{karolínská renesance} -- podpora vzdělanosti a vědy, opět antické ideály
    \item[$-$] mnich \textbf{Alcuin z Yorku} (\href{https://cs.wikipedia.org/wiki/Alcuin}{\faWikipediaW ikipedie})
    \item[$-$] historik \textbf{Pavel Diaconus} (\href{https://cs.wikipedia.org/wiki/Paulus_Diaconus}{\faWikipediaW ikipedie})
    \item[$-$] Petr Pisánský
    \item[$-$] Einhard: \textit{Vita Karoli Magni} (\href{https://cs.wikipedia.org/wiki/Vita_Karoli_Magni}{\faWikipediaW ikipedie})
    \item[$-$] sídlo v Cáchách -- kaple Panny  Marie
    \item[$-$] \textbf{Ludvík Pobožný} (814 -- 840), 3 synové, chce mezi ně rozdělit říši = \textit{ordinatio imperii}
\end{itemize}

\subsection*{Zánik}
\begin{description}
    \setlength\itemsep{0.15em}
    \item[817] \textit{ordinatio imperii}
    \item[843] \textbf{Verdunská smlouva} -- rozdělení říše na Z $\rightarrow$ Francie, V $\rightarrow$ SŘŘ
    \item[Lothar I.] titul císaře, území: S Apeninského pol.
    \item[Karel II. Holý] Západofranská říše -- Francie
    \item[Ludvík II. Němec] Svatá říše římská
\end{description}

\section*{Francie}
\subsection*{Karlovci (do 987)}
\begin{itemize}
    \vspace{-0.5em}
    \setlength\itemsep{0.15em}
    \item[$-$] šlechta si vymohla \textit{léno} (dědictví) na Karlu II.
    \item[$-$] Normanské vévodství (911 -- 1259), pojmenované podle Vikingů (Normané), smlouva mezi \textbf{Karlem III. Prosťáčkem} a vikingským vojevůdcem \textbf{Rollem} toto vévodství ustanovila, po r. 1259 je součást Francie
\end{itemize}

\subsection*{Kapetovci (987 -- 1328)}
\begin{itemize}
    \vspace{-0.5em}
    \setlength\itemsep{0.15em}
    \item[$-$] \textbf{Hugo Kapet} (987 -- 996)
        \begin{itemize}
            \vspace{-0.5em}
            \setlength\itemsep{0.15em}
            \item[$-$] vévoda Île-de-France (centrální část Francie, okolo Paříže)
            \item[$-$] později jmenován králem Francie (\textit{rex francorum})
        \end{itemize}
\end{itemize}

\subsection*{Francouzská společnost}
\begin{itemize}
    \vspace{-0.5em}
    \setlength\itemsep{0.15em}
    \item[$-$] boj královská moc vs. šlechta, panovník vs. církev
    \item[$-$] feudální rozdrobenost
    \item[$-$] klášter Cluny (Clunijské hrabství)
    \item[$-$] \textit{boží příměří} -- zákon zakazující války ve svátky
    \item[$-$] města jsou nezávislá na církvi a šlechtě, jen na králi
\end{itemize}

\section*{Apeninský poloostrov}
\begin{itemize}
    \vspace{-0.5em}
    \setlength\itemsep{0.15em}
    \item[$-$] útoky okolních národů (Arabové, Normani) $\rightarrow$ rozpadnutí na jednotlivé části, feudální rozdrobenost
    \item[$-$] Burgundské království, od 11. st. SŘŘ, ve 14. st. postoupeno Francii
    \item[$-$] Benátská republika -- u moci měšťané, v čele \textit{dóže}, námořní velmoc
    \item[$-$] Papežský stát (\textit{papa} = otec), první biskup sv. Petr, \textit{Konstantinova donace} = měl biskupům zaručovat nadvládu nad Římem, bohužel ne $\rightarrow$ \textit{Pipinova donace} = potvrzuje nároky papežů, Pipin na oplátku uznán za vládce Franské říše; papežové voleni šlechtou, papež Silvestr II.: \textit{renovatio imperii} -- obnovení SŘŘ
\end{itemize}

\section*{Svatá říše římská = SŘŘ (do 1806)}
\subsection*{Karlovci (do 918)}
\begin{itemize}
    \vspace{-0.5em}
    \setlength\itemsep{0.15em}
    \item[$-$] \textbf{Ludvík II. Němec}
    \item[$-$] \textbf{Karel III. Tlustý} chce sjednotit Franskou říši (Bavorsko, Sasko, Frankové, Švábsko, \dots), neúspěch
\end{itemize}

\subsection*{Rod Otonů (919 -- 1023)}
\begin{itemize}
    \vspace{-0.5em}
    \setlength\itemsep{0.15em}
    \item[$-$] \textbf{Jindřich I. Ptáčník} (919 -- 936)
        \begin{itemize}
            \vspace{-0.5em}
            \setlength\itemsep{0.15em}
            \item[$-$] zvolen
            \item[$-$] připojil Lotrinské království
            \item[$-$] Maďaři dělají problémy, útočí
        \end{itemize}
    \item[$-$] SŘŘ sestává z: německé státy, České království, Italské království, Papežský, stát, dočasná území (Švýcarsko, Lucembursko)
    \item[$-$] \textbf{Otto I.} (936 -- 973)
        \begin{itemize}
            \vspace{-0.5em}
            \setlength\itemsep{0.15em}
            \item[962] císařem, \textit{římská císařská jízda}
            \item[955] \textsc{bitva u Lešských polí}, porážka Maďarů
            \item[$-$] expanze: S Itálie, V -- Z Slované
        \end{itemize}
    \item[$-$] \textbf{Otto II.} (973 -- 983)
        \begin{itemize}
            \vspace{-0.5em}
            \setlength\itemsep{0.15em}
            \item[$-$] mírové vztahy s Byzancí
        \end{itemize}
    \item[$-$] \textbf{Otto III.} (983 -- 1002)
        \begin{itemize}
            \vspace{-0.5em}
            \setlength\itemsep{0.15em}
            \item[$-$] chce obnovit říši, \textit{renovatio imperii}
            \item[$-$] papež \textbf{Silvestr II.} -- zavedení arabských číslic, založeno Uherské a Polské arcibiskupství
            \item[$-$] současník \textbf{sv. Vojtěch} -- šíří křesťanství, poslední Slavníkovec, druhý biskup pražský
            \item[$-$] po jejich smrti myšlenka o obnovení říše zaniká
            \item[$-$] začíná \textit{boj o investituru} mezi papežem a králem, kdo bude dominovat, naplno za Sálské dynastie
        \end{itemize}
\end{itemize}

\subsection*{Sálská dynastie (1024 -- 1125)}
\begin{itemize}
    \vspace{-0.5em}
    \setlength\itemsep{0.15em}
    \item[$-$] souboj mezi papežem a králem
    \item[$-$] papež \textbf{Řehoř VII.}
        \begin{itemize}
            \vspace{-0.5em}
            \setlength\itemsep{0.15em}
            \item[(1073)] zvolen papežem
            \item[(1075)] sepsal tzv. \textit{Dictatus papae} (papežská bula) -- chce mít možnost dosazovat i odvolávat biskupy a panovníky
        \end{itemize}
    \item[$-$] král \textbf{Jindřich IV.}
        \begin{itemize}
            \vspace{-0.5em}
            \setlength\itemsep{0.15em}
            \item[$-$] s papežskou bulou nesouhlasí, svolá sezení církevních hodnostářů = \textit{synodu}, sesadí Řehoře, ten ho na oplátku \textit{exkomunikuje} = vyloučí z církve
            \item[1077] Jindřich odchází do Canossy, kde ho Řehoř opět přijme do církve $\rightarrow$ usmíření
            \item[(1084)] Jindřich opět sesadil sesadil Řehoře a jmenuje nového papeže, který ho korunuje císařem
            \item[$-$] u nás současník \textbf{Vratislav II.}, 1085 orvní český král
            \item[1122] \textsc{konkordát wormský} -- v podstatě kompromis, ale ve skutečnosti vítězí církev, král o tom nerozhoduje
        \end{itemize}
\end{itemize}

\subsection*{Štaufská dynastie (1138 -- 1250)}
\begin{itemize}
    \vspace{-0.5em}
    \setlength\itemsep{0.15em}
    \item[$-$] \textbf{Fridrich I. Barbarossa}
        \begin{itemize}
            \vspace{-0.5em}
            \setlength\itemsep{0.15em}
            \item[$-$] další snaha o oslabení církve, neúspěšné
            \item[$-$] chtěl se zmocnit S Itálie: 5 výprav
            \item[1158] Milán, pomáhá \textbf{Vladislav II.} -- opět jmenován králem
            \item[1176] \textsc{bitva u Leguna} výhra spojených severoitalských měst a papeže
            \item[$-$] Jindřich Lev z rodu Welfů, soupeří s J. B., poražen, musí odejít, jeho majetek rozdělen šlechtě
            \item[$-$] syn \textbf{Jindřich VI.} (žena Konstancie, dědička Sicílie)
            \item[$-$] zřídil Moravské markrabství, podléhá jemu $\rightarrow$ vymaní Moravu z vlivu českých knížat
            \item[$-$] pražského biskupa povýšil na říšského biskupa $\rightarrow$ nepodléhá českému králi
            \item[$-$] účast na \textit{třetí křížové výpravě}, kde se utopil
        \end{itemize}
    \item[$-$] \textbf{Fridrich II.} (1197 -- 1250)
        \begin{itemize}
            \vspace{-0.5em}
            \setlength\itemsep{0.15em}
            \item[$-$] syn Jindřicha VI.
            \item[1212] \textsc{Zlatá bula sicilská}, Morava připojena k českým zemím
            \item[1214] \textsc{bitva u Bouvines}, poražení Welfů
            \item[$-$] království vzkvétá, akceptuje muslimy, univerzita v Neapoli (přírodní vědy), spory s papežem
            \item[$-$] po jeho smrti období bezvládí -- \textit{interregnum}
        \end{itemize}
    \item[1257] sbor 7 kurfiřtů (4 světští, 3 církevní) volí krále
    \begin{description}
        \vspace{-0.5em}
        \setlength\itemsep{0.15em}
        \item[světští:] vévoda saský, markrabě braniborský, falckrabě rýnský, český král
        \item[církevní:] arcibiskupové kolínský, trevírský, mohučský
    \end{description}
    \item[1273] tímto způsobem zvolen jako první \textbf{Rudolf Habsburský}
\end{itemize}

\section*{Vikingové}
\subsection*{Obecně}
\begin{itemize}
    \vspace{-0.5em}
    \setlength\itemsep{0.15em}
    \item[$-$] Vikingové = Normané = Dáni = Varjagové
    \item[$-$] S Evropa, germánské kmeny
    \item[$-$] dnešní Dánsko, Norsko, Švédsko -- původní sídla
    \item[$-$] mají \textit{náčelníky}
    \item[$-$] \textit{thing} = shromáždění všech svobodných mužů
    \item[$-$] rybolov, chov dobytka, obchod, zemědělství málo kvůli špatným podmínkám
    \item[$-$] řemeslníci, železené zbraně, lodě \textit{drakary}
    \item[$-$] vikingská expanze:
        \begin{itemize}
            \vspace{-0.5em}
            \setlength\itemsep{0.15em}
            \item[$-$] kvůli špatnému klima, nedstatkui jídla, přemnožení
            \item[793] \textsc{vyplenění kláštera v Lindisfarne} -- začátek
            \item[1066] \textsc{bitva u Hastings} -- konec
        \end{itemize}
\end{itemize}

\subsection*{Norové}
\begin{itemize}
    \vspace{-0.5em}
    \setlength\itemsep{0.15em}
    \item[8. st.] Skotsko, Irsko, Faerské ovy, Shetlandy
    \item[860] objevení Islandu, kolonizace
    \item[982] \textbf{Erik Rudý} v Grónsku (\textit{Grönland} = zelená země)
    \item[cca 1000] jeho syn \textbf{Leif Eriksson} v S Americe -- Vinland, kvůli bojům s původními obyvateli opouštějí
\end{itemize}

\subsection*{Dánové}
\begin{itemize}
    \vspace{-0.5em}
    \setlength\itemsep{0.15em}
    \item[$-$] vyrážejí od 1. pol. 9. st., cíle:
    \begin{itemize}
        \vspace{-0.5em}
        \setlength\itemsep{0.15em}
        \item[$-$] Franská říše (Paříž 845), ZFŘ -- F. ř. ztratila území Normandie, ale nájezdy ustála
        \item[$-$] Anglie (Anglové, Sasové, Jutové -- 7 království = \textit{heptarchie}), v boji proti Vikingům se vždy sjednotí a poté opět rozpadnou
    \end{itemize}
    \item[911] zisk Normandie (v čele Rollo)
    \item[pol. 11. st.] dočasné ovládnutí Anglie: král \textbf{Knut Veliký}
\end{itemize}

\subsection*{Alfréd Veliký (871 -- 900)}
\begin{itemize}
    \vspace{-0.5em}
    \setlength\itemsep{0.15em}
    \item[$-$] král Wessexu, centrum Manchester
    \item[$-$] uměl psát
    \item[878] \textsc{bitva u Edingtonu}, porážka Dánů
    \item[$-$] hranice Dánů a Anglie: Temže, J -- anglosaský, S -- dánský
    \item[$-$] poté se sjednotí a obyvatelstvo se promísí
    \item[$-$] mnich \textbf{Beda Ctihodný}: kronika anglosasů
\end{itemize}

\subsection*{Vilém I. Dobyvatel}
\begin{itemize}
    \vspace{-0.5em}
    \setlength\itemsep{0.15em}
    \item[$-$] před ním král \textbf{Eduard III. Vyznavač}, bezdětný, úpadek království
    \item[$-$] kandidáti na krále: \textbf{Harold II. Godwindson}, \textbf{Vilém}, \textbf{Harald Norský}
    \item[1066] \textsc{bitva u Stanfordského mostu}, Harold porazil Haralda
    \item[1066] \textsc{bitva u Hastings}, Vilém porazil Harolda a založil raně středověkou Anglii
    \item[$-$] \textit{Domesday Book} = evidence pozemků $\rightarrow$ daně
    \item[$-$] tapisérie z Bayeux
\end{itemize}
\hline
\begin{itemize}
    \vspace{-0.5em}
    \setlength\itemsep{0.15em}
    \item[1091] království Sicilské se dostalo pod nadvládu Normanů
    \item[$-$] Varjagové
    \begin{itemize}
        \vspace{-0.5em}
        \setlength\itemsep{0.15em}
        \item[$-$] cesta od Varjagů k Řekům (\textit{iz Varjag v Greki})
        \item[$-$] obchodní kontakt přes Řeky
        \item[862] Rurik ovládl Novgorod
        \item[882] jeho syn Oleg ovládl Kyjev
        \item[$-$] spojením vznikla \textbf{Kyjevská Rus}
    \end{itemize}
\end{itemize}

\subsection*{Státy Vikingů}
\begin{itemize}
    \vspace{-0.5em}
    \setlength\itemsep{0.15em}
    \item[$-$] od 9. -- 11. st.
    \item[$-$] Norské, Dánské, Švédské království
    \item[$-$] christianizace
\end{itemize}



\end{document}
