\documentclass{article}
\usepackage{fullpage}
\usepackage[czech]{babel}
\usepackage{amsfonts}

\title{\vspace{-2cm}Vývoj po druhé světové válce\vspace{-1.7cm}}
\date{}
\author{}

\begin{document}
\maketitle

\begin{itemize}
    \vspace{-0.5em}
    \setlength\itemsep{0.15em}
    \item[$-$] důsledky pro Evropu: placení reparací, Japonsko je okupování USA, Čína vychází vítězně, staré koloniální velmoci ztrácí svoje kolonie, Evropa zničena válkou
    \item[$-$] vymezování sfér vlivu
    \item[$-$] prvně spolupráce v polválečném uspořádání, pak se ale vztahy budou ochlazovat
    \item[$-$] supervelmoci: USA a SSSR, vymezování sfér vlivu
    \item[$-$] dekonolizace: kolonie pomáhaly kolonizátorům vyhrát válku  za příslib nezávislosti při válce
    \item[1945-1946] norimebrský tribunál: probíhaly zde dřív sjezdy NSDAP, teď tribunál se špičkovými nacisty, 24 obžalovaných (komicky málo), Göring vystupoval na tribunálo, měl být potrestán trestem smrti, ale spáchal sebevraždu, většina špičkových nacistů utekla do JAnebo spáchala sebevraždu: 12 trestů smrti, 3 doživotí
    \item[$-$] v dalších letech se konaly další tribunály, ale pořád nepodchyceno
    \item[1946-1948] Tokijský tribunál, mšělo být zúčtováno s japonskými nacisty, ale většina jich stejně utekla nebose zabila
    \item[$-$] celkový proces denacifikace v Německu byl komplikovaný, aprotože ke konci Hitler vtahoval do války mladé kluky, po válce, kdyby se tam provedla denacifikace, nikdo by tam nezůstal $\rightarrow$ denacifikace v Německu neproběhla
    \item[1947] Pařížské mírové smlouvy podepsány s cílem potrestat fašistické státy; mírová konference, která měla vyústit v podpisy se konala v roce 1946, o rok později podepsáno, podepsalo Finsko, Maďarsko, Rumunsko, Bulharsko, Itálie, ale Německo, Rakousko a Japonsko byly dočasně dekonstruovány -- rozloženy, neměly vlastní politiku -- okupační zony
    \item[1951] Sanfranciská mírová smlouva s Japonskem, hranice dle roku 1854
    \item[1955] Rakouská státní smlouva: za to, že Rakousko se zavázalo, že bude neutrální, stáhly se okupační armády, Rakousko rozděleno na 4 okupační zony
    \item[$-$] MacArthur: vedoucí okupační správy Japonska
    \item[$-$] dekonolizace: fáze 1945-56 se týkala JV Asie -- nezávislost získají země francouzské Indočíny, Indie, Indonesie, objevují se nové státy jako třeba Izrael, období 56-65: většina států na Africkém kontinentu nezávislost -- rok Afriky, od roku 1965 dál: Mosambik, Angola a takové záležitosti
    \item[1961] hnutí nezúčastněných: země, bývalé kolonie, které už se nechtěly zapojovat do bloků, jsou proti členství ve vojenských blocích, tapojovat se do mocenských bloků

\end{itemize}

\subsection*{Ohniska napětí}
\begin{itemize}
    \vspace{-0.5em}
    \setlength\itemsep{0.15em}
    \item[$-$] první vypuklo v Německo, kde mají západní mocnosti sjednotit, znovu vytvořit německý stát, jenže se začaly sjednocovat jen západní zony -- americká a britská do tzv. bizonie, pak se přidali Francouzi do trizonie
    \item[1947] Marshallův plán, ČSR musela odmítnout kvůli diktátu Moskvy
    \item[$-$] na západě fungují čtyři základní politické strany, v sovětské sféře jenom jedna
    \item[červenec 1948 / květen 1949] součástí nového vznikajícího němeeckého státu má být součástí i Berlín, dochází k tzv. blokádě západního Berlína -- Sověti zablokovali do těch západních částí, vyřešeno leteckými mosty, bylo tam shozeno vše, co lidi potřebovali
    \item[$-$] ústava, vyhlášení států v roce 1949, vyhlášena Spolková republika Německo s hl. městem Bonn, prezident theodor Hess, kancléř Konrad Adenauer, NDR s hl. městem V Berlínem, Wilhelm Pieck
    \item[$-$] druhá berlínská krize: Německo jediný možný prostor, jak emigrovat na západ -- odliv mozků, tehdejší východní Němci to chtěli zastavit, v roce 1961 v čele SSSR je NIkita Sergejevič Chruščov, USA Kennedy, jejich jednání, která to měla vyřešit, byla neúspěšná, v srpnu 1961 se začala stavět berlínská zeď -- taky zeď hanby, když byla v roce 1989 zbořena, bylo to symbolem konce studené války
    \item[$-$] Řecko, občanská válka 1946-1949, vznbikla tam komunistická oposice, která nesouhlasila s monarchií, vystřídali je Američané, kvůli vzniku opozice byla občanská válka, vyhráli monarchisté, komunisti prohráli exodus; vojenská diktatura řeckých plukovníků, po referendu v roce 1974 vzniká republika
    \item[$-$] Blízký východ -- stát Izrael; zvláštní komise pro Palestinu byla pro vznik státu Izrael, bylo vybráno britské mandátní území, Izrael vyhláše v roce 1948, první premiér
    \item[$-$] Blízký východ -- stát Izrael; zvláštní komise pro Palestinu byla pro vznik státu Izrael, bylo vybráno britské mandátní území, Izrael vyhláše v roce 1948, první premiér David Bengurion; protože byl prostor arabský, začaly války s arabským okolím, problém přetrvává dodnes
    \item[$-$] ČÍna: válčí tam Kuomintang, Mao-ce-tung, vyhráli komunisti, Čínská lidová republika, opozice (Čnakajšek, Kuomintang) se stahuje na Tchaj-wan, ze začátku vztahy s Ruskem pozitivní, ale pak nastoupil Chruščov, kritizoval Stalina, ajenže Mao-ce-tung dělal podobné věci, takže se mu to nelíbilo; Mao-ce-tung: velký skok (ekonomické reformy, které způsobily desítky milionů mrtvých, každá domácnost měla mít pec a vyrábět železnou rudu), 60. léta kulturní revoluce, zaměřeno proti všemu starému
    \item[$-$] Číňané anektovali Tibet a postupně likvidovali tamější svatostánky
    \item[$-$] Číňané se taky zapojili do korejské války v letech 1950 až 1953
    \item[$-$] taky navazují vztahy s Čínou -- Richard NIxon navštívil Čínu
    \item[$-$] Deng Xiaopeng: malý velký muž, nastolil tzv. linii velkého pořádku, reformovali kde co, ale otázka lidských práv nebyla vyřešena, studenti se v roce 1989 domáhali řešení
\end{itemize}

\subsection*{Indie}
\begin{itemize}
    \vspace{-0.5em}
    \setlength\itemsep{0.15em}
    \item[15.8.1947] nezávislost Indie
    \item[$-$] Indie, Z Pákistán, V Pákistán = Bangladéš
    \item[$-$] jako klíč zvoleno náboženství (Indie -- hinduismus, zbytek -- islám)
    \item[$-$] problémy s Kašmírem, probíhají indicko-pákistánské války
    \item[$-$] Mahátma Gándhí: bojoval za nezávislost Indie
\end{itemize}

\subsection*{Korejská válka 25.6.1950-27.7.1953}
\begin{itemize}
    \vspace{-0.5em}
    \setlength\itemsep{0.15em}
    \item[$-$] rozdělení na Severní (orientována na Čínu a SSSR, v čele Kim Ir-Sen) a Jižní Koreu (orientace na západ)
    \item[$-$] zoděleno podle hranice, kam dorazila sovětská vojska
    \item[$-$] po válce se obviňovaly z toho, že není sjednocený Korejský poloostrov, takže Severní Koera se souhlasem Číny a SSSR útočí na tu Jižní, v radě bezpečnosti OSN se prosadilo, že Severní Korea je agresor, musí se vytvořit koalice a pomoct Jižní Koreji
    \item[$-$] Severokorejci vyhnáni, pak naopak
    \item[1951] frontová linie se ustálila, změna v zákopovou válku
    \item[27.7.1953] příměří, nikoliv mírová smlouva, je tam demilitarizované pásmo
    \item[2007] KLDR jaderná zbraň
    \item[$-$] válka vůbec nic nevyřešila
    \item[$-$] nebyla medializována
    \item[$-$] v čele vojsk OSN Američani, seriál MASH
\end{itemize}

\subsection*{Vietnamská válka 1959-1975}
\begin{itemize}
    \vspace{-0.5em}
    \setlength\itemsep{0.15em}
    \item[$-$] narozdíl od korejské v. byla daleko víc medializovaná
    \item[$-$] jedna z tzv. zástupných válek, kdy proti sobě bojují velmoci
    \item[$-$] je tam republika Francouzská indočína -- kolonie
    \item[1945] vznik VDR, pročínsky orientovaný, hl. město Hanoj, vládne Ho Či Min
    \item[$-$] vznik Vietnamu (na jihu), katolický diktátor Ngho Dinh, hl. m. Sajgon, prozápaddně orientované, později vůdce v roce 1960 zavražděn
    \item[1954] \textsc{bitva u Dien Bien Phu}, Francouzi se musí stáhnout
    \item[$-$] tím, že zemřel vietnamský vůdce (atentát, možná i CIA) vznikl nebezpeční, že severní vietnam bude chtít ovládnout i vietnam jižní, vznik Vietkongu, stezky s cílem narušit Vietnamský systém, severní Vietnam ovládne po smrti jižní Vietnam, tomu se snaží zabránit USA, protože sever je prokomunistický, došlo k
    \item[1964] incident v Tokinském zálivu, americké lodě ostřelovýny severními vietnamci, na konci 50. let něco
    \item[$-$] válka hodně medialisovaná, jak se postupně vraceli do USA vojáci, stoupá nespokojenost s válkou
    \item[$-$] na jihu se bojovalo především pozemní válkou, sever byl Američany bombardován, používán Napal, vyvražďování civilního obyvatelstva, používání chemických zbraní
    \item[$-$] presidentská kampaň Richarda Nixona, slibuje, že pokud bude zvolen, ukončí USA účast ve válce takže
    \item[1973] stahují USA svoji armádu z Vietnamu, jenže v této době už nežije ani Ho či min (severovietnamský vůdce), Vietnam se spojuje
    \item[1.5.1975] konec války
    \item[2.7.1976] vznik Vietnamské socialistické republiky, hl. město Hanoj, Sajgon přejmenován
    \item[$-$] \uv{obýváková válka}, protože se na ni dívali lidi v televizi
    \item[$-$] válka způsobila trauma nejen u vojáků, ale v celé americké společnosti
\end{itemize}

\subsection*{Studená válka}
\begin{itemize}
    \vspace{-0.5em}
    \setlength\itemsep{0.15em}
    \item[$-$] začíná kolem roku 46, 47
    \item[$-$] zhoršování vztahů mezi východem a západem, založení organisace Spojených národu, tribunály apod.
    \item[$-$] projevuje se špionáží, pomlouvačnými kampaněmi, vznikají ekonomické, politické, vojenské bloky
    \item[$-$] soupeření ve výzkumu kosmu
    \item[5.3.1946] projev Winstona Churchilla, kdy říkal, že spadla železná opona
    \item[$-$] Evropa rozdělena na dvě sféry, což potvrzuje i
    \item[12.3.1947] Trumanova doktrína, zastavení komunismu na západ, lze k ní přiřadit i
    \item[5.6.1947] Marshallův plán, pokud USA pomůže obnovit Evropským státům, podaří se jim zamezit šíření komunismu na západ, budou tomu více odolní
    \item[$-$] Joseph McCarthy, senátor, mccarthismus, rozpoutal štvavou kampaň chatání komunistů, \uv{hon na čarodějnice}
    \item[4.4.1949] vytvoření NATO, původně zakládalo 12 států
\end{itemize}

\subsection*{Východní blok}
\begin{itemize}
    \vspace{-0.5em}
    \setlength\itemsep{0.15em}
    \item[$-$] SSSR vytváří Informbyro (Jugoslávie vyloučena)
    \item[1949] vytvořena RVHP, ekonomická, hospodářská pospolitost, patřily tam i státy jako Vietnam, Kuba, Mongolsko
    \item[$-$] Varšavská smlouva -- vojenské uskupení, v roce 1968 vystoupila Albánie, protože bylo Československo okupováno, jako protest
\end{itemize}

\section*{Československo}
\begin{itemize}
    \vspace{-0.5em}
    \setlength\itemsep{0.15em}
    \item[$-$] ztráta Podkarpatské Rusi ve prospěch SSSR
    \item[$-$] v Košicích vytvořena vláda národní fronty v čele se Zdeňkem Fierlingerem, vznik unitárního státu
    \item[$-$] postupně se do Prahy vrací jednak košická vláda a přicestoval také Edvard Beneš, byl také v Brně, ve svém projevu mluvil o vylikvidování Němců
    \item[$-$] po válce vytvořen prozatímní parlament, měl tři sta členů, šest politických stran, byly zastoupeny paritně -- tzn. po padesáti, přijal celou řadu dekretů, z Benešových dekretů se staly zákony
    \item[$-$] probíhalo znárodňování, a to především majetku zrádců, kolaborantů, Němců, Maďarů, znárodněny banky, pojišťovny, klíčový průmysl
    \item[$-$] odsun německé menšiny na základě Postupimské konference a Benešových dekretů, kterými zbavili Němce a Maďary občanství -- divoký odsun, asi 800 tisíc lidí
    \item[1946-] státem řízený odsun, odsunut zbytek Němců, vypravovány vlaky, Němci si museli vzít zavazadlo a odjet -- byli odsouvání do zony sovětské a americké, Britové nebrali sovětské Němce
    \item[$-$] mělo dojít ke spojování rodin, ale někteří Němci byli na našem území ponechání, protože buď byli antifašisti, nebo tam měli smíšená manželství, nebo to byli odborníci, kteří v té době byi nenahraditelní
    \item[$-$] odsun Maďarů: dohoda s Maďary na základě smlouvy, moc se to nepovedlo, jen několik desítek tisíc
    \item[1945] národní očista, na zákl. Benešova dekretu, stát měl být očištěn od kolaborantů a zrádů, ne vše ale bylo košer, často vyřizování účtů, Tiso, Tuka, Moravec, Čurda -- byli popraveni, asi 38 tisíc lidí během tří let postaveno před soudy
    \item[$-$] pozemková reforma -- zabavována půda kolaborantům
    \item[$-$] na Slovensku to jde hůř, půda byla pak zájemcům za symbolický poplatek prodávána
    \item[$-$] nové peníze, nová československá měna
    \item[1946] řádné parlamentní volby, situace uklidněna, do těchto voleb vstoupily nové strany -- na Slovensku strana Svobody (levicová) strana práce (pravicová), v Česku vyhráli komunisti, na Slovensku vyhrála strana demokratická, jednoznačně vyhrávají komunisté, vzniká nová vláda v čele s Klementem Gottwaldem, prezident zůstává Beneš
    \item[$-$] o měsíc později probíhají presidentské volby, Beneš znovu zvolen
    \item[$-$]  UNRRA (unřička), dodávky Spojených národů už během války, susšené mléko, konzervy, čokolády apod.
    \item[$-$] když vyhráli komunisté, Gottwaldova vláda vytvořila dvouletku -- jakýsi budovatelský plán, vypracují novou ústavu, zprůmyslnění Slovenska
    \item[$-$] ministr zemědělství Julius Ďuriš začíná mluvit o tom, že by každý měl vlastnit max. 50 ha
    \item[1947] začíná radikalizace, obrovské sucho a neúroda, Stalin posílá do ČS stovky tisíc tun obilí -- to využili komunisté, museli jsme odmítnout nabádku Marshallova plánu, milionářská daň -- ti lidé, kteří přesahovali 1 mil. korun byli zdaněni, sociální demokracie -- vnitrostranická krize, snaha pošpinit nekomunistické strany -- aféry,
    \item[20.2.1948] vyhrocení situace, komunistický ministr vnitra Václav Nosek odvolal v února osm nekomunistických velitelů SNB (sbor národní bezpečnosti), nahradil je komunisty, dvakrát se sešla vláda, demokratičtí ministři podali demisi, 12 ministrů z 26
    \item[$-$] toho začal využívat Gottwwald, tlačí na beneše, aby demisi přijal, že tu vládu doplní svými lidmi
    \item[$-$] kampaň, generální stávka, vznik lidových milic, sjezdy v Praze
    \item[25.2.1948] jednání Beneše, Zápotockého (v čele odborů), Gottwald, Nosek: demise byla přijata
    \item[$-$] Beneše se snažili podpořit studenti
    \item[7.6.1948] abdikace Beneše
    \item[25.2.1948] většinová komunistická vláda, postupně získali komunisté monopol moci
    \item[$-$] Václav Kopecký -- komunista, stalinista, připravoval politické procesy, hlasoval pro tresty smrti
    \item[9.5.1948] vytvořena nová ústava, Ústava ČSR, v ní se sice mluví o tom, že je lidově-demokratická, ale parlament jen jedna komora, zrušen ústavní soud, Slováci mají navíc nějaké orgány (slov. národní rada, sbor pověřenců), stát pořád unitární, ústava asymetrická (dávala Slovákům možnosti navíc), Beneš ji odmítl podepsat
    \item[14.6.] novým prezidentem se stává Klement Gottwald, nová vláda, jejím předsedou se stal Antonín Zápotocký
    \item[$-$] projevů nespokojenosti bylo strašně málo, otázka nad smrtí ministra zahraničí Jana Masaryka, byl zničen
\end{itemize}

\subsection*{Období 1948-1956}
\begin{itemize}
    \vspace{-0.5em}
    \setlength\itemsep{0.15em}
    \item[$-$] komunistická strana Čech a Slovenska se spojila, ČSSD byla sloučena do KSČM, ta pravicová byla nahrazena nějakou formální stranou (Strana Slvoenské obrody), lidovci existovali, ale byli osekáni, stejně tak Národně socialistická strana byla taky osekána
    \item[$-$] nejhorší období
    \item[$-$] represe
    \item[$-$] PTP: pomocné technické prapory, lidi, lidi pracující v rámci vojenské služby -- víceméně pracovní tábory
    \item[$-$] věznice, věznění politických věznů
    \item[$-$] StB: státní bezpečnost
    \item[$-$] Alexej Čepička: ministr národní obrany, Gottwaldův zeť
    \item[$-$] procesy: vykonstruovaná obvinění na lidi, již kritisovali režim
    \item[1949] proces s Heliodorem Píkou, tím tyto procesy začínají
    \item[$-$] procesy s členy Národně-socialistické, její členkou byla i Milada Horáková (1950), Slánský, Babický případ
    \item[$-$] Gottwaldovi psal i Einstein, který pro Horákovou žádal milost
    \item[$-$] nedávno Milada Poledňová Brožová -- jedna z těch prokurátorek -- byla uvězněna
    \item[1952] proces s Rudolfem Slánským, byl to komunista
    \item[$-$] rušení klášterů
    \item[$-$] šíření masové nezákonnosti, komunismus je takový mocenský moloch, který proniká do života a soukromí lidí
    \item[$-$] Emil Zátopek, Jaroslav Heyrovský (polarograf, nositel Nobelovy ceny)
    \item[$-$] po Gottwaldově smrti nastupuje tehdejší předseda vlády jako prezident Antonín Zápotocký
    \item[1953] měnová reforma, spousta lidí ožebračena, okradení šetřílků a spořílků
    \item[$-$] postupně upouštěno od lístkového systému, zavedeny nové maloobchodní ceny
    \item[$-$] po měnové reformě první protesty proti komunistům
    \item[$-$] v této době jsme od r. 49 v RVHP, pak členy Varšavské smlouvy, začínají první pětiletky, zákon o JZD
    \item[$-$]  Spartakiáda -- od 1955 pravidelně po 5 letech jakoby slavnosti
    \item[$-$] Antonín Novotný či co
    \item[1953] zemřel Gottwald a novým prezidentem se stal Antonín Zápotocký, vznikla nová vláda v čele s Širukým nebo tak nějak
    \item[1.6.1953] měnová reforma: výměna peněz v poměru 1 : 5, snižovaly se ceny zboží, ti co měli víc než 5000, měli kurs 1 : 50
    \item[$-$] snížená životní úroveň
    \item[$-$] dochází k prvním nepokojům (v Plzni ve Škodovce)
    \item[$-$] vláda stanovuje nový kurz, chtějí omezit zbrojní průmysl
    \item[$-$] probíhá boj mezi Zápotockým a Novotným
    \item[$-$] na Slovensku se bránili proti zákonu o JZD, prodásledování katolíků
    \item[$-$] na Slovensku se pomalu zvyšovala životní úroveň
    \item[$-$] cokoliv Slováci chtěli (zrovnoprávnění s Čechy), vždy ostrouhali a čeští komunisté to označovali za buržoazní nacionalismus
    \item[1956] v SSSR proběhl 20. sjezd komunistické strany, kde ze strany Chruščova zazněla kritika Stalina, byl kritizován za politické procesy, za kult jeho osobnosti atd., delegace z dalších států byly úplně paf, hodně lidí se chtělo odtrhnout, na sjezdu československých spisovatelů třeba Seifert vystoupil proti systému
    \item[$-$] komunisti to ale ustáli, obětním beránkem se stal Gottwaldův zeť, ministr obrany Čepičkia, který byl odvolán
    \item[$-$] spartakiády nahradily sokolské slety
    \item[1957] po Zápotockým se stává prezidentem Antonín Novotný
    \item[$-$] i když proběhla rehabilitace nebo amnestie politických vězňů, vznikla 11.7.1960 ústava
    \item[$-$] začíná kolabovat třetí pětiletý plán $\rightarrow$ vznik obrodného procesu
\end{itemize}

\subsection*{Obrodná proces}
\begin{itemize}
    \vspace{-0.5em}
    \setlength\itemsep{0.15em}
    \item[$-$] začíná v 60. letech
    \item[$-$] amnestie politických vězňů, rehabilitace
    \item[$-$] klesá cenzura, vydávají se hodnotné tiskoviny: Host, Dějiny a současnost, Mladý svět
    \item[$-$] Miloš Forman, obroda kultury, mIlan Kundera, Josef Škorecký
    \item[$-$] postupně se obnovují politické strany, Junák, Sokol
    \item[$-$] Klub 231, název podle zákona, podle kterého se věznili političtí vězni
    \item[$-$] KAN: klub angažovaných nestraníků -- požadují demokratizaci společnosti
    \item[$-$] postupně se začíná vyštěpovat směr reformních komunistů: Ota Šik (ekonom, chce spojit tržní a řízenou ekonomiku), Dubček (později gen. tajemník), ČErník (předseda vlády), Smrkovský (předseda parlamentu), Kriegel
    \item[1968] místo Novotného je prezidentem Ludvík Svoboda
    \item[$-$] Vondráčková, Neckář, Kubišová
    \item[$-$] Ludvík Vaculík: 2000 slov na sjezdu spisovatelů 27.6.1967
    \item[1968] nový generální tajmeník Dubček a prezident Svoboda
    \item[20./21.8.1968] značná nespokojenost Moskvy s vývojem v Československu: okupace Československa, vojáci tu zůstali, přišli vojáci dalších 5: Polsko, Maďaři, Bulharsko, ..., ..., obsadili klíčové body, uzly, letadla,
    \item[21.8.] čeští komunisté vydávají Provolání ke všemu československému lidu
    \item[$-$] vyzývají, aby Češi zachovali klid
    \item[22.8.] Vysočanský sjezd, kde znovu odsoudili okupaci, je to výraz agrese, porušování práva atd.
    \item[23.-26.8.] postupně čelní představitelé odvlčeni do Moskvy na Moskevská jednání, kde měli podepsat, že s okupací souhlasí, všichni to podepsali kromě Františka Kriegla
    \item[28.8.] mimořádný sjezd KS Slovenska, distancují se od reformních komunistů
    \item[18.10.] mlouva o podmínkách o dočasném pobytu sovětských vojsk v ČSSR -- ten se protáhl až do roku 1991
    \item[27.10.] zákon o federativním uspořádání (platí od ledna 1969)
    \item[16.1.1969] Jan Palach, student filosofické fakulty, chtěl vyburcovat, na Václavském náměstí se upálil; jeho rodiště ve Všetatech, tichý protest v Praze, byl pohřben na Olšanech
    \item[2.1969] Jan zajíc, středoškolák, taky se upálil
    \item[4.4.1969] upálil se dělník Evžen Ploček v Jihlavě
    \item[$-$] nepomáhá to, probíhá normalizace
    \item[$-$] při připomínání srpnových událostí policejní zásahy
    \item[srpen 1969] dva mladí lidé na Moravském nám. zastřeleni
    \item[duben 1969] taj. KSČ Husák
    \item[1970] Poučení z krizového vývoje ve straně a společnosti -- naši komunisti uznali, že udělali chybu a okupace byla správná
    \item[$-$] výrazná sportovkyně Věra Čáslavská
    \item[$-$] samizdat: sami tisknou a vydávají texty
    \item[$-$] Petlice, Česká expedice
    \item[$-$] exilové noviny: Svědectví, Listy, Index, Konfrontace, '68 Publishers
    \item[$-$] mnozí herci nemohli vystupovat na scénách $\rightarrow$ bytové divadlo
    \item[$-$] folkoví zpěváci, kde mezi řádky narážky na systém
    \item[$-$] Pražský výběr: skupina, někdi zakázáni, někdy povoleni
    \item[$-$] underground: The plastic people of the universe
    \item[$-$] folkoví zpěváci: Karel Kryl
    \item[$-$] Charta 77: reakce na to, že ČSSR podepsala pak to spolupráci v něčem a Evropě, prvními mluvčími byli Jiří Hájek, Jan Patočka, Václav Havel, každý rok se mluvčí střídali a obnovovali, upozorňovali na dodržování lidských práv
    \item[$-$] Anticharta: prohlášení umělců, kteří odmítali chartu
    \item[$-$] VONS: výbor na obranu nespravedlivě stíhaných
\end{itemize}

\subsection*{Cesta k sametové revoluci}
\begin{itemize}
    \vspace{-0.5em}
    \setlength\itemsep{0.15em}
    \item[21.8.1988] výročí okupace
    \item[15.-22.ledna 1989] palachův týden
    \item[29.6.1989] petice Několik vět od chartistů, chtějí, aby s nimi komunisté komunikovali, svoboda shromažďování, svoboda slova atd.
    \item[21.8.1989] připomínka okupace
    \item[28.10.1989] taky výročí
    \item[12.11.1989] papež Jan Pavel 2. svatořečí Anežku Českou, byla pověst, že až Anežka Česká bude svatořečena, zavládne v Českých zemích svoboda
\end{itemize}

\subsection*{17. listopad 1989}
\begin{itemize}
    \vspace{-0.5em}
    \setlength\itemsep{0.15em}
    \item[$-$] pochod studentů má připomínat zákaz vasokých škol za protektorátu
    \item[$-$] přes Vyšehrad dorazili na Národní třídu, kde byli studenti zbiti, všechno to vypuklo
    \item[$-$] demonstrace, Strahov, Havel
    \item[$-$] Havel mluví s Adamcem -- tehdejším předsedou vlády
    \item[$-$] Gott a Kryl zpívají Hymnu
    \item[$-$]  ustavení občanského fora
    \item[$-$] rekonstruována vláda Adamce, neúnosné, demise, vytvořena nová vláda s Marianem Čalfou
    \item[$-$] generální stávka
    \item[29.12.] Václav Havel prezidentem
    \item[$-$] ve vile Tugendhat se domlouval rozpad ČEskoslovenska, Havel abdikoval, aby mohl být prezidentem České republiky
\end{itemize}


\subsection*{50.-60. léta}
\begin{itemize}
    \vspace{-0.5em}
    \setlength\itemsep{0.15em}
    \item[1953] zemřel Stalin, nastupuje Chruščov
    \item[$-$] po Trumanovi se v USA stává prezidentem Eisenhower
    \item[$-$] oficiální politika východního bloku: mírové soužití; SSSR se snažil dostat spojence v zemích třetího světa
    \item[1926] dvacátý komunistický sjezd, CHruščov kritizuje politiku Stalina a jeho kult
    \item[$-$] krize východního bloku
    \item[1953] Východoněmecké povstání v Berlíně: stávka stavebních dělníků, přerostla v protivládní demonstrace, potlačeno tanky
    \item[1956] reakce na kritiku Stalina: demonstrace polských dělníků v Poznani, potlačeno, na to reagovali studenti v Maďarsku: Maďarská revoluce, projev solidarity studentů s Poláky, přerostlo v protivládní protisovětské demonstrace, popraven Imre Nagy, důsledek: velké množství Maďarů uteklo do zahraničí
\end{itemize}

\subsection*{Druhá berlínská krize 1958-1961}
\begin{itemize}
    \vspace{-0.5em}
    \setlength\itemsep{0.15em}
    \item[$-$] komunističtí předáci si stěžují, že přes Berlín emigrují lidi z východního bloku, chtějí získat západní Berlín
    \item[$-$] Kennedy to odmítá
    \item[1961] jednání neúspěšná, vygradovalo to ve stavbu berlínské zdi, padla až v listopadu 1989
    \item[$-$] Kennedy: Ich bin ein Berliner
\end{itemize}

\subsection*{Soupeření ve vesmíru}
\begin{itemize}
    \vspace{-0.5em}
    \setlength\itemsep{0.15em}
    \item[$-$] iniciativa Kennedyho, program Apollo
    \item[1968] Američané první na Měsíci
    \item[1957] Sputnik 1, Lajka
    \item[1961] Vostok 1, Jurij Gagarin obletěl Zemi
\end{itemize}

\subsection*{USA}
\begin{itemize}
    \vspace{-0.5em}
    \setlength\itemsep{0.15em}
    \item[1961-1963] John FItzgerald Kennedy
    \item[1963] v kampani zavražděn, pozadí dodnes není známo
    \item[$-$] byl velmi populární i se svojí manželkou Jacqueline, byla považována za jakousi módní ikonu
    \item[1963-1969] Lyndon Johnson: vstup USA do Vietnamské války
    \item[1969-1974] Richard Nixon: navštívil Čínu, vyvedl USA z Vietnamské války, padl kvůli aféře Watergate (výbor pro jeho znovuzvolení nainstaloval špěhovací zařízení do demokratické strany), musel rezignovat
\end{itemize}


\subsection*{Kubánská revoluce, Karibská krize}
\begin{itemize}
    \vspace{-0.5em}
    \setlength\itemsep{0.15em}
    \item[$-$] Kubu získali Američané na konci 19. století, když probíhala španělská válka
    \item[$-$] od roku 1902 Kuba záskala nezávislost, Američané museli odejít
    \item[$-$] mohli si tam vybudovat vojenskou základnu
    \item[1959] na Kubě u moci jakýsi Batista, mluví se o diktatuře, byl svržen levicovými aktivisty, mezi něž patří i Fidel Castro a Ernesto Che Guevara, proběhla kubánská revoluce, Američané odmítali spolupráci, byli tedy vrženi do náručí SSSR, začíná levicová politika, nedodávají Američanům cukr, znárodňování, Američané ekonomicky blokují Kubu, z Kuby odchází emigranti, chtějí zlikvidovat Castra a udělat převrat
    \item[1961] emigranti se neúspěšně vylodili v zátoce Sviní
    \item[1962] kus od Ameriky Chruščov instaluje rakety středního doletu, Američané to zjistili -- začátek kubánské krize
    \item[$-$] intenzivní jednání, Američané blokují námořně Kubu, svět byl na pokraji třetí světové války, nakonec se domluvili, jaderné hlavice byly demontovány
\end{itemize}

\subsection*{Suezská krize}
\begin{itemize}
    \vspace{-0.5em}
    \setlength\itemsep{0.15em}
    \item[$-$] Gamál Násir (egyptský prezident) znárodnil Suezský průplav, Britové a Francouzi to tam začali bombardovat, Izraelci útočí po zemi, OSN se do toho nevložila, museli se stáhnout
\end{itemize}

\subsection*{70.-80: léta východní blok}
\begin{itemize}
    \vspace{-0.5em}
    \setlength\itemsep{0.15em}
    \item[1954-1982] Brežněv: neostalinismus, obrovské peníze na zbrojení, ale všechno ostatní zaostává
    \item[$-$] okuapce Československa, vydal doktrínu, že má právo intervence ve východním bloku
    \item[1982-1984]  Andropov
    \item[1984-1985] Černěnko
    \item[1985] Michail Gorbačov: uvolnění, nechá jednotlivé republiky rozhodvat samy, svoboda slova
\end{itemize}

\subsection*{Východní blok}
\begin{itemize}
    \vspace{-0.5em}
    \setlength\itemsep{0.15em}
    \item[$-$] neostalinismus: Bulharsko, NDR, ČSSR
    \item[$-$] Rumunsko: vypořádávání s Maďary, oslabení cenzury
    \item[$-$] Maďarsko: jezdí se tam na nákupy, protože tam bylo lepší zboží
    \item[$-$] Polsko: drahé potraviny, po roce 1970 velké demostrace, protesty, střelby do demonstrujících, opozice získává vliv, která je zaštítěna katolickou církví (Jan Pavel II.), odborová opozice Solidarita (nakonec zakázána), 1989 polosvobodné volby (zvítězili členové opozice Solidarita, vláda Tedausze Mazowiecki)

\end{itemize}

\subsection*{Západ 70.-80. léta}
\begin{itemize}
    \vspace{-0.5em}
    \setlength\itemsep{0.15em}
    \item[$-$] ostpolitik: vstřícné vůči východnímu bloku, Willy Brandt, Helmut Kohl
    \item[1974] Řecko, Portugalsko, pády diktatury
    \item[1975] Španělsko, umřel Franco
    \item[1974]  rozdělení Kypru
    \item[$-$] IRA (irská republikánská armáda), ETA (balkánská teroristická organizace)
\end{itemize}

\subsection*{USA prezidenti}
\begin{itemize}
    \vspace{-0.5em}
    \setlength\itemsep{0.15em}
    \item[$-$] Ronald Reagan: označoval východní blok za říši zla
    \item[$-$] George Bush rozpoutal válku v zálivu
\end{itemize}

\subsection*{Afgánistán}
\begin{itemize}
    \vspace{-0.5em}
    \setlength\itemsep{0.15em}
    \item[$-$] vytvoření levicové vlády
    \item[$-$] Brežněv tam vpadl, aby posílil vliv levice
    \item[$-$] nezůstalo  bez odezvy
    \item[$-$] proti vládě bojovali Mudžáhidové
    \item[$-$] konflikt medializován, kritizován, chemické látky
    \item[$-$] Gorbačov vojska stáhl
    \item[$-$] pak se k moci dostali tálibánci
    \item[$-$] po ztracení dvojčat zahájili američani operaci trvalá svoboda s tím, že je svrhnou

\end{itemize}

\subsection*{Írán, islamská revoluce}
\begin{itemize}
    \vspace{-0.5em}
    \setlength\itemsep{0.15em}
    \item[$-$] do roku 1979 prozápadní politika, protože u moci byla dyn. Pahlaví
    \item[$-$] Ajatolláh Rúholláh Chomejní řídil islámskou revoluci, dostal se k moci, odříznutí od zbytku světa
    \item[$-$] po něm Allí Chameneí
\end{itemize}

\subsection*{Íránsko-irácká válka}
\begin{itemize}
    \vspace{-0.5em}
    \setlength\itemsep{0.15em}
    \item[$-$] Saddam Hussein
\end{itemize}

\subsection*{Kambodža}
\begin{itemize}
    \vspace{-0.5em}
    \setlength\itemsep{0.15em}
    \item[$-$] teror v 70. letech: Pol Pot, likvidace soukromého vlastnictví
\end{itemize}

\subsection*{Války v zálivu}

\begin{itemize}
    \vspace{-0.5em}
    \setlength\itemsep{0.15em}
    \item[$-$] Irák anektuje Kuvajt (aktivita Saddama Hussajna), OSN zasahuje, Hussaj ignoruje
    \item[$-$] operace pouštní bouře, hussajn se stáhl
\end{itemize}













\end{document}
