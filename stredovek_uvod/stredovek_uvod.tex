\documentclass{article}
\usepackage{fullpage}
\usepackage[czech]{babel}
\usepackage{amsfonts}

\title{\vspace{-2cm}Středověk\vspace{-1.7cm}}
\date{}

\begin{document}
\maketitle
\section*{Periodizace}
\begin{description}
    \vspace{-0.5em}
    \setlength\itemsep{0.15em}
    \item[raný] \textbf{476 -- 11. st.}
    \item[vrcholný] \textbf{11. st. -- 15./16. st.} (1492 -- objevení Ameriky nebo 1453 -- dobytí Cařihradu)
    \item[pozdní] \textbf{16. -- pol. 17. st.} (raný novověk)
\end{description}

\section*{Obecně.}
\vspace{-0.5em}
\begin{itemize}
    \setlength\itemsep{0.15em}
    \item [$-$] \textit{medievalistika} = věda zabývající se středověkem
    \item[$-$] charakteristika: ovlivněno křesťanstvím, konstituce evropských zemí, vznik románské kultury
    \item[$-$] Dionysius Exigus (5. st.) mnich, měl za úkol zjistit, kdy se narodil Ježíš Kristus
\end{itemize}

\section*{Kulturně-náboženská centra}
\begin{description}
    \vspace{-0.5em}
    \setlength\itemsep{0.15em}
    \item[latinská západní kultura:] Germánské, Románské kmeny a západní Slované -- jazykem vzdělanců je \textit{latina}
    \item[byzantsko-slovanská kultura:] Balkán, Malá Asie, V Evropa -- jazykem vzdělanců je \textit{řečtina / staroslověnština}
    \item[islámská arabská kultura:] Blízký Východ, S Afrika, Pyrenejský pol. -- jazykem vzdělanců je \textit{arabština}
\end{description}

\end{document}
