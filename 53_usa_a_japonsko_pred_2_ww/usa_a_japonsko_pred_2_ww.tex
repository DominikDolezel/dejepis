\documentclass{article}
\usepackage{fullpage}
\usepackage[czech]{babel}
\usepackage{amsfonts}

\title{\vspace{-2cm}USA a Japonsko před druhou světovou válkou\vspace{-1.7cm}}
\date{}
\author{}

\begin{document}
\maketitle

\subsection*{USA}
\begin{itemize}
    \vspace{-0.5em}
    \setlength\itemsep{0.15em}
    \item[$-$] mluvíme o tzv. \textit{pozlaceném věku}: ekonomický rozvoj, nárůst obyvatelstva (Spojené státy jsou nejrozvinutější ekonomikou, za nimi Německo i Nelká Nritánie)
    \item[$-$] bráno jako země zaslíbená, země svobody; “selfmademan” člověk, který se sám vypracoval (USA země velkých možností)
    \item[$-$] monopoly, trusty, koncerny (jakoby velké akciovky) x stát
    \item[$-$] největším producentem a vývozcem železa, uhlí, ropy, mědi, stříbra, elektrifikace
    \item[$-$] značky:  \begin{itemize}
        \vspace{-0.5em}
        \setlength\itemsep{0.15em}
        \item[$-$] John Rockefeller, Standard Oil: nafta
        \item[$-$] Andrew Carnegie, Steel Corporation: ocel
        \item[$-$] Henry Ford, Ford Motor Company (první běžící pás)
        \item[$-$] pomáhali zakládat různé instituce, podporovali kulturu atd.
    \end{itemize}
    \item[$-$] vyšší mzdy vůči Evropě $\rightarrow$ větší koupěschopnost, poptávka
    \item[$-$] velkoměsta, NY burza na Wall Street, Pátá avenue
    \item[$-$] olitické strany: republikáni x demokraté
    \item[$-$] republikáni zastupují: průmyslníky, finančníky, obchodníky; hlavní týpek: Theodor Roosevelt
    \item[$-$] demokraté: zemědělce
    \item[$-$] zavedena prohibice (alkohol), jsou proti monopolům, volební právo pro ženy
    \item[$-$] zahraniční politika: úspěšná (\uv{skvělá malá válka}) 1898 americko-španělská válka (Guam, Havajské ostrovy, Filipíny, Portoriko, Kuba)
    \item[(1901-1914)] budování Panamského průplavu  + smlouva o užívání
    \item[(1910)] Panamerická unie, ekonomický a finanční protektorát
    \item[$-$] zlepšení vztahů s VB, zhoršení vztahů s Německem

\end{itemize}

\subsection*{Japonsko}
\begin{itemize}
    \vspace{-0.5em}
    \setlength\itemsep{0.15em}
    \item[$-$] \textit{šógunát}  (u moci je šógun = vojenský velitel, drží výkonnou moc), mají i císaře, ale ten je mimo tu moc (je uctíván jako náboženská autorita)
    \item[$-$] izolace vůči pronikání velmocí (izolacionismus)
    \item[1854] otevření Japonska velmocím: Matthew C. Perry (USA) tam doplul, Japonci museli podepsat Ansejské dohody $\rightarrow$  konec izolace
    \item[1867]  abdikuje poslední šógun
    \item[$-$] císař Mucuhito (Meidži), stěhuje se do Tokia
    \item[$-$] osvícená vláda, reformy: branná povinnost, vojenská flotila, samurajové přišli o moc, přijetí ústavy (Japonsko konstituční monarchií)
    \item[$-$] budování velkých rodinných firem (obchody, banky), zprůmyslnění
    \item[1875]  Japonci dali Rusům J Sachalin výměnou za Kurilské ostrovy (pak si to Jap vezmou zpět)
    \item[1894/95]  čínsko-japonská válka, získali i Taiwan
    \item[$-$] Rusové chtějí Korejský poloostrov
    \item[1901]  anekce korei
    \item[1902]  smlouva s Velkou Británií x Rusko
    \item[1907]  smlouva s Ruskem a Francií

\end{itemize}


\end{document}
