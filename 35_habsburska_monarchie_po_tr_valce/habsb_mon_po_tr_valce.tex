\documentclass{article}
\usepackage{fullpage}
\usepackage[czech]{babel}
\usepackage{amsfonts}

\title{\vspace{-2cm}Habsburská monarchie po třicetileté válce (17.-18. st.)\vspace{-1.7cm}}
\date{}
\author{}

\begin{document}
\maketitle

\section*{Ferdinand III. (1637-1657)}

\begin{itemize}
    \vspace{-0.5em}
    \setlength\itemsep{0.15em}
    \item[$-$] nastupuje po Ferdinandu II. Štýrském, za třicetileté války
    \item[$-$] za svého života nechal korunovat svého syna (Ferdinan IV.), ten však umírá, takže vládne dál
    \item[$-$] vzdělaný, ovládá několik jazyků
    \item[$-$] nakonec po něm nastupuje jeho čtvrtý syn Leopold I.
    \item[1651] \textit{soupis poddaných} po třicetileté válce
    \item[1654] \textit{berní rula} = soupis obcí, katastr $\rightarrow$ lepší výměra daní
\end{itemize}

\section*{Leopold I. (1657-1705)}
\begin{itemize}
    \vspace{-0.5em}
    \setlength\itemsep{0.15em}
    \item[$-$] současník Ludvíka XIV., zároveň soupeř, ale i obdivovatel Ludvíka a krásy, kterou ve Versailles vybudoval
    \item[$-$] v čele Habsburské monarchie, císařem SŘŘ
    \item[$-$] vládne dlouze $\Rightarrow$ povede se mu upevnit absolutismus po třicetileté válce
    \item[$-$] po úbytku obyvatelstva rostou robotnostní povinnosti poddaných
    \item[1680] \textit{robotní patent}, poddaní mohou robotovat maximálně tři dny v týdnu
    \item[(1692-1695)] \textsc{povstání Chodů} (strážci českých jižních hranic, osvobozeni od poddanských povinností) , po existenci monarchie však již žádná hranice neexistuje a nemají smydl, privilegia jsou jim upíráná $\Rightarrow$ povstání za znovuzískání privilegií, v čele \textbf{Jan Sladký Kozina} proti \textbf{Lamingerovi} (Lomikar), povstání končí popravou Koziny
    \item[$-$] čarodějnické procesy
    \item[1663/4] \textsc{války s Turky}, úspěšné, ale Habsburkům dochází peníze $\Rightarrow$ mír
    \item[1683]  \textsc{dvouměsíční obléhání Vídně} Turky, Osmany se podařilo porazit díky Karlu Lotrinskému
    \item[1699] Osmany vyhání \textbf{Evžen Savojský}, \textsc{Karlovický mír} = Osmané se vzdávají Uher
    \item[1701] \textsc{války o dědictví Španělské}
\end{itemize}


\section*{Josef I. (1705-1711)}
\begin{itemize}
    \vspace{-0.5em}
    \setlength\itemsep{0.15em}
    \item[$-$] pokračování \textsc{válek o dědictví Španělské}
    \item[1703] \textsc{posvtání uherských stavů} proti Habsburské nadvládě
    \item[1711] \textsc{Szátmárský mír} mezi Habsburky a uherskými stavy, uherským stavům jsou zachováan privilegie a stavy uznají právo Habsburků na uherský trůn
\end{itemize}

\section*{Karel VI. (1711-1740)}
\begin{itemize}
    \vspace{-0.5em}
    \setlength\itemsep{0.15em}
    \item[$-$] konec války o dědictví španělské
    \item[$-$] války s Turky, neúspěchy
    \item[17.4.1713] \textsc{Pragmatická sankce}: chce zajistit nástupnost i ženským potomkům (vydána před narozením Marie Terezie)
    \item[$-$] Juro Jánošík, přepadal bohaté a dával chudým
\end{itemize}

\section*{Marie Terezie (1740-1780)}
\begin{itemize}
    \vspace{-0.5em}
    \setlength\itemsep{0.15em}
    \item[$-$] manžel \textbf{František Štěpán Lotrinský}, vnuk Karla Lotrinského (porážky Osmanů u Vídně), 16 dětí (patří do Habsbursko-Lotrinské dynastie)
    \item[1740] arcivévodkyně rakouská
    \item[1741] královna uherská
    \item[$-$] pragmatická sankce nebyla ze začátku akceptována $\Rightarrow$
    \item[1740-1748] \textsc{války o dědictví rakouské} = \textsc{války slezské}, odboj vede pruský král \textbf{Fridrich II. Veliký}, Marie Terezie nesouhlasila s korunovací Albrechta českým králem při dobytí Prahy Bavorskem, Saskem a Francií, nakonec vyhnáni $\Rightarrow$ \textsc{berlínský mír}: Habsburkové ztrácejí Slezsko, ale České země jsou osvobozeny a 1743 Marie Terezie se stává českou královnou
    \item[$-$] Francie je ve válkách na straně proti habsburkům
    \item[1745] \textsc{Drážďanský mír}: Pragmatická sankce potvrzena, František I. Štěpán I. je císařem
    \item[1748] definitivní konec válek \textsc{mírem v Cáchách}
    \item[1756-1763] \textsc{sedmiletá válka}: Francie x Anglie, Prusko x Rakousko, přenese se do kolonií, v Indii a Kanadě vítězí Angličané
    \item[$-$] baron Trenk bojoval na straně Marie Terezie, jeho jednotky = \textit{panduři}, loupí
    \item[$-$] \textit{osvícenský absolutismus}, doba reforem, převládá názor, že člověk je schopen udělat stát dokonalejším, reformy
    \item[$-$] centralizace (ve Vídni), snaha zemi ekonomicky povznést
    \item[$-$] \textbf{Václav Antonín Kounic} v čele zahraniční politiky, hrabě \textbf{Leopold Josef Daun} v čele armády, \textbf{Bedřich VIlém Haugwitz} se stará o vnitrostátní politiku
\end{itemize}

\subsection*{Reformy}
\begin{itemize}
    \vspace{-0.5em}
    \setlength\itemsep{0.15em}
    \item[(1748)] 1. tereziánský katastr (soupis půdy poddanské), rustikál $\Rightarrow$ lepší vybírání daní
    \item[(1757)] 2. tereziánský katastr (soupis půdy vrchnostenské), dominikál
    \item[(1754)] 1. sčítání lidu, asi 18,8 milionů, Čechy 2,26 mil.
    \item[$-$] číslování domů, povinná příjmení
    \item[$-$] společné úřady pro celou monarchii: \textit{direktorium} pro finanční politiku a veřejné záležitosti (v čele Haugwitz), poradní státní rada (poradní sbor direktoria)
    \item[$-$] třístupňový státní aparát: ústřední (Vídeň), \textit{gubernia} (v jednotlivých zemích), magistráty (kraje, města)
    \item[(1759)] zřízen \textit{Nejvyšší soudní dvůr} se sídlem ve Vídni
    \item[$-$] \textit{Tereziánský trestní zákoník} mírnější, ale pořád zachovává mučení, \textit{jednací soudní řád}
    \item[$-$] \textit{manufaktury}, v Čechách výroba textilu, skla, hedvábí
    \item[$-$] celní unie mezi Čechami a Rakouskem
    \item[$-$] podpora podnikání, \textit{protekcionalismus} = stát se snaží zamezit dovozu ze zahraničí $\Rightarrow$ vysoké clo
    \item[$-$] budování silnic, poštovního spojení, tolarová a zlatková měna, \textit{bankocetle} = první bankovky na našem území, jednotné míry a váhy
    \item[$-$] pícniny jako potrava pro dobytek $\Rightarrow$ \textsc{jetelová revoluce}, dobytek se ustájí a nezabíjí na zimu
    \item[$-$] dostávají se k nám brambory z Braniborska  
\end{itemize}



\end{document}
