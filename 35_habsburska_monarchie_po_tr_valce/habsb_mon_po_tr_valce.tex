\documentclass{article}
\usepackage{fullpage}
\usepackage[czech]{babel}
\usepackage{amsfonts}

\title{\vspace{-2cm}Habsburská monarchie po třicetileté válce (17.-18. st.)\vspace{-1.7cm}}
\date{}
\author{}

\begin{document}
\maketitle

\section*{Ferdinand III. (1637-1657)}

\begin{itemize}
    \vspace{-0.5em}
    \setlength\itemsep{0.15em}
    \item[$-$] nastupuje po Ferdinandu II. Štýrském, za třicetileté války
    \item[$-$] za svého života nechal korunovat svého syna (Ferdinan IV.), ten však umírá, takže vládne dál
    \item[$-$] vzdělaný, ovládá několik jazyků
    \item[$-$] nakonec po něm nastupuje jeho čtvrtý syn Leopold I.
    \item[1651] \textit{soupis poddaných} po třicetileté válce
    \item[1654] \textit{berní rula} = soupis obcí, katastr $\rightarrow$ lepší výměra daní
\end{itemize}

\section*{Leopold I. (1657-1705)}
\begin{itemize}
    \vspace{-0.5em}
    \setlength\itemsep{0.15em}
    \item[$-$] současník Ludvíka XIV., zároveň soupeř, ale i obdivovatel Ludvíka a krásy, kterou ve Versailles vybudoval
    \item[$-$] v čele Habsburské monarchie, císařem SŘŘ
    \item[$-$] vládne dlouze $\Rightarrow$ povede se mu upevnit absolutismus po třicetileté válce
    \item[$-$] po úbytku obyvatelstva rostou robotnostní povinnosti poddaných
    \item[1680] \textit{robotní patent}, poddaní mohou robotovat maximálně tři dny v týdnu
    \item[(1692-1695)] \textsc{povstání Chodů} (strážci českých jižních hranic, osvobozeni od poddanských povinností) , po existenci monarchie však již žádná hranice neexistuje a nemají smydl, privilegia jsou jim upíráná $\Rightarrow$ povstání za znovuzískání privilegií, v čele \textbf{Jan Sladký Kozina} proti \textbf{Lamingerovi} (Lomikar), povstání končí popravou Koziny
    \item[$-$] čarodějnické procesy
    \item[1663/4] \textsc{války s Turky}, úspěšné, ale Habsburkům dochází peníze $\Rightarrow$ mír
    \item[1683]  \textsc{dvouměsíční obléhání Vídně} Turky, Osmany se podařilo porazit díky Karlu Lotrinskému
    \item[1699] Osmany vyhání \textbf{Evžen Savojský}, \textsc{Karlovický mír} = Osmané se vzdávají Uher
    \item[1701] \textsc{války o dědictví Španělské}
\end{itemize}


\section*{Josef I. (1705-1711)}
\begin{itemize}
    \vspace{-0.5em}
    \setlength\itemsep{0.15em}
    \item[$-$] pokračování \textsc{válek o dědictví Španělské}
    \item[1703] \textsc{posvtání uherských stavů} proti Habsburské nadvládě
    \item[1711] \textsc{Szátmárský mír} mezi Habsburky a uherskými stavy, uherským stavům jsou zachováan privilegie a stavy uznají právo Habsburků na uherský trůn
\end{itemize}

\section*{Karel VI. (1711-1740)}
\begin{itemize}
    \vspace{-0.5em}
    \setlength\itemsep{0.15em}
    \item[$-$] konec války o dědictví španělské
    \item[$-$] války s Turky, neúspěchy
    \item[17.4.1713] \textsc{Pragmatická sankce}: chce zajistit nástupnost i ženským potomkům
\end{itemize}


\end{document}
