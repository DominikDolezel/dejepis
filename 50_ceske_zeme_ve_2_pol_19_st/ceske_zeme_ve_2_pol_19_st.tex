\documentclass{article}
\usepackage{fullpage}
\usepackage[czech]{babel}
\usepackage{amsfonts}

\title{\vspace{-2cm}České země ve druhé polovině 19. století\vspace{-1.7cm}}
\date{}
\author{}

\begin{document}
\maketitle

\subsection*{Česko-německé vyrovnání}
\begin{itemize}
    \vspace{-0.5em}
    \setlength\itemsep{0.15em}
    \item[$-$] \textit{fundamentální články}: pokus o česko-rakouské vyrovnání
    \item[1871]  vyhlášeno německé císařství ve Versailles, Habsburkové se báli války s Německem, chtěli se domluvit s Čechy -- české vyrovnání
    \item[$-$] \textit{fundamentálky}  = samospráva Českých zemí, clastní zemská vláda, navýšené pravomoci
    \item[$-$] František Josef I. slíbil, že se korunuje na čského krále $\rightarrow$ obrovská vina nevole, protože by Maďaři přišli o exklusivitu, protestují Maďaři, Němci $\rightarrow$
    \item[$-$] intervence Bismarcka proti posílení Slovanů kvůli obavám, že by se přiklonili k Rusku
    \item[$\rightarrow$] vyrovnání neproběhlo
\end{itemize}

\subsection*{Česká politika}
\begin{itemize}
    \vspace{-0.5em}
    \setlength\itemsep{0.15em}
    \item[$-$] vyostření vztahů mezi Čechy a Němci (nakupují velkostatky a mají tak volební právo) $\rightarrow$ podvody
    \item[1862] vygradování situace za voleb do Zemského sněmu (tzv \textit{Chabruskové volby})
    \item[$-$] v čele českých politiků \textit{staročeši}, to se nelíbí \textit{mladočechům} (kritisují pasivní resistenci staročechů -- od r. 1863 na protest přestali docházet do říšské rady) $\rightarrow$ dosud jednotná politická scéna se rozestupuje
    \begin{itemize}
        \vspace{-0.5em}
        \setlength\itemsep{0.15em}
        \item[$-$] národní strana staročechů
        \item[$-$] národní strana svobodomyslná (mladočeši, Národní listy bratrů Gregorových)
    \end{itemize}
    \item[1879] konec pasivní resistence, státoprávní ohrazení, vstup zpět do politiky, v čele stále staročeši
    \item[$-$] aktivní podpora vídeňské vlády, jsou loajální, le něco málo za to získají jako ústupek (\textit{drobečková politika}), byo toho však příliš málo $\rightarrow$
    \item[$-$] po příchodu české žádosti na úřad musí úředníci odpovědět
    \item[$-$] Karloferdinandova universite rozdělena na českou a německou část
\end{itemize}

\subsection*{Punktace}
\begin{itemize}
    \vspace{-0.5em}
    \setlength\itemsep{0.15em}
    \item[1890] plán snažící se o česko-německé vyrovnání o 11 bodech
    \item[$-$] domluva s Němci v monarchii, nová úprava územněsprávních jednotekv českém prostoru podle národnostních kriterií
    \item[$-$] mladočeši jsou nespokoejní, staročeši prohrávají ve volbách a odcházejí z politiky
\end{itemize}


\subsection*{Vlády}
\begin{itemize}
    \vspace{-0.5em}
    \setlength\itemsep{0.15em}
    \item[$-$] Eduard Taaffe: snížil povinnost pracovní doby, volebního censu, zavedl nemocenské a úrazové pojištění, měnová reofrma, zlatky nahrazeny korunami
    \item[$-$] Kaimír Badeni: zřízena 5. kurie, všeobecné ale ne rovné volby (různé kurie mají různý hlas), nevolí ženy
    \item[1906] \textit{Max von Beckova reforma} všeobecné reovné volební právo, konec kuriového systému
    \item[1907] volby do říšské rady
    \item[$-$] české politické strany \begin{itemize}
        \vspace{-0.5em}
        \setlength\itemsep{0.15em}
        \item[$-$] národní strana svobodomyslná
        \item[$-$] českoslovanská strana sociálně-demokratická dělnická (dnes SOCDEM, nejstarší strana)
        \item[$-$] československá strana agrární (význam za První republiky, Antonín Švehla)
        \item[$-$] křesťansko-sociální strana (v čele Jan Šrámek)
        \item[$-$] strana národně-sociální (v čele Klofáč)
        \item[$-$] česká strana lidová (inteligence, pokrok, prosazují humanismus, vzdělanost, v čele Tomáš Garrigue Masaryk)
    \end{itemize}
\end{itemize}
\end{document}
