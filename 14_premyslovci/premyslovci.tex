\documentclass{article}
\usepackage{fullpage}
\usepackage[czech]{babel}
\usepackage{amsfonts}

\title{\vspace{-2cm}Přemyslovci\vspace{-1.7cm}}
\date{}
\author{}

\begin{document}
\maketitle
\section*{Přemysl Otakar I.}
\begin{itemize}
    \vspace{-0.5em}
    \setlength\itemsep{0.15em}
    \item[1198] zisk královského titulu
    \item[26. 9. 1212] \textsc{Zlatá bula sicilská} (Fridrich II.) = titul českého krále je dědičný, svobodná volba, navrácena Morava, možnost jmenování biskupů $\rightarrow$ stabilizace
    \item[1216] \textit{primogenitura} = první syn zdědí tituly otce (dříve \textit{seniorát})
    \item[$-$] manželky \textbf{Adléta Míšeňská, Konstancie Uherská}
    \item[$-$] nová mince \textit{brakteát} (denár stále v oběhu)
    \item[$-$] na vlajce nově český lev (nahrazuje černou orlici)
    \item[$-$] sídla Porta coeli, Křivoklát, Zvíkov
\end{itemize}

\section*{Václav I.}
\begin{itemize}
    \vspace{-0.5em}
    \setlength\itemsep{0.15em}
    \item[1241] nájezdy Mongolů v Evropě
    \item[1243] udělení městských privilegií Brnu
    \item[1247] povstání vedené šlechtou a jeho synem \textbf{Přemyslem Otakarem II.}
    \item[1248] \textsc{bitva u Mostu}, Václav I. vítězí a Přemysla uvězní, nakonec se stejně stane králem
    \item[1249] Jihlavský horní zákoník
\end{itemize}

\section*{Přemysl Otakar II.}
\begin{itemize}
    \vspace{-0.5em}
    \setlength\itemsep{0.15em}
    \item[$-$] manželky \textbf{Markéta Babenberská, Kunhuta Uherská}
    \item[$-$] \uv{král železný a zlatý}
    \item[$-$] ovládaná území: Čechy, Morava, Chebsko, Rakousy, Korutany, Kraňsko, Štýrsko, Furlansko
    \item[1254/55] \textsc{Křížová výprava do Pruska} $\rightarrow$ Královec
    \item[$-$] založil 50 měst
    \item[$-$] vznik zemského soudu, Českých desek zemských (úřední dokumenty)
    \item[1273] císařem SŘŘ \textbf{Rudolf Habsburský} $\rightarrow$ nepřátelství s Přemyslem
    \item[1276] Přemysl Otakar II. ztrácí v bitvách rakouské země
    \item[26. 8. 1278] \textsc{bitva na Moravském poli u Suchých Krut}, Přemysl Otakar II. umírá
\end{itemize}

\section*{Václav II.}
\begin{itemize}
    \vspace{-0.5em}
    \setlength\itemsep{0.15em}
    \item[$-$] v době smrti Přemysla Otakara II. mu bylo 7 let $\rightarrow$ vládne \textbf{Ota Braniborský}
    \item[$-$] vězněn, poté vykoupen šlechtou
    \item[$-$] kutání stříbra v Kutné Hoře $\rightarrow$  pražský groš, mincovna Vlašský dvůr
    \item[1300] královské právo horníků
    \item[1300] manželkou \textbf{Eliška Rejčka} (z Polska), zisk titulu polského krále
    \item[1301] zisk titulu uherského krále
    \item[1304] \textbf{Albrecht Habsburský} táhne do Čech, odražen
    \item[$-$] zakládá Zbraslavský klášter
    \item[1305] umírá na tuberkulózu
\end{itemize}

\section*{Václav III.}
\begin{itemize}
    \vspace{-0.5em}
    \setlength\itemsep{0.15em}
    \item[$-$] vzdal se uherské koruny
    \item[$-$] táhl do Polska, aby potlačil povstání Ladislava Lokýtka, v Olomouci zabit $\rightarrow$ Přemyslovci vymírají po meči
\end{itemize}

\section*{Boje o českou korunu}
\begin{itemize}
    \vspace{-0.5em}
    \setlength\itemsep{0.15em}
    \item[$-$] \textbf{Jindřich Korutanský}
    \item[$-$] \textbf{Rudolf Habsburský}
    \item[1310] \textbf{Jan Lucemburský}
\end{itemize}
\end{document}
