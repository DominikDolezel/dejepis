\documentclass{article}
\usepackage{fullpage}
\usepackage[czech]{babel}
\usepackage{amsfonts}

\title{\vspace{-2cm}Totalitní režimy, velká hospodářská krize, Československá republika ve 30. letech, situace před druhou světovou válkou, rozbití ČSR\vspace{-1.7cm}}
\date{}
\author{}

\begin{document}
\maketitle

\section*{Totalitní režimy}
\subsection*{Fašismus}
\begin{itemize}
    \item \textsc{pochod na Řím}, státní převrat 29.10.1922
    \item[1923] volební zákon: když strana získá 25 \%, má nárok na $2/3$ křesel v parlamentu $\rightarrow$ podařilo se to fašistům, ovládli parlament, likvidovali opozici
    \item postupně se moc změnila v diktaturu, likvidace opozice
    \item království zachováno, existoval parlament, ale Velká fašistická rada
    \item Hitler ve 20. letech k Mussolinimu vzhlížel
    \item ekonomický korporativní režim, v čele korporace musí být fašista
    \item navzájem podpora, symbióza s katolíky -- Lateránské smlouvy
\end{itemize}

\subsection*{Španělsko}
\begin{itemize}
    \item po 1. sv. válce: Španělsko neutrální, ve 20. letech chudé a ekonomicky zaostalé, vojenská diktatura v čele s \textbf{Miguelem Primo de Riverem} -- konzervativní; král \textbf{Alfonso}
    \item obrovské státní výdaje, inflace, nebyla pozemková reforma $\rightarrow$ bouření na venkově
    \item[(1925)] vyznamenání Bílého lva, ve 30. letech složil funkci
    \item[1931] vyhlášena republika, církev oddělena od státu, ženy hlasovací právo, nová ústava zaručovala autonomii, Katalánci a Baskové
    \item[1936] Vláda lidové fronty -- komunisti, anarchisti, sociální demokraté
    \item \uv{Den, kdy padl Madrid}
    \item[červenec 1936] válka začala vojenským převratem ve španělském Maroku, nacionalisté, nesouhlas s volbami a současnou vládou
    \item velitelem frankistů \textbf{Francisco Franco}, jeho stoupenci příznivci falangy, režim blízký italskému fašismu, jiná ústava, věznění opozice, cenzura
    \item Hitler poskytl frankistům letecký most z Maroka do Španělska, neměli jinak jak transportovat vojáky a zbraně; dále poskytoval zbraně, muže
    \item zbraně nacionalistům: Německo, Itálie; republikánům: Mexiko, Sovětský svaz, interbrigády -- dobrovolnické sbory z celého světa; politika nevměšování: Francie, Velká Británie
    \item[duben 1937] první nálet na civilní obyvatelstvo v historii: Guernica, vybombardováno Němci, poprvé za cíl zasáhnout civilní obyvatelstvo, německá legie Condor
    \item nedařilo se jim dobýt Madrid, posunuli se na sever, ovládli průmyslovou část země; republikáni oslabení rozpory mezi skupinami
    \item interbrigády musely být rozpuštěny na výzvu Společnosti národů
    \item[1.4.1939] vyhlásil Franco válku za ukončenou
    \item v reakci na bombardování $\rightarrow$ Pablo Picasso obraz Guernica
\end{itemize}

\subsection*{Polsko}
\begin{itemize}
    \item poválečná situace nedobrá, politická roztříštěnost, Pilsudski 1926 premiérem, cíl ozdravit Polsko -- sanace
    \item zahájeno nové období do 1939, pomalé zlepšení situace, ale hospodářská krize těžce dopadla
    \item režim se stával autoritářský, 1935 nová ústava, vojenský rozkaz, prezident velké pravomoce
    \item Josef Beck ministr zahraničí, pol. Rovnováhy, nesmí být v konfliktu zároveň s Německem a Sovětským svazem
\end{itemize}

\subsection*{Maďarsko}
\begin{itemize}
    \item stanovení formy vlády, kdo bude hlava státu -- 1. úkol
    \item prohlášen admirál Miklós Horthy -- království bez krále, monarcha Regent
    \item Karel I. se dvakrát snažil získat trůn, jednou vojensky, neúspěšně
    \item zákon o detronizaci Habsburků
    \item radikální strana Ference Szálasiho: snažil se provést nacionalistický převrat, zavřen, založil národně socialistickou stranu Strana šípových křížů, dostali se k moci na konci války --1944, pak vláda národní jednoty
\end{itemize}

\subsection*{Rakousko}
\begin{itemize}
    \item půjčka od Společnosti národů, obnova pomalá, nestabilita vlád, hospodářská krize $\rightarrow$ autoritativní demokracie
    \item Engelbert Dollfuss (austrofašistická vlastenecká fronta) dostal se k moci legálně, stal se kancléřem, pak rozpustil parlament, vyhlásil vládu 1. strany, pak zavražděn nacisty
    \item Kurt Schuschnigg -- nový kancléř, Hitler si ho pozval na povídání -- ultimátum od Němců $\rightarrow$ ústupky vůči Němcům, nekonečné ústupky, jednal jak chtěl Hitler; pak odmítnul ústupky a podal demisi; chtěl udělat plebiscit o připojení k Německu, ale nestihl se, předtím okupace
    \item novým kancléřem ??? -- požádal Němce o obnovení pořádku, Rakousko začaly obsazovat německé jednotky, byly vítány, 12.3.1938 anšlus
    \item Hitler plebiscit udělal, ale zfalšovaný, výsledek 99 \% pro připojení
    \item květinová válka, vítali Němce květinami
\end{itemize}

\section*{Československo ve 30. letech}

\begin{itemize}
    \item ČSR postihla velká hospodářská krize (ve světě od 1929)
    \item[1929] konec vlády panské koalice, pak předčasné volby -- vláda široké koalice -- nová vláda, v čele František Udržal, krize nebyla řešena, Udržal neudržal
    \item[1933] krize kulminuje, průmyslová výroba poklesla na 60 \%, export na $1/3$, hutnický průmysl na $1/3$, vzrůstá nezaměstnanost, 13 mil. obyvatel, přes 900 tis. až 1 300 tis. nezaměstnaných
    \item nejvíce to odskákali dělníci a zemědělci, krize zasáhla zemědělství, Slovensko a Podkarpatská Rus, nespokojenost na Slovensku, ne autonomie, dopady krize $\rightarrow$ radikalizace
    \item postiženy oblasti s lehkým spotřebním průmyslem papírnictví, sklářství $\rightarrow$ Sudety, sudetští Němci radikalizace, obecně celá společnost
    \item fašistické tendence
    \item slovenská lidová strana, klerofašistická strana, měli sjezd, postavili se za Vojtěcha Tuku, 1928 vydal článek kde říkal že martinská deklarace měla tajný dodatek který nebyl zveřejněn -- Slováci pokud nebudou spokojeni, mají se rozhodnout, co dál -- maďarský agent, uvězněn vládou ČSR
    \item požadavek autonomie, nakonec skutečně -- jednotný autonomistický blok -- požadují autonomii, HSLS + další nacionalistická slovenská strana
    \item hlinkovci se zviditelnili srpen 1933, v Nitře výročí 1100 let od založení 1. kostela v Nitře, v době Velké Moravy, vládní delegace přijela, strhli na sebe slovo Hlinkovci, štvavá řeč proti ČSR
    \item také jsme měli fašisty, Národní liga, čeští nacionalisté, v čele Karel Kramář, satisfakce, ostře proti Němcům, vzorem italský fašismus
    \item existovala Národní obec fašistická, v čele Radola Gajda, nejprve v čele čs armády, degradován, bývalý legionář, národní obec fašistická zorganizovala pokus o převrat fašistický, v Brně v kasárnách Židenice, snaha o převrat jako v Itálii, Německu, v roce 1933 se stal kancléřem Hitler, bylo to trapné, neorganizované, primitivní; pak byl 6 měsíců vězněn; neúspěšně
    \item vlajka -- deklasované živly, čeští fašisté
    \item vláda nereagovala, vládní opatření -- zvýšení dávek, rozdali poukázky, jednorázové akce, rozdávali polévky, mléko, podstatu krize ale neřešili
    \item Udržal se neudržal, vláda padla 1932, další premiér Jan Malypetr 1932--1935, už řešili krizi, státní zásahy do ekonomiky, 240 nařízení
    \item zákony: zmocňovací zákon -- rozšíření pravomocí vlády na úkor parlamentu v hospodářských opatřeních, zákon o protistátní činnosti, umožňoval vládě měnit nestranné soudce, zákon o pozastavení x rozpuštění politických stran, zakázána DNSAP i DNP nejdříve pozastvena činnost pak zakázány; okamžitě se sdružili do jiné strany, sudetoněmecká strana, hemlajnovci; tiskové zákony, mohli pozastavit tisk pro demokracii nebezpečných periodik
\end{itemize}

\subsection*{Složitá zahraniční politika}
\begin{itemize}
    \item fašisující státy, Rakousko, Německo, Itálie, Polsko, Maďarsko
    \item Beneš začal budovat pohraniční opevnění 1932
    \item stále více jasné že versailleský mírový systém bude porušen
    \item Beneš se snažil aktivizovat spolupráci v rámci Malé dohody, pakty porganizační a hospodářský -- Jugoslávie, Rumunsko (spíše královské totalitní režimy)
    \item[1934] Balkánská dohoda --  Turecko, Rumunsko, Řecko, Jugoslávie -- špatné pro nás, táhne to země směrem na Balkán spíše než k nám, později i Bulharsko a Albánie, minus pro nás, vzdalují se Jugoslávci a Rumuni
    \item[1934] ministr zahraničí Beneš se snaží s francouzským Louis Barthou o spolupráci, založen východní pakt -- gros vytvořila Francie, ČSR, SSSR, hráz proti expanzi Hitlera, začali jednat s jugoslávským králem Alexandrem I., nicméně když jednali v Marseille, byl na ně spáchán úspěšný atentát
    \item vstup SSSR do Společnosti národů 1934
    \item[1935] Hitler vyhlásil brannou povinnost, ČSR a Francie smlouvy se Sovětským svazem -- do té doby vlády ignorují Sovětský svaz, de jure stát neuznali, ignorovali ho
    \item dovětek smlouvy -- pomoc nám mohou jen tehdy, pokud nám pomůže Francie -- obava ze SSSR
    \item obrovský problém
    \item[květen 1934] prezidentské volby, TGM, 84 let, abdikoval o rok později
    \item během volby čurbes komunisté, Lenin a ne Masaryk, nakonec vyhozeni od voleb komunističtí poslanci, byli na ně vydány zatykače, utekli do SSSR, Gottwald jediný protikandidát Masaryka, neuspěl
    \item[1933] zakázána DNSAP, nová strana Sudetendeutsche Heimatsfront -- Sudetoněmecká vlastenecká fronta, \textbf{Konrad Henlein}  v čele
    \item volby, jdou pod názvem Sudetoněmecká strana, K Henlein, K H Frank
    \item výsledky voleb průšvih, nejvíc hlasů Němci, 1935 parlamentní volby -- SDP 15,2 \%, agrárníci 14,3 \%, socdem 12,5 \%, KSČ 10,3 \%
    \item podařila se znova vytvořit vláda v čele s agrárníkem Janem Malypetrem, koalice
    \item slovenské strany ne tolik hlasů
    \item agrární strana nová předseda Rudolf Beran, posouvá ji ke krajní pravici, pak nahrazen Milanem Hodžou, drží demokratický proud
    \item[listopad 1935] abdikuje Masaryk, prosinec prezidentské volby, Edvard Beneš, na jeho pozici ministra zahraničí pak K Krofta
    \item kandidát agrárníků u prezidentských voleb, v den voleb ho stáhli
    \item[1936] OH v Berlíně, už se vědělo, prosakovaly informace o tom, co dělá, už první koncentráky, ustál to, olympiáda v Berlíně vyzněla jako oslava německého sportu, Hitler přijal naše sudeťáky, spolčí se
    \item[14.9.1937] Masaryk umírá, chmury
    \item[12.3.1938] anexe Rakouska, nebezpečná hranice, Čechy vykouslé
\end{itemize}

\section*{Mezinárodní situace před druhou světovou válkou}
\subsection*{Japonsko}
\begin{itemize}
    \item císař \textbf{Hirohito} -- stoupencem expanzivní politiky, jeho prvním cílem bylo Mandžusko (bohaté na přírodní zdroje, místo odbytu japonské výroby), 1931 \textit{Mukdenský incident} (zinscenování něčeho simono), samostatný stát Mandžusko (císař Pchu I.)
    \item[1937-1945] \textsc{japonsko-čínská válka} -- začala v červenci 1937 u mostu Marca Pola; masové vraždění civilistů -- 300 tis. obyvatel, skončila kapitulací Japonska v září 1945 (považuje se za součást 2. WW)
    \item[1936] pakt proti Kominterně -- uzavřeno mezi Japonskem a Německem v listopadu 1936, týká se společného zacílení proti SSSR (proti komunismu), v roce 1937 se k paktu připojila Itálie; \uv{tajný}  dodatek o tom, že pokud by jeden stát zaútočil na Rusko, tak ostatní státy nesmí být na straně Ruska, počátek linie Berlín--Řím--Tokio
\end{itemize}

\subsection*{Itálie}
\begin{itemize}
    \item[prosinec 1934] \textsc{válka o Etiopii}, Spojené národy nic neudělaly
    \item[říjen 1935] zahájení války, postup do vnitrozemí Etiopie, v Etiopii císař Haile Salassie, požádal o pomoc společnost národů, zas nic moc neudělali
    \item francouzský premiér Pierre Laval slíbil Itálii kus Etiopie $\rightarrow$ pád francouzské vlády
    \item Etiopie se stala součástí Italské východní Afriky (italské kolonie v Eritrei a Somálsku)
    \item[květen 1939] ocelový pakt mezi Itálií a Německem, zavázali se k vzájemné pomoci v případě války, v roce 1940 se připojilo Rumunsko
\end{itemize}

\subsection*{Politika appeasementu}
\begin{itemize}
    \item politika ústupků, zmírňování a tolerance ve snaze předejít vojenskému či politickému konfliktu
    \item posílení Sudetoněmecké strany (působí pro československo, ale chtějí ho rozbít)
    \item[1935] parlamentní volby -- sudetoněmci nejvíc hlasů, nebyli přizvání k tvorbě vlády, to dělali agrárníci (nemam tušení o čem je řeč a stydím se), premiér Milan Hodža
    \item[1935] prezident Beneš
    \item[1937] program národnostní politiky
    \item[duben 1938] sjezd SdP, z toho vznikly tzv. Karlovarské požadavky -- požadují samosprávu pohraničí
    \item plán \textbf{Fall Grün} -- tajný plán vojenského útoku na Československo
    \item[květen 1938] částečná mobilizace
    \item německá strategie: požadovat nesplnitelné
    \item[září 1938] projev Hitlera na sjezdu NSDAP, sudetští němci se bouří, \dots
    \item ultimátní nóta ČSR aby předala sudety Německu (v rámci zachování míru tzv.), v praze demonstrace asi 250 tis. lidí
    \item chtěli i území od polska (těšínsko?) a maďarska (jižní slovensko)
\end{itemize}

\subsection*{Mnichovská konference}
\begin{itemize}
    \item 29./30. září 1938
    \item zástupci států: Adolf Hitler, Neville Chamberlain, Benito Mussolini, Édouard Daladier
    \item podepsání Mnichovské dohody $\rightarrow$ vyklidit pohraničí do 10. října 1938
    \item[1. října 1938] vznik druhé republiky (Česko--Slovensko)
\end{itemize}

\section*{Druhá republika}

\begin{itemize}
    \vspace{-0.5em}
    \setlength\itemsep{0.15em}
    \item[1.1.1938] R. Beran v časopise Venkovm (časopis agrární strany) vyzývá, aby se ČS vláda naučila komunikovat se sudetoněmeckou stranou, aby vstoupila do vlády
    \item[12.3.] \textit{anschluss} (obsazení, připojení Rakouska Němeci), hranice prodlouženy až na 2000 km
    \item[28.3.] Adolf Hitler se sešel s henleinovci a začali kout své pikle, chtějí klást ČS vládě nesplnitelné požadavky, ČS vláda se snaží jednat, ale henleinovci z jednání odešli
    \item[1.4.] M. Hodža, zehdejší premiér vedl ta jednání
    \item[$-$] \textit{Fall Grün}: plán rozbití Československa, využívá problém se sudeťáky a rostoucí touhu Slováklů po autonomii
    \item[23.-24.4.] \textit{Karlovarský sjezzd}: osm bodů, co chtějí Němci po ČS vládě (zrovnoprávnění české a německé národnosti, uzavření pohraničí, sudetští Němci mají vlastní správu, aby se Němci mohli hlásit k nacismu, odškodnění Němců za to, co jim ČS republika udělala)
    \item[$-$] situaci komplikuje politika appeasementu (tendence ustupovat požadavkům agresora)
    \item[květen] francouzským premiérem Eduard Daladier, taky přistupuje na politiku appeasementu
    \item[$-$] v ČS pohraničí dochází k incidentům, německá armáda se přibližuje, snaží se využít problémů, kjteré přišly s volbami do obecních zastupitelstev (květen-červen)
    \item[15.5.] manifest \textit{Věrní zůstaneme}: vyzývá k obraně Československa, nikdo nám nebude diktovat podmínky, podepsalo ho na milion Čechoslováků
    \item[20.5.] vyhlášená částečná mobilizace
    \item[$-$] o den později byli nachystaní
    \item[$-$] začíná mediální kampaň ,kde vysvětluje, o co jde
    \item[$-$] Němci se stáhli
    \item[3.7.] všesokolský slet, vyzněl protiněmecky
    \item[$-$] Hodžova vláda byli ochotni to vykomunikovat, na návrh Británie Češi přistupují na to, aby na naše území přijela jakási mise, která měla zhodnotit situaci
    \item[3.8.-16.9.] mise lorda \textbf{Waltera Runcimana}, došli k závěru, že soužití Čechů a Němců není možné
    \item[13.9.] pokus o puč jednotky Freikops (sudetské jednotky) po projevu Hitlera v Norimberku, vláda povstání pokračeli a sudetoněmecká strana byla definitivně zakázána
    \item[15.9.] začátek jednání Chamberlaina s Hitlerem v Berchtesgadenu
    \item[$-$] Hitler chce území Československa, kde žije více než 50 \% Němců, Chamberlain souhlasí
    \item[19.9.] ultimátum ČS vládě, aby vyhověla těmto požadavkům
    \item[20.-21. 9.] velvyslanci de Lacroix a Newton, kteří naléhali na přijmutí ultimáta
    \item[21.9.] vláda nakonec souhlasí v 6 ráno, v 5 večer ohlašuje veřejnosti herec Zdeněk Štěpánek, veřejnost se bouří v Praze, vláda podává demisi
    \item[23.9.] vytvoření nové vlády v čele s Janem Syrový, který vyhlásil všeobecnou mobilizaci, probíhá s nadšením, Češi chtějí bránit svou republiku
    \item[22.-23.9.] další jednání Chamberlaina a Hitlerav Bad Godesberg, když se dozvěděl o mobilizaci, požaduje, aby ČS vyhovělo i Polským a dalším požadavkům, Hitler  vyhrožuje, že Česko napadne
    \item[29.-30.9.] situace se tak vyhrotila, že Mussolini zorganizoval \textit{Mnichovskou dohodu}, kde jednali HItler, Mussolini, Chamberlain a Daladier, Čechoslováci nepřizváni, Čechy musí postoupit Hitlerovi sudety v několika vlnách, Československá vláda musí vyhovět Maďarska a Polska
    \item[30.11.] když velvyslanci dorazili do Prahy, Syrového vláda kapitulovala vůči Německu
    \item[$-$] ztráta pohraničí, Poláci chtějí Těšínsko a malé kousky Slovensko (Orava, Spiš, Šariš, Kisúce), což dostali
    \item[2.11.] \textit{Vídeňská arbitráž}: ministr zahraničí Německa a Itálie, co musí odevzdat Čechoslováci Maďarům (jižní a východní Slovensko a Podkarpatská Rus)
    \item[30.9.] \uv{oškubáním} vzniká druhá republika
\end{itemize}

---

\begin{itemize}
    \vspace{-0.5em}
    \setlength\itemsep{0.15em}
    \item[13.3.1939] jednání A. Hitlera a J. Tisa
    \item[14.3.] vyhlášen Slovenský stát, odtrhnuto od Druhé republiky, pražská vláda dostala \textit{Maďarské ultimátum}: buď Češi zmizí z Podkarpatské Rusi, nebo začneme vojenské operace $\rightarrow$ autonomní území
    \item[14.-15.3.] E. Hácha a F. Chvalkovský jednali v Berlíně s Hitlerem a Göringem, Hitler chtěl, aby Hácha požádal o ochranu zbytku Čech, aby byly připojeny ke Třetí říši, Hácha dostal infarkt, ve čtyři hodiny ráno nakonec souhlasil
    \item[15.3.] okupace od 6.00
    \item[$-$] většina lidí schovaných doma, polarizované -- někteří lidé Hitlera vítali, druhá skupina ho nenáviděla
\end{itemize}

\subsection*{Protektorát Čechy a Morava}
\begin{itemize}
    \vspace{-0.5em}
    \setlength\itemsep{0.15em}
    \item[16.3.1939] vyhlášen protektorát Čechy a Morava (čti Hitler podepsal nařízení o vytvoření protektorátu), zůstalo jen 7,5 milionů obyvatel
    \item[18.3.] říšským protektorem jmenován Konstantin von Neurath, státní tajemník K. H. Frank
    \item[$-$] Neurath se jevil málo krutý, proto byl v roce 1941 vystřídán Heinrichem
    \item[$-$] v protektorátu jsme nesměli mít parlament, armádu (jen vládní vojsko), mohli jsme mít presidenta (Hácha byl tedy jen \uv{státní president})
    \item[$-$] měli jsme celkem čtyři protektorátní vlády: Rudolf Beran, Alois Eliáš (vlastenec, udržoval kontakt s Londýnem a odboji, později popraven), Jaroslav Krejčí (kolaborant), Richard Bienert (kolaborant)
    \item[$-$] Hácha se ze začátku snažil protestovat, ale po Heydrychiádě už byl nemocný a začala mu vynechávat psychika, nepoznával lidi, věci apod.
    \item[$-$] po celém protektorátu německé nadpisy, dvojjazyčný systém, germanisace, němčina ve školách, pozměněny české dějiny
    \item[$-$] ostré vystupování proti intelektuálům, v roce 1939 na podzim zavřeny vysoké školy, potřebovali podniky a zemědělce, jezdí se vpravo, říšská marka
\end{itemize}

\subsection*{Slovensko}
\begin{itemize}
    \vspace{-0.5em}
    \setlength\itemsep{0.15em}
    \item[$-$] Tiso patřil mezi umírněný proud, chtěl fašistický stát pod záštitou Německa
    \item[$-$] Vojtěch Tuka patřil k radikálnímu proudu
    \item[$-$] opravdu fašistický stát, posílali židy do koncentráků apod., vyráběli pro říši
    \item[$-$] když napadl Hitler Polsko, Slovensko byla jedna z mála zemí, jež se účastnila
\end{itemize}



\end{document}
