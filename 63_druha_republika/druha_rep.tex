\documentclass{article}
\usepackage{fullpage}
\usepackage[czech]{babel}
\usepackage{amsfonts}

\title{\vspace{-2cm}Druhá republika\vspace{-1.7cm}}
\date{}
\author{}

\begin{document}
\maketitle

\begin{itemize}
    \vspace{-0.5em}
    \setlength\itemsep{0.15em}
    \item[1.1.1938] R. Beran v časopise Venkovm (časopis agrární strany) vyzývá, aby se ČS vláda naučila komunikovat se sudetoněmeckou stranou, aby vstoupila do vlády
    \item[12.3.] \textit{anschluss} (obsazení, připojení Rakouska Němeci), hranice prodlouženy až na 2000 km
    \item[28.3.] Adolf Hitler se sešel s henleinovci a začali kout své pikle, chtějí klást ČS vládě nesplnitelné požadavky, ČS vláda se snaží jednat, ale henleinovci z jednání odešli
    \item[1.4.] M. Hodža, zehdejší premiér vedl ta jednání
    \item[$-$] \textit{Fall Grün}: plán rozbití Československa, využívá problém se sudeťáky a rostoucí touhu Slováklů po autonomii
    \item[23.-24.4.] \textit{Karlovarský sjezzd}: osm bodů, co chtějí Němci po ČS vládě (zrovnoprávnění české a německé národnosti, uzavření pohraničí, sudetští Němci mají vlastní správu, aby se Němci mohli hlásit k nacismu, odškodnění Němců za to, co jim ČS republika udělala)
    \item[$-$] situaci komplikuje politika appeasementu (tendence ustupovat požadavkům agresora)
    \item[květen] francouzským premiérem Eduard Daladier, taky přistupuje na politiku appeasementu
    \item[$-$] v ČS pohraničí dochází k incidentům, německá armáda se přibližuje, snaží se využít problémů, kjteré přišly s volbami do obecních zastupitelstev (květen-červen)
    \item[15.5.] manifest \textit{Věrní zůstaneme}: vyzývá k obraně Československa, nikdo nám nebude diktovat podmínky, podepsalo ho na milion Čechoslováků
    \item[20.5.] vyhlášená částečná mobilizace
    \item[$-$] o den později byli nachystaní
    \item[$-$] začíná mediální kampaň ,kde vysvětluje, o co jde
    \item[$-$] Němci se stáhli
    \item[3.7.] všesokolský slet, vyzněl protiněmecky
    \item[$-$] Hodžova vláda byli ochotni to vykomunikovat, na návrh Británie Češi přistupují na to, aby na naše území přijela jakási mise, která měla zhodnotit situaci
    \item[3.8.-16.9.] mise lorda \textbf{Waltera Runcimana}, došli k závěru, že soužití Čechů a Němců není možné
    \item[13.9.] pokus o puč jednotky Freikops (sudetské jednotky) po projevu Hitlera v Norimberku, vláda povstání pokračeli a sudetoněmecká strana byla definitivně zakázána
    \item[15.9.] začátek jednání Chamberlaina s Hitlerem v Berchtesgadenu
    \item[$-$] Hitler chce území Československa, kde žije více než 50 \% Němců, Chamberlain souhlasí
    \item[19.9.] ultimátum ČS vládě, aby vyhověla těmto požadavkům
    \item[20.-21. 9.] velvyslanci de Lacroix a Newton, kteří naléhali na přijmutí ultimáta
    \item[21.9.] vláda nakonec souhlasí v 6 ráno, v 5 večer ohlašuje veřejnosti herec Zdeněk Štěpánek, veřejnost se bouří v Praze, vláda podává demisi
    \item[23.9.] vytvoření nové vlády v čele s Janem Syrový, který vyhlásil všeobecnou mobilizaci, probíhá s nadšením, Češi chtějí bránit svou republiku
    \item[22.-23.9.] další jednání Chamberlaina a Hitlerav Bad Godesberg, když se dozvěděl o mobilizaci, požaduje, aby ČS vyhovělo i Polským a dalším požadavkům, Hitler  vyhrožuje, že Česko napadne
    \item[29.-30.9.] situace se tak vyhrotila, že Mussolini zorganizoval \textit{Mnichovskou dohodu}, kde jednali HItler, Mussolini, Chamberlain a Daladier, Čechoslováci nepřizváni, Čechy musí postoupit Hitlerovi sudety v několika vlnách, Československá vláda musí vyhovět Maďarska a Polska
    \item[30.11.] když velvyslanci dorazili do Prahy, Syrového vláda kapitulovala vůči Německu
    \item[$-$] ztráta pohraničí, Poláci chtějí Těšínsko a malé kousky Slovensko (Orava, Spiš, Šariš, Kisúce), což dostali
    \item[2.11.] \textit{Vídeňská arbitráž}: ministr zahraničí Německa a Itálie, co musí odevzdat Čechoslováci Maďarům (jižní a východní Slovensko a Podkarpatská Rus)
    \item[30.9.] \uv{oškubáním} vzniká druhá republika
\end{itemize}

---

\begin{itemize}
    \vspace{-0.5em}
    \setlength\itemsep{0.15em}
    \item[13.3.1939] jednání A. Hitlera a J. Tisa
    \item[14.3.] vyhlášen Slovenský stát, odtrhnuto od Druhé republiky, pražská vláda dostala \textit{Maďarské ultimátum}: buď Češi zmizí z Podkarpatské Rusi, nebo začneme vojenské operace $\rightarrow$ autonomní území
    \item[14.-15.3.] E. Hácha a F. Chvalkovský jednali v Berlíně s Hitlerem a Göringem, Hitler chtěl, aby Hácha požádal o ochranu zbytku Čech, aby byly připojeny ke Třetí říši, Hácha dostal infarkt, ve čtyři hodiny ráno nakonec souhlasil
    \item[15.3.] okupace od 6.00
    \item[$-$] většina lidí schovaných doma, polarizované -- někteří lidé Hitlera vítali, druhá skupina ho nenáviděla
\end{itemize}

\subsection*{Protektorát Čechy a Morava}
\begin{itemize}
    \vspace{-0.5em}
    \setlength\itemsep{0.15em}
    \item[16.3.1939] vyhlášen protektorát Čechy a Morava (čti Hitler podepsal nařízení o vytvoření protektorátu), zůstalo jen 7,5 milionů obyvatel
    \item[18.3.] říšským protektorem jmenován Konstantin von Neurath, státní tajemník K. H. Frank
    \item[$-$] Neurath se jevil málo krutý, proto byl v roce 1941 vystřídán Heinrichem
    \item[$-$] v protektorátu jsme nesměli mít parlament, armádu (jen vládní vojsko), mohli jsme mít presidenta (Hácha byl tedy jen \uv{státní president})
    \item[$-$] měli jsme celkem čtyři protektorátní vlády: Rudolf Beran, Alois Eliáš (vlastenec, udržoval kontakt s Londýnem a odboji, později popraven), Jaroslav Krejčí (kolaborant), Richard Bienert (kolaborant)
    \item[$-$] Hácha se ze začátku snažil protestovat, ale po Heydrychiádě už byl nemocný a začala mu vynechávat psychika, nepoznával lidi, věci apod.
    \item[$-$] po celém protektorátu německé nadpisy, dvojjazyčný systém, germanisace, němčina ve školách, pozměněny české dějiny
    \item[$-$] ostré vystupování proti intelektuálům, v roce 1939 na podzim zavřeny vysoké školy, potřebovali podniky a zemědělce, jezdí se vpravo, říšská marka
\end{itemize}

\subsection*{Slovensko}
\begin{itemize}
    \vspace{-0.5em}
    \setlength\itemsep{0.15em}
    \item[$-$] Tiso patřil mezi umírněný proud, chtěl fašistický stát pod záštitou Německa
    \item[$-$] Vojtěch Tuka patřil k radikálnímu proudu
    \item[$-$] opravdu fašistický stát, posílali židy do koncentráků apod., vyráběli pro říši
    \item[$-$] když napadl Hitler Polsko, Slovensko byla jedna z mála zemí, jež se účastnila
\end{itemize}



\end{document}
