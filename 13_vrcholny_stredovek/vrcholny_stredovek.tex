\documentclass{article}
\usepackage{fullpage}
\usepackage[czech]{babel}
\usepackage{amsfonts}

\title{\vspace{-2cm}Vrcholný středověk (11. -- 15. st.)\vspace{-1.7cm}}
\date{}
\author{}

\begin{document}
\maketitle

\section*{Obecně.}
\begin{itemize}
    \vspace{-0.5em}
    \setlength\itemsep{0.15em}
    \item[$-$] osidlování dříve neosidlených území
    \item[$-$] rozšiřování orné půdy $\rightarrow$ větší výnosy, rozvoj řemesla a obchodu
    \item[$-$] peněžní hospodářství,  ve městech univerzity
    \item[$-$] poddanská povstání, morové epidemie (umřela $1 / 3$ obyvatelstva)
    \item[13. st.] církev na vrcholu
    \item[$-$] gotická kultura
    \item[$-$] \textit{stavové} = šlechta, církev, měšťané
    \item[$-$] vpád Mongolů do Evropy
\end{itemize}


\section*{Křížové výpravy = kruciáty (1095 -- 1291)}
\begin{itemize}
    \vspace{-0.5em}
    \setlength\itemsep{0.15em}
    \item[$-$] Levanta vs. Arabové
    \item[$-$] příčiny: dobytí svaté země; \textit{majorát} = dědičné právo, první syn dědí vše, proto se ostatní účastní výprav, aby získali majetek; italská města -- jsou tu Seldžukové, nemohou provozovat obchod
    \item[$-$] vyhlašuje papež
    \item[1071] \textsc{bitva u Mantzikertu}, Seldžukové porážejí Byzantince
    \item[1076] Jeruzalém, Boží hrob
\end{itemize}

\subsection*{První křížová výprava 1095 -- 1099}
\begin{itemize}
    \vspace{-0.5em}
    \setlength\itemsep{0.15em}
    \item[$-$] vyhlásil \textbf{Urban II.}, má problematický vztah s Jindřichem IV.
    \item[1095] \textsc{koncil v Piazenze}, byzantský císař \textbf{Alexios I.} posílá posly, aby papeže požádali o pomoc v boji se Seldžuky
    \item[1095] \textsc{koncil v Klermontu}, vahlášení první křížové výpravy
\end{itemize}





\end{document}
