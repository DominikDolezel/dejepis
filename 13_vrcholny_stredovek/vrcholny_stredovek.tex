\documentclass{article}
\usepackage{fullpage}
\usepackage[czech]{babel}
\usepackage{amsfonts}

\title{\vspace{-2cm}Vrcholný středověk (11. -- 15. st.)\vspace{-1.7cm}}
\date{}
\author{}

\begin{document}
\maketitle

\section*{Obecně.}
\begin{itemize}
    \vspace{-0.5em}
    \setlength\itemsep{0.15em}
    \item[$-$] osidlování dříve neosidlených území
    \item[$-$] rozšiřování orné půdy $\rightarrow$ větší výnosy, rozvoj řemesla a obchodu
    \item[$-$] peněžní hospodářství,  ve městech univerzity
    \item[$-$] poddanská povstání, morové epidemie (umřela $1 / 3$ obyvatelstva)
    \item[13. st.] církev na vrcholu
    \item[$-$] gotická kultura
    \item[$-$] \textit{stavové} = šlechta, církev, měšťané
    \item[$-$] vpád Mongolů do Evropy
\end{itemize}


\section*{Křížové výpravy = kruciáty (1095 -- 1291)}
\begin{itemize}
    \vspace{-0.5em}
    \setlength\itemsep{0.15em}
    \item[$-$] Levanta vs. Arabové
    \item[$-$] příčiny: dobytí svaté země; \textit{majorát} = dědičné právo, první syn dědí vše, proto se ostatní účastní výprav, aby získali majetek; italská města -- jsou tu Seldžukové, nemohou provozovat obchod
    \item[$-$] vyhlašuje papež
    \item[1071] \textsc{bitva u Mantzikertu}, Seldžukové porážejí Byzantince
    \item[1076] Jeruzalém, Boží hrob
\end{itemize}

\subsection*{První křížová výprava 1095 -- 1099}
\begin{itemize}
    \vspace{-0.5em}
    \setlength\itemsep{0.15em}
    \item[$-$] vyhlásil \textbf{Urban II.}, má problematický vztah s Jindřichem IV.
    \item[1095] \textsc{koncil v Piazenze}, byzantský císař \textbf{Alexios I.} posílá posly, aby papeže požádali o pomoc v boji se Seldžuky
    \item[1095] \textsc{koncil v Klermontu}, vyhlášení první křížové výpravy
    \item[(1096)] lidová křížová výprava, nebyla úspěšná
    \item[$-$] poté i rytíři, probojují se do oblasti Levanty, získají svatá místa
    \item[1099] dobytí Jeruzaléma $\rightarrow$ svaté místo
    \item[$-$] křižácké státy podřízené Jeruzalémskému království: Antiochejské knížectví, Tripoliské hrabství, Edesské hrabství
\end{itemize}

\subsection*{Druhá křížová výprava (1147 -- 1149)}
\begin{itemize}
    \vspace{-0.5em}
    \setlength\itemsep{0.15em}
    \item[$-$] francouzský král Ludvík VII., Přemyslovec Vladislav II., SŘŘ KOnrád III.
    \item[$-$] též neúspěšná, úzeí nerozšířeno, sály vyrovnána
    \item[1187] \textsc{bitva u Hattínu} sultán \textbf{Saladin} poráží křižáky, dobyl Jeruzalém a další, konec rovnováhy
\end{itemize}


\subsection*{Třetí křížová výprava (1189 -- 1192)}
\begin{itemize}
    \vspace{-0.5em}
    \setlength\itemsep{0.15em}
    \item[$-$] císař SŘŘ Fridrich Barbarossa, francouzský král Filip II. August, anglický král Richard Lví Srdce
    \item[$-$] Fridrich během tažení utonul
    \item[$-$] též neúspěch, jenom povolení návštěv Jeruzaléma
    \item[$-$] \textsc{znovudobytí strategických bodů Akkonu a Jaffy}
\end{itemize}


\subsection*{Čtvrtá křížová výprava (1202 -- 1204)}
\begin{itemize}
    \vspace{-0.5em}
    \setlength\itemsep{0.15em}
    \item[$-$] původně měla směřovat do Egypta, cíl: oslabit muslimy v tomto prostředí
    \item[$-$] jenže žoldáci zdrženi v Benátkách dóžetem \textbf{Enricem Dandolem}, takže vtrhli a zlikvidovali Byzanc
\end{itemize}


\subsection*{Konec křížových výprav}
\begin{itemize}
    \vspace{-0.5em}
    \setlength\itemsep{0.15em}
    \item[1291] \textsc{pád Akkonu}, konec křížových výprav
    \item[$-$] křižáci se stahují zpět
\end{itemize}


\subsection*{Důsledky}
\begin{itemize}
    \vspace{-0.5em}
    \setlength\itemsep{0.15em}
    \item[$-$] církev na vrcholu, bohatství a moc
    \item[$-$] centralizace, posílené královské moci
    \item[$-$] rozvoj měst na Apeninském poloostrově: Benátky, Janov
    \item[$-$] peněžní hospodářství, nové plodiny, technologie, hrady, gotický sloh
    \item[$-$] rytířská kultura, tři rytířské řády: johanité, templáři, němečtí rytíři
    \item[$-$] prolínání různých kultur
\end{itemize}



\end{document}
