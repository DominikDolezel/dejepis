\documentclass{article}
\usepackage{fullpage}
\usepackage[czech]{babel}
\usepackage{amsfonts}

\title{\vspace{-2cm}Vrcholný středověk (11. -- 15. st.)\vspace{-1.7cm}}
\date{}
\author{}

\begin{document}
\maketitle

\section*{Obecně.}
\begin{itemize}
    \vspace{-0.5em}
    \setlength\itemsep{0.15em}
    \item[$-$] osidlování dříve neosidlených území
    \item[$-$] rozšiřování orné půdy $\rightarrow$ větší výnosy, rozvoj řemesla a obchodu
    \item[$-$] peněžní hospodářství,  ve městech univerzity
    \item[$-$] poddanská povstání, morové epidemie (umřela $1 / 3$ obyvatelstva)
    \item[13. st.] církev na vrcholu
    \item[$-$] gotická kultura
    \item[$-$] \textit{stavové} = šlechta, církev, měšťané
    \item[$-$] vpád Mongolů do Evropy
\end{itemize}


\section*{Křížové výpravy = kruciáty (1095 -- 1291)}
\begin{itemize}
    \vspace{-0.5em}
    \setlength\itemsep{0.15em}
    \item[$-$] výpravy křižáků do Levanty, která je ovládána Araby
    \item[$-$] příčiny: dobytí svaté země; \textit{majorát} = dědičné právo, první syn dědí vše, proto se ostatní účastní výprav, aby získali majetek; italská města -- jsou tu Seldžukové, nemohou provozovat obchod
    \item[$-$] vyhlašuje papež
    \item[1071] \textsc{bitva u Mantzikertu}, Seldžukové porážejí Byzantince
    \item[1076] Jeruzalém, Boží hrob
\end{itemize}

\subsection*{První křížová výprava 1095 -- 1099}
\begin{itemize}
    \vspace{-0.5em}
    \setlength\itemsep{0.15em}
    \item[$-$] vyhlásil \textbf{Urban II.}, má problematický vztah s Jindřichem IV.
    \item[1095] \textsc{koncil v Piazenze}, byzantský císař \textbf{Alexios I.} posílá posly, aby papeže požádali o pomoc v boji se Seldžuky
    \item[1095] \textsc{koncil v Klermontu}, vyhlášení první křížové výpravy
    \item[(1096)] lidová křížová výprava, nebyla úspěšná
    \item[$-$] poté i rytíři, probojují se do oblasti Levanty, získají svatá místa
    \item[1099] dobytí Jeruzaléma $\rightarrow$ svaté místo
    \item[$-$] křižácké státy podřízené Jeruzalémskému království: Antiochejské knížectví, Tripoliské hrabství, Edesské hrabství
\end{itemize}

\subsection*{Druhá křížová výprava (1147 -- 1149)}
\begin{itemize}
    \vspace{-0.5em}
    \setlength\itemsep{0.15em}
    \item[$-$] francouzský král Ludvík VII., Přemyslovec Vladislav II., SŘŘ KOnrád III.
    \item[$-$] též neúspěšná, úzeí nerozšířeno, sály vyrovnána
    \item[1187] \textsc{bitva u Hattínu} sultán \textbf{Saladin} poráží křižáky, dobyl Jeruzalém a další, konec rovnováhy
\end{itemize}


\subsection*{Třetí křížová výprava (1189 -- 1192)}
\begin{itemize}
    \vspace{-0.5em}
    \setlength\itemsep{0.15em}
    \item[$-$] císař SŘŘ Fridrich Barbarossa, francouzský král Filip II. August, anglický král Richard Lví Srdce
    \item[$-$] Fridrich během tažení utonul
    \item[$-$] též neúspěch, jenom povolení návštěv Jeruzaléma
    \item[$-$] \textsc{znovudobytí strategických bodů Akkonu a Jaffy}
\end{itemize}


\subsection*{Čtvrtá křížová výprava (1202 -- 1204)}
\begin{itemize}
    \vspace{-0.5em}
    \setlength\itemsep{0.15em}
    \item[$-$] původně měla směřovat do Egypta, cíl: oslabit muslimy v tomto prostředí
    \item[$-$] jenže žoldáci zdrženi v Benátkách dóžetem \textbf{Enricem Dandolem}, takže vtrhli a zlikvidovali Byzanc
\end{itemize}


\subsection*{Konec křížových výprav}
\begin{itemize}
    \vspace{-0.5em}
    \setlength\itemsep{0.15em}
    \item[1291] \textsc{pád Akkonu}, konec křížových výprav
    \item[$-$] křižáci se stahují zpět
\end{itemize}


\subsection*{Důsledky}
\begin{itemize}
    \vspace{-0.5em}
    \setlength\itemsep{0.15em}
    \item[$-$] církev na vrcholu, bohatství a moc
    \item[$-$] centralizace, posílené královské moci
    \item[$-$] rozvoj měst na Apeninském poloostrově: Benátky, Janov
    \item[$-$] peněžní hospodářství, nové plodiny, technologie, hrady, gotický sloh
    \item[$-$] rytířská kultura, tři rytířské řády: johanité, templáři, němečtí rytíři
    \item[$-$] prolínání různých kultur
\end{itemize}


\section*{Mocenské soupeření mezi Anglií a Francií, 12. -- 15. st.}
\subsection*{Anglie, normandská dynastie}
\begin{itemize}
    \vspace{-0.5em}
    \setlength\itemsep{0.15em}
    \item[1066] zakladatel \textbf{Vilém I. Dobyvatel} poražením \textbf{Harolda II. Godwindsona} v \textsc{bitvě u Hastings}
    \item[$-$] korunován ve Westminsteru
    \item[$-$] Normandie součástí Anglie, protože jej v roce 911 Vikingové dostali od Francouzů
    \item[$-$] Vilém přiděluje přivržencům drobná léna, aby neměli moc velkou moc
    \item[$-$] dělení území na \textit{hrabství}, v jejich čele \textit{šerifové} = královští úředníci, dbají na odvádění daní, soudní pravomoce
    \item[(1086)] \textit{Domesday Book} = soupis veškerého majetku, kde se co nachází $\rightarrow$ později k zavedení daní
\end{itemize}


\subsection*{Dynastie Plantagenetů}
\begin{itemize}
    \vspace{-0.5em}
    \setlength\itemsep{0.15em}
    \item[$-$] \textbf{Jindřich II. Plantagenet}
    \item[$-$] \textbf{Thomas Becket}, Jindřichův přítel, arcibiskup, chce vytvořit církev nezávislost na králi, později zabit
    \item[$-$] \textit{putující soudci} = kontroloři šerifů
    \item[$-$] \textbf{Eleonora Akvitánská}, manželka francouzského krále, po rozvodu se provdala za Jindřicha $\rightarrow$ územní zisky, Francouzi ztácí přístup k moři na západě
    \item[$-$] francouzský král \textbf{Filip II. August} chce území dobýt zpět

    \item[$-$] \textbf{Richard Lví srdce}
    \item[$-$] třetí křížová výprava, při cestě zpět zajat v Rakousku (1192), vykoupen zpět
    \item[$-$] \textsc{válka s Filipem II.}, neúspěch (nic nedobyli), Richard poraněn šípem, umírá
    \item[$-$] Jan Bezzemek
    \item[$-$] jeho bratr
    \item[$-$] \textsc{bitva u Bouvignes} proti Filipovi, ztrácí většinu území ve Francii
    \item[1215] \textit{Magna charta libertatum} (velká listina svobod) = oslabil pozici anglického krále -> zárodky budoucího parlamentu, Velká královská rada -> budoucí horní sněmovna lordů
    \item[$-$] \textbf{Eduard I.} definitivně přiojil Wales
    \item[$-$] \textbf{Eduard II.} jeho manželka ho donutila se vzdát královského titulu ve prospěch Eduarda III., jeho syna
\end{itemize}


\subsection*{Francie, Kapetovci}
\begin{itemize}
    \vspace{-0.5em}
    \setlength\itemsep{0.15em}
    \item[$-$] zakladatel \textbf{Hugo Kapet}, zvolen 987
    \item[$-$] \textbf{Filip II. August} získává území Angličanů vš. Normandie
    \item[$-$] Ludvík IX. Svatý
    \item[$-$] \textbf{Filip IV. Sličný}
    \item[$-$] vrchol centralizace
    \item[$-$] chce zdanit církev -> konflikt s papežem \textbf{Bonifácem VIII.} -> nechal ho zatknout, Bonifác ve vězení umírá
    \item[$-$] na místo papeže dosazuje \textbf{KLimenta V.}, donutí ho, aby církev přesídlila z Říma do Avignonu = \textit{avignonské zajetí}
    \item[$-$] templáři mají na jeho vkus moc majetku -> obviní je z kacířství, zatýká \textbf{Jacquesa de Molay} (poslední velmistr templářů, později upálen), tento řád zrušil, špičky upáleny
    \item[$-$] \uv{černý pátek} $\dots$ zajetí onoho velmistra probehlo v pátek 13.
    \item[$-$] chce získat Flandry, neúspěch
    \item[1302] Filip potřebuje pomoc s dobýváním, začíná svolávat tzv. \textit{generální stavy} = zástupci šlechty, církve a měst, např. rozhodují o daních -> získá si je na svou stranu
\end{itemize}


\subsection*{Příčiny stoleté války}
\begin{itemize}
    \vspace{-0.5em}
    \setlength\itemsep{0.15em}
    \item[$-$] Anglie chce území ve Francii
    \item[$-$] Francie chce Flandry
    \item[1328] vymírají Kapetovci (Karel IV.), provdá svou dceru za \textbf{Filipa z Valois}, proti tomu se vymezili Angličané (Eduard III.), vytváří si nároky na Francouzskou korunu
    \item[$-$] Filip z Valois tím zakládá novou dynastii
\end{itemize}








\end{document}
