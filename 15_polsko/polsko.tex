\documentclass{article}
\usepackage{fullpage}
\usepackage[czech]{babel}
\usepackage{amsfonts}

\title{\vspace{-2cm}Polsko\vspace{-1.7cm}}
\date{}
\author{}

\begin{document}
\maketitle

\subsection*{Přemysl II. Velkopolský}
\begin{itemize}
    \vspace{-0.5em}
    \setlength\itemsep{0.15em}
    \item[$-$] sjednotil Polsko
    \item[1296] umírá
\end{itemize}

\begin{itemize}
    \vspace{-0.5em}
    \setlength\itemsep{0.15em}
    \item[$-$] Václav II.
    \item[$-$] Václav III.
\end{itemize}

\section*{Vladislav I. Lokýtek z Piastovců}
\begin{itemize}
    \vspace{-0.5em}
    \setlength\itemsep{0.15em}
    \item[$-$] opět sjednotil Polsko, králem
\end{itemize}

\section*{Kazimír III. Veliký}
\begin{itemize}
    \vspace{-0.5em}
    \setlength\itemsep{0.15em}
    \item[$-$] dokončil sjednocování Polska
    \item[$-$] poslední Piastovec, po jeho smrti problém s obsazováním polské koruny
\end{itemize}

\section*{Dynastie Anjou}
\subsection*{Ludvík I. Veliký}
\begin{itemize}
    \vspace{-0.5em}
    \setlength\itemsep{0.15em}
    \item[$-$] jeho matka byla sestrou Kazimíra
    \item[$-$] uherským i polským králem $\rightarrow$ sjednocení -- polsko-uherská personální unie
    \item[$-$] dvě dcery: \textbf{Marie} (dědička uherské koruny), \textbf{Hedvika} (dědička polské koruny)
\end{itemize}

\subsection*{Hedvika}
\begin{itemize}
    \vspace{-0.5em}
    \setlength\itemsep{0.15em}
    \item[$-$] vzala si litevského knížete \textbf{Jogalia} $\rightarrow$ dynastie \textbf{Jagellonců}
    \item[(1386)] polsko-litevská personální unie
\end{itemize}

\begin{itemize}
    \vspace{-0.5em}
    \setlength\itemsep{0.15em}
    \item[1410] \textsc{bitva u Grunwaldu} proti řádu německých rytířů, výhra polsko-litevské unie
    \item[(1569 -- 1795)] Lublinská unie: spojení států jako takových
\end{itemize}







\end{document}
