\documentclass{article}
\usepackage{fullpage}
\usepackage[czech]{babel}
\usepackage{amsfonts}

\title{\vspace{-2cm}Revoluce 1848-1849\vspace{-1.7cm}}
\date{}
\author{}

\begin{document}
\maketitle

\begin{itemize}
    \vspace{-0.5em}
    \setlength\itemsep{0.15em}
    \item[$-$] \uv{jaro národů}, navzájem se od sebe inspirují a probíhá několik revolucí
    \item[$-$] touhy po sjednocení, odstranění absolutismu či zrušení nevolnictví
\end{itemize}

\subsection*{Itálie}
\begin{itemize}
    \vspace{-0.5em}
    \setlength\itemsep{0.15em}
    \item[$-$] Apeninský poloostrov je rozdrobený, sever drží Habsburkové, uprostřed papežský stát a na jihu Království obojí Sicílie drženo Bourbony
    \item[$-$] Sardinské království: Sardinie a severozápadní část Pyrenejského poloostrova, hlavní cíl je vytvořit jednotný stát
    \item[$-$] \textit{risorgimento}: obnovení, vzkříšení
    \item[$-$] dva hlavní proudy:
    \begin{itemize}
        \vspace{-0.5em}
        \setlength\itemsep{0.15em}
        \item[$-$] \textit{liberálové}: stačilo by jim sjednocení do ústavní monarchie, budou dcí král má pocházet se sardisnkého království
        \item[$-$] \textit{revoluční demokraté}: Apeninský poloostrov se má sjeednoti to demokratické republiky (členové organisace Mladá Itálie \textbf{Giuseppe Mazzini}, \textbf{Giuseppe Garibaldi})
    \end{itemize}
    \item[leden 1848] \textsc{nepokoje na Sicílii a neapolsku} s cílem vyhnat Bourbony
    \item[bžezen 1848] \textsc{povstání} za cílem sesadit Habsburky
    \item[$-$] v čele snahy o sjednocení Sardinské království v čele se savojskou dynastií, král \textbf{Karel Albert} a šikovný politik hrabě \textbf{Camillo Benso di Cavour}
    \item[(1848)] \textsc{bitva u Custozzy}, prohrává Sardinské království proti Habsburkům (pod vedením maršála Radeckého)
    \item[(1849)] \textsc{bitva u Novarry}, taky prohrává Sardinské království
    \item[$\rightarrow$] pokusy o sjednocení neúspěšné, Karel Albert abdikuje a nastupuje jeho synovec \textbf{Viktor Emanulel II.}
    \item[$-$] radikálové se pokouší na různých místech Apeninského poloostrova vyhlásit republiku, ale neúspěšně, protože Habsburkové jsou v této době silní a na jejich straně je Francie, kde nyní vládne Napoleon III.
    \item[$\rightarrow$] všechny pokusy neúspěšné, Itálie nesjednocena
\end{itemize}

\subsection*{Francie}
\begin{itemize}
    \vspace{-0.5em}
    \setlength\itemsep{0.15em}
    \item[$-$] \textit{Červencová monarchie}, král \textbf{Ludvík Filip}
    \item[1848] zemědělská a obchodní krise
    \item[$-$] velkým problémem je vysoký majetkový census, o politice mohou rozhodovat jen ti nejbrohatší, kterých je zoufale málo, společnost se tedy radikalisuje
    \item[$-$] hlavním cílem revoluce je změna volebního práva a snížení majetkového censu
    \item[$-$] s tímto cílem se konají shromáždění, která jsou spojena s pohoštěním: \textit{bankety}
    \item[únor 1848] král jeden z banketů zakázal $\rightarrow$ začínají demonstrace, staví se barikády, boje na barikádách, během těchto bojů stávající premiér podává demisi
    \item[25.2.1848] \textsc{vyhlášena druhá republika}, král prchá do Londýna, definitivně končí vláda dynastie Bourbonů
    \item[$-$] vznik prozatimní vlády, ta byla nesmírně pestrá, složení ze všech politických proudů, zrušení censury, právo na práci (národní dílny, kde mohou pracovat nezaměstnaní), zrušení otroctví v koloniích, zavedeno všecobecné volební právo
    \item[květen 1848] \textsc{volby do Národního shromáždění}, nedostala se však levice, začínají prosazovat změny, zrušení národních dílen $\rightarrow$ povstání a boje v Paříži = \textit{červnové povstání}
    \item[listopad 1848] nová ústava, kde počítá s posicí presidenta, který má značené pravomoce, volební období 4 roky, hledání vhodného kandidáta
    \item[$-$] prvním presidentem \textbf{Ludvík Bonaparte}, Napoleonův synovec, během napoleonských válek nizozemským králem
    \item[prosinec 1851] posice presidenta prodloužena ze čtyř let na deset
    \item[2.12.1852] pomocí armády na sebe strhává moc a stává se císařem
    \item[$\rightarrow$] konec druhé republiky, začátek druhého císařství, ze začátko vojensko-policejní diktatura, ale postupně se začíná rozvolňovat až do stavu jakési parlamentní monarchie = \textit{bonapartismus}
    \item[1870] \textsc{prusko-francouzská válka}, Francie poražena, Bonaparte zajat a sesazen, vyhlášena třetí republika, ta bude trvat až do roku 1940
\end{itemize}

\subsection*{Německo}
\begin{itemize}
    \vspace{-0.5em}
    \setlength\itemsep{0.15em}
    \item[$-$] Německý spolek: jednotlivé země, Pruska a Rakouského císařství ty části, které dříve patřily do SŘŘ, chtějí se sjednotit
    \item[březen 1848] \textsc{povstání v Berlíně}, pruský král donucen k ústupkům: změna ústavy, reformy, nová liberální vláda, tato vláda vyhlásila volby do celoněmeckého parlamentu v Heidelbergu, vytvořen celoněmecká prozatímní parlament, ten řeší, jakým způsobem se sjednotit
    \item[$-$] dva proudy
    \begin{itemize}
        \vspace{-0.5em}
        \setlength\itemsep{0.15em}
        \item[$-$] \textit{radikální}: chtějí sjednotit do republiky
        \item[$-$] \textit{umírnění}: stačí jim monarchie

    \end{itemize}
    \item[$-$] dvě koncepce:
    \begin{itemize}
        \vspace{-0.5em}
        \setlength\itemsep{0.15em}
        \item[$-$] \textit{koncepce maloněmecká}: sjednocení jen těchto států bez Habsburské monarchie
        \item[$-$] \textit{koncepce velkoněmecká}: sjednocení i s Habsburskou monarchií
    \end{itemize}
    \item[květen 1848] zahájeno jednání ve Frankfurtu nad Mohanem
    \item[červen 1849] vytvořena německá ústava a nová dočasná vláda, německé státy se sjednotí do monarchie, titul císaře nabídli tehdejšímu pruskému králi \textbf{Fridrichu Vilému IV.}, ten to nepřijal $\rightarrow$ propukají další revoluce, které byly postupně rozehnány, až byl nakonec rozehnán i sněm
    \item[$-$] všechny pokusy o sjednocení neúspěšné 
\end{itemize}


\end{document}
