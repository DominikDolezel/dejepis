\documentclass{article}
\usepackage{fullpage}
\usepackage[czech]{babel}
\usepackage{amsfonts}

\title{\vspace{-2cm}Humanismus\vspace{-1.7cm}}
\date{}
\author{}

\begin{document}
\maketitle

\begin{itemize}
    \vspace{-0.5em}
    \setlength\itemsep{0.15em}
    \item[$-$] myšlénkový proud, \textit{humanus} = lidský
    \item[$-$] za prvního představitele považován Francesco Petrarca
    \item[$-$] lidé už se nezabývají jen studiem teologie, ale i studiem odvětví ryze lidských = studia \textit{divina}
    \item[$-$] \textit{ad fontes} = k pramenům (ne jen poslouchat komentáře, ale umět řecky a přečíst si díla sám)
    \item[$-$] základní knihou je bible, všechny knihy vytištěny do roku 1500 = \textit{inkunábule}, jsou nesmírně drahé
    \item[$-$] Kronika trojanská
\end{itemize}

\section*{Filosofie}
\begin{itemize}
    \vspace{-0.5em}
    \setlength\itemsep{0.15em}
    \item[$-$] Lorenzo Valla: mezi prvními říkal, že Konstantinova donace (díky které církev odvozuje svoje právo na Řím) je faleš, že to nepsal Konstantin, spis O slasti (nejlepší dobro pro člověka je slast)
    \item[$-$] Pietro Pomponazzi, Niccolló Machiavelli (vladař)
    \item[$-$] \textit{platonismus} = ve Florencii platonská akademie
\end{itemize}

\section*{Věda}
\begin{itemize}
    \vspace{-0.5em}
    \setlength\itemsep{0.15em}
    \item[$-$] svět není konečný a vše se netočí kolem Země
    \item[$-$] Mikuláš Koperník: heliocentrická teorie, rotace Země kolem své osy
    \item[$-$] Johannes Kepler: Keplerovy zákony, planety neobíhají po kružnici, ale po elipse, zkonstruoval dalekohled
    \item[$-$] Tycho de Brahe: astronom na dvoře Rudolfa II., pohřben v kostele Panny Marie před Týnem
    \item[$-$] Galileo Galilei: dalekohled, Měsíc není pravidelná koule, nakonec pronásledován církví, dožil v domácím vězení, \uv{a přece se točí}, což nejspíš neřekl
    \item[$-$] Giordano Bruno: dominikán, vzdělanec, vesmír je nekonečný, upálen
\end{itemize}

\section*{Anatomie}
\begin{itemize}
    \vspace{-0.5em}
    \setlength\itemsep{0.15em}
    \item[$-$] poskočila znalost lidského těla v rámci prvních pitev
\end{itemize}

\section*{Společenské vědy}
\begin{itemize}
    \vspace{-0.5em}
    \setlength\itemsep{0.15em}
    \item[$-$] Erasmus Rotterdamský: kritizuje nežádoucí jevy ve společnosti, sepsal v díle Chvála bláznivosti
    \item[$-$] Jan Amos Komenský: filosof, pedagog, Labyrint světa a ráj srdce
    \item[$-$] \textit{utopie} = něco neuskutečnitelného v praxi, nereálného
    \item[$-$] Thomas More: Utopia, líčí ideální společnost
    \item[$-$] Thomaso Campanela: utopie sepsána v díle Sluneční stát
\end{itemize}
\section*{Reformace v 16. st. v Německu}
\begin{itemize}
    \vspace{-0.5em}
    \setlength\itemsep{0.15em}
    \item[$=$] snaha o nápravu, obnovení církve
    \item[$-$] žít v chudobě a čistotě, vyřešit papežské schizma, ne svatokupectví, ne korupci v církvi, ne světské moci v církvi
    \item[$-$] \textbf{konciliarismus} = svolávání koncilů
    \item[$-$] \textbf{papalismus} = svolávání kocilů papežem
    \item[$-$] šlechta si dělá zálusk na majetek církve
    \item[$-$] tzv. \textit{první reformace} \textbf{John Wyclif} 14. st.
    \item[$-$] tzv \textit{česká reformace} 15. st. $\rightarrow$ rozdělení církve
    \item[$-$] \textbf{Martin Luther} kazatel, tolog, reformátor
    \begin{itemize}
        \vspace{-0.5em}
        \setlength\itemsep{0.15em}
        \item[1517] přibil na dveře kláštera v Wittenberku 95 tezí proti církvi, kvůli knihtisku se rychle šíří
        \item[$-$] proti odpustkům, kritizuje církev "mezi člověkem a bohem nemusí být nic", přeložil bibli do němčiny, chtěl pod obojí, proti celibátu
        \item[1520] papež \textbf{Lev X.} vydává bulu s hrozbou do kladby \implies Luther veřejně spálil, 1521 papež ho exkomunikoval
        \item[1521] \textbf{Karel V.} vydává \textsc{Wormský edikt}
        \item[$-$] \textbf{Fridrich III. Moudrý}, saský kurfiřt umožnil bydlet na Wartbugu Lutherovi
        \item[1522] \textsc{zakládá lutheránství}
    \end{itemize}
    \item[$-$] Filip Melanchon -- přítel a poradce Luthera
    \item[1524 -- 1526] \textsc{německá selská válka}, cílená proti vrchnosti, chtějí snížit daně a odstranit robotu, lovit ryby v panských rybnících -- sociální, nikoliv náboženské požadavky, spojeno s \textit{chiliasmem}; aktivity nevzdělaných jedinců, významný vůdce: \textbf{Tomáš Müntzer} (kazatel)
    \item[1525] povstání potlačeno v Durynsku, o rok později i v ostatních částech
    \item[1533 -- 34] \textit{Münsterská komuna} -- nejradikálnější program reformačního hnutí, poražen po 16 měsících
    \item[1529] \textsc{říšský sněm ve Špýru} - katolíci se snaží potlačit luterýny $\rightarrow$ bránili se, že hlasovat o něčí víře se nedá $\rightarrow$ označení \textit{protestanti}
    \item[1531] \textit{Šmalkaldský svaz} = vojenské sjednocení protestantů
    \item[1546 -- 47] \textsc{šmalkadská válka}, protestanti proti katolíkům, \textsc{bitva u Mühlbergu}, porážka protestantů
    \item[$-$] lutheránství se ale v SŘŘ šíří dál
    \item[1555] \textsc{Augspurský mír}: čí vláda, toho víra (\textit{cuius regio, eius religio}), Karel V. s tím není ztotožněný $\rightarrow$
    \item[$-$] důsledek: šlechta se může rozhodnout, poddaní mají víru svého pána
    \item[1556] Karel V. odestupuje $\rightarrow$ faktické vítězství protestantů
    \item[$-$] důsledek: SŘŘ je rozdrobená

\end{itemize}

\section*{Šíření luteránství}
\begin{itemize}
    \vspace{-0.5em}
    \setlength\itemsep{0.15em}
    \item[$-$] v těchto státech: Skandinávie (Švédsko, Dánsko), Podabltí (polská města jako Toruň, Poznaň, hanzovní města Talin a Riga), Uhersko (Slovensko, Sedmihradsko), České země (německy mluvící města ve Slezsku a v Čechách)
\end{itemize}


\section*{Švýcarská reformace (Měšťanská reformace)}
\begin{itemize}
  \item[$-$] v této době probíhal švýcarský boj o nezávislost na SŘŘ v 20. letech 16. st., válka mezi jednotlivými švýcarskými kantony (katolické x švýcarské)
  \item[$-$] \textbf{Ulrich Zwingli} (1484-1531) -- vůdce švýcarských kantonů, radikálnější než Luther, důraz na morálku, poddanstvo má právo se vzbouřit, když se šlechta nechová správně, odmítají katolické tradice (přijímání, ikony), pak padl v boji
  \item[$-$] \textbf{Jan Kalvín} (1509-1564) -- teolog francouzského původu, útočiště v Ženevě, vlivy Luthera, Zwingliho, učení předdestinace -- je už o člověku od začátku určeno, zda bude spasen nebo zatracen, ale člověk to neví, takže by se měl chovat správně, v průběhu života se dovídá zda je ten bohem vybraný
  \item[$-$] \textit{kalvinismus} -- důraz na pracovitost, skromnost, individualitu, podnikavost - jakýsi prvopočátek (dejme tomu) kapitalistického, podnikavého mindsetu, pevné morální zásady - prostě se v životě nebav, nebo nebudeš spasen
  \item[$-$] kalvinismus rozšířen v Anglii (významné pro osidlování S Ameriky), Uhersku, ...
\end{itemize}

\section*{Protireformace}
\begin{itemize}
  \item[$-$] Katolíci chtějí zpátky získat ztracené pozice, zabránit šíření nekatolických církví, snaží se o přestavbu katolické církve
  \item[$-$] se souhlasem papeže byl r. 1540 založen nový řád -- \textbf{Jezuité} (Societas Jesu), zakladatel Ignác z Loyoly, přísný řád, odpovídají jen papežovi, účely rekatolizace, misionářství, zřizovali a vedli většinu škol
  \item[$-$] r. 1542 založení \textit{Svaté inkvizice} -- no, inkvizice, dále zakazují některé knihy, píší jejich seznamy (Index librum prohibitorum)
  \item[$-$] \textsc{Tridentský koncil} (1545-1563) -- přelom ve fungovaní katolické církve, odmítnuta radikální reformace, jediná autorita je papež, ale musí se omezit hromadění církevních úřadů v rukou jednotlivce a že duchovní musí být vzdělaní -- biskupské semináře pro výchovu nových kněží
\end{itemize}


\end{document}
