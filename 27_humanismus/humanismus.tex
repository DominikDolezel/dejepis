\documentclass{article}
\usepackage{fullpage}
\usepackage[czech]{babel}
\usepackage{amsfonts}

\title{\vspace{-2cm}Humanismus\vspace{-1.7cm}}
\date{}
\author{}

\begin{document}
\maketitle

\begin{itemize}
    \vspace{-0.5em}
    \setlength\itemsep{0.15em}
    \item[$-$] myšlénkový proud, \textit{humanus} = lidský
    \item[$-$] za prvního představitele považován Francesco Petrarca
    \item[$-$] lidé už se nezabývají jen studiem teologie, ale i studiem odvětví ryze lidských = studia \textit{divina}
    \item[$-$] \textit{ad fontes} = k pramenům (ne jen poslouchat komentáře, ale umět řecky a přečíst si díla sám)
    \item[$-$] základní knihou je bible, všechny knihy vytištěny do roku 1500 = \textit{inkunábule}, jsou nesmírně drahé
    \item[$-$] Kronika trojanská
\end{itemize}

\section*{Filosofie}
\begin{itemize}
    \vspace{-0.5em}
    \setlength\itemsep{0.15em}
    \item[$-$] Lorenzo Valla: mezi prvními říkal, že Konstantinova donace (díky které církev odvozuje svoje právo na Řím) je faleš, že to nepsal Konstantin, spis O slasti (nejlepší dobro pro člověka je slast)
    \item[$-$] Pietro Pomponazzi, Niccolló Machiavelli (vladař)
    \item[$-$] \textit{platonismus} = ve Florencii platonská akademie
\end{itemize}

\section*{Věda}
\begin{itemize}
    \vspace{-0.5em}
    \setlength\itemsep{0.15em}
    \item[$-$] svět není konečný a vše se netočí kolem Země
    \item[$-$] Mikuláš Koperník: heliocentrická teorie, rotace Země kolem své osy
    \item[$-$] Johannes Kepler: Keplerovy zákony, planety neobíhají po kružnici, ale po elipse, zkonstruoval dalekohled
    \item[$-$] Tycho de Brahe: astronom na dvoře Rudolfa II., pohřben v kostele Panny Marie před Týnem
    \item[$-$] Galileo Galilei: dalekohled, Měsíc není pravidelná koule, nakonec pronásledován církví, dožil v domácím vězení, \uv{a přece se točí}, což nejspíš neřekl
    \item[$-$] Giordano Bruno: dominikán, vzdělanec, vesmír je nekonečný, upálen
\end{itemize}

\section*{Anatomie}
\begin{itemize}
    \vspace{-0.5em}
    \setlength\itemsep{0.15em}
    \item[$-$] poskočila znalost lidského těla v rámci prvních pitev
\end{itemize}

\section*{Společenské vědy}
\begin{itemize}
    \vspace{-0.5em}
    \setlength\itemsep{0.15em}
    \item[$-$] Erasmus Rotterdamský: kritizuje nežádoucí jevy ve společnosti, sepsal v díle Chvála bláznivosti
    \item[$-$] Jan Amos Komenský: filosof, pedagog, Labyrint světa a ráj srdce
    \item[$-$] \textit{utopie} = něco neuskutečnitelného v praxi, nereálného
    \item[$-$] Thomas More: Utopia, líčí ideální společnost
    \item[$-$] Thomaso Campanela: utopie sepsána v díle Sluneční stát 
\end{itemize}





\end{document}
