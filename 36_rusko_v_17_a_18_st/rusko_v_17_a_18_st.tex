\documentclass{article}
\usepackage{fullpage}
\usepackage[czech]{babel}
\usepackage{amsfonts}

\title{\vspace{-2cm}Rusko v 17. a 18. století\vspace{-1.7cm}}
\date{}
\author{}

\begin{document}
\maketitle

\begin{itemize}
    \vspace{-0.5em}
    \setlength\itemsep{0.15em}
    \item[$-$] Ivan IV. se prohlásil za prvního cara, Fjodor Ivanovič (vymírá dynastie Rurikovců), Lžidimitrijové, období krize = \textit{smuta} (1587-1613)
    \item[1613] zvolen carem \textbf{Michail Romanov}, jím nastupuje dynastie Romanovců
\end{itemize}

\section*{Michail Romanov, Alexej Romanov (1645-1676)}
\begin{itemize}
    \vspace{-0.5em}
    \setlength\itemsep{0.15em}
    \item[$-$] dlouho pod nadvládou Mongolů, vývoj zastaven, teprve teď obnovují
    \item[$-$] absolutistická vláda = \textit{samoděržaví}
    \item[$-$] velkou moc má pravoslavná církev
    \item[$-$] drtivá většina společnosti nevolníci, zemědělci, zaostalé za Evropou, vše dováženo
    \item[$-$] nízká kulturní, vzdělanostní úroveň obyvatelstva
    \item[$\Rightarrow$] nutné reformy, přístav Archangelsk v zimě zamrzá, chce získat přístup k Baltu nebo Černému moři k otevření obchodu
    \item[$-$] postupně pronikají k Tichému oceánu
    \item[$-$] území Kozáků (obyvatelé ruských stepí) je ovládáno Polsko-litevským státem, pomáhali proti Mongolům $\Rightarrow$
    \item[1648-1654] \textsc{povstání Bohdana Chmelnického}, levobřežní Ukrajina připojena k Rusku
    \item[(1670-1671)] \textsc{selské války} v čele s Stěnkou Razinem, protože poddaní byli nespokojeni
\end{itemize}

\section*{Petr Veliký (1689-1725)}
\begin{itemize}
    \vspace{-0.5em}
    \setlength\itemsep{0.15em}
    \item[$-$] zakladatel moderního ruského státu, obrovské množství reforem
    \item[$-$] nechal mučit svého syna, později ho nechal popravit
    \item[$-$] obdivovatel západu, uznává nové techniky
    \item[1682] dostává se k moci s bratrem Ivanem V., ale vládne jejich sestra Sofie, v sedmnícti letech ji sesadí a vsadí do kláštera $\Rightarrow$ dostává se do čela Ruska, o chvíli později umírá i jeho bratr
    \item[1699] úspěšný ve válkách s Turky, \textsc{dobýjí pevnost Azov}
    \item[1697]  inkognito vyjíždí do západní Evropy, zjišťuje situaci a inspiruje se
    \item[1698] \textsc{povstání střeleckých pluků}, zatímco je v Evropě, chtějí dosadit zpět jeho sestru Sofii
    \item[1700-1721] \textsc{severní válka} vytvoření prtišvédské koalice: Rusko, Dánsko, Polsko, Sasko
    \begin{itemize}
        \vspace{-0.5em}
        \setlength\itemsep{0.15em}
        \item[1700] \textsc{bitva u Narvy}, Švédové porážejí Rusy
        \item[1703] zakládá nové sídelní město Petrohrad, \uv{okno do Evropy}, (1712) hlavní město
        \item[1709]  \textsc{bitva u Poltavy}, švédská vojska vtažena na jih Ruska, kde vojska Petra Velikého vítězí, boje ale pokračují dál
        \item[1721] \textit{Nystadský mír}, Rusko získává přístup k Baltu,  Švédové ztrácí značná území
    \end{itemize}
    \item[$-$] zavedení pravidelné armády, budování válečného námořnictva, Petropavlovská pevnost, školství, budování infrastruktury, zakládání manufaktur, kde pracují nevolníci, sjednocená měna \textit{rubl}
    \item[$-$] rozdělení území na části = \textit{gubernie}
    \item[$-$] zjednodušení písma = \textit{graždanka}  
\end{itemize}


\end{document}
