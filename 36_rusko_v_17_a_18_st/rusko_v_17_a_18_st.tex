\documentclass{article}
\usepackage{fullpage}
\usepackage[czech]{babel}
\usepackage{amsfonts}

\title{\vspace{-2cm}Rusko v 17. a 18. století\vspace{-1.7cm}}
\date{}
\author{}

\begin{document}
\maketitle

\begin{itemize}
    \vspace{-0.5em}
    \setlength\itemsep{0.15em}
    \item[$-$] Ivan IV. se prohlásil za prvního cara, Fjodor Ivanovič (vymírá dynastie Rurikovců), Lžidimitrijové, období krize = \textit{smuta} (1587-1613)
    \item[1613] zvolen carem \textbf{Michail Romanov}, jím nastupuje dynastie Romanovců
\end{itemize}

\section*{Michail Romanov, Alexej Romanov (1645-1676)}
\begin{itemize}
    \vspace{-0.5em}
    \setlength\itemsep{0.15em}
    \item[$-$] dlouho pod nadvládou Mongolů, vývoj zastaven, teprve teď obnovují
    \item[$-$] absolutistická vláda = \textit{samoděržaví}
    \item[$-$] velkou moc má pravoslavná církev
    \item[$-$] drtivá většina společnosti nevolníci, zemědělci, zaostalé za Evropou, vše dováženo
    \item[$-$] nízká kulturní, vzdělanostní úroveň obyvatelstva
    \item[$\Rightarrow$] nutné reformy, přístav Archangelsk v zimě zamrzá, chce získat přístup k Baltu nebo Černému moři k otevření obchodu
    \item[$-$] postupně pronikají k Tichému oceánu
    \item[$-$] území Kozáků (obyvatelé ruských stepí) je ovládáno Polsko-litevským státem, pomáhali proti Mongolům $\Rightarrow$
    \item[1648-1654] \textsc{povstání Bohdana Chmelnického}, levobřežní Ukrajina připojena k Rusku
    \item[(1670-1671)] \textsc{selské války} v čele s Stěnkou Razinem, protože poddaní byli nespokojeni
\end{itemize}

\section*{Petr Veliký (1689-1725)}
\begin{itemize}
    \vspace{-0.5em}
    \setlength\itemsep{0.15em}
    \item[$-$] zakladatel moderního ruského státu, obrovské množství reforem
    \item[$-$] nechal mučit svého syna, později ho nechal popravit
    \item[$-$] obdivovatel západu, uznává nové techniky
    \item[1682] dostává se k moci s bratrem Ivanem V., ale vládne jejich sestra Sofie, v sedmnícti letech ji sesadí a vsadí do kláštera $\Rightarrow$ dostává se do čela Ruska, o chvíli později umírá i jeho bratr
    \item[1699] úspěšný ve válkách s Turky, \textsc{dobýjí pevnost Azov}
    \item[1697]  inkognito vyjíždí do západní Evropy, zjišťuje situaci a inspiruje se
    \item[1698] \textsc{povstání střeleckých pluků}, zatímco je v Evropě, chtějí dosadit zpět jeho sestru Sofii
    \item[1700-1721] \textsc{severní válka} vytvoření prtišvédské koalice: Rusko, Dánsko, Polsko, Sasko
    \begin{itemize}
        \vspace{-0.5em}
        \setlength\itemsep{0.15em}
        \item[1700] \textsc{bitva u Narvy}, Švédové porážejí Rusy
        \item[1703] zakládá nové sídelní město Petrohrad, \uv{okno do Evropy}, (1712) hlavní město
        \item[1709]  \textsc{bitva u Poltavy}, švédská vojska vtažena na jih Ruska, kde vojska Petra Velikého vítězí, boje ale pokračují dál
        \item[1721] \textit{Nystadský mír}, Rusko získává přístup k Baltu,  Švédové ztrácí značná území
    \end{itemize}
    \item[$-$] zavedení pravidelné armády, budování válečného námořnictva, Petropavlovská pevnost, školství, budování infrastruktury, zakládání manufaktur, kde pracují nevolníci, sjednocená měna \textit{rubl}
    \item[$-$] rozdělení území na části = \textit{gubernie}
    \item[$-$] zjednodušení písma = \textit{graždanka}
\end{itemize}

\section*{Kateřina II. Velká (1762-1796)}
\begin{itemize}
    \vspace{-0.5em}
    \setlength\itemsep{0.15em}
    \item[$-$] sesadila svého manžela Petra III. (vnuk Petra Velikého), nechala ho uvěznit, kde byl zavražděn, tzv. \uv{dámská revoluce}
    \item[$-$] doba palácových převratů (6 vladařů za 27 let)
    \item[$-$] lutheránský původ, po příjezdu do Ruska však konvertovala k pravoslaví
    \item[$-$] vládne v duchu osvícenského absolutismu, ale krutá a sexuálně aktivní, literárně činná, milenci z řad člechty
    \item[$-$] \uv{zlatý věk Ruska}, založení Lomonosovy univerzity v Moskvě
    \item[$-$] propagovala očkování proti neštovicím, sama se nechala očkovat
    \item[$-$] centralizace, utužení nevolnictví (\textit{mužici})
    \item[$-$] porážka osmanských Turků, dosáhnutí Černého moře, Krymu: vznik Sevastopolu, Oděsy $\Rightarrow$ kolonizace Ukrajiny, Černomoří, Čukotky, Sachalin, Aljaška
    \item[$-$] \textit{Potěmkinovy vesnice}, měl na starosti kolonizaci, kulisy, aby při návštěvě Kateřiny \uv{bylo na co se dívat}, jeden z jejích milenců
    \item[$-$] \textsc{trojí dělení Polska}
\end{itemize}

\subsection*{Zánik Polska}
\begin{itemize}
    \vspace{-0.5em}
    \setlength\itemsep{0.15em}
    \item[$-$] \uv{zlatá svoboda} šlechty, dělá si co chce, sleduje jenom svoje zájmy
    \item[1386] dědička Hedvika se provdala za litevského knížete Jagiella, spojí se do polsko-litevské personální unie, vládnou Jagellonci
    \item[(1569)] personální unie se ruší, vzniká lublisnká unie, o tři roky později vymírají Jagellonci
    \item[$-$] král je volen šlechtou, \textit{sejm} = zastupitelský orgán (šlechtici, každý z nich má právo veta)
    \item[$-$] drtivá část společnosti znevolněna, národnostně pestré
    \item[$-$] krize využívají sousedé, třeba Švédsko prosazuje svoji dynastii
    \item[$-$] hluboká krize nastává v 18. století za polského krále Augusta III., toho využívá Kateřina, která vojensky prosadila do čela Polska jednoho ze svých stoupenců, nastupuje jako král \textbf{Stanislav II. August}, toho nepřijímá šlechta $\Rightarrow$ sesazen $\Rightarrow$ Rusko se spojí s Rakouskem a Pruskem, domluví se na \textsc{trojím dělením Polska}, každý si vezme asi třetinu, šlechta musí přísahat věrnost
    \item[$-$] šlechta tuší, že je něco špatně, mění své zákony, pokouší se vytvořit novou ústavu (1791), taškařice $\Rightarrow$ druhé dělení
    \item[$-$] po abdikaci Stanislava II. proběhne třetí dělení
    \item[$-$] dělění v letech: 1772 (Rusko, Prusko, Habsburkové), 1793 (Rusko a Prusko), 1795 (Rusko, Prusko, Habsburkové)
    \item[$-$] někdy se též mluví o čtvrtém dělení Polska, což je počátek druhé světové války
\end{itemize}

\subsection*{Vznik Pruska}
\begin{itemize}
    \vspace{-0.5em}
    \setlength\itemsep{0.15em}
    \item[$-$] základem Braniborské markrabství, kde vládne dynastie Hohenzollernové od počátku 15. st.
    \item[$-$] postupně se Braniborské markrabství zvětšuje, připojeno vévodství Pruské, po třicetileté válce patří mezi nejvýznamnější části SŘŘ
    \item[2. pol. 17. st.] kurfiřt Fridrich Vilém: reformy, zakládá manufaktury, posiluje území
    \item[1701] jeho syn, Fridrich III., se v Königsbergu prohlásil za krále (vládne jako Fridrich I.), mluvíme o Pruském království = Braniborsko, Západní Pomořany, Východní Prusko
    \item[$-$] buduje Berlín jako sídelní město, využívá práci nevolníků, většina lutheránu, ale nábožensky tolerantní
\end{itemize}

\subsubsection*{Friedrich I. (1701-1713)}
\begin{itemize}
    \vspace{-0.5em}
    \setlength\itemsep{0.15em}
    \item[$-$] buduje Berlín jako sídelní město, vznik typických barokních zámků
    \item[$-$] velkostatkáři = \textit{junkeři}, většina společnosti, nevolníci
    \item[$-$] zdrojem ekonomiky je vývoz obilí
    \item[$-$] založení pruské univerzity, uchycení lutheránství, ale jsou nábožensky tolerantní vůči ostatním
\end{itemize}

\subsubsection*{Fridrich Vilém I. (1713-1740)}
\begin{itemize}
    \vspace{-0.5em}
    \setlength\itemsep{0.15em}
    \item[$-$] syn Fridricha I., \uv{kaprál na trůně} (celý život chodil v uniformě), militarzoval Prusko
    \item[$-$] branná povinnost pro nevolníky, vojenský drill, poslušnost
    \item[$-$] \textsc{severní válka} mezi Švédkem a protišvédské koalici, Prusko získává malá území
    \item[$-$] rozvoj \textit{merkantilismu}, školství, ale prioritná je poříd militarizace a armáda
\end{itemize}

\subsubsection*{Friedrich II. Veliký (1740-1786)}
\begin{itemize}
    \vspace{-0.5em}
    \setlength\itemsep{0.15em}
    \item[$-$] současník Marie Terezie, za něj zahájeno trojí dělení Polska
    \item[$-$] typicky osvícenský panovník, na jeho dvoře pobýval Voltaire, I. Kant
    \item[$-$] \textit{Edikt} o náboženské svobodě
    \item[$-$] rozvoj obchodu, stavění infrastruktury, zakládání manufaktur, uve do Pruska odborníky ze zahraničí
    \item[$-$] \uv{filosof ze Sanssouci} (rokokový zámek), věnolal se prakticky filosofii, má sloužit tomu státu (ne jako Ludvík XIV.)
    \item[$-$] účastnil se \textsc{slezských válek}, \textsc{sedmileté války}
\end{itemize}


\end{document}
