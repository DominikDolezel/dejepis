\documentclass{article}
\usepackage{fullpage}
\usepackage[czech]{babel}
\usepackage{amsfonts}

\title{\vspace{-2cm}Kultura a vzdělanost vrcholného středověku\vspace{-1.7cm}}
\date{}
\author{}

\begin{document}
\maketitle

\begin{itemize}
    \vspace{-0.5em}
    \setlength\itemsep{0.15em}
    \item[$-$] \textit{scholastika} = filozofické učení, obhajuje teologické teze
    \item[$-$] za začátek považován dvůr Karla Velikého ve vrcholném středověku (8. st.)
    \item[$-$] ze začátku ovlivněno Platónem, poté je dominantní Aristoteles
    \item[$-$] 2 proudy: racionalistický, podle \textbf{Tomáše Akvinského} \textit{tomismus}, \uv{k pravdě lze dojít dvěma způsoby, které se nevylučují: vírou a rozumem}, oproti tomu iracionalistický směr, podle \textbf{Jan Duns Scotus} \textit{scotismus}, \uv{víra je nadřazená rozumu}
    \item[$-$] nositeli jsou mnišské řády: \textit{dominikání} (školství, inkvizice, podle sv. Dominika) vs. \textit{františkáni} (bohatí, vlivní, pokora a láska k rozumu, podle Františka z Assisi)
    \item[$-$] reformní řády (z benediktínů, podle Benedikta z Nursie): \textit{cisterciáci} (kolonizovali a kultivovali půdu, kult Panny Marie, podle města Citeaux ve Francii) a \textit{premonstráti} (kazatelé, podle města Prémonstré ve Francii)
    \item[$-$] školy: při farách a klášterech, ve městech tzv. \textit{partikulární} školy (učí se základy jako číst, psát, počítat), univerzity (studium sedmera svobodných umění: právnická, lékařská, teologická fakulta)
    \item[$-$] na učitele a profesory se vztahuje kanonické právo (sice netuším, co to znamená, ale)
    \item[$-$] nejstarší univerzity: Bologna, Padova, Sorbonna, Salamanca, Cambridge, Oxford, Praha, Krakov, Vídeň, Lipsko
\end{itemize}

\section*{Rytířská kultura}
\begin{itemize}
    \vspace{-0.5em}
    \setlength\itemsep{0.15em}
    \item[$-$] vzniká na jihu Francie někdy ve 12. století, rytířské turnaje
    \item[$-$] ušlechtilé dvorní chování, věrný svému pánovi, chrání slabé, kult ženy
    \item[$-$] \textit{trubadúři} (J Francie), \textit{truvéři} (Francie), \textit{minnesängři} (Německo)
    \item[$-$] Zbraslavská, Dalimilova kronika, cestopis Milion, ...
\end{itemize}


\section*{Gotika}
\begin{itemize}
    \vspace{-0.5em}
    \setlength\itemsep{0.15em}
    \item[$-$] periodizace: \textit{raná} (přechodné období), \textit{vrcholná} (13. -- 14. st.), \textit{pozdní} (15. st.)
    \item[$-$] vzniká ve Francii v okolí Paříže, první stavba Saint Denis, název pejorativní podle Gótů (barbarské), vznikl až za renesance
    \item[$-$] u nás: \textit{přemyslovská} (13. st.), \textit{lucemburská} (14. st.), \textit{jagellonská} (15. st.)
    \item[$-$] stavby \textit{sakrální} a \textit{světské}
\end{itemize}

\subsection*{Architektura}
\begin{itemize}
    \vspace{-0.5em}
    \setlength\itemsep{0.15em}
    \item[$-$] základní znaky: lomený oblouk, žebrová klenba, opěrný systém, vertikalita, bohaté zdobení, členitost, stavěno z kamene, vysoké cihly, tzv, \textit{buchty}
    \item[$-$] \textit{fiála} = špička, jehlan, na kterém ledacos může být, \textit{krakorec} = nosný článek vystupující ze zdi, slouží k podpěře
    \item[$-$] druhy staveb: katedrály, kláštery, kostely, hrady, městská opevnění, městské stavby (radnice, měšťanské domy, mosty)
    \item[$-$] \textit{katedrála} = biskupský nebo arcibiskupský kostel, honosnější, např. Chrám Matky Boží v Paříži
    \item[$-$] Katedrála Sv. Víta. Vojtěcha a Václava: architekt Matyáš z Arrasu, Petr Parléř, Josef Mocker, ale dokončil až ve 20. st. Kamil Hilbert
\end{itemize}

\subsection*{Sochařství}
\begin{itemize}
    \vspace{-0.5em}
    \setlength\itemsep{0.15em}
    \item[$-$] sochy stojí ve volném prostoru, buď ze dřeva nebo z kamene, \textit{polychromie}  = vícebarevné, mají výraz
    \item[$-$] náměty: \textit{madona} = Panna Marie v náručí s Ježíšem, \textit{pieta} =
Panna Marie drží v náručích Ježíše sňatého z kříže
    \item[$-$] socha Svatého Jiří, jak bojuje s drakem

\end{itemize}


\subsection*{Malba}
\begin{itemize}
    \vspace{-0.5em}
    \setlength\itemsep{0.15em}
    \item[$-$] Giotto di Bondone: typicky modrá barva
    \item[$-$] lidé mají individuální rysy, nejsou všichni stejní
    \item[$-$] Přebohaté hodinky vévody z Berry: překrásně zdobená kniha modliteb, vrchol gotické knižní malby
    \item[$-$] u nás: Mistr vyšebrodského, třeboňského oltáře, Mistr Theodorik
    \item[$-$] knižní malba: Velislavova bible, Bible Václava IV.
\end{itemize}




\end{document}
