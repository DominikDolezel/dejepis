\documentclass{article}
\usepackage{fullpage}
\usepackage[czech]{babel}
\usepackage{amsfonts}

\title{\vspace{-2cm}Kultura a vzdělanost vrcholného středověku\vspace{-1.7cm}}
\date{}
\author{}

\begin{document}
\maketitle

\begin{itemize}
    \vspace{-0.5em}
    \setlength\itemsep{0.15em}
    \item[$-$] \textit{scholastika} = filozofické učení, obhajuje teologické teze
    \item[$-$] za začátek považován dvůr Karla Velikého ve vrcholném středověku (8. st.)
    \item[$-$] ze začátku ovlivněno Platónem, poté je dominantní Aristoteles
    \item[$-$] 2 proudy: racionalistický, podle \textbf{Tomáše Akvinského} \textit{tomismus}, \uv{k pravdě lze dojít dvěma způsoby, které se nevylučují: víru a rozum}, oproti tomu iracionalistický směr, podle \textbf{Jan Duns Scotus} \textit{scotismus}, \uv{víra je nadřazená rozumu}
    \item[$-$] nositeli jsou mnišské řády: \textit{dominikání} (školství, inkvizice, podle sv. Dominika) vs. \textit{františkáni} (bohatí, vlivní, pokora a láska k rozumu, podle Františka z Assisi)
    \item[$-$] reformní řády (z benediktínů, podle Benedikta z Nursie): \textit{cisterciáci} (kolonizovali a kultivovali půdu, kult Panny Marie, podle města Citeaux ve Francii) a \textit{premonstráti} (kazatelé, podle města Prémonstré ve Francii)
    \item[$-$] školy: při farách a klášterech, ve městech tzv. \textit{partikulární} školy (učí se základy jako číst, psát, počítat), univerzity (studium sedmera svobodných umění: právnická, lékařská, teologická fakulta)
    \item[$-$] na učitele a profesory se vztahuje kanonické právo (sice netuším, co to znamená, ale)
    \item[$-$] nejstarší univerzity: Bologna, Padova, Sorbonna, Salamanca, Cambridge, Oxford, Praha, Krakov, Vídeň, Lipsko
\end{itemize}

\end{document}
