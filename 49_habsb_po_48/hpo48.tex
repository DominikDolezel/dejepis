\documentclass{article}
\usepackage{fullpage}
\usepackage[czech]{babel}
\usepackage{amsfonts}

\title{\vspace{-2cm}Rakouské císařství po roce 1848\vspace{-1.7cm}}
\date{}
\author{}

\begin{document}
\maketitle

\begin{itemize}
    \vspace{-0.5em}
    \setlength\itemsep{0.15em}
    \item[$-$] revoluce přinesla změnu císaře a zrušení poddanství, všichni si mají být rovni
    \item[4. 3. 1849] revoluci ukončuje oktrojovaná \textit{Stadionova ústava}, která má respektovat zrušení poddanství a občanská práva
    \item[31. 12. 1851] je však k ničemu, protože během krátké chvíle je \textit{Silvestrovskými patenty} obnoven absolutismus $\rightarrow$ \textit{bachovský absolutismus}
    \item[$-$] národnostní útlak, zejména těch, kteří se účastnili revoluce
\end{itemize}

\subsection*{Domácí politika}
\begin{itemize}
    \vspace{-0.5em}
    \setlength\itemsep{0.15em}
    \item[$-$] občané vyloučeni ze správy státu
    \item[$-$] dokončení průmyslové revoluce
    \item[$-$] \textit{Živnostenský řád} umožňuje svobodu podnikání
    \item[$-$] vznik nových bank, burz, textilní, zemědělský, potravinářský průmysl
    \item[1873] hospodářská krize, kterou přežili jen ti silnější
    \item[$-$] zrovnoprávnění dosud jen tolerovaných náboženství (\textit{Tolerančním patentem}): lutheránství, kalvinismus a pravoslaví
\end{itemize}

\subsection*{Zahraniční politika}
\begin{itemize}
    \vspace{-0.5em}
    \setlength\itemsep{0.15em}
    \item[(1853-1856)] \textsc{krymská válka} otevřeně nevystupuje, zachovává neutralitu, dostává se do mezinárodní isolace
    \item[1859] \textsc{bitva u Solferina a Magenty}, kdy se Italové snaží vyhnat Habsburky z jejich území, Rakousko je poraženo, odvoláni významní politici
    \item[20.10.1860] tím končí absolutismus, Bach je odvolán, František vydává \textit{Říjnový diplom}, slibuje ústavnost a liberalismus, zřízení parlamentu
    \item[26.2.1861] vydává \textit{Únorovou} = \textit{Schmerlingovu} ústavu, konstituční monarchie, zakládá říšskou radu o dvou komorách (horní -- panská, členy jmenuje císař, dolní -- poslanecká sněmovna, do které se poslanci volí skrze zemské sněmy), kurijní volební systém (volí se v rámci čtyř kurijí, později přibyla ještě pátá), pro muže všeobecné volební právo, ale ne rovné (zezačátku volili jen ti majetní)
    \item[3.7.1866] \textsc{bitva u Sadové}, konec \textsc{prusko-rakouské války}, na straně Prusů je Itálie, Italové od poraženého rakouská získávají Benátsko
    \item[$-$] díky těmto fiaskům musí Rakušáci vyslyšet požadavky Maďarů $\rightarrow$ přistupují na \textit{dualismus}, tzv. Rakousko-Uherské vyrovnání
    \item[17.2.1867] G. Ardassy uherským ministerským předsedou, vytváří si vlastní ústavu, vzniká personální unie Rakousko-Uhersko (Předlitavsko a Zalitavsko)
    \item[$-$] František Josef I. korunován na uherského krále
\end{itemize}

\subsection*{Rakousko-Uhersko}
\begin{itemize}
    \vspace{-0.5em}
    \setlength\itemsep{0.15em}
    \item[$-$] císař František Josef I., společná tři ministerstva (financí, zahraničí, války), delegace jsou společný zákonodárný orgán, kde jsou zastoupeni jak Rakušáci tak Uhři
\end{itemize}

\section*{50.-60. léta 19. století}
\begin{itemize}
    \vspace{-0.5em}
    \setlength\itemsep{0.15em}
    \item[$-$] pronásledováni: František Palacký (stáhl se z politiky), Karel Havlíček Borovský (skončil v Brixenu), Josef Kajetán Tyl (vyhnán z Prahy, musel žít jako kočovný herec na venkově)
    \item[1863] Češi se přidali k politice \textit{pasivní resistence}, to jest protestování po Vídeňském centralismu, protest spočíval v tom, že nedocházeli do Říšské rady
    \item[$-$] i přes politickou censuru vzniká řada uměleckých děl: Babička, Kytice
    \item[1861] \textit{Únorová ústava} říká, že jsme konstituční monarchií, to jsou momenty, kdy začíná ožívat politika a společenský život v českých zemích
    \item[$-$] malíři Josef Mánes (výzdoba Staré radnice), Karel Purkyně, Jaroslav Čermák, Max Švabinský, Josef Hlávka (podporoval studenty)
    \item[$-$] spolek Hlahol (pěvecký): Smetana, Sokol (tělovýchovný): Tyrš, Fügner, Prozatímní divadlo (odpolední česká představení), Umělecká beseda (sdružuje umělce nejrůznějších oborů), Spolek českých lékařů, májovci
\end{itemize}

\subsection*{Dualismus}
\begin{itemize}
    \vspace{-0.5em}
    \setlength\itemsep{0.15em}
    \item[1866] porážka Rakušanů v \textsc{bitvě u Sadové}, došlo k narovnání s Maďary, ti si vypracovali svoji ústavu
    \item[$-$] tři stejná ministerstva, různé vlády
    \item[$-$] \textit{Prosincová ústava}, ta byla vypracována pro Předlitavsko
    \item[$-$] Čechům se to nelíbí, začínají protestovat
    \item[1867] delegace na národopisnou výstavu do Moskvy, přijetí u cara Alexandra II.
    \item[1867] při přesunu českých korunovačních klenot z Vídně do Prahy demonstrace
    \item[16.5.1868] položen základní kámen Národního divadla
    \item[1868] tábory lidu Čechů na důležitých historických místech, odmítají platit daně  
\end{itemize}





\end{document}
