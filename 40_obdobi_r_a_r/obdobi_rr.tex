\documentclass{article}
\usepackage{fullpage}
\usepackage[czech]{babel}
\usepackage{amsfonts}

\title{\vspace{-2cm}Období restaurace a rovnováhy\vspace{-1.7cm}}
\date{}
\author{}

\begin{document}
\maketitle

\subsection*{Znaky}
\begin{itemize}
    \vspace{-0.5em}
    \setlength\itemsep{0.15em}
    \item[$-$] obnova původních monarchií
    \item[$-$] národní a národně osvobozenecké hnutí, vymanění z nadvlády Turků, nezávislost portugalských a španělských kolonií
    \item[$-$] průmyslová revoluce (už od 18. st. ve Velké Británii)
    \item[$-$] modernizace státní správy
    \item[$-$] snaha prosadit občanská práva (Rusko), proti absolutismu
\end{itemize}

\subsection*{Německý spolek}
\begin{itemize}
    \vspace{-0.5em}
    \setlength\itemsep{0.15em}
    \item[$-$] bývalá SŘŘ, Napoleon z něj udělal Rýsnký spolek, potom Německý spolek
    \item[$-$] už za napoleonských válek vzniká Německé národní hnutí, cíl: vytvořit jednotné Německo
    \item[$-$] účastní se jich studenti a profesoři univerzit
    \item[1817] slavnost ve Wartburgu k výročí vystoupení Martina Luthera (1517)
    \item[1819] Metternichovi se to vůbec nelíbí, pozval si do Karlových Varů zástupce německých států, vzniklo \textit{Karlovarské usnesení}, vlastenecké spolky zrušeny, díky tomu mají německé univerzity omezeny akademické svobody, dány pod dozor
    \item[30. léta 19. st.] první pokus o sjednocení Německa neúspěšný, takže se znovu snaží scházet ve spolku Mladé Německo
\end{itemize}

\subsection*{Apenincký poloostrov}
\begin{itemize}
    \vspace{-0.5em}
    \setlength\itemsep{0.15em}
    \item[$-$] celý rozdrobený, sever drží Habsburkové, střed je papežský stát, na jihu vládnou Bourboni (království Obojí Sicílie)
    \item[$-$] vzniká hnutí \textit{risorgimento} s cílem sjednotit Apeninský poloostrov, v jejich čele jsou \textit{karbonáři} (aktivisté, kteří se scházeli mezi výrobci uhlí, v čele SIlvio Pellico)
    \item[1820] \textsc{povstání v Neapolsku} (část království Obojí Sicílie), král přijal ústavu
    \item[1820] \textsc{povstání v Sardinském království}
    \item[1821] Habsburkové okamžitě posílají intervenční armádu, povstání jsou potalčena, účastníci pochytáni a vězněni
    \item[$-$] vzniká spolek Mladá Itálie s cílem sjednoti itálii a vyhnat cizí dynastie
\end{itemize}

\subsection*{Španělsko}
\begin{itemize}
    \vspace{-0.5em}
    \setlength\itemsep{0.15em}
    \item[$-$] už za napoleonských válek boje španělských kolonií za nezávislost
    \item[$-$] potom obrovská finanční krize
    \item[$-$] pokračování v absolutistickém způsobu vlády
    \item[1819] prodej Floridy za 5 milionů dolarů
    \item[1820] v čele revolucionářů liberální důstojníci Rafael del Riego
    \item[$-$] vypracována ústava, král zajat, liberální vláda, volby do \textit{kortesu} (parrlamentu)
    \item[1823] francouzská intervenční armáda potlačuje zárodky povstání
\end{itemize}

\subsection*{Rusko}



\end{document}
