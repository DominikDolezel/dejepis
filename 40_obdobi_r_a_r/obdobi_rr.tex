\documentclass{article}
\usepackage{fullpage}
\usepackage[czech]{babel}
\usepackage{amsfonts}

\title{\vspace{-2cm}Období restaurace a rovnováhy\vspace{-1.7cm}}
\date{}
\author{}

\begin{document}
\maketitle

\subsection*{Znaky}
\begin{itemize}
    \vspace{-0.5em}
    \setlength\itemsep{0.15em}
    \item[$-$] obnova původních monarchií
    \item[$-$] národní a národně osvobozenecké hnutí, vymanění z nadvlády Turků, nezávislost portugalských a španělských kolonií
    \item[$-$] průmyslová revoluce (už od 18. st. ve Velké Británii)
    \item[$-$] modernizace státní správy
    \item[$-$] snaha prosadit občanská práva (Rusko), proti absolutismu
\end{itemize}

\subsection*{Německý spolek}
\begin{itemize}
    \vspace{-0.5em}
    \setlength\itemsep{0.15em}
    \item[$-$] bývalá SŘŘ, Napoleon z něj udělal Rýsnký spolek, potom Německý spolek
    \item[$-$] už za napoleonských válek vzniká Německé národní hnutí, cíl: vytvořit jednotné Německo
    \item[$-$] účastní se jich studenti a profesoři univerzit
    \item[1817] slavnost ve Wartburgu k výročí vystoupení Martina Luthera (1517)
    \item[1819] Metternichovi se to vůbec nelíbí, pozval si do Karlových Varů zástupce německých států, vzniklo \textit{Karlovarské usnesení}, vlastenecké spolky zrušeny, díky tomu mají německé univerzity omezeny akademické svobody, dány pod dozor
    \item[30. léta 19. st.] první pokus o sjednocení Německa neúspěšný, takže se znovu snaží scházet ve spolku Mladé Německo
\end{itemize}

\subsection*{Apeninský poloostrov}
\begin{itemize}
    \vspace{-0.5em}
    \setlength\itemsep{0.15em}
    \item[$-$] celý rozdrobený, sever drží Habsburkové, střed je papežský stát, na jihu vládnou Bourboni (království Obojí Sicílie)
    \item[$-$] vzniká hnutí \textit{risorgimento} s cílem sjednotit Apeninský poloostrov, v jejich čele jsou \textit{karbonáři} (aktivisté, kteří se scházeli mezi výrobci uhlí, v čele SIlvio Pellico)
    \item[1820] \textsc{povstání v Neapolsku} (část království Obojí Sicílie), král přijal ústavu
    \item[1820] \textsc{povstání v Sardinském království}
    \item[1821] Habsburkové okamžitě posílají intervenční armádu, povstání jsou potalčena, účastníci pochytáni a vězněni
    \item[$-$] vzniká spolek Mladá Itálie s cílem sjednoti itálii a vyhnat cizí dynastie
\end{itemize}

\subsection*{Španělsko}
\begin{itemize}
    \vspace{-0.5em}
    \setlength\itemsep{0.15em}
    \item[$-$] už za napoleonských válek boje španělských kolonií za nezávislost
    \item[$-$] potom obrovská finanční krize
    \item[$-$] pokračování v absolutistickém způsobu vlády
    \item[1819] prodej Floridy za 5 milionů dolarů
    \item[1820] v čele revolucionářů liberální důstojníci Rafael del Riego
    \item[$-$] vypracována ústava, král zajat, liberální vláda, volby do \textit{kortesu} (parlamentu)
    \item[1823] francouzská intervenční armáda potlačuje zárodky povstání
\end{itemize}

\subsection*{Rusko}
\begin{itemize}
    \vspace{-0.5em}
    \setlength\itemsep{0.15em}
    \item[1825] Alexandr I. zemřel, nový car \textbf{Mikuláš I.}
    \item[$-$] nové důstojnické požadavky: odstranění absolutismu (\textit{samoděržaví}), ústava; inspirace ve Francii
    \item[14.12.1825] odmítli přísahat carovi věrnost $\rightarrow$ hovoříme o \textsc{povstání děkabristů}, podpora však nebyla dostatečně široká $\rightarrow$ potrestáni
\end{itemize}


\subsection*{Balkánský poloostrov}
\subsubsection*{Srbsko}$-$
\begin{itemize}
    \vspace{-0.5em}
    \setlength\itemsep{0.15em}
    \item[$-$] Srbsko součástí Osmanské říše
    \item[1815-\textbf{1817}] \textsc{druhé srbské povstání}, v čele Miloš Obrenovič, úspěšné, získalo autonomii
    \item[(1829)] \textit{Drinopolský mír}, Osmané přiznávají srbskou autonomii
\end{itemize}

\subsubsection*{Řecko}
\begin{itemize}
    \vspace{-0.5em}
    \setlength\itemsep{0.15em}
    \item[1821] \textsc{povstání} za nezávislost
    \item[$-$] velmoci na straně Řeků (Rusko, VB, Francie), ale Metternich proti
    \item[$-$] \textbf{Alexander} d \textbf{Demeterius Ypsilanti}, vůdce řeckého povstání, byli vězněni v Osvětimi
    \item[$-$] \textit{Drinopolský mír}, kterým Osmané uznávají nezávislost Řecka, stává se z něj monarchie
    \item[1828] první řecký král Otto I.
    \item[1830] získávají autonomii i Moldavské a Valašské knížectví, zárodky budoucího Rumunska
    \item[$-$] kolaps Osmanské říše
\end{itemize}

\subsection*{Latinská Amerika}
\begin{itemize}
    \vspace{-0.5em}
    \setlength\itemsep{0.15em}
    \item[$-$] kolonie mají Španělé a Portugalci, začínají propukat boje za autonomii, využívají obsazení Španělska Napoleonem
    \item[$-$] první fáze povstání za napoleonských válek, vůdce \textbf{Simon Bolívar}
    \item[$-$] druhá fázel, úspěšná, koloniemi zůstaly jen Portoriko a Kuba
    \item[$-$] Portugalci ztrácí nadvládu nad Brazílií
\end{itemize}

\subsection*{Francie}
\begin{itemize}
    \vspace{-0.5em}
    \setlength\itemsep{0.15em}
    \item[1814] Ludvík XVIII., restaurace Bourbonů, konstituční monarchie
    \item[1824] do čela Francie bratr Ludvíka, Karel X., konzervativní, opírá se o monarchisty, absolutismus
    \item[25.-26.7.1830] \textit{Ordonance ze Saint-Cloud}: omezení politických práv, cenzura,  rozpuštění sněmovny $\rightarrow$ nespokojenost
    \item[26.7.1830] \textsc{červencová revoluce}, k moci se dostávají liberálové, sepisují novou ústavu, Karel utíká do Anglie
    \item[$-$] novým králem zvolený Ludvík Filip, Bourbon z orelánské větvě, toto období se označuje jako červencová monarchie, trvá až do roku 1848, dokud nevznikne republika, ekonomická prosperita
\end{itemize}

\subsection*{Belgie}
\begin{itemize}
    \vspace{-0.5em}
    \setlength\itemsep{0.15em}
    \item[$-$] Vídeňským kongresem spojen do Nizozemského království
    \item[$-$] prosazuje centralismus (vláda z Nizozemí), kalvinismus, nizozemština na úkor francouzštiny
    \item[25.8.1830] červencová revoluce ve Francii inspirovala Belgičany, v Bruselu vypuklo \textsc{povstání} s cílem odtrhnout se od Nizozemí, úspěch, Belgie se stává samostatným královstvím, prvním králem Leopold I. Belgický, jedna z nejliberálnějších ústav, společně s Anglií útočištěm politických emigrantů
\end{itemize}

\subsection*{Polsko}
\begin{itemize}
    \vspace{-0.5em}
    \setlength\itemsep{0.15em}
    \item[$-$] součást Ruska, tzv. \uv{Kongresovka}, díky Vídeňskému kongresu je připojena k Rusku personální unií
    \item[$-$] nadvláda Rusů se jim nelíbí
    \item[1830] \textsc{povstání} s cílem vybojování nezávislosti na Rusku, centrem je Varšava, vyhánějí ruské úředníky z Varšavy, svolali Sejm, sesadili Mikuláše I., šel jim pomáhat Karel Hynek Mácha, dopadlo velice špatně
    \item[1831] Poláci poraženi, dobyta Varšava, zrušena univerzita, zrušena autonomie, rozpuštěno polské vojsko, násilná rusifikace
    \item[1863] další neúspěšné povstání
\end{itemize}


\subsection*{Velká Británie}
\begin{itemize}
    \vspace{-0.5em}
    \setlength\itemsep{0.15em}
    \item[$-$] od roku 1714 vládne Hannoverská dynastie
    \item[1838] za vlády Viktorie I. rozšíření tzv. \textit{Hnutí chartistů}, tajné volby bez majetkového censu
    \item[1839] parlament zamítá
    \item[$-$] Irové chtějí zrovnoprávnění katolíků s protestanty, odtrhnutí od Velké Británie chce hnutí Mladé Irsko, 1848 neúspěšná revoluce, ve 40. letech plíseň brambor, hladomor, emigrace
\end{itemize}



\end{document}
