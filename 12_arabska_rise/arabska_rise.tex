\documentclass{article}
\usepackage{fullpage}
\usepackage[czech]{babel}
\usepackage{amsfonts}

\title{\vspace{-2cm}Arabská říše\vspace{-1.7cm}}
\date{}
\author{}

\begin{document}
\maketitle

\begin{itemize}
    \vspace{-0.5em}
    \setlength\itemsep{0.15em}
    \item[$-$] Semité, na Arabském poloostrově
    \item[$-$] \textit{beduín} = člen kmenu, \textit{šajcha} = vůdce kmenu
    \item[$-$] Velká karavanní obchodní cesta (mezi Orientem a Čínou)
    \item[$-$] postupně vznik aristokracie a šlechty
    \item[$-$] \textbf{Mekka} = náboženská i obchodní křižovatka
    \item[6. st.] začínají ohrožovat Etiopané a Íránci $\rightarrow$ sjednotitel \textbf{Mohamed} z rodiny Kurajšovců
\end{itemize}

\section*{Mohamed (570 -- 632)}
\begin{itemize}
    \vspace{-0.5em}
    \setlength\itemsep{0.15em}
    \item[$-$] oženil se s vdovou Chadidže
    \item[(613)] začíná hlásat islám, předává učení Alláha
    \item[$-$] zdroje islámu: judaismus, křesťanství, jednotlivá kmenová polyteistická náboženství
    \item[$-$] \textit{súry} = Mohamedova zjevení, seskládány do posvátného textu = \textit{Korán}
    \item[$-$] \textit{Sunna} = posvátný text o Mohamedových činech
    \item[$-$] 5 pilířů islámu: pouť do Mekky, v měsíci ramadánu půst, modlit se pětkrát denně, víra v jediného boha, dávat almužnu chudým
    \item[$-$] \textit{džihád} = povinnost šířit islám mečem a ohněm
    \item[622] \textit{hidžra} = \uv{vystěhování}, odešel od Mediny, kde založil první muslimskou obec, počátek muslimského kalendáře
    \item[630] \textsc{dobytí Mekky} Mohamedem
    \item[632] smrt
\end{itemize}

\section*{Chalífát}
\begin{itemize}
    \vspace{-0.5em}
    \setlength\itemsep{0.15em}
    \item[$-$] po Mohamedově smrti jsou jeho (\textit{chalífové}) nástupci volení
    \item[$-$] bujná expanze
    \item[$-$] teokratický stát: spojuje je náboženství, vláda je propojena s vírou
    \item[$-$] otázka nástupnictví: rozdělení na \textit{šíity} (uznávají jako panovníky jen Mohamedovy přímé potomky) a \textit{sunnity} (stačí většinová shoda) 
\end{itemize}



\end{document}
