\documentclass{article}
\usepackage{fullpage}
\usepackage[czech]{babel}
\usepackage{amsfonts}

\title{\vspace{-2cm}Arabská říše\vspace{-1.7cm}}
\date{}
\author{}

\begin{document}
\maketitle

\begin{itemize}
    \vspace{-0.5em}
    \setlength\itemsep{0.15em}
    \item[$-$] Semité, na Arabském poloostrově
    \item[$-$] \textit{beduín} = člen kmenu, \textit{šajcha} = vůdce kmenu
    \item[$-$] Velká karavanní obchodní cesta (mezi Orientem a Čínou)
    \item[$-$] postupně vznik aristokracie a šlechty
    \item[$-$] \textbf{Mekka} = náboženská i obchodní křižovatka
    \item[6. st.] začínají ohrožovat Etiopané a Íránci $\rightarrow$ sjednotitel \textbf{Mohamed} z rodiny Kurajšovců
\end{itemize}

\section*{Mohamed (570 -- 632)}
\begin{itemize}
    \vspace{-0.5em}
    \setlength\itemsep{0.15em}
    \item[$-$] oženil se s vdovou Chadidže
    \item[(613)] začíná hlásat islám, předává učení \textbf{Alláha}, považoval se za posledního proroka
    \item[$-$] zdroje islámu: judaismus, křesťanství, jednotlivá kmenová polyteistická náboženství
    \item[$-$] \textit{súry} = Mohamedova zjevení, seskládány do posvátného textu = \textit{Korán}
    \item[$-$] \textit{Sunna} = posvátný text o Mohamedových činech
    \item[$-$] 5 pilířů islámu: pouť do Mekky, v měsíci ramadánu půst, modlit se pětkrát denně, víra v jediného boha, dávat almužnu chudým
    \item[$-$] \textit{džihád} = povinnost šířit islám mečem a ohněm
    \item[622] \textit{hidžra} = \uv{vystěhování}, odešel od Mediny, kde založil první muslimskou obec, počátek muslimského kalendáře
    \item[630] \textsc{dobytí Mekky} Mohamedem
    \item[632] smrt
\end{itemize}

\section*{Chalífát}
\begin{itemize}
    \vspace{-0.5em}
    \setlength\itemsep{0.15em}
    \item[$-$] po Mohamedově smrti jsou jeho (\textit{chalífové}) nástupci volení
    \item[$-$] bujná expanze
    \item[$-$] teokratický stát: spojuje je náboženství, vláda je propojena s vírou
    \item[$-$] otázka nástupnictví: rozdělení na \textit{šíity} (uznávají jako panovníky jen Mohamedovy přímé potomky) a \textit{sunnity} (stačí většinová shoda, dnes víc)
\end{itemize}

\section*{Umajjovci (661 -- 749)}
\begin{itemize}
    \vspace{-0.5em}
    \setlength\itemsep{0.15em}
    \item[$-$] nové hlavní město \textbf{Damašek}
    \item[$-$] expanze na sever do Arménie, Byzanc, Kypr, do Z Indie
    \item[711] na Pyrenejském poloostrově, kde vybudovali pevnost Tárikova skála
    \item[$-$] nejvýznamější centrum na Pyr. pol.: \textbf{Cordoba}
    \item[732] \textsc{v bitvě u Poitiers} a \textsc{u Tours} Karel Martell zastavil postup Arabů
\end{itemize}


\section*{Abbásovci (749 -- 1258)}
\begin{itemize}
    \vspace{-0.5em}
    \setlength\itemsep{0.15em}
    \item[$-$] \textit{vezír} = zástupce chalífy
    \item[$-$] území se postupně rozpadá na jednotlivé \textit{emiráty}
    \item[$-$] nové hlavní město \textbf{Bagdád}
    \item[$-$] \textbf{Hárún ar-Rašíd}, jeden z nejvýznamnějších chalífů, současník Karla Velikého, podpra vědy a kultury
    \item[$-$] \textit{iktá} = obdoba léna, ale nikdy dědičné
    \item[1071] \textsc{bitva u Mantzikertu, vpád Seldžuků} $\rightarrow$ zánik Arabské říše
    \item[$-$] \textit{reconquista} = \uv{znovudobytí}: od 11. století evropští křesťané vytlačují Araby
    \item[1492] v Evropě definitivně poraženi v \textsc{bitvě u Granady}
\end{itemize}


\section*{Kultura}
\begin{itemize}
    \vspace{-0.5em}
    \setlength\itemsep{0.15em}
    \item[$-$] centra Bagdád, Damašek, Alexandrie, Cordoba
    \item[$-$] přinesli nám poznatky Aristotela
    \item[$-$] geografie: Ibn Fadlán, \textbf{Ibrahím ibn Jákúb} (10. st., zastavil se v Praze, kterou pozitivně hodnotí), Ibn Ibrisi
    \item[$-$] astronomie, nástroje na pozorování hvězd
    \item[$-$] matematika, arabské číslice
    \item[$-$] fyzika, optické přístroje, léky
    \item[$-$] medicína, oční lékařství, brýle
    \item[$-$] broskve, růže, lilie
    \item[$-$] filosofie, Ibn Síná (Avicenna), Ibn Rušd (Averroes)
    \item[$-$] Pohádky tisíce a jedné noci
    \item[$-$] řemeslo: textilnictví, tolecké dýky, damastencké meče, sklo, porcelán, voňavky, papír
    \item[$-$] mešita, minaret, muezzín (kdo svolává na mši), oslí hřbet = typ oblouku
    \item[$-$] paláce v Cordobě a Granadě
    \item[$-$] mozaiky, arabesky = nepřítomnost figur na kresbách, afigurálnost, kaligrafie = krasopis
\end{itemize}




\end{document}
