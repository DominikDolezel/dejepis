\documentclass{article}
\usepackage{fullpage}
\usepackage[czech]{babel}
\usepackage{amsfonts}

\title{\vspace{-2cm}Vídeňský kongres, období restaurace a rovnováhy,\\ revoluce v letech 1848 a 1849\vspace{-1.7cm}}
\date{}
\author{}

\begin{document}
\maketitle


\section*{Vídeňský kongres 1814-1815}
\begin{itemize}
    \vspace{-0.5em}
    \setlength\itemsep{0.15em}
    \item[$-$] cíl: uspořádat Evropu po napoleonských válkách
    \item[$-$] \textit{tančící kongres}, přes 100000 účastníků, 700 diplomatů, zastoupení všechn evropských zemí
    \item[$-$] hlavní velmoci: Rusko (car Alexandr I.), Prusko (král Friedrich Vilém III.), Rakousko (Franntišek I., kancléř Metternich) a Británie (vévoda Wellington), nicméně určitý hlas má i Francie (ministr zahraničí de Talleyrand)
    \begin{enumerate}
        \vspace{-0.5em}
        \setlength\itemsep{0.15em}
        \item restaurace zaniklých dynastií
        \item systém rovnováhy sil v Evropě, žádný stát nemá převahu
        \item státy spolu budou spolupracovat
        \item územní změny na mapě Evropy:
        \begin{itemize}
            \vspace{-0.5em}
            \setlength\itemsep{0.15em}
            \item[$-$] Prusko (největší německy mluvící země) i Rusko (velkovévodství Varšavské, Finsko) získává značná území
            \item[$-$] Rakousko získává území v severní části Apeninského poloostrova
            \item[$-$] Francie: vrací se k hranicím před rokem 1792
            \item[$-$] Velká Británie získává přístup k důležitým strateigkým bodům (Cejlon, Mys Dobré naděje)
            \item[$-$] spojení Švédska a Norska do personální unie (bude trvat až do roku 1905), Švédským králem maršál Bernadotte
            \item[$-$] Nizozemské království: spojení severního a jižního Nizozemí
            \item[$-$] z Rýnského spolku vznikl Německý spolek -- 39 států, zástupci těchto států se schází na sněmu ve Frankfurtu nad Mohanem
            \item[$-$] Itálie zůstává rozdrobená
            \item[$-$] Švýcarsko: konec sesterské Napoleonovy repubnliky, obnovena konfederace 21 kantonů, kongres garantuje neutralitu
        \end{itemize}
        \item \textit{kvietismus} = snaha o neměnnost těchto domluvených výsledků
        \item svatá aliance: spolupráce Ruska, Rakouska a Pruska vnášet do politiky různé křesťanské principy
    \end{enumerate}
\end{itemize}

\section*{Období restaurace a rovnováhy}

\subsection*{Znaky}
\begin{itemize}
    \vspace{-0.5em}
    \setlength\itemsep{0.15em}
    \item[$-$] obnova původních monarchií
    \item[$-$] národní a národně osvobozenecké hnutí, vymanění z nadvlády Turků, nezávislost portugalských a španělských kolonií
    \item[$-$] průmyslová revoluce (už od 18. st. ve Velké Británii)
    \item[$-$] modernizace státní správy
    \item[$-$] snaha prosadit občanská práva (Rusko), proti absolutismu
\end{itemize}

\subsection*{Německý spolek}
\begin{itemize}
    \vspace{-0.5em}
    \setlength\itemsep{0.15em}
    \item[$-$] bývalá SŘŘ, Napoleon z něj udělal Rýsnký spolek, potom Německý spolek
    \item[$-$] už za napoleonských válek vzniká Německé národní hnutí, cíl: vytvořit jednotné Německo
    \item[$-$] účastní se jich studenti a profesoři univerzit
    \item[1817] slavnost ve Wartburgu k výročí vystoupení Martina Luthera (1517)
    \item[1819] Metternichovi se to vůbec nelíbí, pozval si do Karlových Varů zástupce německých států, vzniklo \textit{Karlovarské usnesení}, vlastenecké spolky zrušeny, díky tomu mají německé univerzity omezeny akademické svobody, dány pod dozor
    \item[30. léta 19. st.] první pokus o sjednocení Německa neúspěšný, takže se znovu snaží scházet ve spolku Mladé Německo
\end{itemize}

\subsection*{Apeninský poloostrov}
\begin{itemize}
    \vspace{-0.5em}
    \setlength\itemsep{0.15em}
    \item[$-$] celý rozdrobený, sever drží Habsburkové, střed je papežský stát, na jihu vládnou Bourboni (království Obojí Sicílie)
    \item[$-$] vzniká hnutí \textit{risorgimento} s cílem sjednotit Apeninský poloostrov, v jejich čele jsou \textit{karbonáři} (aktivisté, kteří se scházeli mezi výrobci uhlí, v čele SIlvio Pellico)
    \item[1820] \textsc{povstání v Neapolsku} (část království Obojí Sicílie), král přijal ústavu
    \item[1820] \textsc{povstání v Sardinském království}
    \item[1821] Habsburkové okamžitě posílají intervenční armádu, povstání jsou potalčena, účastníci pochytáni a vězněni
    \item[$-$] vzniká spolek Mladá Itálie s cílem sjednoti itálii a vyhnat cizí dynastie
\end{itemize}

\subsection*{Španělsko}
\begin{itemize}
    \vspace{-0.5em}
    \setlength\itemsep{0.15em}
    \item[$-$] už za napoleonských válek boje španělských kolonií za nezávislost
    \item[$-$] potom obrovská finanční krize
    \item[$-$] pokračování v absolutistickém způsobu vlády
    \item[1819] prodej Floridy za 5 milionů dolarů
    \item[1820] v čele revolucionářů liberální důstojníci Rafael del Riego
    \item[$-$] vypracována ústava, král zajat, liberální vláda, volby do \textit{kortesu} (parlamentu)
    \item[1823] francouzská intervenční armáda potlačuje zárodky povstání
\end{itemize}

\subsection*{Rusko}
\begin{itemize}
    \vspace{-0.5em}
    \setlength\itemsep{0.15em}
    \item[1825] Alexandr I. zemřel, nový car \textbf{Mikuláš I.}
    \item[$-$] nové důstojnické požadavky: odstranění absolutismu (\textit{samoděržaví}), ústava; inspirace ve Francii
    \item[14.12.1825] odmítli přísahat carovi věrnost $\rightarrow$ hovoříme o \textsc{povstání děkabristů}, podpora však nebyla dostatečně široká $\rightarrow$ potrestáni
\end{itemize}


\subsection*{Balkánský poloostrov}
\subsubsection*{Srbsko}
\begin{itemize}
    \vspace{-0.5em}
    \setlength\itemsep{0.15em}
    \item[$-$] Srbsko součástí Osmanské říše
    \item[1815-\textbf{1817}] \textsc{druhé srbské povstání}, v čele Miloš Obrenovič, úspěšné, získalo autonomii
    \item[(1829)] \textit{Drinopolský mír}, Osmané přiznávají srbskou autonomii
\end{itemize}

\subsubsection*{Řecko}
\begin{itemize}
    \vspace{-0.5em}
    \setlength\itemsep{0.15em}
    \item[1821] \textsc{povstání} za nezávislost
    \item[$-$] velmoci na straně Řeků (Rusko, VB, Francie), ale Metternich proti
    \item[$-$] \textbf{Alexander} d \textbf{Demeterius Ypsilanti}, vůdce řeckého povstání, byli vězněni v Osvětimi
    \item[$-$] \textit{Drinopolský mír}, kterým Osmané uznávají nezávislost Řecka, stává se z něj monarchie
    \item[1828] první řecký král Otto I.
    \item[1830] získávají autonomii i Moldavské a Valašské knížectví, zárodky budoucího Rumunska
    \item[$-$] kolaps Osmanské říše
\end{itemize}

\subsection*{Latinská Amerika}
\begin{itemize}
    \vspace{-0.5em}
    \setlength\itemsep{0.15em}
    \item[$-$] kolonie mají Španělé a Portugalci, začínají propukat boje za autonomii, využívají obsazení Španělska Napoleonem
    \item[$-$] první fáze povstání za napoleonských válek, vůdce \textbf{Simon Bolívar}
    \item[$-$] druhá fázel, úspěšná, koloniemi zůstaly jen Portoriko a Kuba
    \item[$-$] Portugalci ztrácí nadvládu nad Brazílií
\end{itemize}

\subsection*{Francie}
\begin{itemize}
    \vspace{-0.5em}
    \setlength\itemsep{0.15em}
    \item[1814] Ludvík XVIII., restaurace Bourbonů, konstituční monarchie
    \item[1824] do čela Francie bratr Ludvíka, Karel X., konzervativní, opírá se o monarchisty, absolutismus
    \item[25.-26.7.1830] \textit{Ordonance ze Saint-Cloud}: omezení politických práv, cenzura,  rozpuštění sněmovny $\rightarrow$ nespokojenost
    \item[26.7.1830] \textsc{červencová revoluce}, k moci se dostávají liberálové, sepisují novou ústavu, Karel utíká do Anglie
    \item[$-$] novým králem zvolený Ludvík Filip, Bourbon z orelánské větvě, toto období se označuje jako červencová monarchie, trvá až do roku 1848, dokud nevznikne republika, ekonomická prosperita
\end{itemize}

\subsection*{Belgie}
\begin{itemize}
    \vspace{-0.5em}
    \setlength\itemsep{0.15em}
    \item[$-$] Vídeňským kongresem spojen do Nizozemského království
    \item[$-$] prosazuje centralismus (vláda z Nizozemí), kalvinismus, nizozemština na úkor francouzštiny
    \item[25.8.1830] červencová revoluce ve Francii inspirovala Belgičany, v Bruselu vypuklo \textsc{povstání} s cílem odtrhnout se od Nizozemí, úspěch, Belgie se stává samostatným královstvím, prvním králem Leopold I. Belgický, jedna z nejliberálnějších ústav, společně s Anglií útočištěm politických emigrantů
\end{itemize}

\subsection*{Polsko}
\begin{itemize}
    \vspace{-0.5em}
    \setlength\itemsep{0.15em}
    \item[$-$] součást Ruska, tzv. \uv{Kongresovka}, díky Vídeňskému kongresu je připojena k Rusku personální unií
    \item[$-$] nadvláda Rusů se jim nelíbí
    \item[1830] \textsc{povstání} s cílem vybojování nezávislosti na Rusku, centrem je Varšava, vyhánějí ruské úředníky z Varšavy, svolali Sejm, sesadili Mikuláše I., šel jim pomáhat Karel Hynek Mácha, dopadlo velice špatně
    \item[1831] Poláci poraženi, dobyta Varšava, zrušena univerzita, zrušena autonomie, rozpuštěno polské vojsko, násilná rusifikace
    \item[1863] další neúspěšné povstání
\end{itemize}


\subsection*{Velká Británie}
\begin{itemize}
    \vspace{-0.5em}
    \setlength\itemsep{0.15em}
    \item[$-$] od roku 1714 vládne Hannoverská dynastie
    \item[1838] za vlády Viktorie I. rozšíření tzv. \textit{Hnutí chartistů}, tajné volby bez majetkového censu
    \item[1839] parlament zamítá
    \item[$-$] Irové chtějí zrovnoprávnění katolíků s protestanty, odtrhnutí od Velké Británie chce hnutí Mladé Irsko, 1848 neúspěšná revoluce, ve 40. letech plíseň brambor, hladomor, emigrace
\end{itemize}

\section*{Revoluce v letech 1848-1849}


\begin{itemize}
    \vspace{-0.5em}
    \setlength\itemsep{0.15em}
    \item[$-$] \uv{jaro národů}, navzájem se od sebe inspirují a probíhá několik revolucí
    \item[$-$] touhy po sjednocení, odstranění absolutismu či zrušení nevolnictví
\end{itemize}

\subsection*{Itálie}
\begin{itemize}
    \vspace{-0.5em}
    \setlength\itemsep{0.15em}
    \item[$-$] Apeninský poloostrov je rozdrobený, sever drží Habsburkové, uprostřed papežský stát a na jihu Království obojí Sicílie drženo Bourbony
    \item[$-$] Sardinské království: Sardinie a severozápadní část Pyrenejského poloostrova, hlavní cíl je vytvořit jednotný stát
    \item[$-$] \textit{risorgimento}: obnovení, vzkříšení
    \item[$-$] dva hlavní proudy:
    \begin{itemize}
        \vspace{-0.5em}
        \setlength\itemsep{0.15em}
        \item[$-$] \textit{liberálové}: stačilo by jim sjednocení do ústavní monarchie, budou dcí král má pocházet se sardisnkého království
        \item[$-$] \textit{revoluční demokraté}: Apeninský poloostrov se má sjeednoti to demokratické republiky (členové organisace Mladá Itálie \textbf{Giuseppe Mazzini}, \textbf{Giuseppe Garibaldi})
    \end{itemize}
    \item[leden 1848] \textsc{nepokoje na Sicílii a neapolsku} s cílem vyhnat Bourbony
    \item[bžezen 1848] \textsc{povstání} za cílem sesadit Habsburky
    \item[$-$] v čele snahy o sjednocení Sardinské království v čele se savojskou dynastií, král \textbf{Karel Albert} a šikovný politik hrabě \textbf{Camillo Benso di Cavour}
    \item[(1848)] \textsc{bitva u Custozzy}, prohrává Sardinské království proti Habsburkům (pod vedením maršála Radeckého)
    \item[(1849)] \textsc{bitva u Novarry}, taky prohrává Sardinské království
    \item[$\rightarrow$] pokusy o sjednocení neúspěšné, Karel Albert abdikuje a nastupuje jeho synovec \textbf{Viktor Emanulel II.}
    \item[$-$] radikálové se pokouší na různých místech Apeninského poloostrova vyhlásit republiku, ale neúspěšně, protože Habsburkové jsou v této době silní a na jejich straně je Francie, kde nyní vládne Napoleon III.
    \item[$\rightarrow$] všechny pokusy neúspěšné, Itálie nesjednocena
\end{itemize}

\subsection*{Francie}
\begin{itemize}
    \vspace{-0.5em}
    \setlength\itemsep{0.15em}
    \item[$-$] \textit{Červencová monarchie}, král \textbf{Ludvík Filip}
    \item[1848] zemědělská a obchodní krise
    \item[$-$] velkým problémem je vysoký majetkový census, o politice mohou rozhodovat jen ti nejbrohatší, kterých je zoufale málo, společnost se tedy radikalisuje
    \item[$-$] hlavním cílem revoluce je změna volebního práva a snížení majetkového censu
    \item[$-$] s tímto cílem se konají shromáždění, která jsou spojena s pohoštěním: \textit{bankety}
    \item[únor 1848] král jeden z banketů zakázal $\rightarrow$ začínají demonstrace, staví se barikády, boje na barikádách, během těchto bojů stávající premiér podává demisi
    \item[25.2.1848] \textsc{vyhlášena druhá republika}, král prchá do Londýna, definitivně končí vláda dynastie Bourbonů
    \item[$-$] vznik prozatimní vlády, ta byla nesmírně pestrá, složení ze všech politických proudů, zrušení censury, právo na práci (národní dílny, kde mohou pracovat nezaměstnaní), zrušení otroctví v koloniích, zavedeno všecobecné volební právo
    \item[květen 1848] \textsc{volby do Národního shromáždění}, nedostala se však levice, začínají prosazovat změny, zrušení národních dílen $\rightarrow$ povstání a boje v Paříži = \textit{červnové povstání}
    \item[listopad 1848] nová ústava, kde počítá s posicí presidenta, který má značené pravomoce, volební období 4 roky, hledání vhodného kandidáta
    \item[$-$] prvním presidentem \textbf{Ludvík Bonaparte}, Napoleonův synovec, během napoleonských válek nizozemským králem
    \item[prosinec 1851] posice presidenta prodloužena ze čtyř let na deset
    \item[2.12.1852] pomocí armády na sebe strhává moc a stává se císařem
    \item[$\rightarrow$] konec druhé republiky, začátek druhého císařství, ze začátko vojensko-policejní diktatura, ale postupně se začíná rozvolňovat až do stavu jakési parlamentní monarchie = \textit{bonapartismus}
    \item[1870] \textsc{prusko-francouzská válka}, Francie poražena, Bonaparte zajat a sesazen, vyhlášena třetí republika, ta bude trvat až do roku 1940
\end{itemize}

\subsection*{Německo}
\begin{itemize}
    \vspace{-0.5em}
    \setlength\itemsep{0.15em}
    \item[$-$] Německý spolek: jednotlivé země, Pruska a Rakouského císařství ty části, které dříve patřily do SŘŘ, chtějí se sjednotit
    \item[březen 1848] \textsc{povstání v Berlíně} (hlavní město Pruského království), pruský král donucen k ústupkům: změna ústavy, reformy, nová liberální vláda, tato vláda vyhlásila volby do celoněmeckého parlamentu v Heidelbergu, vytvořen celoněmecká prozatímní parlament, ten řeší, jakým způsobem se sjednotit
    \item[$-$] dva proudy
    \begin{itemize}
        \vspace{-0.5em}
        \setlength\itemsep{0.15em}
        \item[$-$] \textit{radikální}: chtějí sjednotit do republiky
        \item[$-$] \textit{umírnění}: stačí jim monarchie

    \end{itemize}
    \item[$-$] dvě koncepce:
    \begin{itemize}
        \vspace{-0.5em}
        \setlength\itemsep{0.15em}
        \item[$-$] \textit{koncepce maloněmecká}: sjednocení jen těchto států bez Habsburské monarchie, části Polska a pod.
        \item[$-$] \textit{koncepce velkoněmecká}: sjednocení i s Habsburskou monarchií
    \end{itemize}
    \item[květen 1848] zahájeno jednání ve Frankfurtu nad Mohanem
    \item[červen 1849] vytvořena německá ústava a nová dočasná vláda, německé státy se sjednotí do monarchie, titul císaře nabídli tehdejšímu pruskému králi \textbf{Fridrichu Vilému IV.}, ten to nepřijal $\rightarrow$ propukají další revoluce, které byly postupně rozehnány, až byl nakonec rozehnán i sněm
    \item[$-$] všechny pokusy o sjednocení neúspěšné
\end{itemize}

\subsection*{Habsburská monarchie v době předbřeznové}
\begin{itemize}
    \vspace{-0.5em}
    \setlength\itemsep{0.15em}
    \item[$-$] \textit{doba předbřeznová} = doba před vládou \textbf{Františka I.} (II.), syn Leopolda II., tedy doby před rokem 1848
    \item[$-$] Habsburská moanrchie velice národnostě pestrá: Němci, Češi, Poláci, Maďaři, Jugoslávci
    \item[$-$] zpátečnická politika, snaha nastolit absolutismus, opora o armádu a tajnou policii
    \item[$-$] František posledním císařem SŘŘ
    \item[1804] získává dědičný titul císař rakouský, král český a uherský
    \item[$-$] \textit{metternichovský absolutismus}, symbolem absolutismu je \textbf{Klemenc Lothar Metternich}
    \item[$-$] rakouská monarchie v době vídeňskéh kongresu prosazovala kvietismus, rovnováhu
    \item[1811] vydán \textit{všeobecný občanský zákoník}, zákoník moderní doby, základy právního moderního státu, politické ani občanské svobody nejsou
    \item[1811] státní bankrot: v rámci kontinentální blokády nedostatek surovin, vyšší daně $\rightarrow$ tisknou se peníze
    \item[$-$] provázeno českým národním obrozením, dba průmyslové revoluce: zavádění parních strojů do výroby, střet německých a českých zájmů
    \item[1791] první průmyslová výstava v Klementinu
    \item[$-$] František Josef Gerstner na českém území jako první zkonstruoval parní stroj
    \item[$-$] Josef Božek: parní kočár, první paroloď u nás
    \item[$-$] Josef Ressel: lodní šroub
    \item[$-$] bratranci Veverkovi: ruchadlo (vylepšený pluh, nejen kypší, ale i obrací)
    \item[$-$] první koňská železnice: z Českých Budějovic do Linze, využíváná k přepravě osob a soli
    \item[$-$] první parní železnice u nás z Vídně přes Moravu do Haliče (Polsko), odbočka do Brna v roce 1839
    \item[$-$] Brno získává přezdívku \textit{moravský Manchester}
\end{itemize}

\subsection*{Ferdinand I. Dobrotivý}
\begin{itemize}
    \vspace{-0.5em}
    \setlength\itemsep{0.15em}
    \item[(1836)] poslední král český
    \item[$-$] slabomyslný, zpomalenější, ale ovládal pět jazýků a měl značné znalosti v botanice či technických oborech
    \item[$-$] fysicky slabý, epilepsie $\rightarrow$ není příliš schopen vlády, byla mu dána poručnická rada složena z Metternicha, jeho strýce arcivévody Ludvíka a hraběte Františka Antonína Kolovrata, rada je však nejednotná
\end{itemize}

\subsection*{Rakouské císařství v letech 1848-1849}
\begin{itemize}
    \vspace{-0.5em}
    \setlength\itemsep{0.15em}
    \item[$-$] Ferdinand I. Dobrotivý, metternichovský absolutismus
    \item[$-$] Habsburská monarchie národnostně pestrá
    \item[(13.) březen 1848] ve Vídni Ferdinand propustil Metternicha, slib ústavy, vešla v platnost oktrojovaná ústava, nebyl však o ní zájem, odmítnuta
    \item[$-$] začínají se prosazovat jednotlivé národy v rámci monarchie
    \item[březen 1848] povstání v Pešti (generace maďarských básníků), cíl je odtrhnout se od Habsburské nadválády
    \item[11.3.1848] \textsc{shromáždění Pražanů} ve Svatováclavských lázních, vytvoření Svatováclavského výboru, dohoda o vypracování petice s požadavky českéhonároda směrem k císaři, nakonec jich bylo vypracováno několik a jedna z nich byla poslána císaři
    \item[$-$] požadavky: pevnější spojení zemí koruny české, aby české království mělo vlastní směr, zrovnoprávnění Čechů s Němci, v podstatě počátek formování českého politického programu
    \item[duben 1848] Svatováclavský výbor přejmenován  na Náropdní výbor, mělk zastupovat české veřejné mínění, zejméně zástupci české člechty a měšťanstva
    \item[8.4.1848] Ferdinand I. vydává jako odpověď na tuto petici \textit{Kabinetní list}, vew kterém slibuje splnění požadavků
    \item[$-$] rozštěpení na dva proudy:
    \begin{itemize}
        \vspace{-0.5em}
        \setlength\itemsep{0.15em}
        \item[$-$] liberálové (Palacký), zrovnoprávgnění češtiny s němčinou, větší spojení českých zemí, vlastní sněm
        \item[$-$] radikálové (Sabina), spíše obecné požadavky, aby se nižší vrstvy mohly podílet na politickém životě
    \end{itemize}
    \item[$-$] oba proudy se shodují na \textit{austroslavismu}: jakýsi jejich program, chtějí, aby monarchie byla federací, ve které by žily národy vedle sebe rovnoprávně a rovnocenně
    \item[$-$] \textit{panslavismus}: společné zájmy Slovanů v monarchii, jejich cílem je dosáhnout austroslavismu
    \item[2.-12.6.1848] \textsc{slovanský sjezd}: schválili \textit{Manifest k národům evropským}, více nestihli, protože v Praze propuklo povstání, v manifestu požadují rovnost Slovanů v rámci Habsburské monarchie, jakousi federaci
    \item[$-$] pražské povstání, které přerušilo sněm, trvalo asi týden, lidé z Koňského trhu (Václavského náměstí) narazil na císařskou armádu, tehdy byl jejím velitel Alfred von Windischgrätz, do bojů se zapojili studenti a intelektuálové
    \item[$-$] rozehnán slovanský sjezd, výbor rozpuštěn, pražské povstání poraženo
    \item[květen 1848] \textsc{revoluce ve Vídni}, císař utíká do Innsbrucku
    \item[22.7.1848] svolán \textsc{ústavodárný říšský sněm}, cíl: vypracovat ústavu pro monarchii, každý z národů má své zástupce, zrušeno poddanství a robota
    \item[7.9.1848] císař schvaluje zrušení poddanství za náhradu
    \item[říjen 1848] revoluce ve Vídni potlačena generálem Windischgrätzem, císař utíká do Olomouce, vzniká nová vláda
    \item[$-$] ústavodárný sněm přenesen do Kroměříže, kde pokračují jednání
    \item[2.12.1848] císař Ferdinand I. abdikuje ve prospěch svého synovce Františka Josefa I., který nastupuje jako osmáctiletý, záhy je ukončeno i jednání ústavodárného sněmu
    \item[(1849)] tajně připravované povstání radikálů v Praze -- májové spiknutí -- bylo potlačeno
    \item[březen 1849] nová oktrojovaná tzv. \textit{Stadionova}  ústava, o dva roky byl \textit{Silvestrovskými patenty} obnoven absolutismus, tomu se pak říká \textit{bachovský absolutismus}
\end{itemize}



\end{document}
