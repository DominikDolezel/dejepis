\documentclass{article}
\usepackage{fullpage}
\usepackage[czech]{babel}
\usepackage{amsfonts}

\title{\vspace{-2cm}Vikingové\vspace{-1.7cm}}
\date{}
\author{}

\begin{document}
\maketitle
\begin{itemize}
    \vspace{-0.5em}
    \setlength\itemsep{0.15em}
    \item[$-$] Vikingové = Normané = Dáni = Varjagové
    \item[$-$] S Evropa, germánské kmeny
    \item[$-$] dnešní Dánsko, Norsko, Švédsko -- původní sídla
    \item[$-$] mají \textit{náčelníky}
    \item[$-$] \textit{thing} = shromáždění všech svobodných mužů
    \item[$-$] rybolov, chov dobytka, obchod, zemědělství málo kvůli špatným podmínkám
    \item[$-$] řemeslníci, železené zbraně, lodě \textit{drakary}
    \item[$-$] vikingská expanze:
        \begin{itemize}
            \vspace{-0.5em}
            \setlength\itemsep{0.15em}
            \item[$-$] kvůli špatnému klima, nedstatkui jídla, přemnožení
            \item[793] \textsc{vyplenění kláštera v Lindisfarne} -- začátek
            \item[1066] \textsc{bitva u Hastings} -- konec
        \end{itemize}
\end{itemize}

\section*{Norové}
\begin{itemize}
    \vspace{-0.5em}
    \setlength\itemsep{0.15em}
    \item[8. st.] Skotsko, Irsko, Faerské ovy, Shetlandy
    \item[860] objevení Islandu, kolonizace
    \item[982] \textbf{Erik Rudý} v Grónsku (\textit{Grönland} = zelená země)
    \item[cca 1000] jeho syn \textbf{Leif Eriksson} v S Americe -- Vinland, kvůli bojům s původními obyvateli opouštějí
\end{itemize}

\section*{Dánové}
\begin{itemize}
    \vspace{-0.5em}
    \setlength\itemsep{0.15em}
    \item[$-$] vyrážejí od 1. pol. 9. st., cíle:
    \begin{itemize}
        \vspace{-0.5em}
        \setlength\itemsep{0.15em}
        \item[$-$] Franská říše (Paříž 845), ZFŘ -- F. ř. ztratila území Normandie, ale nájezdy ustála
        \item[$-$] Anglie (Anglové, Sasové, Jutové -- 7 království = \textit{heptarchie}), v boji proti Vikingům se vždy sjednotí a poté opět rozpadnou
    \end{itemize}
    \item[911] zisk Normandie (v čele Rollo)
    \item[pol. 11. st.] dočasné ovládnutí Anglie: král \textbf{Knut Veliký}
\end{itemize}

\section*{Alfréd Veliký (871 -- 900)}
\begin{itemize}
    \vspace{-0.5em}
    \setlength\itemsep{0.15em}
    \item[$-$] král Wessexu, centrum Manchester
    \item[$-$] uměl psát
    \item[878] \textsc{bitva u Edingtonu}, porážka Dánů
    \item[$-$] hranice Dánů a Anglie: Temže, J -- anglosaský, S -- dánský
    \item[$-$] poté se sjednotí a obyvatelstvo se promísí
    \item[$-$] mnich \textbf{Beda Ctihodný}: kronika anglosasů
\end{itemize}

\section*{Vilém I. Dobyvatel}
\begin{itemize}
    \vspace{-0.5em}
    \setlength\itemsep{0.15em}
    \item[$-$] před ním král \textbf{Eduard III. Vyznavač}, bezdětný, úpadek království
    \item[$-$] kandidáti na krále: \textbf{Harold II. Godwindson}, \textbf{Vilém}, \textbf{Harald Norský}
    \item[1066] \textsc{bitva u Stanfordského mostu}, Harold porazil Haralda
    \item[1066] \textsc{bitva u Hastings}, Vilém porazil Harolda a založil raně středověkou Anglii
    \item[$-$] \textit{Domesday Book} = evidence pozemků $\rightarrow$ daně
    \item[$-$] tapisérie z Bayeux
\end{itemize}
\hline
\begin{itemize}
    \vspace{-0.5em}
    \setlength\itemsep{0.15em}
    \item[1091] království Sicilské se dostalo pod nadvládu Normanů
    \item[$-$] Varjagové
    \begin{itemize}
        \vspace{-0.5em}
        \setlength\itemsep{0.15em}
        \item[$-$] cesta od Varjagů k Řekům (\textit{iz Varjag v Greki})
        \item[$-$] obchodní kontakt přes Řeky
        \item[862] Rurik ovládl Novgorod
        \item[882] jeho syn Oleg ovládl Kyjev
        \item[$-$] spojením vznikla \textbf{Kyjevská Rus}
    \end{itemize}
\end{itemize}

\section*{Státy Vikingů}
\begin{itemize}
    \vspace{-0.5em}
    \setlength\itemsep{0.15em}
    \item[$-$] od 9. -- 11. st.
    \item[$-$] Norské, Dánské, Švédské království
    \item[$-$] christianizace
\end{itemize}




\end{document}
