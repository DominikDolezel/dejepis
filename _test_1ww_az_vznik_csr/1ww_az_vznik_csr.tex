\documentclass{article}
\usepackage{fullpage}
\usepackage[czech]{babel}
\usepackage{amsfonts}

\title{\vspace{-2cm}První světová válka, ruské revoluce, vznik ČSR\vspace{-1.7cm}}
\date{}
\author{}

\begin{document}
\maketitle

\section*{První světová válka}

-- doplnit od Hanky --

\subsection*{Záminka, vypuknutí války}
\begin{itemize}
    \vspace{-0.5em}
    \setlength\itemsep{0.15em}
    \item[28.6.1914] \textsc{atentát na Františka Ferdinanda d'Esteho} v Sarajevu, František je následník Rakousko-uherského trůnu, záměrně si vybral na návštěvu den pořážky Srbska na Kosově poli, což pro ně bylo citlivé $\rightarrow$ vlna nenáisti vůši němu, nacionalista \textbf{Gabrilo Princip} (člen tajné srbské studentské nacionalistické organizace \textbf{Mladá Bosna}) na něj a jeho choť spáchá atentát, zbraně jim dodalo hnutí Černá ruka, na něj však nebyli nijak napojeni
    \item[(23.7.1914)] Rakousko-Uhersko dává Srbsku \textit{Červencové ultimátum}  (jako odvetu za atentát na Františka Ferdinanda d'Esteho) formulované tak, aby jej Srbsko nepřijalo
    \item[28.7.1914] \textsc{Rakousko-Uhersko vyhlásilo válku Srbsku}, považováno za počátek první světové války $\rightarrow$ vznik srbské fronty
    \item[1.8.1914] \textsc{Německo vyhlásilo válku Rusku}, Německo dalo ultimátum Belgii (v němž požadovalo volný prhůchod svých vojsk přes Belgii, Francie a Anglie potvrdily, že v případě německé agrese poskytnou Belgii vojenskou pomoc), belgická vláda ultimátum odmítla $\rightarrow$ \textsc{Německo vyhlašuje válku Belgii}, \textsc{Anglie vyhlašuje válku Německu}
    \item[6.8.1914] \textsc{Rakousko-Uhersko vyhlásilo válku Rusku}
    \item[$-$] Itálie zůstává neutrální
\end{itemize}

\subsection*{Charakteristika}
\begin{itemize}
    \vspace{-0.5em}
    \setlength\itemsep{0.15em}
    \item[$-$] hlavním bojištěm je Evropa, dála ale i Afrika, Asie a oceány
    \item[$-$] zemřelo asi 10 milionů lidí, dvojnásobek zraněn
    \item[$-$] z počátku se označovala jako \textit{velká válka}, až později jako první světová
    \item[$-$] ženy mnohdy nahrazovaly mužské profese, protože byli ve válce
    \item[$-$] nejrůznější vynálezy, motorizace armády, roste význam zbrojního, těžkého průmyslu
    \item[$-$] zájmy civilního obyvatelstva jsou podružné zakázkám jako výroba zbraní, zásobování fronty atd. $\rightarrow$ přeměna v centrální řízenou ekonomiku
\end{itemize}

\subsection*{Západní fronta}
\begin{itemize}
    \vspace{-0.5em}
    \setlength\itemsep{0.15em}
    \item[$-$] Německo se snaží porazit Francii
    \item[$-$] německý \textit{Schlieffenův plán} vytvořen už na začátku století, měla to být \uv{blesková válka}, mělo dojít k izolaci Velké Británie, chtěli docílit dobytí Evropy
    \item[20.8.1914] Němci dobývají Brusel
    \item[9.1914] Němci na francouzské hranici, postupují rychle na francouzské území, téměř až k Paříži
    \item[5.9.-15.9.1914] \textsc{boje na řece Marně} (zázraky na Marně), Francouzům se podařilo Němce zatlačit zpátky, zvrat ve válce, padá naděje na bleskovou válku, problém pro Němce
    \item[$-$] po této události bylo nesmírně složité, aby jakákoliv strana kamkoliv postoupila
    \item[10.-11.1914] \textit{běh k moři}, Francouzi i Němci chtějí obklíčit toho druhého a dobýt území směrem k moři, Němci dobyli ještě Antverpy, Flanderské vánoční příměří (o Vánocích se boje zastavily)
    \item[3.1915] francouzská \textsc{ofenzíva u Champagni}
    \item[4.1915] \textsc{bitva u Yper}, chlor (první použití bojového otravného plynu)
    \item[2.-12.1916]  \textsc{bitva u Verdunu}, 600 000 padlých, Němci neúspěšní $\rightarrow$ střídání v německém velení
    \item[7.-11.1916] \textsc{bitva na řece Sommě}, 1 300 000 padlých, neúspěch na straně Dohody,
    \item[1917] \textbf{Philipe Pétaine}  velitelem francouzské armády, po válce odsouzen za velezradu kvůli kolaboraci s Německem
    \item[$-$] Německo vede ofenzívu na západní frontě
    \item[7.-8.1918] \textsc{druhá bitva na Marně}, poslední bitva na západní frontě první světové války, Němci poraženi
    \item[$-$] prolomení Siegfriedovy linie (Německé opevnění, obdoba francouzské Maginotovy linie)
\end{itemize}

\subsection*{Balkán, srbská fronta}
\begin{itemize}
    \vspace{-0.5em}
    \setlength\itemsep{0.15em}
    \item[$-$] po vyhlášení války Rakouskem-Uherskum Srbsku se na jejich stranu přidá Černá Hora, Rakušané mají špatnou armádu, a proto jsou zatlačeni až za Dunaj
    \item[září 1915] \textsc{vstup Bulharů na balkánskou frontu} s cílem získat další území na stranu Rakouska-Uherska, během krátké doby jsou Srbové poraženi, jejich jednotky jsou převezeny na ostrov Corfu a odsud poté do Řecka, kde bojovali na straně Řecka, které bojovalo na straně Dohody
    \item[$-$] později poražena i Černá Hora
    \item[$-$] důsledky: Rakousko-Uhersko  spojeno s Bulhary a s osmanskými Turky, spojení centrálních mocností, mohly si pomáhat
    \item[(1915-1916)] \textit{Operace Gallipoli}: pokus Dohody vylodit se v Gallipoli, jejich cílem bylo získat Dardanely a dostat se do Černého moře a pomoct zásobovat Rusy, neúspěšné, politické zemětřesení u Britů
    \item[srpen 1916] vstup Rumunska do války na straně Dohody, během roka byli poraženi
    \item[1917] vstup Řeků taky na straně Dohody, ti však poraženi nebyli
    \item[$-$] manifest \textit{Mým národům}, tím císař oznamoval, že jsme vstoupili do války
    \item[9.1918]  ofenziva Dohody $\rightarrow$ kapitulace Bulharska, německá vojska utekla
    \item[29.9.1918]  mír s Bulharskem
    \item[30.9.1918] protože bylo Rakousko-Uhersko v rozkladu (samostatnost ČSR a státu Slovinců, Chorvatů a Srbů), pořádali uhersští představitelé o příměří
\end{itemize}

\subsection*{Východní fronta}
\begin{itemize}
    \vspace{-0.5em}
    \setlength\itemsep{0.15em}
    \item[15.8.1914] Rusové zahájili ofenzívu do Východního Pruska, která mohla bát nebezpečná pro celé Prusko, ale Němcům se podařilo tento útok zastavit a v několika bitvách Rusy porazit. Tím zachránili Prusko před invazí a vytvořili si podmínky pro postup v roce 1915.
    \item[1.8.1914] Německo vyhlásilo válku Rusku
    \item[6.8.1914] Rakousko-Uhersko vyhlásilo válku Rusku
    \item[$-$] německý útok (Paul von Hindenburg, Erich Ludendorff), Rusové několikrát poraženi, pro Rusy katastrofa
    \item[$-$] Rakušanům se nedaří: v čele Rusů generál \textbf{Brusilov}, na úkor Rakouska pronikl až k hranicím dnešního Slovenska ke Krakovu
    \item[1915] proto naplánovaly Centrální mocnosti tzv. \textit{Gorlický průlom} $\rightarrow$ posouvají hranici dál na východ
    \item[$-$] na západní frontě obrovské bitvy, tím, že by Rusové kapitulovali, mohli by armády přesunout na západní frontu
    \item[1916] Rusové se pokusili v rámci \textit{Brusilovovy ofensivy} posunout linii, podařilo se jim to trochu
    \item[$-$] dohodovým zemím měli pomoct Rumuni, ale ti byli během pár měsíců úplně zlikvidováni
    \item[1917] vstup Řecka do války, otevření Soluňské fronty, kam byli dováženi Srbové, kteří byli už dřív poraženi
\end{itemize}

\subsection*{Jižní (italská) fronta}
\begin{itemize}
    \vspace{-0.5em}
    \setlength\itemsep{0.15em}
    \item[$-$] mezi Italy a Rakušáky a Italy a Němci
    \item[23.5.1915] Itálie vyhlašuje válku Rakousku-Uhersku, o rok později Německu
    \item[duben 1915] dohodové země slibují Italům území, která chtějí v \textit{Londýnské smlouvě}
    \item[podzim 1917] Italové poraženi \textsc{u Caporetta}, linie se posunula na řeku Piavu
    \item[10.-11.1918] rozhodující \textsc{bitva u Vittorio Veneta} mezi Rakouskem-Uherskem a Dohodou, prolomení italské fronty na řece Piavě, Italové vítězí
    \item[26.10.1918] Karel I. požádá o separátní mír a okamžité příměří
    \item[27.10.1918]  Gyula Andrássy, R-U přijímá mírové podmínky
    \item[3.11.1918] příměří mezi Itálií a Rakouskem-Uherskem
    \item[6.11.1918] demobilizace armády
    \item[11.11.1918]  Karel I. opouští Schönbrünn
\end{itemize}

\subsection*{Turecké fronty}
\begin{itemize}
    \vspace{-0.5em}
    \setlength\itemsep{0.15em}
    \item[$-$] kavkazská fronta (Osmani x Rusové), mezopotámská fronta (Britové x Osmani), syrsko-palestinská fronta (Osmani x Briti)
    \item[$-$] do roku 1915 Rusové vítězí
    \item[$-$] genocida Arménů, Turci jich zlikvidovali asi milion a půl
    \item[$-$] mezopotámská fronta se posouvá na úkor Osmanů
    \item[$-$] na syrsko-palestinské frontě jsou Osmané zpočátku úspěšní, jde jim o Suezský průplav, krátce se jim to povedlo, pak jsou zatlačeni zpět Brity
\end{itemize}

\subsection*{Boje na moři}
\begin{itemize}
    \vspace{-0.5em}
    \setlength\itemsep{0.15em}
    \item[$-$] proti němcům především Velká Británie (a Japonsko)
    \item[$-$] dohodové země úspěšné, získají přístup k německým koloniím v Africe (bitvy: u Helgolandu, Dogger Bank, Skaggeraku, Jutska), zkrátka v Severním moři
    \item[1917] \textsc{německá ponorková válka}: jsou úspěšní, potápějí nepřátelské lodě
    \item[(7.5.1915)] Němci potopili loď s Američany (Lusitania), to byl jeden z důvodů proč se pak USA zapojily do války
\end{itemize}

\subsection*{Politické změny a situace}
\begin{itemize}
    \vspace{-0.5em}
    \setlength\itemsep{0.15em}
    \item[1916] britský \uv{válečný kabinet}: \textbf{David Lloyd George}, klíčovou postavou mnoha reforem, které položily základ sociálního státu
    \item[1916] čtrnáct bodů W. Wilsona válčícím státům (stanovit cíle války), stalo se východiskem při mírových vyjednáváních
    \item[21.11.1916] Rakousko-Uhersko: smrt Františka Josefa I. (na zápal plic), nastupuje \textbf{Karel I.} (manželka Zita Parmská), snažil se zastavit válku mírovými jednáními (na jejímž vypuknutí neměl žádný podíl), též chce dosáhnout sociálního smíru v hroutící se monarchii
    \item[$-$]  bratr Zity Parmské, belgický princ \textbf{Sixtus Bourbonský} zprostředkovává utajovaná mírová jednání s Francií, ale zjistilo se to $\rightarrow$ \textit{Sixtova aféra} $\rightarrow$  nová smlouva s Německem (obnovení spojenectví), změny ve francouzské vládě
    \item[1917] Německo vede neomezenou ponorkovou válku
    \item[6.4.1917] \textsc{vstup USA do války}
    \item[3.3.1918] \textit{Brestlitevský mír} potvrzuje vítězstvá ústředních mocností na východní frontě bolševické revoluci a následné kapitulaci Ruska
    \item[5.10.1918] německá žádost o příměří, Němeci posílají W. Wilsonovi souhlas s vyjednáváním podle Wilsonových čtrnácti bodů, Wilson požaduje \begin{itemize}
        \vspace{-0.5em}
        \setlength\itemsep{0.15em}
        \item[$-$] ústup německých vojsk ze všech obsazených území
        \item[$-$] zastavení válečných aktivit na moři
        \item[$-$] abdikaci císaře Viléma II.
    \end{itemize}
    \item[9.11.1918]  Vilém II. abdikuje
    \item[11.11.1918] v 11 hodin se stává účinným tzv. \textit{příměří z Compiègne} mezi Německem a Dohodou, považováno za konec první světové války (dnes slavíme den veteránů)
\end{itemize}

\subsection*{Výsledky}
\begin{itemize}
    \vspace{-0.5em}
    \setlength\itemsep{0.15em}
    \item[$-$] vyhrála Dohoda $\rightarrow$ zánik Rakouska-Uherska (rozpad na 5 nástupnických států: ČSR, Rakousko, Maďarsko, Polsko, království Srbů, Chorvatů a Slovinců)
    \item[$-$] poražené státy: Německo, Rakousko, Maďarsko, Turecko, Bulharsko
    \item[$-$] z Německa vznikne Výmarská republika
    \item[$-$] Rusko ztratí území: Finsko, Poblatí, Litva, Bělorusko, Ukrajina připadnou Polsku, Besarabie připadne Rumunům
    \item[$-$] Osmané: zrušen sultanát, stávají se republikou Turecko
    \item[$-$] Belgie obnovena v původním rozsahu
    \item[$-$] ekonomickým vítězem jsou Spojené státy, Evropa už není na vrcholu
\end{itemize}

\section*{Ruské revoluce}
\begin{itemize}
    \vspace{-0.5em}
    \setlength\itemsep{0.15em}
    \item[$-$] pro Rusko válka velice zničující pro už tak zaostalý stát, kolaps ekonomiky, váznutí zásobování
    \item[$-$] hladové stávky, demonstrace
    \item[leden 1905] \textit{Krvavá neděle}: namísto toho, aby střelba do davu před Zimním palácem paralyzovala povstání, propukla další
    \item[$-$] vydání \textit{Říjnového manifestu}, kterým se tehdejší car \textbf{Mikuláš II.}  povolil parlamentu a zavedl volby
    \item[$-$] G. J. \textbf{Rasputin}: dostal se do přízně carské rodiny, presentoval se jako lidový léčitel a mystik, zachánil následníka trůnu Alexeje, který trpěl hemofilií (nemoc krve)
    \item[$-$] politické strany
    \begin{itemize}
        \vspace{-0.5em}
        \setlength\itemsep{0.15em}
        \item[$-$] \textit{eseři}: zástupci zemědělců, agrární program, po revoluci chtějí sesadit cara, nevadí jim se uchylovat k atentátům
        \item[$-$] \textit{kadeti}: liberálové, jejich cílem je konstituční monarchie
        \item[$-$] \textit{sociální demokraté}: dvě frakce: \textit{menševici} (reformní soc. demokraté, chtějí nastolit pořádek skrz reformy), \textit{bolševici} (v čele Lenin, ruští komunisté, navazují na ideologii Marxe, jejich cílem je socialistická revoluce -- jak se však ukáže nikoliv socialistické společnosti, ale svrhnutí moci na sebe)
    \end{itemize}
\end{itemize}

\subsection*{Únorová revoluce 1917}
\begin{itemize}
    \vspace{-0.5em}
    \setlength\itemsep{0.15em}
    \item[23.2.] stávka v Petrohradě
    \item[27.2.] republika, svržen car
    \item[$-$] podle juliánského kalendáře, byla až v březnu gregoriánského (našeho) kalendáře
    \item[$-$] Mikuláš II. byl sesazen, i když se snažil ještě abdikovat ve prospěch Michala, neúspěšné
    \item[$-$] vznik prozatímní liberálně-demokratické vlády, v čele kníže Lvov, ale paralelně s ní vznikaly tzv. \textit{sověty}, bylo jich několik, ta nejvýznamější v tehdejším hlavním městě Petrohradě, ze začátku zástupci eserů a menševiků $\rightarrow$ dvouvládí
    \item[$-$] bolševici se snaží ovládnout sověty a přes ně se dostat k moci
    \item[$-$] autonomie Finska, Estonska, nezávislost Polska
    \item[$-$] nespokojenosti využívá Lenin (žijící v exilu), jeden z vůdců bolševiků
    \item[duben 1917] Lenin se dostává do Ruska díky Němcům, kteří mu chtěli pomoct, aby ukončil válku na východní frontě, odstranil prozatímní vládu
    \item[červenec] rekonstrukce vlády jako důsledek protestů, nový premiér Alexandr Kerenský
    \item[$-$] nespokojenosti toho, že Rusko pořád válčí využívají sověti, Lenin se vrátl do exilu do Finska, ale další bolševik Trockij připravuje revoluci
    \item[září] pokus o nastolení vojenské diktatury, Lavr Kornikov, především díky bolševické agitaci se k němu nepřidali vojáci
\end{itemize}

\subsection*{Říjnová revoluce}
\begin{itemize}
    \vspace{-0.5em}
    \setlength\itemsep{0.15em}
    \item[25.10.1917] poslední kapka, které vedla k říjnové revoluci, přijel tam i z Finska Lenin
    \item[$-$] nejprve obsadili v Petrohradu klíčové body jako pošty, nádraží, přístav, mosty, banky, vtrhli do Zimního paláce, kde sídlila prozatímní vláda, všechny ministry kromě Kerenského pozatýkali a strhli an sebe moc
    \item[$-$] Aurora: výstřel z tohoto křižníku údajně zahájil revoluci
\end{itemize}

\subsection*{V. I. Lenin}
\begin{itemize}
    \vspace{-0.5em}
    \setlength\itemsep{0.15em}
    \item[26. října] zasedání všeruského sjezdu, kde vydali dekrety: do čela Ruska jde nová vláda, tzv. Rada lidových komisařů v čele s Leninem, dále okamžité uzavření míru, zkonfiskují velkostatkářům a církvi bez náhrady půdu
    \item[listopad] volby do ústavodárného shromáždění, azvítězili eseři, ale Rusko bylo i tak vyhlášeno za demokraticko-federativní republiku, o den později Lenin toto shromáždění zrušil a začíná vláda bolševik (vláda jedné strany)
    \item[$-$] období, kdy budou postupně ovládat celý prostor probíhá občanská válka (Rudá armáda vedena Trockým proti stoupencům minulého režimu -- bílí)
    \item[3.3.1918] Brestlitevský mír, Trockij tam vyjednával
    \item[1924] umírá
    \item[$-$] manželka Naděžda Krupská
\end{itemize}

\section*{Vznik ČSR}
\begin{itemize}
    \vspace{-0.5em}
    \setlength\itemsep{0.15em}
    \item[$-$] Česko, Slovensko, Podkarpatská Rus
    \item[28.7.1914] František Josef I. v manifestu \textit{Mým národům} ohlašuje, že monarhcie je ve válce
    \item[25.7.1914]uzavřena Říšská rada (parlament)
    \item[$-$] změny v ekonomice, orientace na militarismus, zaveden přídělový systém
    \item[$-$] Češi musí bojovat za monarchii na východní frontě
    \item[$-$] rekace českých politiků:
    \begin{itemize}
        \vspace{-0.5em}
        \setlength\itemsep{0.15em}
        \item[$-$] ti, co chodili do Říšské rady nebyli pro to, aby se měla monarchie rozbít, ještě po válce vydali stanovisko, že se má zůstat v monarchii
        \item[$-$] skupina kolem Karla Kramáře: začal začal spojovat vznik Československa s Rusy, chtěl mít Československo jako monarchii
        \item[$-$] skupina kole Tomáše Garriguea Masaryka
    \end{itemize}
\end{itemize}

\subsection*{Tomáš Garrigue Masaryk}
\begin{itemize}
    \vspace{-0.5em}
    \setlength\itemsep{0.15em}
    \item[$-$] do politiky se zapojil po první světové válce, kdy mu bylo přes šedesát
    \item[$-$] vystudoval sociologii, studoval gymnasium v Brně, ve Vídni filosofickou fakultu
    \item[$-$] po studiích začíná přednášet na české části pražské Karlovy university
    \item[$-$] habilitoval se prací \textit{Sebevražda}
    \item[$-$] při svém pobytu v Lipsku se seznámil s Američankou Charlotte Garrigou, s níž se v USA oženil a poté se vrátili do Čech
    \item[$-$] založil stranu lidovou, následně přejmenovanou na pokrokovou
    \item[$-$] už před válkou se zapsal do povědomí: přiklonil se k vědcům, kteří považovali RKZ za falsa
    \item[$-$] zapojil se do \textit{Hilsneriády}, kde upozorňoval proti antisemitismu
    \item[$-$] po vypouknutí války odchází do Itálie a následně do Švýcarska
    \item[4.7.1915] v Curychu vystupuje při příležitosti 500. výročí upálení Mistra Jana Husa
    \item[6.7.1915] přesouvá se do Ženevy, kde říká, že se má Československo odtrhnout od monarchie a že Češi a Slováci mají mít svůj vlastní stát
    \item[září 1915] do Ženevy odchází jeho žák, Edvard Beneš, aby mohl Beneš nastoupit do funkce presidenta, prosadil Masaryk, že funkce presidenta je už od 35 let
    \item[$-$] zakládají Československý zahraniční komitet
    \item[únor 1916] poté se přesouvají do Francie a zakládají Českou národní radu (ČNR), to je politický ilegální orgán, tři hlavní působící: Masaryk, Beneš, Štefánik
    \item[$-$] úkolem ČNR bylo přesvědčit dohodové země, aby souhlasily s tím, že se Rakousko-Uhersko rozpadne na nástupnické státy
    \item[říjen 1916] \textit{Cleavelandská dohoda}, byla podepsána mezi Čechy a Slováky, znamená, že chceme vlastní federativní stát
\end{itemize}

\subsection*{České země během války}
\begin{itemize}
    \vspace{-0.5em}
    \setlength\itemsep{0.15em}
    \item[$-$] zaveden ostrý protičeský kurs, zrušena svoboda slova, censura
    \item[$-$] Habsburkové se potřebují soustředit na válku a ne řešit zlobivé Čechy -- bylo zatknuto několik českých politiků, někteří dokonce za protirakouskou politiku byli odsouuzeni k trestu smrti, zachránila je amnestie
    \item[březen 1915] vznik \textit{Maffie}: snaží se mezi občany rozšiřovat povědomí o tom, že by bylo dobré mít svůj vlastní stát a udávat informace Dohodě o dění na území Rakouska-Uherska, spolupracovali s nimi i Slováci
    \item[$-$] proti Maffii stáli čeští politici, kteří dřív docházeli do Říšského sněmu, sami si vytvořili \textit{Český svaz poslanců ŘR}: prohalšují, že budou poslušní Habsburkům
\end{itemize}

\subsection*{Ruské legie}
\begin{itemize}
    \vspace{-0.5em}
    \setlength\itemsep{0.15em}
    \item[$-$] jedním z nástrojů, jak Čechy v rámci monarhcie zviditelnit, byly tzv. ruské legie
    \item[$-$] na území cizích států působí legionáři: české jednotky, které bojovaly na straně Dohody
    \item[$-$] měly i propagační význam -- propagovaly myšlenky samostatného Československého státu
    \item[(12.8.1914)] car Mikuláš povolil vznik české družiny v Kyjevě
    \item[$-$] postupné vytváření jednotky československé
    \item[$-$] do Ruska přijel i Masaryk s cílem jednotku zorganizovat, což se mu podařilo
    \item[(2.7.)1917] \textsc{bitva u Zborova}, první úspěch českých legií na úkor Rakousko-Uherské armády;
    \item[(7.11.1917)] Bolševický převrat: Masaryk nařizuje, aby se Čechoslováci do situace v Rusku nevměšovali (\textit{ozbrojená neutralita}), mají bojovat maximálně v zájmu Čechoslováků
    \item[$-$] Čechoslováci se mají přes tzv. trassibiřskou magistrálu (vlak) přesunout na západní frontu, kde by bojovali na straně Dohody, těžká operace (Rusáci je chtěli odzbrojit, vylodění Japonců)
    \item[$-$] Rusové vnímají Čechoslováky velmi negativně, protože obsadili transsibiřskou magistrálu, čímž jim zabránili rozšiřovat revoluci na Sibiř
    \item[listopad 1918] doba, kdy existuje československý stát -- legionáře navštívil Rastislav Štefánik a vyřídil jim, aby pokračovali v bojích v Rusku na straně intervenčních armád, ti však nechtěli
    \item[$-$] ukořistění části ruského pokladu, který ukradl Kolčak
    \item[prosinec 1919] první transport legionářů z Vladivostoku
    \item[listopad 1920] dokončení evakuace -- tomuto návratu se říká \textit{transibiřská anabáze}
    \item[$-$] generálové Syrový, Krejčí, Čeček
    \item[$-$] v Rusku bojovalo více než 10 000 českých vojáků
\end{itemize}

\subsection*{Československé legie}
\begin{itemize}
    \vspace{-0.5em}
    \setlength\itemsep{0.15em}
    \item[$-$] vznikla v rámci francouzské cizinecké legie až na konci války
    \item[$-$] říkali si \textit{rota Nazdar}
    \item[$-$] zapsali se v bitvách u Terronu, Arrasu, Vousieres, Chemin des Dames
    \item[$-$] Itálie: boje na Piavě, Doss Alto
    \item[$-$] Srbsko
    \item[$-$] přispělo k myšlence vzniku samostatného československého státu
\end{itemize}

\subsection*{Rakousko-Uhersko 1916-1917}
\begin{itemize}
    \vspace{-0.5em}
    \setlength\itemsep{0.15em}
    \item[21.11.1916] prasynovec Františka Josefa I. stanovil novou politiku, široká politická amnestie (zachánení Kramáře, Rašína)
    \item[$-$] vedl tajná jednání s Dohodou s cílem uzavřít mír, byla prozrazena
    \item[$-$] Český svaz: sdružuje politiky, kteří chtjěí být součástí monarchie
    \item[$-$] proti setrvání v Habsburské monarchii protestuje česká inteligence, sepisuje \textit{manifest českých spisovatelů}, autorem je Jaroslav Kvapil, podepsalo jej 222 osob, obracejí na české poslance, aby hájili zájmy českého národa, jinak ať odejdou
    \item[17.5.1917] poslanci se nejspíše pod vlivem manifestu začali hlásit k získáni autonomie a připojení Slovenska k Čechu
    \item[6.1.1918] \textit{Tříkrálová deklarace}: habsurkové ji označili za velezrádný dokument, čeští politici se v ní hlásí k vytvoření samostatného státu
    \item[13.4.] v Praze v Obecním domě se sešli významní představitelé a Alois jirásek tu předčítal slavnostní přísahu v tom smyslu, aby vznikl samostatný československý stát
\end{itemize}

\subsection*{Vojenské vzpoury 1918}
\begin{itemize}
    \vspace{-0.5em}
    \setlength\itemsep{0.15em}
    \item[$-$] Boka Kotorská (v Černé Hoře) bouřili se tam Čechoslováci, už nechtjěí válčit, Kragujevac (Srbsko), Piava
    \item[30.5.] \textit{Pitsburská dohoda}: podepisuje ji masaryk v USA, už se nemluví federaci, ale o autonomii Slovenska, což nebylo dodrženo
    \item[$-$] postupně Dohodové velmoci uznaly, že má vzniknout budoucí Československo, uznaly československou národní radu za oficiální politický orgán
\end{itemize}

\subsubsection*{Československo}
\begin{itemize}
  \item[13. 7.] se poskládal Národní výbor dle posledních voleb (Kramář, Švehla, Rašín, Klofáč, Soukup) -- připravovali se, že převezmou moc až skončí válka, pod nimi funguje
  \item[od srpna] \textit{Zemská hospodářská rada}: otázka hospodářství, měnové odluky, železnic
  \item[14. 10.] uspořádána generální stávka nějakou socialistickou radou (taky pod N výborem) proti vývozu potravin z Československa, současně vznikla čs. prozatimní vláda, tedy že se ČNR (Masaryk, Beneš, Čvehla) pasuje na vládu
  \item[16.10.1918] Karel I. přichází s manifestem \textit{Mým věrným národům rakouským}, snažil se nabídnout národům v rámci monarchie federalizaci, nicméně už je pozdě
  \item[18. 10.] \textit{Washingtonská deklarace} od Masaryka Wilsonovi zveřejněna v Paříži -- na jakých základech bude čs. stát vybudován (republika, demokracie), stejného dne přichází tzv. \textit{Wilsonova nóta}, že souhlasí s rozpadem Rakouska a vznikne ČSR
  \item[24. 10.] zrušena smlouva s Německem, 25.10. zhroucení italské fronty, 26.10. Karel žádá o separátní mír, 27.10. posílá Gyula Andrassy nótu že přijímají mírové podmínky, 28.10. \textit{zákon o zřízení samostatného čes\-ko\-slo\-ven\-ské\-ho státu} (z pera Aloise Rašína), krátce po poledni tedy vyhlášena republika, všichni důležití byli v Ženevě, tam jednali s politiky v emigraci kdo bude president, jaká bude vláda atd., známí jsou ti, kteří zůstali u nás: \textit{muži 28. října} -- Alois Rašín, Jiří Stříbrný, Antonín Švehla, František Soukup, Vavro Šrobár
  \item[30. 10.] sešli se zástupci slovenských politiků ve svatem Martine -- \textit{Martinská deklarace}: Slovenská národní rada, vyjádřili se, že společně s Čechy chtějí vytvořit čs. stát
  \item[13. 11.] \textit{Prozatimní ústava}, národní výbor zvětšili, rozšířili o Slováky -- Revoluční národní shromáždění, všechno se to odehrálo v Obecním domě
  \item[14. 11.] se v Thunovském paláci sešlo RNS, prohlásili, že ČS bude republika, prezidentem zvolili v nepřítomnosti Masaryka, sesadili Habsburky a jmenovlai vládu v čele s Karlem Kramářem
  \item[21.12.1918] TGM se dostává do ČSR
\end{itemize}



\end{document}
