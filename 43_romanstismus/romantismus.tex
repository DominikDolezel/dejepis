\documentclass{article}
\usepackage{fullpage}
\usepackage[czech]{babel}
\usepackage{amsfonts}

\title{\vspace{-2cm}Romantismus\vspace{-1.7cm}}
\date{}
\author{}

\begin{document}
\maketitle

\begin{itemize}
    \vspace{-0.5em}
    \setlength\itemsep{0.15em}
    \item[$-$] divoký, nespoutaný, hrdina je často vyděděnec, emoce, návrat ke středověkým slohům
    \item[$-$] kolébkou je Anglie, poté se šíří dál
    \item[$-$] \textit{historismus}: vrací se do historie, inspiruje se jí
    \item[$-$] historizující slohy: novogitoka, novorománský sloh, novorenesanční, novobarokní
    \item[$-$] novogotické památky: parlament v Budapešti, v Londýně, katedrála v Edinburghu, staroanglická venkovská sídla
    \item[$-$] přírodní krajinářský park: Central Park, Bolognský lesík, Lednice; obsahují stavbičky či umělé zříceniny
    \item[$-$] hrádek u Nechanic, Hluboká, Konopiště, Kokořín, chrám Sv. Víta, Karlštejn
    \item[$-$] novorenesanční památky: Rudolfinum, Národní divadlo, Národní museum
    \item[$-$] sochaři: Václav Levý
    \item[$-$] malířství: Eugéne Delacroix, zříceniny, temné, tajuplné, Francisco Goya
    \item[$-$] malířství u nás: Antonín a Josef Mánes
\end{itemize}

\end{document}
