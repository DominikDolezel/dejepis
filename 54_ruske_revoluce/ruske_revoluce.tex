\documentclass{article}
\usepackage{fullpage}
\usepackage[czech]{babel}
\usepackage{amsfonts}

\title{\vspace{-2cm}Ruské revoluce\vspace{-1.7cm}}
\date{}
\author{}

\begin{document}
\maketitle

\begin{itemize}
    \vspace{-0.5em}
    \setlength\itemsep{0.15em}
    \item[$-$] pro Rusko válka velice zničující pro už tak zaostalý stát, kolaps ekonomiky, váznutí zásobování
    \item[$-$] hladové stávky, demonstrace
    \item[leden 1905] \textit{Krvavá neděle}: namísto toho, aby střelba do davu před Zimním palácem paralyzovala povstání, propukla další
    \item[$-$] vydání \textit{ŘÍjnového manifestu}, kterým se povolilparlamentu a zavedení voleb
    \item[$-$] G. J. Rasputin: dostal se do přízně carské rodiny, presentoval se jako lidový léčitel a mystik, zachánil následníka trůnu Alexeje, který trpěl hemofilií (nemoc krve)
    \item[$-$] politické strany
    \begin{itemize}
        \vspace{-0.5em}
        \setlength\itemsep{0.15em}
        \item[$-$] \textit{eseři}: zástupci zemědělců, agrární program, po revoluci chtějí sesadit cara, nevadí jim se uchylovat k atentátům
        \item[$-$] \textit{kadeti}: liberálové, jejich cílem je konstituční monarchie
        \item[$-$] \textit{sociální demokraté}: dvě frakce: \textit{menševici} (reformní soc. demokraté, chtějí nastolit pořádek skrz reformy), \textit{bolševici} (v čele Lenin, ruští komunisté, navazují na ideologii Marxe, jejich cílem je socialistická revoluce -- jak se však ukáže nikoliv socialistické společnosti, ale svrhnutí moci na sebe)
    \end{itemize}
\end{itemize}

\subsection*{Únorová revoluce 1917}
\begin{itemize}
    \vspace{-0.5em}
    \setlength\itemsep{0.15em}
    \item[23.2.] stávka v Petrohradě
    \item[27.2.] republika, svržen car
    \item[$-$] podle juliánského kalendáře, byla až v březnu gregoriánského (našeho) kalendáře
    \item[$-$] Mikuláš II. byl sesazen, i když se snažil ještě abdikovat ve prospěch Michala, neúspěšné
    \item[$-$] vznik prozatímní liberálně-demokratické vlády, v čele kníže Lvov, ale paralelně s ní vznikaly tzv. \textit{sověty}, bylo jich několik, ta nejvýznamější v tehdejším hlavním městě Petrohradě, ze začátku zástupci eserů a menševiků $\rightarrow$ dvouvládí
    \item[$-$] bolševici se snaží ovládnout sověty a přes ně se dostat k moci
    \item[$-$] autonomie Finska, Estonska, nezávislost Polska
    \item[$-$] nespokojenosti využívá Lenin (žijící v exilu), jeden z vůdců bolševiků
    \item[duben 1917] Lenin se dostává do Ruska díky Němcům, kteří mu chtěli pomoct, aby ukončil válku na východní frontě, odstranil prozatímní vládu
    \item[červenec] rekonstrukce vlády jako důsledek protestů, nový premiér Alexandr Kerenský
    \item[$-$] nespokojenosti toho, že Rusko pořád válčí využívají sověti, Lenin se vrátl do exilu do Finska, ale další bolševik Trockij připravuje revoluci
    \item[září] pokus o nastolení vojenské diktatury, Lavr Kornikov, především díky bolševické agitaci se k němu nepřidali vojáci
\end{itemize}

\subsection*{Říjnová revoluce}
\begin{itemize}
    \vspace{-0.5em}
    \setlength\itemsep{0.15em}
    \item[25.10.] poslední kapka, které vedla k říjnové revoluci, přijel tam i z Finska Lenin
    \item[$-$] nejprve obsadili v Petrohradu klíčové body jako pošty, nádraží, přístav, mosty, banky, vtrhli do Zimního paláce, kde sídlila prozatímní vláda, všechny ministry kromě Kerenského pozatýkali a strhli an sebe moc
    \item[$-$] Aurora: výstřel z tohoto křižníku údajně zahájil revoluci
\end{itemize}

\subsection*{V. I. Lenin}
\begin{itemize}
    \vspace{-0.5em}
    \setlength\itemsep{0.15em}
    \item[26. října] zasedání všeruského sjezdu, kde vydali dekrety: do čela Ruska jde nová vláda, tzv. Rada lidových komisařů v čele s Leninem, dále okamžité uzavření míru, zkonfiskují velkostatkářům a církvi bez náhrady půdu
    \item[listopad] volby do ústavodárného shromáždění, azvítězili eseři, ale Rusko bylo i tak vyhlášeno za demokraticko-federativní republiku, o den později Lenin toto shromáždění zrušil a začíná vláda bolševik (vláda jedné strany)
    \item[$-$] období, kdy budou postupně ovládat celý prostor probíhá občanská válka (Rudá armáda vedena Trockým proti stoupencům minulého režimu -- bílí)
    \item[3.3.1918] Brest-litevský mír, Trockij tam vyjednával
    \item[1924] umírá
    \item[$-$] manželka Naděžda Krupská  
\end{itemize}




\end{document}
