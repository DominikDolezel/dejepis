\documentclass{article}
\usepackage{fullpage}
\usepackage[czech]{babel}
\usepackage{amsfonts}

\title{\vspace{-2cm}Jagellonci\vspace{-1.7cm}}
\date{}
\author{}

\begin{document}
\maketitle

\begin{itemize}
    \vspace{-0.5em}
    \setlength\itemsep{0.15em}
    \item[1471] zemský sněm zvolil 1471 \textbf{Vladislava Jagellonského} (na doporučení Jiřího z Poděbrad)
\end{itemize}

\section*{Vladislav II. Jagellonský (1471 – 1516)}
\begin{itemize}
    \vspace{-0.5em}
    \setlength\itemsep{0.15em}
    \item[$-$] Jiří se dostal do sporu s papežskou stolicí: chtěl aby byla dodržována bazilejská kompaktáta, papež je zrušil, dal Jiřího do klatby a vyhlásil na něj křížovou výpravu $\rightarrow$
    \item[1469] \textsc{bitva u Vilémova}, Jiří vyhrál, Matyáš Korvín se ovšem přesto nechá korunovat králem, drží Moravu a Slezsko
    \item[$\Rightarrow$] do 1471 drží Vladislav II. jen Čechy
    \item[$-$] bojoval s Matyášem, k ničemu to nevedlo, jen se potvrdil status quo
    \item[1479] \textit{Olomoucké dohody}: oba budou mít titul krále českého, až jeden z nich zemře, přejde titul na druhého z nich
    \item[1490] Matyáš Korvín umírá; uherští stavové souhlasili, že králem Uher bude Vladislav II. Jagellonský $\rightarrow$ českouherská personální unie (jen do roku 1526)
    \end{itemize}

\noindent Pozn.: v Uhrách měli silný vliv magnáti, zvolili si za krále Matyho Korvína, aby ochránil zemi před Osmany, ten je ale silný až moc, \uv{zkrotil} je; v Uhrách byl tehdy na dvoře spisovný jazyk čeština, silný vliv bratříčků

\begin{itemize}
    \vspace{-0.5em}
    \setlength\itemsep{0.15em}
    \item[$-$] vláda: slabá, sídlil v Budíně (v Uhrách), silní stavové, stavovská monarchie
    \item[$-$] dochází ke změnám v hospodářství: šlechta začala podnikat; vznikají velkostatky, zakládají sklárny, hutě, rybníky (významní zakldatelé rybníků: Josef Štěpánek Netolický, Jakub Krčín z Jelčan -- založil Rožmberk), pěstuje vinné révy
    \item[$-$] rozvíjí se města co založila šlechta -- poddanská, konkurenti královská města, šlechta chce zbavit královská města místa na zemském sněmu (královská města posílila během husitských válek)
    \item[$-$] začali jsme znovu obchodovat se zahraničím, obnova po husitských válkách, vystupujeme z izolace
    \item[$-$] roste počet poddaných $\rightarrow$ řada povstání:
    \begin{enumerate}
        \vspace{-0.5em}
        \setlength\itemsep{0.15em}
        \item \textbf{Dalibor z Kozojed}: zeman města na Litoměřicku, ujal se nevolníků, kteří utíkali od svého pána ze sousedního města, poté je odmítal vydat; byl odsouzen, umístěn do vězení na věž, které se po něm říká Daliborka
        \item \textbf{Dóžovo povstání}: velké povstání v Uhrách, chystala se křížová výprava proti Osmanům, nachystaná armáda se změnila ve velké protifeudální povstání, v čele \textbf{Jiří Dóža}, krutě pobito
    \end{enumerate}
    \item[$-$] Vladislav II. je katolík, podporuje katolíky, narůstá napětí, především v Praze $\Rightarrow$
    \item[1483] \textsc{2. pražská defenestrace}, lidé různě po Praze zaútočili na katolické konšely i na katolické kostely opravené po husitských válkách
    \item[1485] náboženský smír v Kutné Hoře; mezi katolíky a kališníky, cílem klid v zemi
    \item[1500] \textit{Vladislavské zřízení}; listina, kterou vydala šlechta, chtěla zbavit královská města práv a míst na zemském sněmu, chtěla hospodářská privilegia pro poddanská města
    \item[1517] \textit{Svatováclavská smlouva}: \uv{kompromis}, ale výhodný pro šlechtu, dostala vlastně vše co chtěla, ale stavové z královských měst zůstanou na zemském sněmu

\end{itemize}


\section*{Ludvík Jagellonský (1516 – 1526)}
\begin{itemize}
    \vspace{-0.5em}
    \setlength\itemsep{0.15em}
    \item[$-$] nastupuje 10letý, anarchie, slabá vláda
    \item[$-$] sestra Anna Jagellonská
    \item[1526] zahynul v \textsc{bitvě u Moháče}, boj proti Osmanům, které vedl Sulejman Nádherný


\end{itemize}

\section*{Kultura}
\begin{itemize}
    \vspace{-0.5em}
    \setlength\itemsep{0.15em}
    \item[$-$] Klaudiánova mapa: neodpovídá realitě, sever je dole
    \item[$-$] Matěj Rejsek: Prašná brána, chrám Sv. Barbory v Kutné Hoře
    \item[$-$] Benedikt Ried (Rejt): Vladislavský sál na Pražském hradě
    \item[$-$] Antonín Pilgram: kostel Sv. Jakuba, portál staré brněnské radnice
\end{itemize}



\end{document}
