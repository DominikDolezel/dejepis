\documentclass{article}
\usepackage{fullpage}
\usepackage[czech]{babel}
\usepackage{amsfonts}

\title{\vspace{-2cm}Sámova říše (623/29 -- 658/59), zkrátka 7. st.\vspace{-1.7cm}}
\date{}
\author{}

\begin{document}
\maketitle
\begin{itemize}
    \vspace{-0.5em}
    \setlength\itemsep{0.15em}
    \item[$-$] Slované
    \begin{itemize}
        \vspace{-0.5em}
        \setlength\itemsep{0.15em}
        \item[$-$] původní sídla mezi Vislou a Dněprem
        \item[$-$] jdou na území, kde dříve byli Germáni
        \item[$-$] přichází ve vlnách v 6. st.
    \end{itemize}
    \item[$-$] \textbf{Kosmas} (1045 -- 1125): literární fikce
    \item[$-$] byzantští historici: \textbf{Prokopios}, \textbf{Jordanes}, Slované $\sim$ Venedové, Autové
    \item[$-$] nejvýznamější zdroj informací: \textit{Historia Francorum}, autor \textbf{Fredegar}
    \item[$-$] polyteismus (Perun, Dažbog, Veles, ...)
    \item[$-$] obživa: pěstování olivovin, luštěnin, keramika pražského typu -- na hrnčířském kruhu, rybolov, \textit{brtnictví} -- sběr včelího medu
    \item[$-$] kmenoví náčelníci
    \item[$-$] \textit{polozemnice} -- příbytek zapuštěný do země
    \item[576] Avaři přichází do Maďarska = Panonie
    \item[(620)] protiavarské povstání, svrhnutí nadvlády Avarů
    \item[$-$] \textbf{Sámo} (623 -- 658)
        \begin{itemize}
            \vspace{-0.5em}
            \setlength\itemsep{0.15em}
            \item[$-$] několikrát porazil Avary, proto si ho zvolili za \uv{krále}
        \end{itemize}
    \item[631] \textsc{bitva u Wogastisburgu}, poražení \textbf{Dagoberta z Meroveovců} (Franská ř.)
    \item[$-$] \textbf{Dagobert I.} si myslel, že na kupcích, kteří prochází přes s. ř. je pácháno zlo, proto záminka k bivě u W.
    \item[$-$] problém s Avary končí na sklonku 8. st. díky tažení \textbf{Karla Velikého}, 3 tažení, zbavení Evropy nebezpečí Avarů
\end{itemize}
\end{document}
