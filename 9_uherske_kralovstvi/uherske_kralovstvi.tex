\documentclass{article}
\usepackage{fullpage}
\usepackage[czech]{babel}
\usepackage{amsfonts}

\title{\vspace{-2cm}Uherské království\vspace{-1.7cm}}
\date{}
\author{}

\begin{document}
\maketitle

\section*{Obecně.}
\begin{itemize}
    \vspace{-0.5em}
    \setlength\itemsep{0.15em}
    \item[896] příchod ugrofinských kočovníků pod vedením \textbf{Arpáda}
    \item[$-$] 7 staromaďarských kmenů
    \item[955] \textsc{bitva na řece Lechu} zastavení Maďarů Čechy a Otou I., stáhnou se do Panonie
    \item[965] \textbf{Gejza z Arpádovců} zakládá první Uherské knížectví, opírá se o křesťnství, centrum: Ostřihom
\end{itemize}

\section*{Vajko = Štěpán I. Svatý (997 -- 1038)}
\begin{itemize}
    \vspace{-0.5em}
    \setlength\itemsep{0.15em}
    \item[$-$] syn Gejzy
    \item[1000] první Uherský král
    \item[1000] v Ostřihomi zřízeno arcibiskupství
    \item[$-$] vyhnal Boleslava Chrabrého ze Slovenska
    \item[$-$] připojuje území Slovenska, Sedmihradsa
    \item[$-$] \textit{komitáty} = \textit{župy} = základní správní jednotka, \textit{župani} = královsští úředníci
    \item[$-$] prohlášen za svatého, svatoštěpánské korunovační klenoty
\end{itemize}

\section*{11. -- 12. st.}
\begin{itemize}
    \vspace{-0.5em}
    \setlength\itemsep{0.15em}
    \item[$-$] připojení Slovenska Slovinska, Slavonie, Chorvatska, Dalmácie
\end{itemize}

\section*{13. st.}
\begin{itemize}
    \vspace{-0.5em}
    \setlength\itemsep{0.15em}
    \item[1222] \textsc{Zlatá bula} Ondřeje II. = osvobození církve a šlechty od daní
    \item[$-$] Béla IV.: boje proti Tatarům
    \item[$-$] stát se stabilizoval díky německé kolonizaci
    \item[1260] \textsc{bitva u Kressenbrunnu} Béla proti Přemyslu Otakaru II., Béla prohrál  
\end{itemize}


\end{document}
