\documentclass{article}
\usepackage{fullpage}
\usepackage[czech]{babel}
\usepackage{amsfonts}

\title{\vspace{-2cm}Vývoj po druhé světové válce\vspace{-1.7cm}}
\date{}
\author{}

\begin{document}
\maketitle

\begin{itemize}
    \vspace{-0.5em}
    \setlength\itemsep{0.15em}
    \item[$-$] důsledky pro Evropu: placení reparací, Japonsko je okupování USA, Čína vychází vítězně, staré koloniální velmoci ztrácí svoje kolonie, Evropa zničena válkou
    \item[$-$] vymezování sfér vlivu
    \item[$-$] prvně spolupráce v polválečném uspořádání, pak se ale vztahy budou ochlazovat
    \item[$-$] supervelmoci: USA a SSSR, vymezování sfér vlivu
    \item[$-$] dekonolizace: kolonie pomáhaly kolonizátorům vyhrát válku  za příslib nezávislosti při válce
    \item[1945-1946] norimebrský tribunál: probíhaly zde dřív sjezdy NSDAP, teď tribunál se špičkovými nacisty, 24 obžalovaných (komicky málo), Göring vystupoval na tribunálo, měl být potrestán trestem smrti, ale spáchal sebevraždu, většina špičkových nacistů utekla do JAnebo spáchala sebevraždu: 12 trestů smrti, 3 doživotí
    \item[$-$] v dalších letech se konaly další tribunály, ale pořád nepodchyceno
    \item[1946-1948] Tokijský tribunál, mšělo být zúčtováno s japonskými nacisty, ale většina jich stejně utekla nebose zabila
    \item[$-$] celkový proces denacifikace v Německu byl komplikovaný, aprotože ke konci Hitler vtahoval do války mladé kluky, po válce, kdyby se tam provedla denacifikace, nikdo by tam nezůstal $\rightarrow$ denacifikace v Německu neproběhla
    \item[1947] Pařížské mírové smlouvy podepsány s cílem potrestat fašistické státy; mírová konference, která měla vyústit v podpisy se konala v roce 1946, o rok později podepsáno, podepsalo Finsko, Maďarsko, Rumunsko, Bulharsko, Itálie, ale Německo, Rakousko a Japonsko byly dočasně dekonstruovány -- rozloženy, neměly vlastní politiku -- okupační zony
    \item[1951] Sanfranciská mírová smlouva s Japonskem, hranice dle roku 1854
    \item[1955] Rakouská státní smlouva: za to, že Rakousko se zavázalo, že bude neutrální, stáhly se okupační armády, Rakousko rozděleno na 4 okupační zony
    \item[$-$] MacArthur: vedoucí okupační správy Japonska
    \item[$-$] dekonolizace: fáze 1945-56 se týkala JV Asie -- nezávislost získají země francouzské Indočíny, Indie, Indonesie, objevují se nové státy jako třeba Izrael, období 56-65: většina států na Africkém kontinentu nezávislost -- rok Afriky, od roku 1965 dál: Mosambik, Angola a takové záležitosti
    \item[1961] hnutí nezúčastněných: země, bývalé kolonie, které už se nechtěly zapojovat do bloků, jsou proti členství ve vojenských blocích, tapojovat se do mocenských bloků

\end{itemize}

\subsection*{Ohniska napětí}
\begin{itemize}
    \vspace{-0.5em}
    \setlength\itemsep{0.15em}
    \item[$-$] první vypuklo v Německo, kde mají západní mocnosti sjednotit, znovu vytvořit německý stát, jenže se začaly sjednocovat jen západní zony -- americká a britská do tzv. bizonie, pak se přidali Francouzi do trizonie
    \item[1947] Marshallův plán, ČSR musela odmítnout kvůli diktátu Moskvy
    \item[$-$] na západě fungují čtyři základní politické strany, v sovětské sféře jenom jedna
    \item[červenec 1948 / květen 1949] součástí nového vznikajícího němeeckého státu má být součástí i Berlín, dochází k tzv. blokádě západního Berlína -- Sověti zablokovali do těch západních částí, vyřešeno leteckými mosty, bylo tam shozeno vše, co lidi potřebovali
    \item[$-$] ústava, vyhlášení států v roce 1949, vyhlášena Spolková republika Německo s hl. městem Bonn, prezident theodor Hess, kancléř Konrad Adenauer, NDR s hl. městem V Berlínem, Wilhelm Pieck
    \item[$-$] druhá berlínská krize: Německo jediný možný prostor, jak emigrovat na západ -- odliv mozků, tehdejší východní Němci to chtěli zastavit, v roce 1961 v čele SSSR je NIkita Sergejevič Chruščov, USA Kennedy, jejich jednání, která to měla vyřešit, byla neúspěšná, v srpnu 1961 se začala stavět berlínská zeď -- taky zeď hanby, když byla v roce 1989 zbořena, bylo to symbolem konce studené války
    \item[$-$] Řecko, občanská válka 1946-1949, vznbikla tam komunistická oposice, která nesouhlasila s monarchií, vystřídali je Američané, kvůli vzniku opozice byla občanská válka, vyhráli monarchisté, komunisti prohráli exodus; vojenská diktatura řeckých plukovníků, po referendu v roce 1974 vzniká republika
    \item[$-$] Blízký východ -- stát Izrael; zvláštní komise pro Palestinu byla pro vznik státu Izrael, bylo vybráno britské mandátní území, Izrael vyhláše v roce 1948, první premiér
    \item[$-$] Blízký východ -- stát Izrael; zvláštní komise pro Palestinu byla pro vznik státu Izrael, bylo vybráno britské mandátní území, Izrael vyhláše v roce 1948, první premiér David Bengurion; protože byl prostor arabský, začaly války s arabským okolím, problém přetrvává dodnes
    \item[$-$] ČÍna: válčí tam Kuomintang, Mao-ce-tung, vyhráli komunisti, Čínská lidová republika, opozice (Čnakajšek, Kuomintang) se stahuje na Tchaj-wan, ze začátku vztahy s Ruskem pozitivní, ale pak nastoupil Chruščov, kritizoval Stalina, ajenže Mao-ce-tung dělal podobné věci, takže se mu to nelíbilo; Mao-ce-tung: velký skok (ekonomické reformy, které způsobily desítky milionů mrtvých, každá domácnost měla mít pec a vyrábět železnou rudu), 60. léta kulturní revoluce, zaměřeno proti všemu starému
    \item[$-$] Číňané anektovali Tibet a postupně likvidovali tamější svatostánky
    \item[$-$] Číňané se taky zapojili do korejské války v letech 1950 až 1953
    \item[$-$] taky navazují vztahy s Čínou -- Richard NIxon navštívil Čínu
    \item[$-$] Deng Xiaopeng: malý velký muž, nastolil tzv. linii velkého pořádku, reformovali kde co, ale otázka lidských práv nebyla vyřešena, studenti se v roce 1989 domáhali řešení
\end{itemize}


\end{document}
