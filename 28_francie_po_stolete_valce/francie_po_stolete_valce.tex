\documentclass{article}
\usepackage{fullpage}
\usepackage[czech]{babel}
\usepackage{amsfonts}

\title{\vspace{-2cm}Francie Habsburkové\vspace{-1.7cm}}
\date{}
\author{}

\begin{document}
\maketitle

\section*{Francie po stoleté válce (1337-1453)}
\begin{itemize}
    \vspace{-0.5em}
    \setlength\itemsep{0.15em}
    \item[$-$] vládnou panovníci z roku Valois
    \item[$-$] jediní, kdo ve Francii pracovali, byli měšťané
    \item[$-$] po stabilizaci Francie přichází expanzivní politika, cílem je Apeninský poloostrov
\end{itemize}

\section*{Karel VIII. (1483-1498)}
\begin{itemize}
    \vspace{-0.5em}
    \setlength\itemsep{0.15em}
    \item[$-$] pokusil se získat Neapolské království, kde kdysi vládla fran. dynastie z Anjou
    \item[$-$] vyfoukl mu ho \textbf{Ferdinand Aragonský} (zakladatel Španělska), na jeho stranu se totiž přiklonila tzv. \textit{benátská liga} (Maxmilián Habsburský, papež, Benátky, Milán)
\end{itemize}

\section*{Ludvík XII. (1498-1515)}
\begin{itemize}
    \vspace{-0.5em}
    \setlength\itemsep{0.15em}
    \item[$-$] dočasně získal Milánské vévodství
    \item[$-$] poté se musí stáhnout pryč, získají to Habsburkové
    \item[$\Rightarrow$] války mezi Francií a Habsburky pokračují
\end{itemize}

\section*{František I. (1515-1547)}
\begin{itemize}
    \vspace{-0.5em}
    \setlength\itemsep{0.15em}
    \item[$-$] válčil se španělským králem \textbf{Karlem V.}
    \item[$-$] \textit{kníže renesance}, na jeho dvoře pobývali třeba Leonardo, Michelangelo, Rafael, Rabelais, Tizian
    \item[$-$] úředním jazykem ve Francii se stává francouzština, rozšiřuje svoji knihovnu
    \item[$-$] Habsburkové v této době drží obrovská území, položí základy Habsburské mnohonárodnostní monarchie, \uv{Habsburkové vládnou říši, které slunce nezapadá}
    \item[$(1525)$]\textsc{bitva u Pávie}, František I. proti Karlu V., vyhrává Karel, František byl zajat
\end{itemize}


\section*{Náboženské války (1562-1589)}
\begin{itemize}
    \vspace{-0.5em}
    \setlength\itemsep{0.15em}
    \item[$-$] mezi katolickou církví a protestanty = \textit{hugenoti}
    \item[$-$] vyvrcholí za tzv. tří Jindřichů, proot se jim říka války tří Jindřichů
    \item[1562] zmasakrování hugenotů ve Wassy, počátek válek
    \item[$-$] u moci je \textbf{Jindřich III.}, v čele katolíků \textbf{Jindřich de Guise}, šel Jindřichovi III. po krku, sám se chce stát králem,
    \item[$\Rightarrow$] 1588 Jindřichem III. zavražděn
    \item[$-$] vůdce hugenotů \textbf{Jindřich de Bourbon}, princ navarský
    \item[23. srepn 1572] busieness nápad: oženit Jindřicha Bourbonského s \textbf{Markétou z Valois} (princezna) = \textit{bartolomějská noc}, ale války dál pokračují
    \item[1589] zlomový rok, kdy byl král Jindřich III. zavražděn $\Rightarrow$ konec rogu Valois, nastupuje dynastie \textbf{Bourbon}
    \item[1593] konvertoval Jindřich Bourbonský ke katolicismu, 1594 korunován na Jindřicha IV.
\end{itemize}


\section*{Jindřich IV. de Bourbon (1594-1610)}
\begin{itemize}
    \vspace{-0.5em}
    \setlength\itemsep{0.15em}
    \item[1598] \textsc{edikt nantský} zrovnoprávňuje katolíky a hugenoty, mohou do státní správy
    \item[$-$] budování infrastruktury, kladení důrazu na venkov, proniknutí Francouzů do Kanady, dočasné snížení daní
    \item[$-$] i po konvertování ke katolicismu podporuje protestanty
    \item[1610] zavražděn katolickým mnichem
    \item[$-$] po Markétě (bezdětné manželství) další sňatek s \textbf{Marií Medicejskou}, syn \textbf{Ludvík XIII.}, ten je při jeho smrti ještě nezletilý, nějakou dobu tedy vládne Marie jako regentka
\end{itemize}


\section*{Ludvík XIII.}
\begin{itemize}
    \vspace{-0.5em}
    \setlength\itemsep{0.15em}
    \item[$-$] dobytí pevnosti La Rochelle, hugenoti tam byli vyhladověni
    \item[$-$] manželka Anna Rakouská
    \item[$-$] kardinál \textbf{Richelieu}, nejdříve na straně Marie, když se však ujal moci Ludvík, byl plně oddán jemu, působil jako první ministr, po přiklonění k Ludvíkovi se distancoval od Habsburků
    \item[1635] \textsc{třicetiletá válka} proti Habsburkům, do ní Francii zavedl právě Richelieu
\end{itemize}



\end{document}
