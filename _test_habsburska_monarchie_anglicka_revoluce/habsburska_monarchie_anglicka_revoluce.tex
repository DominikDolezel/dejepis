\documentclass{article}
\usepackage{fullpage}
\usepackage[czech]{babel}
\usepackage{amsfonts}

\title{\vspace{-2cm}Habsburská monarchie, anglická revoluce\vspace{-1.7cm}}
\date{}
\author{}

\begin{document}
\maketitle

\section*{Habsburská monarchie}

\subsection*{Ferdinand I. }
\begin{itemize}
    \vspace{-0.5em}
    \setlength\itemsep{0.15em}
    \item[$-$] manželka \textbf{Anna Jagellonská} (sestra Ludvíka Jagellonského), bratr Karla V., synovci Johany Šílené
    \item[1515] \textit{Vídeňská smlouva} mezi Habsburky a Jagellonci o nástupnictví, když jeden vymře, nastupuje druhý
    \item[$-$] náboženská otázka, při jeho nástupu 10 \% katolíků, jinak lutheráni, kalvíni, jednota bratrská, kališníci, ale Habsburkové chtějí katolicismus a centralizaci ve Vídni
    \item[1526] král český
    \item[+] Slovensko, Z Uhry, Chorvatsko; zbytek Turci
    \item[(1546/7)] \textsc{1. protiuherský odboj}: stavové odmítli žádost o pomoc Ferdinanda proti luteránství $\Rightarrow$
    \item[$-$] \textsc{Šmalkadská válka}: luteráni poraženi Habsburky (Karel V.), poté i na území Česka (Ferdinand I.)\footnote{I přes Karlovo vítězství nedokázal zvrátit postup reformace a vydání \textsc{Augspurského míru} v roce 1555, který znamenal konec náboženských válek v SŘŘ a fakt, že poddaní mají stejnou víru jako jejich pán (\textit{cuius regio, eius religio}). To Karel považoval za tak velkou prohru, že odstoupil.} $\rightarrow$ Ferdinand do měst dosadil svoje lidi, odevzdali pokuty i zbraně, byl zřízen \textit{apelační} = odvolací soud
    \item[$\Rightarrow$] \begin{itemize}
        \vspace{-0.5em}
        \setlength\itemsep{0.15em}
        \item[$-$] jednota bratrská -- nepříjemní $\rightarrow$ odchází
        \item[$-$] kališníci novoultrakvisté $\rightarrow$ k lutheránům, staroultrakvisté $\rightarrow$ ke katolicismu
        \item[$-$] jezuité (od 1556) Klementinum, arcibiskupství v Praze
    \end{itemize}
\end{itemize}


\subsection*{Maxmilián II. (1564-1576)}
\begin{itemize}
    \vspace{-0.5em}
    \setlength\itemsep{0.15em}
    \item[$-$] syn Ferdinanda I., manželka \textbf{Marie Španělská}, vzdělaný, tolerantní k náboženství
    \item[1562] přísaha věrnosti katolíkům
    \item[1575] volnost vyznání (zatím jen ústní): \textit{Česká konfese} (\textit{Confessio Bohemica}) $\rightarrow$ lutheráni, ultrakvisté a další jsou spokojení, protože mohou vyznávat svoji víru
\end{itemize}

\subsection*{Rudolf II.}
\begin{itemize}
    \vspace{-0.5em}
    \setlength\itemsep{0.15em}
    \item[$-$] syn Maxmiliána II. a jeho sestřenice $\Rightarrow$ pravdepodobne kvuli incestu pozorovatelne psychicke i fyzicke problemy u Rudolfa
    \item[1583] přestěhoval sídlo z Vídně do Prahy $\Rightarrow$ rozkvět Prahy
    \item[$-$] pomáhala mu \uv{španělská strana} -- katolíci, lidé vyslaní papežem
    \item[$-$] katolici ve vysokých pozicích $\Rightarrow$ lidem se to nelíbilo


    \item[$-$] problém s Osmany, kteří ohrožují Evropu, problém se svým ctižádostivým bratrem
    \item[1606] uzavřel s Osmany křehký mír za 200k dukátů (draze), vydržel 20 let
    \item[$-$] bratr \textbf{Matyáš} je místodržitelem Uher, získá si podporu uherských stavů, dále rakouských stavů a moravských stavů $\Rightarrow$ 1608 \textsc{vstoupí do Čech}, donutí Rudolfa podepsat \textit{Libeňskou smlouvu} (25. 6. 1608) -- Matyáš drží Rakousy, král Uherský, Morava, zbylé spravuje Rudolf $\Rightarrow$ Rudolf má titul českého krále, (ovládá Lužici a Slezsko), je nadále císařem, současně souhlasí s tím, že Matyáš bude budoucím českým králem

    \item[(9. 7.) 1609] čeští stavové toho využijí a donutí Rudolfa podepsat \textit{Maiestas Rudolfina} = náboženská svoboda pro všechny (tedy nejen pro vrchnost)
    \item[$-$] stavům svěřena Pražská univerzita, vytvořen sbor 30 defenzorů, kteří kontrolují dodržování náboženské svobody

    \item[1611] pasovský biskup \textbf{Leopold} pošle na žádost Rudolfa žoldáky, kteří měli pomoct Rudolfovi nastolit katolicismus, proti Matyášovi se stavy však nemá šanci $\Rightarrow$ toho roku Rudolf II. abdikuje, 1612 umírá
    \item[$\Rightarrow$]  Matyáš králem a císařem
    \item[1617]  sídlem opět Vídeň

    \item[$-$] známý pro svou rudolfinskou sbírku umění, většinu sebrali Švédové

    \item[$-$] dále se zajímal o alchymii, astronomii, astrologii, zval si na dvůr učence
    \begin{itemize}
        \vspace{-0.5em}
        \setlength\itemsep{0.15em}
        \begin{multicols}{2}
        \item[$-$] Jehuda Löw ben Becalel (prý stvořil Golema, napsal dílo Codex Gigas)
        \item[$-$] Hans von Aachen (malíř, pomáhal Rudolfovi díla nakupovat)
        \item[$-$] Bartolomeus Spranger
        \item[$-$] Giuseppe Arcimboldo (zeleninové ksichty)
        \item[$-$] Johannes Kepler
        \item[$-$] Tycho de Brahe (astronom)
        \item[$-$] Adrian de Vries (sochař, jeho díla jsou ve Valdštejnské zahradě)
        \item[$-$] Mordechaj Maisel (architekt, synagoga)
        \end{multicols}
    \end{itemize}

\end{itemize}


\subsection*{Matyáš Habsburský (1611-1619)}
\begin{itemize}
    \vspace{-0.5em}
    \setlength\itemsep{0.15em}
    \item[$-$] sídlí ve Vídni, bratr Rudolfa II.
    \item[$-$] od 1608 král Uherský, 1611 český, 1612 císař SŘŘ po Rudolfově smrti
    \item[$-$] český stát spravován \textit{katolickými místodržícími}, rekatolizační politika, Rudolfův majestát často porušován
    \item[1617] \textit{Oňatova smlouva}, španělská habsburská větev se zřekla nároku na Habsburskou monarchii $\Rightarrow$ i čeští stavové souhlasí s tím, že se po Matyášově smrti stane králem jejich bratranec \textbf{Ferdinand Štýrský} $\Rightarrow$ vyřešení následnického problému
    \item[$-$]  vzrůstá nespokojenost českých stavů s rekatolizací, scházejí se, posílají Matyášovi petice
    \item[23.5.1618] \textsc{třetí pražská defenestrace}, v čele Jindřich Matyáš Turn, zinscenovali improvizovaný soud, tři osoby vylétly ven z okna: místodržící Vilém Slavata z Chlumu, Jaroslav Bořita z Martinic, písař Fabricius $\Rightarrow$  počátek třicetileté války, protikatolického odboje
    \item[24.5.1618] vytvořena prozatímní protihabsburská vláda 30 direktorů, pouze Nizozemí stavům peněžně pomáhá
    \item[$-$] vojsko špatně placeno, na straně císaře však kvalitní armáda
    \item[1619] Matyáš umírá, nastupuje jeho bratranec
\end{itemize}


\subsection*{Ferdinand II.}
\begin{itemize}
    \vspace{-0.5em}
    \setlength\itemsep{0.15em}
    \item[$-$] zastánce katolicismu, stavové to odmítají, zvolí v srpnu 1619 nového krále \textbf{Fridricha Falckého}, na trůnu jen asi jednu zimu = zimní král, první kalvinista v čele českého státu
    \item[8.11.1620] \textsc{bitva na Bílé hoře} pořážka českých stavů uherskými vojsky, která jsou lépe finančně zajištěna a motivována, Fridrich Falcký prchá do Bratislavy, potom do Haagu
    \item[$\Rightarrow$ ] zlomový moment v českých dějinách, zatýkání účastníků stavovského odboje
    \item[21.6.1621] \textsc{staroměstská exekuce} = poprava odbojářů, celkem 27 lidí, Jan Mydlář, Kryštof Harant z Polžic a Bezdružic,  Jan Jesenský, kat jan Mydlář
    \item[$-$] \textit{Obnovené zřízení zemské} pro Čechy od 1627 a pro Moravu 1628, Habsburkové mají dědičné právo na český trůn, jediné katolické náboženství, návrat jezuitů, obnovení zemského sněmu, který schvaluje třeba daně, má velice omezené pravomoci, podlomení měšťanského stavu, má jenom jeden hlas celkově, němčina zrovnoprávněna s němčinou, inkolát uděluje jen šlechta, rozhoduje jen panovník
    \item[$-$] \textit{inkolát} = udělení příslušnosti ke šlechtě
    \item[$-$] \textit{Mandát Ferdinanda II.} = nekatolická šlechta i měšťanstvo musí odejít, kdo nekonvertuje na katolicismus musí taky odejít $\Rightarrow$ \textit{exulantismus}

\end{itemize}

\section*{Třicetiletá válka}
\begin{itemize}
    \vspace{-0.5em}
    \setlength\itemsep{0.15em}
    \item[$-$] periodizace:
    \begin{enumerate}
        \vspace{-0.5em}
        \setlength\itemsep{0.15em}
        \item válka česká 1618-1620
        \item válka falcká 1621-1623
        \item válka dánská 1625-1629
        \item válka švédská 1630-1635
        \item válka švédsko-francouzská 1635-1648
    \end{enumerate}
    \item[$-$] název války vždy podle oponenta Habsburků

\end{itemize}

\subsection*{Válka česká}
\begin{itemize}
    \vspace{-0.5em}
    \setlength\itemsep{0.15em}
    \item[$-$] skončila bitvou na Bílé hoře porážkou českých stavů, Fridrich Falcký utekl
\end{itemize}

\subsection*{Válka falcká}
\begin{itemize}
    \vspace{-0.5em}
    \setlength\itemsep{0.15em}
    \item[$-$] boje na území SŘŘ, protestanti a Nizozemí (v čele Fridrich Falcký) proti katolíkům (v čele s Maxmiliánem Bavorským) a španělští Habsburkové
    \item[$-$] vítězství Habsburků díky Maxmiliánu Habsburskému, sebere Fridrichovi Horní a Dolní Falc, stává se kurfiřtem v SŘŘ
\end{itemize}


\subsection*{Válka dánská}
\begin{itemize}
    \vspace{-0.5em}
    \setlength\itemsep{0.15em}
    \item[$-$] v čele tehdejší dánsko-norský král \textbf{Kristián IV.} s Nizozemím, Anglií, finanční pomoc Francie, Fridrich Falcký = protihabsburská koalice proti císaři a katolické lize (Maxmilián Habsburský, Albrecht z Valdštejna)
    \item[$-$] díky kardinálu Richelieuovi přispívá i Francie a ostatní nepřítelé Habsburků
    \item[1626] už rozhodnuto díky rozhodující bitvě \textsc{u Dessavy}, vyhrávají Habsburkové
    \item[$\Rightarrow$ ] Albrecht z Valdštejna získává území, centrem jeho panství je Jičín, stává se velitelem všech císařských vojsk
    \item[1629] \textit{restituční edikt}, vše, co katolická církev ztratila po roce 1555 se jí vrátí $\Rightarrow$ kurfiřti tlačí na císaře, aby byl Albrecht propuštěn z vojenských služeb, protože se báli, že jim bude vnucovat katolicicmus $\Rightarrow$
    \item[1630] Albrecht z Valdštejna zbaven pozice v armádě
    \item[1629] \textit{mírová smlouva v Lübecku}, konec války bez reparací a bez územních zisků a ztrát, rozpad protihabsburské koalice

\end{itemize}

\subsection*{Válka děcka švédská}
\begin{itemize}
    \vspace{-0.5em}
    \setlength\itemsep{0.15em}
    \item[$-$] protihabsburská koalice: \textbf{Gustav II. Adolf} (švédský král), Sasko, Braniborsko, Nizozemí, Francie, Rusko prot Habsburkům, v čele maršál Jan Tilly
    \item[$-$] Švédi vtrhli přes Bavorsko do Čech
    \item[1631] v \textsc{bitvě u Breitenfeldu} poráží Habsburky $\Rightarrow$ Albrecht z Valdštejna o rok později povolán zpět do armády $\Rightarrow$
    \item[1632] v  \textsc{bitvě u Lützenu} Habsburkové v čele s Albrechtem vítězí, Gustav II. Adolf zde umírá
    \item[1634] Albrecht II. z Valdštejna zavražděn, protože se císař domníval, že usiluje o českou korunu
    \item[$-$] i po jeho smrti Habsburkové vítězí dál
    \item[1635] \textit{Pražský mír}, Sasko získává Horní Lužici, protože přešli na stranu Habsburků
\end{itemize}

\subsection*{Válka francouzsko-švédská}
\begin{itemize}
    \vspace{-0.5em}
    \setlength\itemsep{0.15em}
    \item[$-$] protihabsburská koalice: Švédsko, Nizozemí a Francie proti Habsburkům, Dánsku, Braniborsku a Sasku
    \item[$-$] ničivá válka, tažení a drancování
    \item[$-$] Portugalsko se připojí na stranu Francie proti Španělsku, Francie vyhraje a Portugalsko získává nezávislost na Španělesku
    \item[$-$] gen. Torstenson, Königsmark, Wrangel
    \item[1645] \textsc{bitva u Jankova} vítězí Švédové, chtějí projít přes Moravu a dobýt Vídeň, chtěli se spojit se Sedmihradským vévodou, ale toho na svou stranu získali Habsburkové $\Rightarrow$ tažení neúspěšné
    \item[(3.5.-23.8.) 1645] obléhání Brna, trvalo čtyři měsíce, v čele Francouz Ludvík Raduit de Souches a město před Švédy ubránil
    \item[1648] obléhání Prahy
    \item[24.10.1648] \textit{Vestfalský mír} dědicové Fridricha Falckého získali Dolní Falc, Severní Nizozemí a Švýcarsko nezávislost, SŘŘ zůstává rozdrobená, náboženská otázka vrácena do roku 1624
\end{itemize}


\section*{Anglická revoluce}

\subsection*{Jakub I. (1603-1625)}
\begin{itemize}
    \vspace{-0.5em}
    \setlength\itemsep{0.15em}
    \item[$-$] nastupuje po Alžbětě I., personální unie Anglie a Skotska
    \item[$-$] podporuje podnikání, rozvoj obchodu a rybolovu na úkor Nizozemí
    \item[$-$] jeho dcera Alžběta byla manželkou Fridricha Falckého
    \item[$-$] za něj vypukne třicetiletá válka, působí Shakespeare, Francis Bacon (filozof, vědec, spisovatel)
\end{itemize}

\subsection*{Karel I. (1625-1649)}
\begin{itemize}
    \vspace{-0.5em}
    \setlength\itemsep{0.15em}
    \item[$-$] syn Jakuba I., snaží se vládnout bez parlamentu, prosazuje absolutismus
    \item[1629] rozpuštěn parlament
    \item[$-$] mimořádné daně nebo dávky bez souhlasu parlamentu, půjčky na měšťanech
    \item[$-$] uplatňuje panovnický monopol na prodej mýdla, vína, uhlí
    \item[$-$] tyto aktivity iniciovaly vytvoření opozice, která je rekrutována z řad kalvinismu:
    \begin{itemize}
        \vspace{-0.5em}
        \setlength\itemsep{0.15em}
        \item[$-$] \textit{puritáni} se snaží očistit církev v duchu kalvinismu, hlavní jádro opozice, významní vlastníci půdy, díky svým názorům pronásledováni
        \item[$-$] \textit{presbyteriáni} připouští zachování konstituční monarchie, třeba bohatí bankéři, obchodníci, chtějí změnit hierarchii stávající církve, chtejí, aby v jejím čele byli volení \textit{presbyteři}
        \item[$-$] \textit{independenti} vůdcem je Oliver Cromwell, chtějí republiku, třeba majitelé manufaktur, menší obchodníci, chtějí volný výklad bible nezávislý na presbyterech
        \item[$-$] \textit{levelleři} chtějí rovnost mužů, vypracovali ústavu (ta nevešla v platnost), chtějí republiku, prosazují náboženskou svobodu
    \end{itemize}
    \item[$-$] ve Skotsku je kalvinismus, Irsko je katolické, Stuartovci tu násilím zaváději anglikánskou církev
    \item[1639] \textsc{Skotské povstání}, skotská armáda vpadla ze severu do Anglie, Karel svolá parlament a chce, aby odsouhlasil daně, aby mohl vytvořit armádu, která porazí Skoty, parlament však odmítá poslušnost a chce potrestat královy rádce, během 14 dnů Karel parlament opět rozpustil = \textit{krátký parlament}
    \item[1640] Karel opět svolává parlament, postupně ho ovládnou presbyteři, parlament nebyl rozpuštěn až do roku 1653 = \textit{dlouhý parlament}
    \item[1642] pokus Karla I. zatknout vůdce opozice v parlamentu, neúspěšné, utíká z Londýna do Osxfordu, začíná otevřená občanská válka
    \item[$-$] stoupenci krále: vyznavači anglikánské církve, stoupenci parlamentu: nová šlechta, kalvinisté, Londýn, Cromwellova armáda (budována na základě dobrovolnosti)
    \item[1644] \textsc{válka u Marston Moor} obrat ve válce, vyhrává armáda parlamentu, velí jí Cromwell
    \item[1645] \textsc{bitva u Naseby} vítězí parlament, král prchá do Skotska
    \item[1647] SKoti krále vydávají parlamentu
    \item[$-$] byl stvořen nový parlament, do kterého se dostanou radikálové (independenti, levelleři) vs. umírnění (presbyteři), radikálové vyhráli $\Rightarrow$
    \item[30.1.1649] král popraven
\end{itemize}

\subsection*{Oliver Cromwell (1649-1658)}
\begin{itemize}
    \vspace{-0.5em}
    \setlength\itemsep{0.15em}
    \item[1649] Anglie republikou, stanul v čele republiky po smrti Karla I.
    \item[1653-58]  rozehnal parlament, nastolil vojenskkou diktaturu $\Rightarrow$ \textit{lord protektor}
    \item[$-$] posílení moci, oporou armáda a podnikatelé
    \item[1651] \textit{navigační akta} = do Anglie může dovážet zboží buď anglická loď, nebo ta, ze které to zboží je $\Rightarrow$ Nizozemci nemohou, protože zboží jen přeprodávají
    \item[$-$] po jeho smrti otevřen prostor k obnovení monarchie
\end{itemize}

\subsection*{Karel II. (1660-1685)}
\begin{itemize}
    \vspace{-0.5em}
    \setlength\itemsep{0.15em}
    \item[1660] stoupenci monarchie v čele s generálem Monckem obnovují monarchii, nastupuje syn Karla I. ze Stuartovců
    \item[1679] \textit{Habeas Corpus Act} = nelze někoho věznit bez doloženého obvinění
    \item[1665-1666] mor (100 000 obětí) a požár v Londýně , díky požáru se Londýn zbavil morové epidemie
\end{itemize}

\subsection*{Jakub II. (1685-1688)}
\begin{itemize}
    \vspace{-0.5em}
    \setlength\itemsep{0.15em}
    \item[$-$] bratr Karla II., prosazuje absolutismus, sesazen parlamentem
    \item[$-$] parlament se obrací na \textbf{Viléma III. Oranžského} (zeť Jakuba II.), dochází k dočasnému spojení Nizozemí a Anglie, vylodí se v Anglii $\Rightarrow$
    \item[1688] \textsc{Slavná revoluce}, nastupuje Vilém Oranžský, chce se odvděčit parlamentu $\Rightarrow$
    \item[1689] \textit{Bill of Rights} = jednoznačně se dělí o moc s parlamentem, nemůže rozhodovat bez jeho souhlasu
    \item[$-$] v dolní sněmovně \textit{torryové} (konzervativní, velkostatkáři), \textit{whigové} (liberální, podnikatelé, buržoazie)
    \item[$-$] zrušení cenzury, prostor pro veřejné mínění
    \item[1701] Vilém nemá následníka trůnu $\Rightarrow$ \textit{Dekret o nástupnictví}, nastoupí Hannoverská dynastie, před nimi však ještě
    \item[$-$] \textbf{Anna Stuartovna} (1702-1714), za její vlády 1707 z personální unie vzniká jednotný stát SPoijené království Velké Británie, 1713 zisk Gibraltaru
\end{itemize}

\subsection*{Jiří I. Hannoverský}
\begin{itemize}
    \vspace{-0.5em}
    \setlength\itemsep{0.15em}
    \item[$-$] Hannoversko dočasně spojeno s Anglií, personální unie
    \item[$-$] z německého prostředí $\Rightarrow$ problém komunikace krále a parlamentu $\Rightarrow$ krále zastupuje premiér, král tedy moc nevládne, roste vliv parlamentu
\end{itemize}




\end{document}
