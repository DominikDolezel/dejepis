\documentclass{article}
\usepackage{fullpage}
\usepackage[czech]{babel}
\usepackage{amsfonts}

\title{\vspace{-2cm}Totalitní režimy\vspace{-1.7cm}}
\date{}
\author{}

\begin{document}
\maketitle

od Hanky
Fašismus, nacismus, stalinismus
+++
-pochod na řím, státní převrat 29.10.1922
-1923 volební zákon – když strana získá 25 %, má nárok na 2/3 křesel v parlamentu  podařilo se to fašistům, ovládli parlament, likvidovali opozici
-postupně se moc změnila v diktaturu, likvidace opozice
-království zachováno, existoval parlament, ale Velká fašistická rada
-Hitler ve 20. letech k Mussolinimu vzhlížel
-ekonomický korporativní režim, v čele korporace musí být fašista
-navzájem podpora, symbióza s katolíky – Lateránské smlouvy

• Po 1.sv. válce : španělsko neutrální, ve 20. letech chudé a ekonomicky zaostalé, vojenská diktatura v čele s Miguel Primo de Riverem – konzervativní; král Alfonzo
• Obrovské státní výdaje, inflace, nebyla pozemková reforma  bouření na venkově
• 1925 vyznamenání bílého lva, ve 30. letech složil funkci
• Čím dál méně lidí chce monarchii, 1931 vyhlášena republika, církev oddělena od státu, ženy hlasovací právo, nová ústava zaručovala autonomii, katalánci a baskové
• 1936 Vláda lidové fronty – komunisti, anarchisti, sociální demokraté
• „Den, kdy padl Madrid“
• Válka začala vojenským převratem ve španělském Maroku, červenec 1936, nacionalisté, nesouhlas s volbami a současnou vládou
• Velitelem frankistů Francisco Franco, jeho stoupenci příznivci falagy, režim blízký italskému fašismu, jiná ústava, věznění opozice, cenzura
• Hitler poskytl frankistům letecký most z Maroka do Španělska, neměli jinak jak transportovat vojáky a zbraně; dále poskytoval zbraně, muže
• Zbraně nacionalistům: Německo, Itálie; republikánům: Mexiko, Sovětský svaz, interbrigády – dobrovolnické sbory z celého světa; politika nevměšování: Francie, Velká Británie
• První nálet na civilní obyvatelstvo v historii: Guernica, duben 1937, vybombardováno Němci, poprvé za cíl zasáhnout civilní obyvatelstvo, německá legie Condor
• Nedařilo se jim dobýt Madrid, posunuli se na sever, ovládli průmyslovou část země; republikáni oslabení rozpory mezi skupinami
• Interbrigády musely být rozpuštěny na výzvu Společnosti národů
• 1.4. 1939 vyhlásil Franco válku za ukončenou
• V reakci na bombardování  Pablo Picasso obraz Guernica

POLSKO
    • Poválečná situace nedobrá, politická roztříštěnost, Pilsudski 1926 premiérem, cíl ozdravit polsko – sanace
    • Zahájeno nové období do 1939, pomalé zlepšení situace, al ehospodářská krize těžce dopadla
    • Režim se stával autoritářský, 1935 nová ústava, vojenský rozkaz, prezident velké pravomoce
    • Josef Bek min zahraničí, pol. Rovnováhy, nesmí být v konfliktu zároveň s německem a sove svazem

MAĎARSKO
    • Stanovení formy vlády, kdo bude hlava státu – 1. úkol
    • Prohlášen admirál Mikklos Horthy – království bez krále, monarcha Regent
    • Karel I. Se dvakrát snažil získat trůn, jednou vojensky, neúspěšně
    • Zákon o detronizaci habsburků
    • Radikální strana ference szalasiho : snažil se provést nacionalistický převrat, zavřen, založil národně socialistickou stranu strana šípových křížů, dostali se k moci na konci války -1944, pak vláda národní jednoty

RAKOUSKO
    • Půjčka od společnosti národů,obnova pomalá, nestabilita vlád, hosp. krize  autoritativní demokracie
    • Engelbert dollfuss (austrofašistická vlastenecká fronta) dostal se k moci legálně, stal se kancléřem, pak rozpustil parlament, vyhlásil vládu 1. strany, pak zavražděn nacisty
    • Kurt Schuschnigg – nový kancléř, Hitler si ho pozval na povídání - ultimátum od němců  ústupky vůči Němcům, nekonečné ústupky, jednal jak chtěl Hitler; pak odmítnul ústupky a podal demisi; chtěl udělat plebiscit o připojení k Německu, ale nestihl se, předtím okupace
    • Novým kancléřem ??? - požádal Němce o obnovení pořádku, rakousko začaly obsazovat německé jednotky, byly vítány, 12.3. 1938 anšlus
    • Hitler plebiscit udělal, ale zfalšovaný, výsledek 99 % pro připojení
    • Květinová válka, vítali Němce květinami

ČESKOSLOVENSKO VE 30. LETECH

    • Čsr postihla velká hospodářská krize
    • Od r. 1929
    • 1929 konec vlády panské koalice, pak předčasné volby – vláda široké koalice – nová vláda, v čele františek udržal, krize nebyla řešena, udržal neudržal
    • Krize kulminuje až 1933, průmyslová výroba poklesla na 60 %, export na 1/3, hutnický průmysl na 1/3, vzrůstá nezaměstnanost, 13 mil obyvatel, přes 900k až milion 300k nezaměstnaných
    • Nejvíce to odskákali dělníci a zemědělci, krize zasáhla zem+ědělství, slovensko a podkarpatská rus, nespokojenost na slovensku, ne autonomie, dopady krize  radikalizace
    • Postiženy oblasti s lehkým spotřebním průmyslem papírnictví, sklářství  Sudety, sudetští němci radikalizace, obecně celá společnost
    • Vzrostl nacionalismus, vláda nechala střílet do demonstrantů, Mostecká stávka, několik mrtvých, vláda střílí do nezaměstnaných, všeobecná nespokojenost, odpor intelektuálů
    • Sudetští němci se organizovali v rámci DNSAP – sesterská NSDAP, dále DNP – organizují se němci
    • Hlinkovci – Andrej Hlinka, v čele HSLS – hlinkova slovenská lidová strana, klerofašistická strana, měli sjezd, postavili se za Vojtěcha tuku, 1928 vydal článek kde říkal že martinská deklarace měla tajný dodatek který nebyl zveřejněn – Slováci pokud nebudou spokojeni, mají se rozhodnout, co dál – maďarský agret, uvězněn vládou ČSR
    • Požadavek autonomie, nakonec skutečně – jednotný autonomistický blok – požadují autonomii, HSLS + další nacionalistická slovenská strana
    • Hlinkovci se zviditelnili srpen 1933, v nitře výročí 1100 let od založení 1. kostela v Nitře, v době velké moravy, vládní delegace přijela, strhli na sebe slovo hlinkovci, štvavá řeč proti čsr
    • Také jsme měli fašisty, národní liga, čeští nacionalisté, v čele karel kramář, satisfakce, ostře proti Němcům, vzorem italský fašismus
    • Existovala Národní obec fašistická, v čele Radola Gajda, nejprve v čele čs armády, degradován, bývalý legionář, národní obec fašistická zorganizovala pokus o převrat fašistický, v brně v kasárnách židenice, snaha o převrat jako v itálii, německu, v roce 1933 se stal kancléřem hitler, bylo to trapné, neorganizované, primitivní; pak byl 6 měsíců vězněn; neúspěšně
    • Vlajka – deklasované živly, čeští fašisté
    • Vláda nereagovala, vládní opatření – zvýšení dávek, rozdali poukázky, jednorázové akce, rozdávali polévky, mléko, podstatu krize ale neřešili
    • Udržal se neudržal, vláda padla 1932, další premiér Jan Malypetr 1932-1935, už řešili krizi, státní zásahy do ekonomiky, 240 nařízení
    • Zákony: zmocňovací zákon – rozšíření pravomocí vlády na úkor parlamentu v hospodářských opatřeních, zákon o protistátní činnosti, umožňoval vládě měnit nestranné soudce,zákon o pozastavení x rozpuštění politických stran, zakázána DNSAP i DNP nejdříve pozastvena činnost pak zakázány; okamžitě se sdružili do jiné strany, sudetoněmecká strana, hemlajnovci; Tiskové zákony, mhli pozastavit tisk pro demokracii nebezpečných periodik

Složitá zahraniční politika
    • Fašisující státy, rakousko, německo, itálie, poláci, maďarsko
    • Beneš začla budovat pohraniční opevnění 1932
    • Stále více jasné že versailleský mírový systém bude porušen
    • Beneš se snažil aktivizovat spolupráci v rámci malé dohody, pakty porganizační a hospodářský – jugoslávie, rumunsko (spíše královské totalitní režimy)
    • Balkánská dohoda – 1934 turecko, rumunsko, řecko, jugoslávie – špatné pro nás, táhne to země směrem na balkán spíše než k nám, později i bulharsko a albánie, minus pro nás, vzdalují se jugoslávci a rumuni
    • 1934 ministr zahraničí beneš se snaží s francouszkým louis barthou o spolupráci, založen východní pakt – gros vytvořila frnacie, čsr, sssr, hráz proti expanzi Hitlera, začali jednat s jugoslávským králem alexandrem I. , nicméně když jednali v marseille, byl na ně spáchán úspěšný atentát
    • Vstup sssr do společnosti národů 1934
    • 1935 hitler vyhlásil branou povinnost, čsr a francie smlouvy se sovětským svazem – do té doby vlády ignorují sovětský svaz, de jure stát neuznali, ignorovali ho
    • Dovětek smlouvy – pomoc nám mohou jen tehdy, ppkud nám pomůže francie – obava ze sssr

    • Obrovský problém

    • Květen 1934 prezidentské volby, TGM, 84 let, abdikoval o rok později
    • Během volby čurbes komunisté, lenin a ne masaryk, nakonec vyhozeni od voleb komunističtí poslanci, byli na ně vydány zatykače, utekli do sssr, gotwald jediný protikandidát Masaryka, neuspěl
    • V roce 1933 zakázána DNSAP, nová strana sudetendeutsche heimatsfront – sudetoněmecká vlastenecká fronta, konrad henlein v čel e
    • Volby, jdou pod názvem sudetoněmecká strana, k henlein, k h frank
    • Výsledky voleb průšvih, nejvíc hlasů němci, 1935 parlamentní volby – SDP 15, 2 %, agrárníci 14, 3, socdem 12,5, ksč 10, 3
    • Podařila se znova vytvořit vláda v čele s agrárníkem janem malypetrem, koalice
    • Slovenské strany ne tolik hlasů
    • Afrární strana nová předseda Rudolf Beran, posouvá ji ke krajní pravici, pak nagrazen milanem hodžou, drží demokratický proud
    • 1935 listopad abdikuje masaryk , prosinec prezidentské volby, edvard beneš, na jeho pozici ministra zahraničí pak K Krofta
    • Kandidát agrárníků u prezidentských voleb, v den voleb ho stáhli

    • 1936 OH v Berlíně, už se vědělo, prosakovaly informace o tom, co dělá, už první koncentráky, ustál to, olympiáda v berlíně vyzněla jako oslava německého sportu, hitler přijal naše sudeťáky, spolčí se
    • Masaryk 14.9.1937 umírá, chmury
    • 12.3. 1938 anexe rakouska, nebezpečná hranice, čechy vykouslé
    •

\subsection*{Itálie}
\begin{itemize}
    \vspace{-0.5em}
    \setlength\itemsep{0.15em}
    \item[$-$] Mussolini postupně zakládá fašistickou stranu PNF
    \item[$-$] působil v novinách
    \item[1921] fašisté se dostávají do parlamentu
    \item[27./29.10.1922] \textsc{pochod na Řím}, státní převrat, Mussolini tam přijel až později vlakem, donutil stávajícího premiéra \textbf{Luigi Facta}, aby abdikoval
    \item[29.10.1922] Beniro Mussolini premiérem
    \item[$-$] volební zákon: strana, která získá alespoň 25 \% hlasů, má nárok na 2/3 křesel v parlamentu
    \item[1925] otevřená diktatura, Velká fašistická rada, likvidace oposice
    \item[1926] teror, de iure je zachován parlament, ale se vším musí souhlasit tzv. Velká fašistická rada, které předsedal Mussolini
    \item[11.2.1929] \textit{lateránské smlouvy} (konkordát = smlouva mezi státem a papežem), navzájem se uznali, Vatikán je nedotknutelný a nezávislý a Řím je hlavní stát Itálie
\end{itemize}

\subsection*{Německo}

\subsection*{Polsko}
\begin{itemize}
    \vspace{-0.5em}
    \setlength\itemsep{0.15em}
    \item[$-$] Pilsudski, chtěl ozdravit Polsko (\uv{sanace})
    \item[$-$] pomalé zlepšení situace, ale pořád se optýkají s velkou hospodářskou krizí
    \item[$-$] režim je čím dál tím víc autoritářský, k tomu přispěla i nová ústava (1935)
    \item[$-$] ministrem zahraničí Josef Beck, který prosazoval politiku rovnováhy (že ne nesmí dostat do konfliktu s Německem a Sovětským svazem)
\end{itemize}

\subsection*{Maďarsko}
\begin{itemize}
    \vspace{-0.5em}
    \setlength\itemsep{0.15em}
    \item[$-$] rozhodnutí nepovolat císaře, který měl nárok an korunu, místo toho Miklós Horthy, který se stal regentem, Maďarsko je královstvím
    \item[$-$] Karel I. Habsburský se chce ujmout trůnu, neúspěšně
    \item[$-$] zákon o detronizaci Habsburků, aby tam nemohli vládnout
    \item[$-$] F. Szalási udělal vzpouru, zavřeli ho a pak založil Stranu šípových křížů, později se dostala i do vlády
\end{itemize}

\subsection*{Rakousko}
\begin{itemize}
    \vspace{-0.5em}
    \setlength\itemsep{0.15em}
    \item[$-$] vzali si půjčku od Společenství národů, obnova však probíhala pomalu
    \item[$-$] časté střídání vlád, navíc byly nechopné, hospodářská krize
    \item[$-$] Engelbert Dollfuss vyhlásil vládu jedné strany (Austrofašistické vlastenecké fronty), byl zavražděň nacisty
    \item[$-$] Kurt Schnuschnigg, kancléř, snažil se navázat kontakty s Německem, předali mu ultimátum, musel obsadit ministry podle jejich představ, později se zalekl německé mobilizace a podal demisi
    \item[$-$] Seyss-Inquart Rakušané vítají německé jednotky a o den později se připojil k nacistickému Německu (anšlus), Hitlerův první krok k obnovení německé říše
\end{itemize}





\end{document}
