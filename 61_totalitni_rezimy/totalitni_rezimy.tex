\documentclass{article}
\usepackage{fullpage}
\usepackage[czech]{babel}
\usepackage{amsfonts}

\title{\vspace{-2cm}Totalitní režimy\vspace{-1.7cm}}
\date{}
\author{}

\begin{document}
\maketitle

\subsection*{Itálie}
\begin{itemize}
    \vspace{-0.5em}
    \setlength\itemsep{0.15em}
    \item[$-$] Mussolini postupně zakládá fašistickou stranu PNF
    \item[$-$] působil v novinách
    \item[1921] fašisté se dostávají do parlamentu
    \item[27./29.10.1922] \textsc{pochod na Řím}, státní převrat, Mussolini tam přijel až později vlakem, donutil stávajícího premiéra \textbf{Luigi Facta}, aby abdikoval
    \item[29.10.1922] Beniro Mussolini premiérem
    \item[$-$] volební zákon: strana, která získá alespoň 25 \% hlasů, má nárok na 2/3 křesel v parlamentu
    \item[1925] otevřená diktatura, Velká fašistická rada, likvidace oposice
    \item[1926] teror, de iure je zachován parlament, ale se vším musí souhlasit tzv. Velká fašistická rada, které předsedal Mussolini
    \item[11.2.1929] \textit{lateránské smlouvy} (konkordát = smlouva mezi státem a papežem), navzájem se uznali, Vatikán je nedotknutelný a nezávislý a Řím je hlavní stát Itálie
\end{itemize}

\subsection*{Německo}

\subsection*{Polsko}
\begin{itemize}
    \vspace{-0.5em}
    \setlength\itemsep{0.15em}
    \item[$-$] Pilsudski, chtěl ozdravit Polsko (\uv{sanace})
    \item[$-$] pomalé zlepšení situace, ale pořád se optýkají s velkou hospodářskou krizí
    \item[$-$] režim je čím dál tím víc autoritářský, k tomu přispěla i nová ústava (1935)
    \item[$-$] ministrem zahraničí Josef Beck, který prosazoval politiku rovnováhy (že ne nesmí dostat do konfliktu s Německem a Sovětským svazem)
\end{itemize}

\subsection*{Maďarsko}
\begin{itemize}
    \vspace{-0.5em}
    \setlength\itemsep{0.15em}
    \item[$-$] rozhodnutí nepovolat císaře, který měl nárok an korunu, místo toho Miklós Horthy, který se stal regentem, Maďarsko je královstvím
    \item[$-$] Karel I. Habsburský se chce ujmout trůnu, neúspěšně
    \item[$-$] zákon o detronizaci Habsburků, aby tam nemohli vládnout
    \item[$-$] F. Szalási udělal vzpouru, zavřeli ho a pak založil Stranu šípových křížů, později se dostala i do vlády
\end{itemize}

\subsection*{Rakousko}
\begin{itemize}
    \vspace{-0.5em}
    \setlength\itemsep{0.15em}
    \item[$-$] vzali si půjčku od Společenství národů, obnova však probíhala pomalu
    \item[$-$] časté střídání vlád, navíc byly nechopné, hospodářská krize
    \item[$-$] Engelbert Dollfuss vyhlásil vládu jedné strany (Austrofašistické vlastenecké fronty), byl zavražděň nacisty
    \item[$-$] Kurt Schnuschnigg, kancléř, snažil se navázat kontakty s Německem, předali mu ultimátum, musel obsadit ministry podle jejich představ, později se zalekl německé mobilizace a podal demisi
    \item[$-$] Seyss-Inquart Rakušané vítají německé jednotky a o den později se připojil k nacistickému Německu, Hitlerův první krok k obnovení německé říše
\end{itemize}





\end{document}
