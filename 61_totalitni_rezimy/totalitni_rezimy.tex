\documentclass{article}
\usepackage{fullpage}
\usepackage[czech]{babel}
\usepackage{amsfonts}

\title{\vspace{-2cm}Totalitní režimy\vspace{-1.7cm}}
\date{}
\author{}

\begin{document}
\maketitle

\subsection*{Itálie}
\begin{itemize}
    \vspace{-0.5em}
    \setlength\itemsep{0.15em}
    \item[$-$] Mussolini postupně zakládá fašistickou stranu PNF
    \item[$-$] působil v novinách
    \item[1921] fašisté se dostávají do parlamentu
    \item[27./29.10.1922] \textsc{pochod na Řím}, státní převrat, Mussolini tam přijel až později vlakem, donutil stávajícího premiéra \textbf{Luigi Facta}, aby abdikoval
    \item[29.10.1922] Beniro Mussolini premiérem
    \item[$-$] volební zákon: strana, která získá alespoň 25 \% hlasů, má nárok na 2/3 křesel v parlamentu
    \item[1925] otevřená diktatura, Velká fašistická rada, likvidace oposice
    \item[1926] teror, de iure je zachován parlament, ale se vším musí souhlasit tzv. Velká fašistická rada, které předsedal Mussolini
    \item[11.2.1929] \textit{lateránské smlouvy} (konkordát = smlouva mezi státem a papežem), navzájem se uznali, Vatikán je nedotknutelný a nezávislý a Řím je hlavní stát Itálie
\end{itemize}

\subsection*{Německo}



\end{document}
