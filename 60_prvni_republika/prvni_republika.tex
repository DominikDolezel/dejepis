\documentclass{article}
\usepackage{fullpage}
\usepackage[czech]{babel}
\usepackage{amsfonts}

\title{\vspace{-2cm}\vspace{-1.7cm}}
\date{}
\author{}

\begin{document}
\maketitle

\begin{itemize}
  \item Československo muselo obsadit pohraničí vojenskou silou, sudety, němci
  \item národností složení -- víc němců než slováků, myšlenka čechoslovakismu -- češi a slováci dvě větve jednoho národa
  \item české země nejindustriálnější částí RU monarchie, sklářství, těžší průmysl, potravinářství, pivovarnictví
  \item slovensko už bylo agrární a podkarpatská rus ta byla úplně ztracená
  \item jsou důležité měnové reformy -- muselo dojít k rychlé měnové odluce od rakouské koruny, za toto všechno děkujeme legendárnímu českému ekonomovi dru. Aloisi Rašínovi, měl obrovské znalosti, až chodící ekonomická encyklopedie, snažil se aby české země neměly vysokou inflaci (i na úkor zaměstanosti), že se musí uskromnit lidé aby se nastartovala ekonomika -- to mu udělalo pár nepřátel, tedy zejména anarchokomunistů -- nakonec ho zastřelili
  \item čtvrtinu rakouských bankovek stáhl z oběhu -- měna posílila, byla zavedena československá koruna v poměru 1:1, povedlo se to rychle, do r. 1922 vyřešil inflaci, položil základy státních rezerv (pokladů)
  \item další důležitou reformou byla ta pozemková -- poválečný parlament zrušil šlechtické tituly, také byly omezeny statky církví a velkostatkářů -- nad 150 ha museli půdu prodat, lesy připadly především státu, tato půda která byla rozprodána (asi 28\% rozlohy ČSR), zásluha Antonína Švehly
  \item sociální situace nicméně ideální nebyla, byli sirotci, vdovy, byla nezaměstanost, nedostatek výrobků, potravin -- typická poválečná krisička
  \item 14.11.1918 v Thunovském paláci na Revolučním národním shromáždění byla zvolena vláda Všenárodní koalice v čele s Karlem Kramářem, bylo asi 20 stran v obecném povědomí: agrárníci (Švehla -- vnitro), sociální demokraté (Fr. Soukup -- spravedlnost), národní demokracie (mladočeši, pravice, Kramář -- předseda, Rašín -- finance), národní socialisté (Václav Klofáč -- obrana), čsl strana lidová (Jan Šrámek -- bez portfeje), slováci (v. šrobár -- ministr slovenska), původně i MRŠ -- min. války (ale atentát z dílny Edvarda Beneše ho utnul (vtip(?)))
  \item potom proběhly volby do obecních zastupitelstev, zvítězila socdem, za nimi agrárníci, v důsledku tohoto Kramář podává demisi, vzniká vláda tzv. rudozelené koalice v čele Vlastimil Tusar
  \item 28.1.1919 založena MUNI, 27.6.1919 Komenského univerzita v Bratislavě (nojo slováci mají výšku fakt super)
  \item 29.2.1920 schválena Ústavní listina republiky Československé -- \uv{Masarykova ústava}, tři šložky, dvoukom. parl., jako dnes ale jiné počty posl. a sen. a jiné věkové limity a délky mandátů
  \item u presidenta už se počítalo s tím, že by mohl nastoupit dr. Edvard Beneš, takže tam byl limit od 35 let
  \item zřízeny nezávislé soudy, ústavní soud (toto trochu výjimka myslim)
  \item v dubnu 1920 probíhají první regulerní parlamentní volby, ústávají socdem, takže druhá Tusarova vláda
  \item joo Masaryk, byly volby presidentské regulerní, v roce 1920 bezkonkurenčně vítězí Tomáš Garrigue Masaryk
\end{itemize}

\end{document}
