\documentclass{article}
\usepackage{fullpage}
\usepackage[czech]{babel}
\usepackage{amsfonts}

\title{\vspace{-2cm}Sjednocovací procesy, Francie, Rusko a Francie ve druhé polovině 19. století\vspace{-1.7cm}}
\date{}
\author{}

\begin{document}
\maketitle

\section*{Sjednocovací procesy}

\subsection*{Itálie}

\begin{itemize}
    \vspace{-0.5em}
    \setlength\itemsep{0.15em}
    \item[$-$] po revolucích v letech 1848 a 1849 rozdrobena, dva proudy
    \item[$-$] Sardinské království, král Viktor Emannuel II., má k ruce hraběte Camilla Bensa di Cavour
    \item[$-$] království Obojí Sicílie pod nadvládou Bourbonů
    \item[$-$]  v čele sjednocovacího procesu je Sardinské království, chtějí sjednotit Itálii pod jejich vládou
    \item[1859] \textsc{války s Rakouskem} (u Magenty a Solferina), Napoleon se spojí se Sardinským královstvím za slib území, vítězí Sardinské království, připojuje si Lombardii, Benátsko zatím zůstává Habsburkům
    \item[$-$] po tomto úspěchu následuje řada povstání
    \item[1860] na základě plebiscitu území pod nadvládou Habsburků taky připojena k Sardinskému království
    \item[$-$] Napoleon se trochu lekl, Cavour se dohodnul s Francouzi a předává jim Nice (jako odměnu za to, že se s nimi spojili)
    \item[1860] \textsc{povstání zemědělců na Sicílii}, na pomoc jim připlul Giuseppe Garibaldi s tisícovkou dobrovolníků, kterému se postupně podařilo ovládnout celé království Bourbonů (Obojí Sicílie)
    \item[$-$] Garibaldi se stává diktátorem
    \item[$-$] po plebiscitu se i království Obojí Sicílie stává součástí Sardinského království
    \item[1861] ve městě Turín vyhlášeno Italské království, tady taky zasedl první italský parlament, králem zůstává Viktor Emannuel II., součástí není ani Benátsko, ani papežský stát
    \item[1866] \textsc{prusko-rakouská válka}, na stranu Pruska se postavilo Italské království, Rakušané jsou sami, Rakušané prohrávají, v důsledku se Benátsko připojuje k Italskému království
    \item[1870] dobytí papežského státu a Říma, protože jeho ochráncem byl Napoleon III., který byl sesazen
    \item[1871] Řím hlavním městem Italského království
    \item[$-$] papeži vyhrazen Vatikán, který se musí zříci světské moci, vztahy Vatikánu  s Italským státem narovnány až ve 20. letech 20. století, do té doby se navzájem neuznávali
    \item[$-$] královským sídlem Kvirinálský palác
    \item[$-$] rozdíly severu a jihu
\end{itemize}

\subsection*{Německo}
\begin{itemize}
    \vspace{-0.5em}
    \setlength\itemsep{0.15em}
    \item[$-$] po revoluci 1848 přetrvává rozdrobenost, překážka rozvoje
    \item[$-$] rozvíjení strojírenského průmyslu, těžby uhlí, chemického průmyslu, rozvoj obchodu
    \item[$-$] \textit{junkeři} = velcí vlastníci půdy
    \item[1861] v čele sjednocování Prusko: Vilém I. z dynastie Hohenzollernů
    \item[$-$] ve volbách do zemského sněmu zvítězili liberálové a demokraté, s nimiž nesympatisoval, proto nechal parlament rozpustit
    \item[$-$] vládu nechal vzniknout uměle, vytvořil si svoji vládu neuznanou parlamentem, první ministr Otto von Bismarck, který chce \uv{sjednotit Německo krví a železem}, jeho oporou jsou junkeři
    \item[1864] \textsc{německo-dánská válka}, Dánové obsazují  Šlesvicko a Hostýnsko (na severu Německa), to se jim nelíbí, nato vypuká válka, chtějí oblast zpět, na straně Němců bojuje i Rakousko, Dánové prohrávají a území ztrácí, za odměnu Rakousku dávají Holštýnsko
    \item[1866] \textsc{prusko-rakouská válka}, na straně Prusů Itálie
    \item[3.7.1866] v \textsc{bitvě u Sadové} Prusové definitivně vítězí
    \item[$-$] Rakousko je vyřezeno ze sjednocovacího procesu Německého spolku
    \item[$-$] Bismarck prosazuje maloněmeckou koncepci sjednocování, vzniká Severoněmecký spolek
    \item[$-$] už i liberálové jsou na straně Bismarcka
    \item[$-$] nová ústava, v čele Pruský král, Říšský sněm, spolková rada, všeobecné hlasovací právo
    \item[$-$] zbývá připojit jih, což je velký problém pro Francii, protože jim roste mocný soused
    \item[1870-1871] Bismarck vyprovokoval \textsc{prusko-francouzskou válku}, jde o to, kdo obsadí španělský trůn, pruský král Vilém I. (z dyn. Hohenzollernů) upraví depeši, Nepoleon vyhlásil válku

\end{itemize}

\subsubsection*{Prusko-francouzská válka}
\begin{itemize}
    \vspace{-0.5em}
    \setlength\itemsep{0.15em}
    \item[$-$] konec Napoleona, Francie republikou
    \item[18.1.1871] obrovská potupa, Němci si ve Versailles vyhlásili sjednocený stát Německé císařství
    \item[$-$] Francie musí platit reparace 5 miliard franků ročně
\end{itemize}

\begin{itemize}
    \vspace{-0.5em}
    \setlength\itemsep{0.15em}
    \item[$-$] po sjednocení Němců tam začíná obrovský ekonomický rozvoj, založení firmy Krupp
    \item[$-$] soustředění dělníků v sociálně-demokratické straně dělnické, jsou pronásledování
    \item[1890] po odchodu Bismarcka konečně končí jejich pronásledování
    \item[$-$] značný vliv má šlechta, problémem je i církev: na severu protestanti, na jihu katolíci, se kterými měl Bismarck problémy, zahájil proti nim tažení = \textit{Kulturkampf}
    \item[$-$] snaha získat kolonie, úspěšně
    \item[1888] novým císařem Vilém II., pokračuje ve zbrojení, proniká do Afriky
\end{itemize}

\section*{Francie}


\begin{itemize}
    \vspace{-0.5em}
    \setlength\itemsep{0.15em}
    \item[$-$] spojeno s Ludvíkem Bonapartem
    \item[20.12.1848] Ludvík Bonaparte se stává presidentem
    \item[1851] na deset let má plnou moc, tato posice je potvrzena plebiscitem, diktatura
    \item[2.12.1852] Napoleon III. císařem
    \item[$-$] vypracována nová ústava, děcka, potvrzovala existenci parlamentu, ale ze začátku neměl žádné pravomoce, všechno dělal Napoleon
    \item[$-$] parlamentní monarchie vytvářena postupně, ze začátku vládne absolutisticky, omezeny občanské svobody (shromažďovací právo)
    \item[$-$] oporou Napoleona jsou bohaté vrstvy, armáda, policie, církev, zemědělci
\end{itemize}

\subsection*{Domácí politika}
\begin{itemize}
    \vspace{-0.5em}
    \setlength\itemsep{0.15em}
    \item[$-$] ralisace průmyslové revoluce, ve velkém se staví továrny, železnice, silnice, velkolepá výstavba v Paříži
    \item[$-$] zmírnění cla na dovážené britské produkty
    \item[$-$] obava z toho, že by se dělníci mohli bouřit, zákaz stávek, chce vyřešit nezaměstnanost $\rightarrow$ podpora výstavby, urbanisace
    \item[$-$] přestavbu vede architekt Haussmann, typická zástavba Paříže pochází právě z této doby
\end{itemize}

\subsection*{Druhé francouzské císařství}
\begin{itemize}
    \vspace{-0.5em}
    \setlength\itemsep{0.15em}
    \item[$-$] koloniální výboje, i když sliboval mír, finančné náročné
    \item[1853-1856] \textsc{krymská válka}, Francouzi na straně Osmanů (Sardinského království) a Velké Británie proti Rusku
    \item[$-$] výprava do Sýrie a Egypta, stavba Suezského průplavu
    \item[1859] podpora Sardinského království v bojích s Habsburky, za odměnu získává Savojsko a Nice
    \item[$-$] proniká do Číny a Indočíny (Vietnam, Laos, Kambodža)
    \item[$-$] v Mexiku vyhlašuje císařství, kam v roce 1861 poslal expedici, protože mexičané nespláceli své dluhy, do čela Mexika prosazuje Maxmiliána Habsburského, nakonec armádu stahuje, Maxmilián zatčen a popraven
    \item[1870-1871] \textsc{válka s Pruskem}, záminka: kdo nastoupí na španělský trůn, Bismarck záměrně upravil Enžskou depeši, vyprovokoval ho k vyhlášení války; pro Napoleona končí tragicky, Francouzi poraženi u Méty, v \textsc{bitvě u Sedanu} kapituluje většina francouzské armády a Napoleon je zajat
    \item[4.9.1870] Francie vyhlášena republikou
\end{itemize}


\subsection*{Pařížská komuna}
\begin{itemize}
    \vspace{-0.5em}
    \setlength\itemsep{0.15em}
    \item[$-$] Francouzi se nechtějí vzdát Alsaska a Lotrinska
    \item[18.1.1871] Němci si vyhlásí ve Versailles Německé císařství v čele s Vilémem I.
    \item[$-$] ještě v lednu Paříž kapituluje, následně podepisují s Němci příměří a předběžnou mírovou smlouvu, ve které Francouzi uznávají ztrátu Alsaska a Lotrinska a francouzská armáda je odzbrojena
    \item[$-$] v Paříži zůstávají ozbrojené jednotky, tzv. \textit{národní gardy}, která chránila Paříž, vláda je chtěla odzbrojit, zbraně si však koupili za vlastní prostředky a to se jim nelíbilo $\rightarrow$ vyhlásili \textit{Pařížskou komunu}
    \item[$-$] levicové síly ovládly Paříž: dělníci, anarchisté
    \item[$-$] Pařížská komuna zavádí pevné ceny potravin, bezplatnou výuku či lékařskou péči, rovnoprávnost žen, povinný podíl dělníků na řízení podniků
    \item[květen 1871] francouzské vládě se podařilo s Němci uzavřít mír, ztratila Alsasko a Lotrinsko, ohradili se proti Komuně
    \item[$-$] poslední boje probíhaly u zdi Pére-Laichaise, kde byla Komuna definitivně zlikvidována, její členové deportováni či popraveni
\end{itemize}


\subsection*{Třetí republika}
\begin{itemize}
    \vspace{-0.5em}
    \setlength\itemsep{0.15em}
    \item[1875] vypracování nové ústavy, všeobecné hlasovací právo, bezplatné vzdělávání a. j.
    \item[$-$] proti republice se ze začátku neúspěšně ohrazují monarchisté, postupně se stabilizuje
    \item[$-$] dvě politické strany:
    \begin{itemize}
        \vspace{-0.5em}
        \setlength\itemsep{0.15em}
        \item[$-$] \textit{radikálové}: na konci století vůdčí politickou stranou, chtějí demokratisaci, odpojení církve od státu, oslabení kompetencí presidenta
        \item[$-$] \textit{socialisté}: na začátku nejednotní, postupně se spojí do jedné strany
    \end{itemize}
    \item[$-$] \textit{Dreifusova aféra}: odsouzen z proněmecké špionáže, odpor intelektuálů, inspirováno antisemitismem
    \item[$-$] na problém antisemitismu poukazuje Émile Zola
    \item[$-$] Eiffelova věž, basilika Sacré-Coeur
    \item[$-$] Louis Blériot přeletěl La Mancheský průliv
\end{itemize}

\subsection*{Kultura}
\begin{itemize}
    \vspace{-0.5em}
    \setlength\itemsep{0.15em}
    \item[$-$] impresionismus: Claude Monet (považován za zakladatele), Piére-Auguste Renoir, Auguste Rodin (sochař)
    \item[$-$] kubismus: Georges Braque (jeden ze zakladatelů kubismu)
    \item[$-$] secese: Alfons Mucha
    \item[$-$] realismus: Gustav Flaubert, Guy da Maupassant
    \item[$-$] naturalismus: Émile Zola
    \item[$-$] symbolismus: Paul Verlain, Arthur Rimbaud
    \item[$-$] postimpresionismus: Paul Cézanne (otec moderního umění), Vincenc van Gogh, Paul Gauguin
    \item[$-$] pointilismus: Georges Seurat
    \item[$-$] expresinismus: Vasilij Kandinskij, Edvard Munch
    \item[$-$] surrealismus: Salvador Dalí, Joan Miró
\end{itemize}


\section*{Rusko}

\begin{itemize}
    \vspace{-0.5em}
    \setlength\itemsep{0.15em}
    \item[$-$] vládnou zde Romanovci, Alexandr I. (vnuk Kateřiny II.)
    \item[$-$] zisk Finska po Vídeňském kongresu
    \item[$-$] při nástupu Mikuláše proti němu mladí důstojníci zorganizovali \textit{povstání Děkabristů}, neúspěšné
    \item[$-$] Rusko je v této době agrární zemí, vyváží obilí, velké množství nevzdělaných nevolníků
    \item[$-$] samoděržaví, pravoslavné náboženství, většinu majetků drží venkovská šlechta
    \item[$-$] Mikuláš I. se snaží podporovat vysokoškolské vzdělání, dosahuje na něj však jenom vysoká šlechta, většina lidí je však pořád nevzdělaná
    \item[$-$] kontrola názorů poddaných, k čemuž sloužila tajná policie $\rightarrow$ jen málo lidí v opozici, mezi nimi byli třeba \textit{slavjanofilové} (kritizovali absolutismus, Rusko se má vrátit ke svým slovanským kořenům a hodnotám, mělo by jít cestou něco mezi absolutismem a liberalismem), \textit{západníci} (vidí vzor na západě, kritizují zaostalost Ruska, jednoznačně preferují liberalismus a demokracii, nejradikálnější ze západníků se potom dostali i do emigrace)


\end{itemize}

\subsection*{Zahraniční politika}
\begin{itemize}
    \vspace{-0.5em}
    \setlength\itemsep{0.15em}
    \item[$-$] potlačil povstání v Polsku, po porážce nastává rusifikace Polska a emigrace Poláků
    \item[1849] pomáhá Habsburkům porazit Uhry u \textsc{Világoše}
    \item[$-$] východní otázka: Rusové chtějí ovládnout přes Černé moře úžiny Bospor a Dardanely a dostat ke Středozemnímu moři, jenže tam jsou Osmané
    \item[$-$] Rusové vědí, že Osmané jsou pro ně samotné moc velký oříšek, proto navrhli Britům, jestli se k nim nepřipojí, nechtějí však
    \item[1853] proto Rusové začají podnikat sami, obsazují Moldavské a Valašské knížectví
    \item[$-$] ruský admirál Nachimov zničil téměř celou tureckou flotilu, jsou ohroženy zájmy Britů -- třeba dovoz obilí
    \item[$-$] do války se na stranu Osmanů přidávají Britové, Francie, Sardinské království
    \item[$-$] drtivá část bojů se odehrála na Krymu
    \item[1856] konec války, mír není pro Rusy pozitivní, protože ztrácí část Besarabie a kontrolu nad Dunajskou plavbou, Bospor a Dardanely byly prohlášeny za neutrální
    \item[$-$] car Mikuláš I. umírá
    \item[$-$] ruská armáda je neschopná, mají zaostalé zbraně, špatné velení, zprávy se přenášejí velmi pomalu, vázne zásobování
\end{itemize}


\subsection*{Alexandr II. (1855-1881)}
\begin{itemize}
    \vspace{-0.5em}
    \setlength\itemsep{0.15em}
    \item[$-$] nový car, pochopil, že tudy cesta nevede
    \item[1861] ruší nevolnictví, mohou se vykoupit z pozice nevolníků nebo si to odpracovat
    \item[$-$] různé reformy školství, soudnictví, správy, armády
    \item[$-$] \textsc{povstání Poláků}, které Rusové opět potlačili
    \item[$-$] vznik nové opozice, říkají si \textit{národníci}, chtějí vyprovokovat masové povstání, které smete cara, hlavním nositelem revoluce má být venkovský lid (národ), mezi nimi i teroristické organisace (Narodnaja volja)
    \item[1881] atentáty na Alexandra, celkem 7, zasáhl ho až ten poslední, kdy umírá v kočáře, na který dopadla bomba
    \item[1867] prodej Aljašky Spojeným státům americkým
\end{itemize}

\subsection*{Alexandr III. (1881-1894)}
\begin{itemize}
    \vspace{-0.5em}
    \setlength\itemsep{0.15em}
    \item[$-$] protože byl Alexandr zavražděn v důsledku atentátu, utužují nevolnictví
    \item[$-$] jinak vede mírovou politiku
\end{itemize}

\subsection*{Mikuláš II. }
\begin{itemize}
    \vspace{-0.5em}
    \setlength\itemsep{0.15em}
    \item[$-$] Rusko expanduje: upevnění pozice v Kavkazu, další zájem o Afghanistán a o Persii (dnešní Írán), jenže tato území chtějí i Britové
    \item[$-$] Britové vyhlašují protektorát v Afghanistánu, nezávislá zůstává část Persie, jednu třetinu získávají Britové a tu druhou Rusové
    \item[$-$] za francouzské peníze probíhá industrializace Ruska (investoři)
    \item[$-$] neúspěšný zájem o Čínu
    \item[1904-1905] \textsc{rusko-japonská válka}, Rusové poraženi u Mukdenu, Cušimě a ztrácí přístav Port Arthur
\end{itemize}

\subsubsection*{Domácí politika}
\begin{itemize}
    \vspace{-0.5em}
    \setlength\itemsep{0.15em}
    \item[$-$] zprůmyslnění Ruska pomáhá zahraniční kapitál, především Moskva a Petrohrad, jinak pořád vyváží zemědělské produkty
    \item[$-$] politické strany vznikají v souvislosti s tím, že roste počet dělníků, sociální demokraté:
    \begin{itemize}
        \vspace{-0.5em}
        \setlength\itemsep{0.15em}
        \item[$-$] \textit{bolševici}, radikálové vedení Leninem, preferují proletářskou revoluci
        \item[$-$] \textit{menševici} standardní západní pojetí marxismu-leninismu
    \end{itemize}
    \item[$-$] dále kadeti a eseři
    \item[22.1.1905] \textsc{krvavá neděle} G. Gapon v Zimnín paláci, strážci začínají střílet do davu, to vyprovokovalo řadu dalších nepokojů a stávek
    \item[$-$] \textit{Říjnovým manifestem} je Rusům poprvé zajištěno volební právo a zřízen zástupní sbor = \textit{Duma}, ekonomické reformy umožňující podnikání
    \item[$-$] po zklidnění situace další utužování absolutismu
    \item[1917] \textsc{únorová revoluce} sesazen, exportován na Sibiř do Tobolska, jakmile se bolševici dostali k moci, převezli celou jeho ordinu do Jekatěrinburgu, kde je bolševici do jednoho vyvraždili
\end{itemize}

\section*{Viktoriánská Anglie}

\begin{itemize}
    \vspace{-0.5em}
    \setlength\itemsep{0.15em}
    \item[$-$] královnou se stává až v roce 1837
    \item[1876] císařovnou Indie
    \item[2. pol. 19. st.] dovršení průmyslové revoluce, v této době nejvyspělejší země světa, v téže době se do popředí dostává i Německo a USA
    \item[$-$] zejména textilní průmysl, koloniální velmoc
    \item[$-$] \uv{královna moří a oceánů}, britské impérium tvoří pětinu světa
    \item[$-$] nejsilnější vojenské i obchodní loďstvo
    \item[$-$] demokratický parlamentarismus, čím dál víc větší roli hraje vláda, která je odpovědná parlamentu, síla královny postupně klesá
    \item[1865] volební reforma, volebná právo majhí všichni vlastníci nemovitostí
    \item[1851] \textit{Londýnská světová výstava}, první světová výstava, využívá dlouhého období míru, v londýnském křišťálovém paláci se presentovaly různé státy, nejvelkolepější měli právě Angličané
\end{itemize}

\subsection*{Rodinný život}
\begin{itemize}
    \vspace{-0.5em}
    \setlength\itemsep{0.15em}
    \item[$-$] manželem její bratranec, princ Albert
    \item[$-$] poslední panovnice z Hannoverské dynastie
    \item[$-$] 9 dětí, 42 vnoučat, manželství po celé Evropě, nazývána \uv{evropskou babičkou}
    \item[$-$] její syn Eduard VII. nastoupí už jako dynastie Windsor
\end{itemize}

\subsection*{Vláda}
\begin{itemize}
    \vspace{-0.5em}
    \setlength\itemsep{0.15em}
    \item[$-$] sídlila v nově vybudovaném Buckinghanském paláci
    \item[$-$] rádcem její strýc, tehdejší belgický král Leopold
    \item[$-$] během její vlády na ni bylo neúspěšně spácháno několik atentátů
\end{itemize}

\subsection*{Irsko}
\begin{itemize}
    \vspace{-0.5em}
    \setlength\itemsep{0.15em}
    \item[1845] postiženo neúrodou brambor, více než milion životů a emigrace dalšího milionu lidí, tehdy turecký sultán nabídl větší peněžní pomoc než Anglie
\end{itemize}

\begin{itemize}
    \vspace{-0.5em}
    \setlength\itemsep{0.15em}
    \item[1861] smrt prince Alberta, po celý zbytek života nosí černý šat jako symbol
    \item[1901] umírá
\end{itemize}


\subsection*{Politická situace}
\begin{itemize}
    \vspace{-0.5em}
    \setlength\itemsep{0.15em}
    \item[$-$] ve vládě se střídají konzervativci (\textit{thoriové}) a liberálové (\textit{whigové}), vzniká také dělnická strana: \textit{Labour party}
    \item[$-$] na konci století se konstituuje \textit{hnutí sufrežetek} -- ženy, jež usilují o volební právo
    \item[$-$] těsně před první světovou válkou je premiérem    David George
    \item[$-$] zrušení obilních zákonů, jež chránily anglický trh před dovozem levného obilí ze zahraničí, to bylo nevýhodné pro městské obyvatelstvo, proti těmto zákonům vystupují jak whigové, tak jeden z toryovských premiérů
    \item[$-$] \uv{královna moří a oceánů}, nejvýznamnější kolonií je Indie, která byla pod Anglickou správou do roku 1704, ta byla zrušena kvůli povstání Sikhů?, poté to není Východoindická společnost, ale  něco , aerlfjvbasfbvasdlkh viktorie je císařství, Viktorie císařovna
    \item[$-$] Britové z Indie pronikají i do okolních zemí jak Afghánistánu a Persie, nad Afghánistánem vyhlásili protektorát a Persii třetina nezávislá, třetina Rusové a třetina Britové
    \item[$-$] ještě za existence Východoindické společnosti problém vyvážení opia do Číny, Číňani jsou závislí na opiu, vypuknutí \textsc{opiových válek}, britové vyhráli, Čína má otevřít pět svých přístavů, Hong Kong přišel pod britskou správu
    \item[$-$] v dalších dvou opiových válkách se k Británii připojila i Francie, opět úspěšné
    \item[1865] radikální volební reforma, mohou volit všichni, kteří mají nějakou nemovitost
    \item[$-$] liberálové prosazují školní docházku, omezují privilegia anglikánské církve, zatahovali liberalismus do ekonomiky, konzervativci omezují pracovní dobu na maximálně 56 hodin týdně, snaží se zlepšit životní péči, díky nim taky Britové drží Kypr, protože podporovali expanzní politiku
    \item[$-$] \textsc{búrské války} s Holanďany v Jižní Africe, vyhrávají Britové
    \item[$-$] na zákoladě snah kanady a Austrálie o nezávislost jim Británie udělila status dominia, to jest status větší nezávislosti a samosprávy
    \item[$-$] problematická situace s Irskem, kteří dlouho usilovali o odtržení, až po první světové válce se v roce 1921 se Severní a Jižní Irsko oddělilo
\end{itemize}





\end{document}
