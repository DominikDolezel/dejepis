\documentclass{article}
\usepackage{fullpage}
\usepackage[czech]{babel}
\usepackage{amsfonts}

\title{\vspace{-2cm}Renesance\vspace{-1.7cm}}
\date{}
\author{}

\begin{document}
\maketitle

\begin{itemize}
    \vspace{-0.5em}
    \setlength\itemsep{0.15em}
    \item[$-$] 14. – 16. st.
    \item[$\approx$] humanismus; znovuzrození, návrat k antické kultuře
    \item[$-$] rozkvět severoitalských měst (Florencie, Benátky, Janov, Řím)
    \item[$-$] \textbf{antropocentrismus} = obrat k člověku a jeho schopnostem
    \item[$-$] nárůst světské moci, rozvoj poznání, obchodu
    \item[$-$] italové se seskupuji do politicko-vojenských skupin = \textbf{guelfové} (za SSŘ) a \textbf{ghibelini} (ZA papeže)
\end{itemize}

\begin{enumerate}
    \vspace{-0.5em}
    \setlength\itemsep{0.15em}
    \item \textit{Torecento} (14. st.)
        \begin{itemize}
            \vspace{-0.5em}
            \setlength\itemsep{0.15em}
            \item[$-$] na pomezí mezi středověkem a renesancí
            \item[$-$] cílem je vytvořit silnou společnost a sjednotit Itálii
            \item[$-$] Dante Alighieri (spisovatel, politik); Cola di Rienzo, Francesco Petrarca (chtěli sjednotit Apeninský polostrov); Giovanni Boccaccio: Dekameron; Giotto di Bondone
        \end{itemize}
    \item \textit{Quattrocento} (15. st.)
    \begin{itemize}
        \vspace{-0.5em}
        \setlength\itemsep{0.15em}
        \item[$-$] pochopili, že sjednotit je nemožné, nejvýznamějším městem je Florencie:
        \item[$-$] Cosimo de Medici (republikánské zřízení = \textbf{signoire})
        \item[$-$] Lorenzo de Medici (největší rozkvět Florencie)
        \item[$-$] Filippo Brunelleschi (dóm ve Florencii)
        \item[$-$] Donatello: bronzový David, Sandro Botticelli: Zrození Ven(o)uše, Primavera
        \item[$-$] architektura: naprosto horizontální, prvky antiky, geometričnost, symetrie, \textbf{arkády} = podloubí, \textbf{sala terrena} = přízemní místnost otevřená do zahrady, \textbf{rustika} = zdivo z neopracovaných kvádrů
        \item[] přelom 15. a 16. st.:
        \item[$-$] Leonardo da Vinci: vynálezce, malíř, architekt, sochař, působil zejména ve FLorencii a pak u rodu Sforzů v Miláně, zemřel ve Francii; Madonna v jeskyni, dáma s hranostajem, Svatá Anna, Vitruvius = dokonale proporční člověk, Mona Lisa, Poslední večeře Páně (v Miláně)
        \item[$-$] Raffael Santi: Sixtinská Madonna (ideál tělesné a duševní krásy), Athénská škola, výzdoba Vatikánu
        \item[$-$] Michelangelo: sochař, malíř, architekt, ale psal i básně, spojen s Florencií a Římem; stavba Svatopetrské baziliky, Sixtinská kaple (soukromá kaple jednoho z papežů), palác Farnese v Římě, palác Senátorů v Římě, socha Mojžíše, Pieta, Davida
    \end{itemize}
    \item \textit{Cinquecento} (16. st.)
    \begin{itemize}
        \vspace{-0.5em}
        \setlength\itemsep{0.15em}
        \item[$-$] spojeno s Benátkami
        \item[$-$] malíři znají anatomii lidského těla a nebojí se přiznat lidskou nahotu, realistické obrazy
        \item[$-$] Masaccio (patří do kvatročenta wtf)
        \item[$-$] Tizian: Venuše urbinská, Apollo a Marsyas (v arcibiskupském zámku v Kroměříži), autoportrét
        \item[$-$] Tintoretto
    \end{itemize}
\end{enumerate}


\end{document}
