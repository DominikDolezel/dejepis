\documentclass{article}
\usepackage{fullpage}
\usepackage[czech]{babel}
\usepackage{amsfonts}

\title{\vspace{-2cm}Velká Morava\vspace{-1.7cm}}
\date{}
\author{}

\begin{document}
\maketitle
\begin{itemize}
    \vspace{-0.5em}
    \setlength\itemsep{0.15em}
    \item[$-$] vznik v první třetině devátého st.
    \item[$-$] \textit{Moravia Magna}
\end{itemize}

\section*{Vznik}
\begin{itemize}
    \vspace{-0.5em}
    \setlength\itemsep{0.15em}
    \item[$-$] Moravské knížectví, \textbf{Mojmír}
    \item[$-$] Nitranské knížectví, \textbf{Pribina}
    \item[$-$] křesťanství
    \item[$-$] Mojmír pokřťen v Pasově biskupem \textbf{Reginhardem}
    \item[833] Mojmír ovládl území Pribiny, t. j. Nitranské knížectví, vyhnal ho $\rightarrow$ základy VM
    \item[$-$] informace máme od: \textbf{Konstantina Porfyrogennéta} -- byzantský císař, vládl v 1. pol. 10. st.
    \item[$-$] \textit{Letopisy (anály) Fuldské}
\end{itemize}

\section*{Mojmír I. (833 -- 846)}
\begin{itemize}
    \vspace{-0.5em}
    \setlength\itemsep{0.15em}
    \item[853] Verdunská smlouva, rozpad Franské říše, Frankové velkým nebezpečím
    \item[$-$] uznává svrchovanost východofanského panovníka
    \item[$-$] platil tribut, aby Frankové neúzočili
\end{itemize}

\section*{Rastislav I. (846 -- 870)}
\begin{itemize}
    \vspace{-0.5em}
    \setlength\itemsep{0.15em}
    \item[$-$] nechce platit tribut, chce VM vymanit z franského vlivu
    \item[$-$] vyhnal franské a bavorské duchovenstvo (protože šíří v latině)
    \item[(861)] obrací se na papeže \textbf{Mikuláše I.}, jestli by mu neposlal nějaké duchovenstvo, odmítá
    \item[(863)] obrací se na \textbf{Michala III.}, ten mu pošle \textbf{KOnstantina} (vystudoval teologii, filozofii, literaturu) a \textbf{Metoděje} (právnické vzdělání), pochází ze Soluně
\end{itemize}

\section*{Bratři ze Soluně}
\begin{itemize}
    \vspace{-0.5em}
    \setlength\itemsep{0.15em}
    \item[$-$] vytvořili \textit{hlaholici}
    \item[$-$] přeložení částí bible
    \item[$-$] mluví staroslověnsky
    \item[$-$] Proglas = předzpěv k Evangeliu, Kyjevské listy
    \item[$-$] Zákon sudnyj ljudem
    \item[$-$] zakládají církevní školy
    \item[$-$] přineseení ostatků sv. Klimenta
    \item[$-$] jejich působení trnem v oku tzv. \textit{trojjazyčníkům} (lidé vedoucí bohoslužby v latině, řečtině a hebrejštině), kteří na ně útočí $\rightarrow$ vydají se do Říma
    \item[867] Řím, staroslověnština potvrzena jako plnoprávný bohoslužebný jazyk (Hadrián II.)
    \item[(869)] Konstantin -- Cyril umírá
    \item[$-$] na Moravu se vrací jen Metoděj, který byl potvrten MoravskoůPanonským arcibiskupem
    \item[880] \textit{bula Industria tuae} opět potvrzuje platnost staroslověnštiny
\end{itemize}

\section*{Svatopluk (871 -- 894)}
\begin{itemize}
    \vspace{-0.5em}
    \setlength\itemsep{0.15em}
    \item[$-$] Rastislavův synovec
    \item[$-$] skonšil ve vězení, na jeho místo dosazeni Vilém a Englšalk
    \item[$-$] oni požádali Svatopluka, aby jim pomohl v bojích, ten však přeběhl k Moravanům, porazili a vyhnali franky
    \item[(874)] \textsc{Forcheimský mír}, Svatopluk slibuje, že bude odvádět mírový tribut Frankům, za odměnu získá nezávislost
    \item[$-$] v této době se vrací Metoděj, překlad bible do staroslověnštiny, později zničeno
    \item[(883)] křest Bořivoje a Ludmily
    \item[$-$] nitranský biskup \textbf{Wiching} chce na VM latinské bohoslužby
    \item[$-$] územní rozsah: Čechy, morava, lužice, slezsko, krakovsko, panonie MAPA PROSÍM, max. územní rozsah VM
    \item[(885)] smrt Metoděje, jejich žáci vyhnáni do Bulharska, na jeho ísto latinští kazatelé
\end{itemize}

\section*{Mojmír II. (894 -- 906)}
\begin{itemize}
    \vspace{-0.5em}
    \setlength\itemsep{0.15em}
    \item[$-$] období vnitřních rozkladů
    \item[$-$] útoky z Váchodofranské říše
    \item[$-$] odtrhnutí Čech od VM
    \item[906/7] vpád Maďarů $\rightarrow$ konec VM
\end{itemize}

\section*{Kultura}
\begin{itemize}
    \vspace{-0.5em}
    \setlength\itemsep{0.15em}
    \item[$-$] \textit{hradiště} -- centra řemesla, obchodu (Staré město u Uherského Hradiště, Valy u Mikulčic, Děvín, \dots)
    \item[$-$] řemeslo, šperkařství
    \item[$-$] křesťané $\rightarrow$ kostrové hroby
    \item[$-$] architekrura: dlouhé kostely (Kopčany -- kostel sv. Markéty), baziliky, rotundy
    \item[$-$] \textit{gombíky} = knoflíky
    \item[$-$] hrnčířský kruh
    \item[$-$] směnný obchod, později první platidla \textit{platýnky}, kůže, hřivny železa
\end{itemize}




\end{document}
