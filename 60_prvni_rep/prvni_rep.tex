\documentclass{article}
\usepackage{fullpage}
\usepackage[czech]{babel}
\usepackage{amsfonts}

\usepackage{fontspec}
\usepackage{sectsty}
\newfontfamily\Kapitan{Kapitan-Medium}
\allsectionsfont{\Kapitan}

\setmainfont{OpenSans}

\title{\vspace{-2cm}\Kapitan První republika\vspace{-1.7cm}}
\date{}
\author{}

\begin{document}
\maketitle
\begin{itemize}
    \vspace{-0.5em}
    \setlength\itemsep{0.15em}
    \item[28.10.1918] vyhlášena republika
    \item[$-$] ČSR vzešla na základě toho, že zaniká Rakousko-Uhersko, je pro nás zásadní, aby byl Versailleský mírový systém dodržován, dostali jsme Hlučínsko a nějaká území od nového Rakouska
    \item[$-$] \textit{Trianonská smlouva} s Maďarskem: získání Zakarpatské Rusi a Slovenska
    \item[$-$] Malá dohoda: ČSR, Rumunsko a budoucí Jugoslávie, orientace na Francii
    \item[$-$] president Masaryk, ministr zahraničí Edvard Beneš
    \item[$-$] i po vyhlášední státu trvalo dva roky, než se Československ árepublika konstituovala
    \item[11.11.1918] problám po vyhlášení státu Poláků, kteří obsadili značnou část Těšínska, argumentovali tím ,že se tam hovoří většinově polsky, ale už dlouho toto území patřilo k českému království, Poláci chtěli, abi tamní obyvatelé volili do parlamentu, měli být i povoláni do poslké armády; na to zareagovali Češi, především legionáři tzv.
    \item[(23.-30.1.1919)] {\Kapitan sedmidenní válkou} $\rightarrow$ zabrání Těšínska, velmoci donutily se ČSR stáhnout
    \item[1920] \textit{arbitráž ve Spa}, dohoda Dohody, kudy povede hranice mezi ČSR a Polskem $\rightarrow$ neurčila to válka, ale velmoci; Bohumínsko-košická dráha je na našem území, a zůstaly tam i doly, zbytek mají Poláci, došlo i k rozpůlení Těšínska na český a polský
    \item[$-$] POláci do nikdy nezkousli a po podepsání Mnichovské dohody si Těšín hned nárokovali
    \item[$-$] další problém je se Slovenskem, Maďaři odtamtud nechtěli
    \item[$-$] bylo vytvořeno ministerstvo pro Slovensko (ministrem Vavro Šrobár)
    \item[leden 1919] Češi museli vojensky obsadit Slovensko
    \item[$-$] na území Maďarska se vytvořila bolševická vláda Maďarská republika rad, Čechoslováci překročili hranice Maďarska s cílem porazit tuto vládu, to se nepodařilo a Maďaři šli do protiútoku a obsadili dokonce kus Slovenska, v části Slovenska zase vznikla Slovenská republika rad
    \item[$-$]  na popud Francie Maďaři vyhnáni a Slovenská republika rad byla zrušena
    \item[1920] situace vyřešena \textit{Červnovou mírovou smlouvou} v Trianonu, kde byly stanoveny hranice mezi Slovenskem a Maďarskem
    \item[$-$] to se nelíbí Maďarům ,protože ztratili Slovensko, zůstává tam maďarské obyvatelstvo
    \item[$-$] za Mnichovské dohody Maďaři část Slovenska získali a obsadili
    \item[říjen 1918] sešel se Masaryk s Rusíny ve Spojených státech, kde se dohodli na  tom, že Rusíni chtějí být součástí Českoslovesnké republiky, musí to ještě potvrdit oficiální politická reprezentace
    \item[8.5.1919] centrem Podkarpatské Rusi je Užhorod, tam se potvrdilo, že to bude součást ČSR
    \item[$-$] pro ČSR spíše nevýhodné, území je velmi zaostalé, na tomto území prochází maďarizace, žádné vzdělání
\end{itemize}

\subsection*{České pohraničí}
\begin{itemize}
    \vspace{-0.5em}
    \setlength\itemsep{0.15em}
    \item[$-$] už před první světovou válkou se používá termín \textit{sudety}
    \item[$-$] nějakých 25 \% obyvatelstva ČSR jsou menšiny
    \item[$-$] po vyhlášení republiky byly vytvořeny čtyři německé provincie (Jihlava, Olomouc, Brno byly také silné německé enklávy)
    \item[$-$]
\end{itemize}


\end{document}
