\documentclass{article}
\usepackage{fullpage}
\usepackage[czech]{babel}
\usepackage{amsfonts}

\title{\vspace{-2cm}Sjednocení Itálie\vspace{-1.7cm}}
\date{}
\author{}

\begin{document}
\maketitle

\begin{itemize}
    \vspace{-0.5em}
    \setlength\itemsep{0.15em}
    \item[$-$] po revolucích v letech 1848 a 1849 rozdrobena, dva proudy ()
    \item[$-$] Sardinské království, král Viktor Emannuel II., má k ruce hraběte Camilla Bensa di Cavour
    \item[$-$] království Obojí Sicílie pod nadvládou Bourbonů
    \item[$-$]  v čele sjednocovacího procesu je Sardinské království, chtějí sjednotit Itálii pod jejich vládou
    \item[1859] \textsc{války s Rakouskem} (u Magenty a Solferina), Napoleon se spojí se Sardinským královstvím za slib území, vítězí Sardinské království, připojuje si Lombardii, Benátsko zatím zůstává Habsburkům
    \item[$-$] po tomto úspěchu následuje řada povstání
    \item[1860] na základě plbisticu území pod nadvládou Habsburků taky připojena k Sardinskému království
    \item[$-$] Napoleon se trochu lekl, Cavour se dohodnul s Francouzi a předává jim Nice (jako odměnu za to, že se s nimi spojili)
    \item[1860] \textsc{povstání zemědělců na Sicílii}, na pomoc jim připlul Giuseppe Garibaldi s tisícovkou dobrovolníků, kterému se postupně podařilo ovládnout celé království Bourbonů (Obojí Sicílie)
    \item[$-$] Garibaldi se stává diktátorem
    \item[$-$] po plebiscitu se i království Obojí Sicílie stává součástí Sardinského království
    \item[1861] ve městě Turín vyhlášeno Italské království, tady taky zasedl první italský parlament, králem zůstává Viktor Emannuel II.
    \item[1866] \textsc{prusko-rakouská válka}, na stranu Pruska se postavilo Italské království, Rakušané jsou sami, Rakušané prohrávají, v důsledku se Benátsko připojuje k Italskému království
    \item[1870] dobytí papežského státu a Říma, protože jeho ocháncem byl Napoleon III., který byl sesazen
    \item[1871] Řím hlavním městem Italského království
    \item[$-$] papeži vyhrazen Vatikán, který se musí zříci světské moci, vztahy Vatikánu  s Italským státem narovnány až ve 20. letech 20. století, do té doby se navzájem neuznávali
    \item[$-$] královským sídlem Kvirinálský palác
    \item[$-$] rozdíly severu a jihu
\end{itemize}

\subsection*{Německo}
\begin{itemize}
    \vspace{-0.5em}
    \setlength\itemsep{0.15em}
    \item[$-$] po revoluci 1848 přetrvává rozdrobenost, překážka rozvoje
    \item[$-$] rozvíjení strojírenského průmyslu, těžby uhlí, chemického průmyslu, rozvoj obchodu
    \item[$-$] \textit{junkeři} = velcí vlastníci půdy
    \item[1861] v čele sjednocování Prusko z dynastie Hohenzollernů
    \item[$-$] ve volbách do zemského sněmu zvítězili liberálové a demokraté, s nimiž nesympatisoval, proto nechal parlament rozpustit
    \item[$-$] vládu nechal vzniknout uměle, vytvořil si svoji vládu neuznanou parlamentem, první ministr Otto von Bismarck, který chce \uv{sjednotit Německo krví a železem}, jeho oporou jsou junkeři
    \item[1864] \textsc{německo-dánská válka}, D8nové obsazují  Šlesvicko a Hostánsko (na severu Německa), to se jim nelíbí, nato vypuká válka, chtějí oblast zpět, na straně Němců bojuje i Rakousko, Dánové prohrávají a území ztrácí, za odměnu Rakousku dávají Holštýnsko
    \item[1866] \textsc{prusko-rakouskí válka}, na straně Prusů Itálie
    \item[3. 7. 1866] v \textsc{bitvě u Sadové} Prusové definitivně vítězí
    \item[$-$] Rakousko je vyřezeno ze sjednocovacího procesu Německého spolku
    \item[$-$] Bismarck prosazuje maloněmeckou koncepci sjednocování, vzniká Severoněmecký spolek
    \item[$-$] už i liberálové jsou na straně Bismarcka
    \item[$-$] nová ústava, v čele Pruský král, Říšský sněm, spolková rada, všeobecné hlasovací právo
    \item[$-$] zbývá připojit jih, což je velký problém pro Francii, protože jim roste mocný soused
    \item[1870-1871] Bismarck vyprovokoval \textsc{prusko-francouzskou válku}, jde o to, kdo obsadí španělský trůn, pruský král Vilém I. chce, aby se Hohenzollernové vzdali nároku na španělský trůn, jakási depeše byla Vilémem upravena tak, že Nepoleon vyhlásil válku

\end{itemize}

\subsection*{Prusko-francouzská válka}
\begin{itemize}
    \vspace{-0.5em}
    \setlength\itemsep{0.15em}
    \item[$-$] konec Napoleona, Francie republikou
    \item[18.1.1871] obrovská potupa, Němci si ve Versailles vyhlásili sjednocení stát Německé císařství
    \item[$-$] Francie musí platit reparace 5 miliard franků ročně
\end{itemize}


\end{document}
