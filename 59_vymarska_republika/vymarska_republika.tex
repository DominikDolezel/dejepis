\documentclass{article}
\usepackage{fullpage}
\usepackage[czech]{babel}
\usepackage{amsfonts}

\usepackage{fontspec}
\usepackage{sectsty}
\newfontfamily\Kapitan{Kapitan-Medium}
\allsectionsfont{\Kapitan}

\setmainfont{OpenSans}

\title{\vspace{-2cm}\Kapitan Výmarská republika\vspace{-1.7cm}}
\date{}
\author{}

\begin{document}
\maketitle

\begin{itemize}
  \vspace{-0.5em}
  \setlength\itemsep{0.15em}
  \item[9.11.1918] zrušeno císařství, vyhlášena demokratická republika
  \item[leden 1919] volby do parlamentu (do této doby prozatímní vláda s předs. Friedrichem Ebertem /SD/)
  \item[únor 1919] parlament se sejde ve Výmaru, zde jde sepsána i ústava
  \item[srpen 1919] ústava, je prezident (první Fr. Ebert), kancléř (ekvivalentní s naším předsedou parlamentu) s vládou, dvoukomorový parlament (Říšský sněm -- volení, Říšská rada -- jmenování zemskými vládami podle místní příslušnosti), hlavním městem Berlín
  \item[$-$] četná povstání (Bavorská republika rad, \dots)
  \item[březen 1920] \textsc{Kappův puč}  (Wolfgang Kapp) v Berlíně, pokus o pravicový puč, nesouhlas s rozpuštěním armády
  \item[1921] stanovena částka reparací na 132 mld. marek, což je astronomické číslo
  \item[$-$] vláda s tím nesouhlasí, volí metodu tzv. \textit{pasivní resistence}:  nic nevyráběli, aby jim to Dohoda nezabavila apod., Dohoda obsadí Porúří (1923), aby si reparace vydobili
  \item[$-$] i z těchto důvodu německá marka čelí hyperinflaci (1923 litr mléka stojí 4,2 milionů marek)
  \item[8./9.11.1923] \textsc{mnichovský pivní puč}: v Mnichově v nějaké pivnici byly \uv{bavorské politické špičky}, Hitler tam nakráčel s SA, chtěl po nich něco, oni mu nic nedali, SA s Hitlerem šli pochodovat do ulic, pak je všechny zatkli
  \item[$-$] Hitler je vězněn, v této době sepisuje \textbf{Mein Kampf}
  \item[září 1923] odstoupeno od taktiky pasivní resistence, nový kancléř
  \item[listopad 1923] měnová reforma
  \item[1924] \textit{Dawsův plán}: Američané chtěli z Němců reparace dostat, pokud německá ekonomika nebude fungovat tak nikdy nic nezaplatí, čili platby se zmírnily, formou různých půjček, úvěrů nastartovali německou ekonomiku
  \item[1925] nový prezident \textbf{Paul von Hindenburg}
  \item[192] \textit{Locarnský garanční pakt}: Německo slibuje, že jeho západní hranice jsou neměnné (ale ne ty východní)
  \item[1929] \textit{Youngův plán}: reparace se snižují na čtvrtinu
  \item[1929-33] \textit{Velká hospodářská krize}: krachla newyorská burza, je to v krachu
  \item[1932] \textit{Laussanská konference}: reparace odpuštěny, němci musí zaplatit aspoň 4 mld. (ani to se nestane)
  \item[$-$] vrcholem meziválečným mírových snah je \textit{Briand-Kellogův pakt}  -- 60 států se zapojilo, v případě porušení měly být třeba sankce, ale těžce to nefunguje
\end{itemize}

\end{document}
