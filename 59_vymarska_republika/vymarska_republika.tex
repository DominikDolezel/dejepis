\documentclass{article}
\usepackage{fullpage}
\usepackage[czech]{babel}
\usepackage{amsfonts}

\usepackage{fontspec}
\usepackage{sectsty}
\newfontfamily\Kapitan{Kapitan-Medium}
\allsectionsfont{\Kapitan}

\setmainfont{OpenSans}

\title{\vspace{-2cm}\Kapitan Výmarská republika\vspace{-1.7cm}}
\date{}
\author{}

\begin{document}
\maketitle

\begin{itemize}
  \vspace{-0.5em}
  \setlength\itemsep{0.15em}
  \item 9.11.1918 bylo zrušeno císařství, vyhlášena demokratická republika
  \item leden 1919 volby do parlamentu (do této doby prozatimní vláda s předs. Friedrichem Ebertem /SD/)
  \item únor 1919 parlament se sejde ve Výmaru, ze jde sepsána i ústava
  \item srpen 1919 ústava, je prezident (první Fr. Ebert), kancléř (ekviv. s předsedou parlamentu) s vládou, dvoukomorový parlament (Říšský sněm -- volení, Říšská rada -- jmenování zemskými vládami podle místní příslušnosti), hlavním městem Berlín
  \item četná povstání (Bavorská republika rad, ...)
  \item březen 1920 Kappův (Wolfgang Kapp) puč -- v Berlíně, pokus o pravicový puč, nesouhlas s rozpuštěním armády
  \item 1921 stanovena částka reparací na 132 mld. marek, což je astronomické číslo
  \item vláda s tím nesouhlasí, volí metodu tzv. pasivní resistence -- nic nevyráběli, aby jim to dohoda nezabavila apod., dohoda obsadí porúří (1923), aby si reparace vydobili
  \item i z těchto důvodu německá marka čelí hyperinflaci (1923 litr mléka stojí 4,2 milionů marek)
  \item 8./9.11.1923 -- mnichovský pivní puč -- v Mnichově v nějaké pivnici byly \uv{bavorské politické špičky}, Hitler tam nakráčel s SA, chtěl po nich něco, oni mu nic nedali, SA s Hitlerem šli pochodovat do ulic, pak je všechny zatkli
  \item Hitler je vězněn, v této době sepisuje Mein Kampf
  \item září 1923 odstoupeno od taktiky pasivní resistence, nový kancléř
  \item listopad 1923 měnová reforma
  \item 1924 Dawsův plán -- američané chtěli z němců reparace dostat, pokud německá ekonomika nebude fungoat tak nikdy nic nezaplatí, čili platby se zmírnily, formou různých půjček, úvěrů nastartovali německou ekonomiku
  \item 1925 nový prezident Paul von Hindenburg
  \item 1925 -- Locarnský garanční pakt -- Německo slibuje, že jeho západní hranice jsou neměnné (ale ne ty východní :trollface:)
  \item 1929 Youngův plán -- reparace se snižují na čtvrtinu
  \item 1929-33 Velká hospodářská krize -- krachla newyorská burza, je to v krachu (v piči)
  \item 1932 Laussanská konference -- tak na reparace pečeme, zaplaťte aspoň 4 mld. (ani to se nestane)
  \item  vrcholem meziválečným mírových snah je Briand-Kellogův pakt -- 60 států se zapojilo, v případě porušení měly být třeba sankce, ale těžce to nefunguje
\end{itemize}

\end{document}
