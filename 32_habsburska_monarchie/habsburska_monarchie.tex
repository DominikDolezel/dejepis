\documentclass{article}
\usepackage{fullpage}
\usepackage[czech]{babel}
\usepackage{amsfonts}
\usepackage{multicol}
\setlength{\columnsep}{0.8cm}

\title{\vspace{-2cm}Habsburská monarchie\vspace{-1.7cm}}
\date{}
\author{}

\begin{document}
\maketitle

\section*{Ferdinand I. (1526-1648)}
\begin{itemize}
    \vspace{-0.5em}
    \setlength\itemsep{0.15em}
    \item[$-$] \textit{Vídeňská smlouva} mezi Habsburky a Jagellonci o nástupnictví, když jeden vymře, nastupuje druhý
    \item[$-$] náboženská otázka, při jeho nástupu 10 \% katolíků, jinak lutheráni, kalvíni, jednota bratrská, kališníci
    \item[1526]
    \item[]  ----- chybí -----
\end{itemize}


\section*{Maxmilián II. (1564-1576)}


\section*{Rudolf II.}
\begin{itemize}
    \vspace{-0.5em}
    \setlength\itemsep{0.15em}
    \item[$-$] syn Maxmiliána II. a jeho sestřenice $\Rightarrow$ pravdepodobne kvuli incestu pozorovatelne psychicke i fyzicke problemy u Rudolfa
    \item[1583] přestěhoval sídlo z Vídně do Prahy $\Rightarrow$ rozkvět Prahy
    \item[$-$] pomáhala mu \uv{španělská strana} -- katolíci, lidé vyslaní papežem
    \item[$-$] katolici ve vysokých pozicích $\Rightarrow$ lidem se to nelíbilo


    \item[$-$] problém s Osmany, kteří ohrožují Evropu, problém se svým ctižádostivým bratrem
    \item[1606] uzavřel s Osmany křehký mír za 200k dukátů (draze), vydržel 20 let
    \item[$-$] bratr \textbf{Matyáš} je místodržitelem Uher, získá si podporu uherských stavů, dále rakouských stavů a moravských stavů $\Rightarrow$ 1608 \textsc{vstoupí do Čech}, donutí Rudolfa podepsat \textit{Libeňskou smlouvu} (25. 6. 1608) -- Matyáš drží Rakousy, král Uherský, Morava, zbylé spravuje Rudolf $\Rightarrow$ Rudolf má titul českého krále, (ovládá Lužici a Slezsko), je nadále císařem, současně souhlasí s tím, že Matyáš bude budoucím českým králem

    \item[(9. 7.) 1609] čeští stavové toho využijí a donutí Rudolfa podepsat \textit{Maiestas Rudolfina} = náboženská svoboda pro všechny (tedy nejen pro vrchnost)
    \item[$-$] stavům svěřena Pražská univerzita, vytvořen sbor 30 defenzorů, kteří kontrolují dodržování náboženské svobody

    \item[1611] pasovský biskup \textbf{Leopold} pošle na žádost Rudolfa žoldáky, kteří měli pomoct Rudolfovi nastolit katolicismus, proti Matyášovi se stavy však nemá šanci $\Rightarrow$ toho roku Rudolf II. abdikuje, 1612 umírá
    \item[$\Rightarrow$]  Matyáš králem a císařem
    \item[1617]  sídlem opět Vídeň

    \item[$-$] známý pro svou rudolfinskou sbírku umění, většinu sebrali Švédové

    \item[$-$] dále se zajímal o alchymii, astronomii, astrologii, zval si na dvůr učence
    \begin{itemize}
        \vspace{-0.5em}
        \setlength\itemsep{0.15em}
        \begin{multicols}{2}
        \item[$-$] Jehuda Löw ben Becalel (prý stvořil Golema, napsal dílo Codex Gigas)
        \item[$-$] Hans von Aachen (malíř, pomáhal Rudolfovi díla nakupovat)
        \item[$-$] Bartolomeus Spranger
        \item[$-$] Giuseppe Arcimboldo (zeleninové ksichty)
        \item[$-$] Johannes Kepler
        \item[$-$] Tycho de Brahe (astronom)
        \item[$-$] Adrian de Vries (sochař, jeho díla jsou ve Valdštejnské zahradě)
        \item[$-$] Mordechaj Maisel (architekt, synagoga)
        \end{multicols}
    \end{itemize}

\end{itemize}


\section*{Matyáš Habsburský (1611-1619)}
\begin{itemize}
    \vspace{-0.5em}
    \setlength\itemsep{0.15em}
    \item[$-$] sídlí ve Vídni, bratr Rudolfa II.
    \item[$-$] od 1608 král Uherský, 1611 český, 1612 císař SŘŘ po Rudolfově smrti
    \item[$-$] český stát spravován \textit{katolickými místodržícími}, rekatolizační politika, Rudolfův majestát často porušován
    \item[1617] \textit{Oňatova smlouva}, španělská habsburská větev se zřekla nároku na Habsburskou monarchii $\Rightarrow$ i čeští stavové souhlasí s tím, že se po Matyášově smrti stane králem jejich bratranec \textbf{Ferdinand Štírský} $\Rightarrow$ vyřešení následnického problému
    \item[$-$]  vzrůstá nespokojenost českých stavů s rekatolizací, scházejí se, posílají Matyášovi petice
    \item[23.5.1618] \textsc{třetí pražská defenestrace}, v čele Jindřich Matyáš Turn, zinscenovali improvizovaný soud, tři osoby vylétly ven z okna: místodržící VIlém Slavata z Chlumu, Jaroslav Bořita z Martinic, písař Fabricius $\Rightarrow$  počátek třicetileté války, protikatolického odboje
    \item[24.5.1618] vytvořena prozatímní protihabsburská vláda 30 direktorů, pouze Nizozemí stavům peněžně pomáhá
    \item[$-$] vojsko špatně placeno, na straně císaře však kvalitní armáda
    \item[1619] Matyáš umírá, nastupuje jeho bratranec
\end{itemize}


\section*{Ferdinand II.}
\begin{itemize}
    \vspace{-0.5em}
    \setlength\itemsep{0.15em}
    \item[$-$] zastánce katolicismu, stavové to odmítají, zvolí v srpnu 1619 nového krále \textbf{Fridricha Falckého}, na trůnu jen asi jednu zimu = zimní král, první kalvinista v čele českého státu
    \item[8.11.1620] \textsc{bitva na Bílé hoře} pořážka českých stavů uherskými vojsky, která jsou lépe finančně zajištěna a motivována, Fridrich Falcký pechá do Bratislavy, potom do Haagu
    \item[$\Rightarrow$ ] zlomový moment v českých dějinách, zatýkání účastníků stavovského odboje
    \item[21.6.1621] \textsc{staroměstská exekuce} = poprava odbojářů, celkem 27 lidí, Jan Mydlář, Kryštof Harant z Polžic a Bezdružic,  Jan Jesenský, kat jan Mydlář
    \item[$-$] \textit{Obnovené zřízení zemské} pro čechy od 1627 a pro moravu 1628, Habsburkové mají dědičné právo na český trůn, jediné katolické náboženství, návrat jezuitů, obnovení zemského sněmu, který schvaluje třeba daně, má velice omezené pravomoci, podlomení měšťanského stavu, má jenom jeden hlas celkově, němčina zrovnoprávněna s němčinou, inkolát uděluje jen šlechta, rozhoduje jen panovník
    \item[$-$] \textit{inkolát} = udělení příslušnosti ke šlechtě
    \item[$-$] \textit{Mandát Ferdinanda II.} = nekatolická šlechta i měšťanstvo musí odejít, kdo nekonvertuje na katolicismus musí taky odejít $\Rightarrow$ \textit{exulantismus}

\end{itemize}



\section*{Třicetiletá válka}
\begin{enumerate}
    \vspace{-0.5em}
    \setlength\itemsep{0.15em}
    \item válka česká 1618-1620
    \item válka falcká 1621-1623
    \item válka dánská 1625-1629
    \item válka švédská 1630-1635
    \item válka švédsko-francouzská 1635-1648
\end{enumerate}

\end{document}
