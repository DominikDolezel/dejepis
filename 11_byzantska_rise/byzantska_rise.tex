\documentclass{article}
\usepackage{fullpage}
\usepackage[czech]{babel}
\usepackage{amsfonts}

\title{\vspace{-2cm}Byzantská říše\vspace{-1.7cm}}
\date{}
\author{}

\begin{document}
\maketitle

\section*{Obecně.}
\begin{itemize}
    \vspace{-0.5em}
    \setlength\itemsep{0.15em}
    \item[$-$] Východořímská říše (vznik 395), od 12. st. Byzantská
    \item[$-$] Konstantin I. založil Byzantion
    \item[$-$] považují se za následníky Římanů
    \item[$-$] pojítkem křesťanství, liturgickým jazykem \textit{řečtina}
    \item[$-$] města: \textbf{Konstantinopol} (správní centrum), Antiochea, Níkaáia
    \item[$-$] textilnictví, zlatnictví, hedvábí, mince
    \item[$-$] absolutismus
\end{itemize}

\section*{Justinián I. (527 -- 565)}
\begin{itemize}
    \vspace{-0.5em}
    \setlength\itemsep{0.15em}
    \item[$-$] nejvýznamnější panovník
    \item[$-$] chce obnovit území cele Římské říše $\rightarrow$ expanzivní politika
    \item[$-$] maželka \textbf{Theodora}
    \item[$-$] vykupoval mír tributem Novoperské říši
    \item[532] \textsc{povstání Niká}, ustál
    \item[$-$] expanze: Apeninský pol. Středomořské ostrovy, S Afriky; obnovil území Římské říše
    \item[$-$] \textit{Corpus iuris civilis} (svod občanského práva) -- soupis římského práva
    \item[$-$] theokratický absolutismus, považoval se za \uv{prodlouženou ruku boha na zemi}
    \item[529] zákaz pohanských škol (i té v Athénách)
    \item[$-$] bazilika San Vitale v Ravenně s krásnými mozajkami
    \item[$-$] \textit{césaropapismus} = způsob vlády, není jen v čele moci světské, ale i církevní
\end{itemize}

\section*{Velké schizma}
\begin{itemize}
    \vspace{-0.5em}
    \setlength\itemsep{0.15em}
    \item[$-$] \textbf{Mikuláš I.} vs. \textbf{Fótios}, navzájem se sesadaili
    \item[1054] \textsc{Velké schizma}: křesťanská církev se roztrhla na dvě části (Z -- papež, latina; V -- ortodoxní, Kristus je lidské podstaty)
    \item[$-$] misie: Velká Morava 863, Bulharsko, Kyjevská Rus 988
\end{itemize}

\section*{7. -- 11. století, krize}
\begin{itemize}
    \vspace{-0.5em}
    \setlength\itemsep{0.15em}
    \item[$-$] ze severu příchod Bulharů pod vedením chána \textbf{Asparucha}
    \item[$-$] z jihu příchod Arabů
    \item[$-$] \textit{ikonoklasmus} = obrazoborecké hnutí, odmítá zobrazování svatých v kostelech atd.
    \item[1071] \textsc{bitva u Mantzikertu}: seldžúkští Turci porazili byzantskou armádu
    \item[1204] \textsc{čtvrtá křížová výprava}, vtrhli do Byzance, zlikvidovali Cařihrad, na místě Byzance vzniká latinské císařství, Byzantinci se ztáhli do Nikajského císařství
    \item[1261] obnovení Byzantské říše
    \item[1453] vyvrácení Byzantské říše Turky
\end{itemize}

\section*{Kultura}
\begin{itemize}
    \vspace{-0.5em}
    \setlength\itemsep{0.15em}
    \item[$-$] architektura: půdorys tvar kříže, mozaiky, hodně zlata
    \item[$-$] výborné školství, vysoká úroveň práva, medicíny, přírodních věd
    \item[$-$] literatura: Jordanes, Prokopios, Jan Zlatoústý
    \item[$-$] byzantský sloh: sv. Sofie v Kyjevě, San Vitale v Ravenně, \dots
\end{itemize}




\end{document}
