\documentclass{article}
\usepackage{fullpage}
\usepackage[czech]{babel}
\usepackage{amsfonts}

\title{\vspace{-2cm}Byzantská říše\vspace{-1.7cm}}
\date{}
\author{}

\begin{document}
\maketitle

\section*{Obecně.}
\begin{itemize}
    \vspace{-0.5em}
    \setlength\itemsep{0.15em}
    \item[$-$] Východořímská říše, od 12. st. Byzantská
    \item[$-$] Konstantin I. založil Byzantion
    \item[395] východořímská říše
    \item[$-$] 12. stol. Byzantská říše
    \item[$-$] považují se za následníky Římanů
    \item[$-$] pojítkem křesťanství, liturgickým jazykem řečtina
    \item[$-$] města: Konstantinopol (správní centrum), Antiochea, Níkaáia
    \item[$-$] textilnictví, zlatnictví, hedvábí, mince
    \item[$-$] absolutismus
\end{itemize}

\section*{Justinián I (527 -- 565)}
\begin{itemize}
    \vspace{-0.5em}
    \setlength\itemsep{0.15em}
    \item[$-$] nejvýznamější panovník
    \item[$-$] chce obnovit území cele Římské říše -> ecpanzivní politika
    \item[$-$] maželka Theodora
    \item[$-$] vykupoval mír tributem Novoperské říši
    \item[532] \textsc{povstání Niká}, ustál
    \item[$-$] expanze: Apeninský pol. Středomořské ostrovsy, S Afriky; obnovil úzení Římské říše
    \item[$-$] \textit{Corpus iuris civilis} (svod občanského práva) -- soupis římského práva
    \item[$-$] theokratický absolutismus, považoval se za \uv{prodlouženou ruku boha na zemi}
    \item[529] zákaz pohanských škol (i tu v Athénách)  
\end{itemize}


\end{document}
