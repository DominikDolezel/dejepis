\documentclass{article}
\usepackage{fullpage}
\usepackage[czech]{babel}
\usepackage{amsfonts}

\title{\vspace{-2cm}Kultura a vzdělanost v 17. a 18. století\vspace{-1.7cm}}
\date{}
\author{}

\begin{document}
\maketitle

\section*{Náboženství}

\begin{itemize}
    \vspace{-0.5em}
    \setlength\itemsep{0.15em}
    \item[$-$] Evropa díky reformaci nábožensky nejdenotná (protestanti, katolicismus)
    \item[$-$] díky protireformaci katolická církev opět posiluje, hromadí majetek $\Rightarrow$ zvedání kritiky posílené církve
    \item[$-$] dva proudy: \textit{jansenismus} (podle Cornelia Jansena, zastoupení především v katolických zemích -- Francie, klášter Port Royal, Blaise Pascal), \textit{pietismus} (v protestantských zemích, mezi chudšími): požadují hluboký, bezprostřední vztah k bohu, upřímnost víry
\end{itemize}

\section*{Věda}
\begin{itemize}
    \vspace{-0.5em}
    \setlength\itemsep{0.15em}
    \item[$-$] druhá polovina 17. století: vědecká revoluce v Nizozemí, Francii, Británii
    \item[$-$] je potřeba provést pokusy pro potvrzení teorie
    \item[$-$] Galileo Galilei (dalekohled), Blais Pascal (jednotka tlaku, první mechanický kalkulátor), Isaac Newton (základy moderní fyziky), Prokop Diviš (bleskosvod), William Harvey (krevní oběh), Antoine-Laurent de Lavoisier (uspořádání chemických prvků, neuchytilo se), Jan Ámos Komenský
\end{itemize}

\section*{Filosofie}
\begin{itemize}
    \vspace{-0.5em}
    \setlength\itemsep{0.15em}
    \item[$-$] dva hlavní proudy, \textit{racionalismus} = zdrojem objektivního poznání je rozum (Descartes), \textit{empirismus} = naše poznání je založeno na základě smyslové zkušenosti (Hobbes)
\end{itemize}

\section*{Osvícenství}
\begin{itemize}
    \vspace{-0.5em}
    \setlength\itemsep{0.15em}
    \item[$-$] duchovní filosofický směr, vzniká na sklonku 17. století, vzniká v Nizozemí, ale typický je ve Francii v 18. st.
    \item[$-$] víra v rozum, vymezují se proti tmářství, věří v lidské schopnosti, kritici katolického pronásledování a netolerance, začínají mluvit o právech člověka, myšlenka rovnosti před zákonem, dělba státní moci na tři složky, odklon od náboženství
    \item[$-$] \textit{deismus} = bůh existuje, ale jenom jako stvořitel světa a dále už do dění světa nezasahuje
    \item[$-$] \textit{mechaničtí materialisté} = existuje jenom hmota a \underline{nic} jiného, žádný bůh ani duchovno, L. O. de la Metrie


\end{itemize}

\section*{Francie}
\begin{itemize}
    \vspace{-0.5em}
    \setlength\itemsep{0.15em}
    \item[$-$] Francois Maria Arout pracuje pod pseudonymem Voltaire, protože byl dvakrát vězněn a nechtěl poškodit svého otce, který se živil jako notář
    \item[$-$] Charles Lousi Montesquie: O duchu zákonů, chce rozdělit státní moc na tři složky
    \item[$-$] Jean Jacques Rousseau: manželka se nemá starat o děti, ale o něj, svoje děti poslal do sirotčince, lidi zkazil soukromý majetek, ale alespoň lze dosáhnout rovnosti před zákonem, kritik soukromého vlastnictví
    \item[$-$] Encyklopedie -- soubor veškerého tehdejšího vědění
\end{itemize}
\section*{Ekonomie}
\begin{itemize}
    \vspace{-0.5em}
    \setlength\itemsep{0.15em}
    \item[$-$] Francois Quesnay: zdrojem bohatství státu má být zemědělství, na rozdíl od merkantilismu stát nemá do ekonomiky zasahovat = \textit{fyziokratismus}
    \item[$-$] Adam Smith: přišel s myšlenkou volného trhu, paradox hodnoty: věci, které nejvíce potřebujeme ke svému životu, je levnější než to, co nepotřebujeme
\end{itemize}

\section*{Ostatní poznatky}
\begin{itemize}
    \vspace{-0.5em}
    \setlength\itemsep{0.15em}
    \item[$-$] od Algebry se odděluje geometrie, vzniká deskriptivní geometrie
    \item[$-$] začíná se úspěšně očkovat proti neštovicím
    \item[$-$] Carl Linné
    \item[$-$] Luigi Galvani
    \item[$-$] Alessandro Volta
    \item[$-$] Fyzikové, elektřina
    \item[$-$] bratři Montgolfierové – horkovzdušný balón

\end{itemize}

\section*{Baroko (17.-18. století)}
\begin{itemize}
    \vspace{-0.5em}
    \setlength\itemsep{0.15em}
  \item[$-$] začátek konec 16. st. V Itálii
  \item[$-$] široký rozsah, projevuje se i v koloniích, v Evropě a Jižní a Střední Americe
  \item[$-$] nástroj katolické církve jak lákat lid, protireformace
  \item[$-$] oblouk, nesmírně zdobné, mnoho zlata, baculaté ženy a andělíčci, iracionalismus
  \item[$-$] častá náboženská tématika

\end{itemize}

\section*{Architektura}
\subsection*{Itálie}

\begin{itemize}
    \vspace{-0.5em}
    \setlength\itemsep{0.15em}
    \item[$-$] Giovanni Bernini (Vatikán), Svatopetrské náměstí a kolonáda, stolec sv. Petra, nejen architekt, i sochař: fontána medusy
\item[$-$] Francesco Borromini: lateránská bazilika, kostel sv. Iva (Řím)
\end{itemize}

\subsection*{Francie}
\begin{itemize}
    \vspace{-0.5em}
    \setlength\itemsep{0.15em}
  \item[$-$] Jules Mansart: podílel se na Louvru, Versailles
  \end{itemize}

\subsection*{Rakousko, Čechy}
\begin{itemize}
    \vspace{-0.5em}
    \setlength\itemsep{0.15em}
  \item[$-$] Jan Fischer z Erlachu: kašna Parnas, zámek Vranov nad Dyjí, Schönbrunn
  \item[$-$] v Praze chrám sv. Mikuláše, vytvořili otec a syn Dientzenhoferové
  \item[$-$] Carlo Lurago: Klementinum v Praze
  \item[$-$] Jan Blažej Santini Eichel: Zelená hora, Křtiny, Rajhrad
  \item[$-$] Loretta v Praze, zvonkohra
  \item[$-$] morový sloup v Olomouci
  \item[$-$] Valdštejnský palác, dnes sídlo senátu
  \item[$-$] Špilberk (dnes barokní podoba, vznikal dřív)
  \item[$-$] dnešní Mahenova knihovna, Schrattenbachův palác
  \item[$-$] většina kostelů v Brně
  \item[$-$] Svatý Kopeček u Olomouce
  \item[$-$] Sv. Mikuláš v Praze
  \item[$-$] selské baroko – Holašovice

\end{itemize}

\subsection*{Malířství}
\begin{itemize}
    \vspace{-0.5em}
    \setlength\itemsep{0.15em}
    \item[$-$] Nizozemí: Petr Paul Rubbens, Rembrandt
  \item[$-$] Francie: Poussin
  \item[$-$] Španělsko: Velázquez
  \item[$-$] Itálie: Carravaggio
  \item[$-$] u nás: Škréta, Kupecký, Václav Hollar

\end{itemize}



\section*{Sochařství}
\begin{itemize}
    \vspace{-0.5em}
    \setlength\itemsep{0.15em}
    \item[$-$] Itáli: \textbf{Lorenzo Bernini} (autor některých soch na Karlově mostě)
    \item[$-$] České země: \textbf{Brokoff}, \textbf{Braun}
    \item[$-$] pěstován kult Jana Nepomuckého, poznáme ho podle hvěz nad hlavou
\end{itemize}

\section*{Hudba, literatura}
\begin{itemize}
    \vspace{-0.5em}
    \setlength\itemsep{0.15em}
    \item[$-$] Antonio Vivaldi
    \item[$-$] Johann Sebastian Bach, Händel, Monteverdi
    \item[$-$] u nás Mysliveček, Adam Michna z Otradovic
    \item[$-$] historik Bohuslav Balbín, Adam Michna z Otradovic
\end{itemize}

\section*{Rokoko}
\begin{itemize}
    \vspace{-0.5em}
    \setlength\itemsep{0.15em}
    \item[$-$] ve Francii se ve 2. třetině 18. století baroko vyvine do rokoka (za fran. \textit{roccaile}, mušle)
    \item[$-$] znaky interiéru: obrazce mušlí, lehké, pastelové barvy
    \item[$-$] dámy nosí paruky, napudrované, vyzývavé, boty na podpatku, užívání radostných stránek života
    \item[$-$] vznik míšeňského porcelánu
    \item[$-$] malíř Jean-Honoré \textbf{Fragonard}
    \item[$-$] hudba: Haydn, Mozart
    \item[$-$] literatura: Torquato di Tasso, John Milton
    \item[$-$] zámek Sanssouci v Postupimi (Německo), palác Kinských v Praze
\end{itemize}



\end{document}
