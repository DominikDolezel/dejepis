\documentclass{article}
\usepackage{fullpage}
\usepackage[czech]{babel}
\usepackage{amsfonts}

\title{\vspace{-2cm}Kultura a vzdělanost v 17. a 18. století\vspace{-1.7cm}}
\date{}
\author{}

\begin{document}
\maketitle

\section*{Náboženství}

\begin{itemize}
    \vspace{-0.5em}
    \setlength\itemsep{0.15em}
    \item[$-$] Evropa díky reformaci nábožensky nejdenotná (protestanti, katolicismus)
    \item[$-$] díky protireformaci katolická církev opět posiluje, hromadí majetek $\Rightarrow$ zvedání kritiky posílené církve
    \item[$-$] dva proudy: \textit{jansenismus} (podle Cornelia Jansena, zastoupení především v katolických zemích -- Francie, klášter Port Royal, Blaise Pascal), \textit{pietismus} (v protestantských zemích, mezi chudšími): požadují hluboký, bezprostřední vztah k bohu, upřímnost víry
\end{itemize}

\section*{Věda}
\begin{itemize}
    \vspace{-0.5em}
    \setlength\itemsep{0.15em}
    \item[$-$] druhá polovina 17. století: vědecká revoluce v Nizozemí, Francii, Británii
    \item[$-$] je potřeba provést pokusy pro potvrzení teorie
    \item[$-$] Galileo Galilei (dalekohled), Blais Pascal (jednotka tlaku, první mechanický kalkulátor), Isaac Newton (základy moderní fyziky), Prokop Diviš (bleskosvod), William Harvey (krevní oběh), Antoine-Laurent de Lavoisier (uspořádání chemických prvků, neuchytilo se), Jan Ámos Komenský
\end{itemize}

\section*{Filosofie}
\begin{itemize}
    \vspace{-0.5em}
    \setlength\itemsep{0.15em}
    \item[$-$] dva hlavní proudy, \textit{racionalismus} = zdrojem objektivního poznání je rozum (Descartes), \textit{empirismus} = naše poznání je založeno na základě smyslové zkušenosti (Hobbes)
\end{itemize}

\section*{Osvícenství}
\begin{itemize}
    \vspace{-0.5em}
    \setlength\itemsep{0.15em}
    \item[$-$] duchovní filosofický směr, vzniká na sklonku 17. století, vzniká v Nizozemí, ale typický je ve Francii v 18. st.
    \item[$-$] víra v rozum, vymezují se proti tmářství, věří v lidské schopnosti, kritici katolického pronásledování a netolerance, začínají mluvit o právech člověka, myšlenka rovnosti před zákonem, dělba státní moci na tři složky, odklon od náboženství
    \item[$-$] \textit{deismus} = bůh existuje, ale jenom jako stvořitel světa a dále už do dění světa nezasahuje
    \item[$-$] \textit{mechaničtí materialisté} = existuje jenom hmota a \underline{nic} jiného, žádný bůh ani duchovno, L. O. de la Metrie


\end{itemize}

\section*{Francie}
\begin{itemize}
    \vspace{-0.5em}
    \setlength\itemsep{0.15em}
    \item[$-$] Francois Maria Arout pracuje pod pseudonymem Voltaire, protože byl dvakrát vězněn a nechtěl poškodit svého otce, který se živil jako notář
    \item[$-$] Charles Lousi Montesquie: O duchu zákonů, chce rozdělit státní moc na tři složky
    \item[$-$] Jean Jacques Rousseau: manželka se nemá starat o děti, ale o něj, svoje děti poslal do sirotčince, lidi zkazil soukromý majetek, ale alespoň lze dosáhnout rovnosti před zákonem, kritik soukromého vlastnictví
\end{itemize}
\section*{Ekonomie}
\begin{itemize}
    \vspace{-0.5em}
    \setlength\itemsep{0.15em}
    \item[$-$] Francois Quesnay: zdrojem bohatství státu má být zemědělství, na rozdíl od merkantilismu stát nemá do ekonomiky zasahovat
    \item[$-$] Adam Smith: přišel s myšlenkou volného trhu, paradox hodnoty: věci, které nejvíce potřebujeme ke svému životu, je levnější než to, co nepotřebujeme
\end{itemize}


\end{document}
