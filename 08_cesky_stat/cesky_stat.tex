\documentclass{article}
\usepackage{fullpage}
\usepackage[czech]{babel}
\usepackage{amsfonts}

\title{\vspace{-2cm}Český stát\vspace{-1.7cm}}
\date{}
\author{}

\begin{document}
\maketitle
\section*{Přírodní podmínky}
\begin{itemize}
    \vspace{-0.5em}
    \setlength\itemsep{0.15em}
    \item[$-$] nížiny, husté a nepropustné lesy $\rightarrow$ přirozená hranice
    \item[$-$] \textit{trojpolní systém} -- lada, ozim, jař
    \item[$-$] keramika, textilie, směnný obchod
    \item[$-$] půdu vlastní šlechta, lidé si ji pronajímají za \textit{rentu}
    \item[$-$] \textit{rustikál} = půda, kterou si vesničané pronajímají od šlechty
    \item[$-$] \textit{dominikál} = půda, kterou vlastní šlechta (zámky, statky krále)
    \item[$-$] kolem vesnic pastviny ve společném majetku obce
    \item[$-$] obchodníci v podhradí
\end{itemize}

\section*{Osídlení}
\begin{itemize}
    \vspace{-0.5em}
    \setlength\itemsep{0.15em}
    \item[$-$] Charváti (V Čechy) -- Libice nad Cidlinou, vládnou Slavníkovci
    \item[$-$] Čechové (střední Čechy) -- Budeč, Levý Hradec, vládnou Přemyslovci
    \item[$-$] Přemyslovci název od Přemysla Oráče, postupně sjednotí ostatní kmeny
\end{itemize}

\begin{itemize}
    \vspace{-0.5em}
    \setlength\itemsep{0.15em}
    \item[poč. 9. st.] \textbf{Karel Veliký} vtrhne do Čech, vynutí si \textit{tribut} = poplatek za to, že na ně nebude útočit
    \item[845] bylo v Řezně pokřtěno 14 českých kmenových knížat
    \item[2. pol. 9. st.] sjendocovací proces: prostor Čech ovládl kmen Čechů, vládnou Přemyslovci
    \item[$-$] legendární knížata: Přemysl, Nezamysl, Mnata, Vojen, Vnislav, Křesomysl, Neklan, Hostivít, \dots
\end{itemize}

\section*{Bořivoj (867 -- 894)}
\begin{itemize}
    \vspace{-0.5em}
    \setlength\itemsep{0.15em}
    \item[$-$] manželka Ludmila
    \item[$-$] zástupce Svatupluka -- knížete VM v Čechách
    \item[883] křest
    \item[$-$] stavba rotundy sv. Klimenta -- první křesťanský kostel v Čechách
    \item[$-$] začátek stavby Pražského hradu, kostelík Panny Marie
    \item[$-$] synové: Spytihněv a Vratislav
\end{itemize}

\section*{Spytihněv (894 -- 915)}
\begin{itemize}
    \vspace{-0.5em}
    \setlength\itemsep{0.15em}
    \item[$-$] dostává se na trůn, když Svatopluk umírá
    \item[$-$] odtrhne Čechy od Velké Moravy
    \item[$-$] rotunda na Budči sv. Petra a Pavla
    \item[$-$] budování Pražského hradu
    \item[$-$] základy státní správy
\end{itemize}

\section*{Vratislav (915 -- 921)}
\begin{itemize}
    \vspace{-0.5em}
    \setlength\itemsep{0.15em}
    \item[$-$] manželka Drahomíra (nechala zavraždit Ludmilu, prohlášena za svatou)
    \item[$-$] dva synové: Václav a Boleslav
    \item[$-$] úspěšné boje s Maďary
\end{itemize}

\section*{Svatý Václav (921 -- 935)}
\begin{itemize}
    \vspace{-0.5em}
    \setlength\itemsep{0.15em}
    \item[$-$] nastupuje mladý
    \item[$-$] o vliv na Václava bojují matka Drahomíra s babičkou Ludmilou
    \item[$-$] Drahomíra nechala Ludmilu uškrtit šálou
    \item[$-$] současník Jindřich I. Ptáčník, pokračování v placení mírového tributu
    \item[28.9.935] Boleslav ho poslal do Mladé Boleslavi, u kostela na něj zaútočil mečem, zabili ho jeho družiníci
    \item[$-$] popsáno v Gumpoldově legendě
    \item[$-$] prohlášen za svatého, patron českého národa
\end{itemize}

\section*{Boleslav I. (935 -- 972) Ukrutný}
\begin{itemize}
    \vspace{-0.5em}
    \setlength\itemsep{0.15em}
    \item[$-$] stabilizace země, hospodářství, vybírání daní
    \item[$-$] současník Otty I.
    \item[$-$] nezávislost na SŘŘ, nechce odvádět tribut
    \item[$-$] \textsc{bitva u Lešských polí}, porážka Maďarů
    \item[$-$] expanze do Moravy, Slezska, Krakovska, povodí řeky Váhu
    \item[$-$] budování hradské správy, v rukou kastelánů
    \item[$-$] \textit{stříbrný denár} = první přemyslovská mince
    \item[$-$] děti: Doubravka (provdala se za Měška z rodu Piastovců) -- zakladatelé polského státu; Mlada (abatyše benediktinek při sv. Jiří); Boleslav
\end{itemize}

\section*{Boleslav II. (972 -- 999) Pobožný}
\begin{itemize}
    \vspace{-0.5em}
    \setlength\itemsep{0.15em}
    \item[973] vznik biskupství v Praze, první biskup Dětmar, druhý sv. Vojtěch ze Slavníkovců (pohřben v Hnězdně), později zřízeno arcibiskupství: arcibiskup Radim
    \item[$-$] expanze na Východ až ke Lvovu
    \item[$-$] podpora příchodu benediktínů, klášter v Břevnově
    \item[28.9.955] \textsc{vyvraždění Slavníkovců v Libici} při oslavách sv. Václava
\end{itemize}

\section*{Krize Českého státu}
\begin{itemize}
    \vspace{-0.5em}
    \setlength\itemsep{0.15em}
    \item[$-$] boj o moc: \textbf{Boleslav III. Ryšavý} je krutý, neschopný vs. \textbf{Jaromír} vs. \textbf{Oldřich}
    \item[$-$] této situace využívá \textbf{Boleslav Chrabrý} (syn Doubravky a Měška, polský král), Vladivoj posílá vládnout do Čech, umírá
    \item[$-$] poté Jaromír, následně Oldřich, který vyvede Čechy z krize
\end{itemize}

\section*{Oldřich (1012 -- 1033)}
\begin{itemize}
    \vspace{-0.5em}
    \setlength\itemsep{0.15em}
    \item[$-$] potká Boženu, s ní má syna Břetislava
    \item[$-$] vyhnal Poláky z našeho území
    \item[(1019)] znovu definitivně připojil Moravu
    \item[$-$] Sázavský klášter, benediktíni, slovanská liturgie
\end{itemize}

\section*{Břetislav I. (1035 -- 1055), český Achilles}
\begin{itemize}
    \vspace{-0.5em}
    \setlength\itemsep{0.15em}
    \item[$-$] stabilizace Českého státu
    \item[$-$] manželka Jitka ze Svinibrodu, unesl si ji z kláštera
    \item[1039] \textsc{polské tažení}, dobyl Slezsko a Krakovsko; velká kořist
    \item[$-$] nad hrobem Vojtěcha v Hnězdně pronesl tzv. \textit{Břetislavovy dekrety} = nejstarší český právní dokument (křesťanský způsob života)
    \item[1054] \textit{stařešinský řád} = nejstarší člen rodu má právo na vládu; pro zbylé zřídil na Moravě tzv. \textit{údělná knížectví} -- Brněnské, Olomoucké, Znojemské
    \item[$-$] Rajhradský klášter
    \item[$-$] několik bitev proti císaři SŘŘ Jindřichu III., neúspěšné
    \item[$-$] pět synů
\end{itemize}

\section*{Vratislav II. (1061 -- 1092)}
\begin{itemize}
    \vspace{-0.5em}
    \setlength\itemsep{0.15em}
    \item[$-$] sídlo přesunul na Vyšehrad
    \item[$-$] bazilika sv. Petra a Pavla na Vyšehradě
    \item[$-$] nejstarší pražská rotunda -- rotunda sv. Martina
    \item[$-$] zřízení olomouckého biskupství
    \item[1085] královský titul od Jindřicha IV. (za pomoc v boji o investituru), jako první český král
    \item[$-$] Kodex vyšehradský
    \item[$-$] po jeho smrti probíhají boje o moc mezi jeho dětmi
\end{itemize}

\section*{Soběslav I.}
\begin{itemize}
    \vspace{-0.5em}
    \setlength\itemsep{0.15em}
    \item[$-$] Vratislavův nejmladší syn
    \item[1126] v \textsc{bitvě u Chlumce u Nakléřovského} zastavil vpád vojáků SŘŘ pod vedením Lothara, budoucího císaře
    \item[$-$] buduje opevnění Pražského hradu
    \item[$-$] rotunda sv. Jiří na Řípu
\end{itemize}

\section*{Vladislav II. (1140 -- 72)}

\begin{itemize}
    \vspace{-0.5em}
    \setlength\itemsep{0.15em}
    \item[$-$] vnuk Vratislava II.
    \item[$-$] druhý český král
    \item[1147] druhá křížová výprava
    \item[1158] zisk královského titulu od Fridricha Barbarossy
    \item[$-$] manželka Judita, iniciovala stavbu Juditinu mostu
    \item[$-$] rotunda sv. Kateřiny ve Znojmě
    \item[$-$] reformní řády: premonstráti, cisterciáci, johanité -- vychází z benediktínů
    \item[$-$] vznik řady klášterů
    \item[1172] se vzdává svého královského titulu ve prospěch syna Bedřicha
\end{itemize}

\begin{itemize}
    \vspace{-0.5em}
    \setlength\itemsep{0.15em}
    \item[1182] Fridrich Barbarossa využívá situace, odtrhl Moravu -- markrabství moravské (Konrád II. Ota), je přimo podřízená jemu, povýšil biskupa pražského na říšské kníže, též podléhá jemu
    \item[1189] \textit{Statuta Conradi} = kodifikace zvykového práva Konrádem II. Otou, přiznal šlechtě dědictví půdy
\end{itemize}

\section*{Přemysl Otakar I. (1197 -- 1230)}
\begin{itemize}
    \vspace{-0.5em}
    \setlength\itemsep{0.15em}
    \item[$-$] stabilizace země
    \item[1212] \textsc{zlatá bula sicilská} = definitivně Přemyslovcům přiznán dědičný titul krále, anulováno odtržení Moravy, český král si může vybírat biskupy
    \item[$-$] manželka KOnstancie Uherská, spojována se vznika kláštera Porta Coeli
\end{itemize}

\end{document}
