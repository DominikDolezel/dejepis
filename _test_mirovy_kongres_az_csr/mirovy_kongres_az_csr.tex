\documentclass{article}
\usepackage{fullpage}
\usepackage[czech]{babel}
\usepackage{amsfonts}

\title{\vspace{-2cm}Pařížský mírový systém, poválečné Rusko a Německo, První~republika\vspace{-1.7cm}}
\date{}
\author{}

\begin{document}
\maketitle

\section*{Pařížský mírový systém}

\subsection*{Versailleská smlouva}
\begin{itemize}
  \item nedaleko Paříže se v palácích Versailleského komplexu domlouvají mírové podmínky, oficiálně začínají jednání 18. 1. 1919 (tedy přesně 48 let potom, co zde bylo vyhlášeno Německé císařství)
  \item účastní se především tzv. \textit{velká trojka} -- David Lloyd George (prem. UK), Woodrow Wilson (pres. USA), Georges Clemenceau (prem. Fr.), dále Vittorio Orlando (prem. It.) a Nobuaki Makino (prem. Jap.), popř. se scházeli v deseti, kde byli tito a jejich ministři zahraničí, bez pozvání zástupci poražených zemí a sovětů
  \item[7. 5. 1919] předložili body mírové smlouvy, Němci z toho byli zničení, něm. vláda to odmítla a podala demisi, Němci to nakonec stejně 28. 6. 1919 podepsali
  \item podmínky pro Německo
  \begin{itemize}
    \item museli vrátit Francii Alsasko-Lotrinsko
    \item museli postoupit Belgii Eupén-Malmedy
    \item levý břeh Rýnu bude 15 let okupován, 50 km na pravém břehu je demilitarizovaná zóna
    \item Němci postupují Polsku Poznaňsko, polský koridor
    \item postupují Memel mandátu, potom Litvě, Gdaňsk postupují mandátu
    \item postupují Hlučínsko Československu
    \item Německo ztrácí všechny kolonie
    \item zrušení všeobecné branné povinnosti, prof. armáda max 100k mužů, redukce válečného loďstva, zákaz letectva, rozpuštění generálního štábu
    \item 10 let dodávat uhlí Francii, Belgii, Itálii
    \item museli postoupit část Šlesvicka Dánsku
    \item peněžní reparace
    \item Sársko jako mandát
    \item dále referendum v jižním Prusku a v Horním Slezsku, ty ale vyhrává Německo
  \end{itemize}
\end{itemize}

\subsection*{Saint-Germainská smlouva }
\begin{itemize}
  \item[10. 9. 1919] smlouva s Rakouskem, musí to být republika, muselo uznat nástupnické státy
  \item postupují Valticko a Vitorazsko ČSR, Jižní Tyrolsko, Istrii, Terst Itálii, Halič Polsku a Bukovinu Rumunsku
  \item zákaz sjednocení s Německem
  \item reparace, armáda do 30k vojáků
\end{itemize}

\subsection*{Trianonská smlouva s Maďarskem}
\begin{itemize}
    \vspace{-0.5em}
    \setlength\itemsep{0.15em}
    \item[(4. 6.) 1920] podepsána v Trianonnu s Maďarskem
    \item[$-$] ztráta 70 \% území a 60 \% obyvatelstva
    \item[$-$] Maďaři ztratili: Slovensko, Podkarpatskou Rus, museli uznat existenci násupnických států
    \item[$-$] po konferenci plebiscit
    \item[$-$] smlouva vyvolala obrovskou nespokojenost
\end{itemize}

\subsection*{Neuillyská smlouva s Bulharskem}
\begin{itemize}
    \vspace{-0.5em}
    \setlength\itemsep{0.15em}
    \item[27. 11. 1919] územní ztráty ve prospěch sousedů (jižní část Řecku -- ztracení přístupu k Egejskému moři)
\end{itemize}

\subsection*{Sévreská smlouva s Tureckem}
\begin{itemize}
    \vspace{-0.5em}
    \setlength\itemsep{0.15em}
    \item[10. 8. 1920] podepsána, ztráta 4 / 5 území, naprosto nepřijatelná
    \item[$-$] Turci dosáhli revize díky generálovi Mustafu Kemalovi
    \item[$-$] ze sultanátu se stává republika v čele s Mustafou Kemalem
    \item[$-$] vznik sekulárního státu, ženy mají volební právo
\end{itemize}

\subsection*{Blízký východ}
\begin{itemize}
    \vspace{-0.5em}
    \setlength\itemsep{0.15em}
    \item[$-$] vytvoření dvou mandátních území: britský a francouzský mandát
    \item[$-$] jednání o tom, kdy tu vzniknou nezávislá území
    \item[$-$] eskalace napětí mezi židy a araby
\end{itemize}

\subsection*{Společenství národů}
\begin{itemize}
    \vspace{-0.5em}
    \setlength\itemsep{0.15em}
    \item[10. 1. 1920] vznik na Pařížské konferenci
    \item[$-$] hlavní cíl: zabránit dalším válkám, to se nepovedlo, poté rozpuštěna
    \item[$-$] sídlem byla Ženeva, dnes sídlo OSN
    \item[$-$] otec myšlenky americký president Woodrow Wilson
    \item[$-$] na půdě Společenství národů aktivní tehdy ministr zahraničí Edvard Beneš
\end{itemize}

\subsection*{Malá Dohoda}
\begin{itemize}
    \vspace{-0.5em}
    \setlength\itemsep{0.15em}
    \item[$-$] političtí reprezentanti Československa, Jugoslávie a Rumunska sbližuje strach z Maďarska, Karla I. (možnost Velkých Uher, o což se pokusí) se sblížili už na Pařížské konferenci
    \item[$-$] organisace společného postupu těchto zemí
\end{itemize}

\subsection*{Washingtonská mírová konference}
\begin{itemize}
    \vspace{-0.5em}
    \setlength\itemsep{0.15em}
    \item[$-$] měla uspořádat poměry na Dálném východě
    \item[$-$] potvrzení držav v Tichomoří
    \item[$-$] Číně byl vnucen princip otevřených dvěří (cíl Ameriky)
    \item[$-$] poměr tonáží válečných lodí v Tichomoří
    \item[$-$] říká se mu \textit{Versaillesko-Washingtonský systém}
    \item[$-$] problémy: absence organizace, která by pořádky udržovávala, příliš přísná k poraženým, vznik mnohonárodnostních států (problém s menšinami), Němci (Hitler) to později využije k tomu, aby napravil křivdu z Versailles
\end{itemize}

\section*{Poválečné Rusko}

\begin{itemize}
    \vspace{-0.5em}
    \setlength\itemsep{0.15em}
    \item[$-$] zástupci Ruska nebyli pozváni na Pařížskou smlouvu
    \item[1917/18-1920/21] \textsc{občanská válka}: jeden z vůdců bolševiků Lev Davidovič Trocký proti neruským národnostem, bělogvardějcům (bílí -- monarchisté, vysocí církevní hodnostáři, bohatí) a intervenčním armádám (vojáci z dohodových zemí, kteří se snaží pomoci bolševické oposici, aby se znovuotevřela, zahynulo minimálně 10 milionů lidí východní fronta)
    \item[(8. 3. 1917)] vytvoření komunistické strany z bolševiků
    \item[(12. 3. 1917)] Moskva hlavním městěm místo Petrohradu
    \item[(17. 7. 1918)] poprava cara v Jekatěrinburgu
    \item[1918-1920] \textsc{válka s Polskem}, kteří ztratili území na úkor Ruska, chtějí obnovit svoje území tak, jak bylo před trojím dělením Polska -- Rusko iniciuje válku
    \item[1919] \textit{Kominterna}: komunistická internacionála, odsud se měly šířit socialistické myšlenky
    \item[30. 12. 1922] přejmenování na SSSR: Bělorusko, Ukrajina, Kavkazské republiky a postupné rozšiřování území
\end{itemize}

\subsection*{Občanská válka}
\begin{itemize}
    \vspace{-0.5em}
    \setlength\itemsep{0.15em}
    \item[$-$] hladomor, kvůli němu taky zemřelo hodně lidí
    \item[$-$] bolševiky úspěšně vede gen. Trockij, postupně ovládli centra
    \item[$-$] docházelo k bojům s legionáři
    \item[$-$] bělogvardějci
    \item[$-$] celé Rusko nastaveno na válečný komunismus -- vše bylo podřízeno vítězství bolševiků
    \item[2.-3.1921] nespokojení s tím, co dělali vojáci: vzpoura námořníků v Kronštadtu (nelíbilo se jím počínání bolševiků), tvrdě potlačeno, symbolický konec občanské války
    \item[$-$] Michael Tuchačevskij -- jeden z velitelů rudé armády, který byl zlikvidován Stalinem ve třicátých letech, taky provedl reorganizaci rudé armády
\end{itemize}

\subsection*{Rusko-polská válka (1918-1920)}
\begin{itemize}
    \vspace{-0.5em}
    \setlength\itemsep{0.15em}
    \item[1918] obnovení Polska, mělo za cíl se obnovit pokud možno jako před trojím dělením Polska (vzali si jej Rusko, Prusko a Habsburkové)
    \item[$-$] Poláci postupovali směrem na Rusko, obsadili Kyjev
    \item[leden 1918] \textsc{útok rudé armády} na Polsko (generál Michail Tuchačevskij)
    \item[srpen 1920] zázrak \textsc{na Visle}, Poláci porazili Rusy (polský generál Józef Pilsudski)
    \item[září 1920] \textsc{bitva na řece Němenu}, Poláci získali část Litvy
\end{itemize}

\subsection*{Ekonomika}
\begin{itemize}
    \vspace{-0.5em}
    \setlength\itemsep{0.15em}
    \item[$-$] Józef Pilsudski: táhne polsko doprava, opírání o armádu, jeden z jeho ministrů Beck (zahraničí, ostře vystupoval proti Čechoslovákům v otázce Těšínska, snažil se o spolupráci s Němci)
    \item[$-$] \textit{válečný komunismus} během války, všechna výroba se soustředí na vítězství bolševiků, povolžský hladomor (smrt 5 milionů lidí)
    \item[(1921)] NEP: nová ekonomická politika, strůjcem je Nikolaj Bucharin, později popraven, v Rusku vznikají soukromé podniky, příliv zahraničního kapitálu, základ ekonomiky má být zemědělství, industrializace
    \item[1929] direktivní ekonomika, první pětiletý plán
    \item[$-$] \textit{kolektivizace}: združstevnění, vznik zemědělských družstev, likvidace velkých majitelů půdy, zakládání \textit{kolchozů} (venkovská zemědělská družstva)
    \item[1919] \textit{gulag}: (glavnoje upravlenije lagerej), pracovní tábory, končí tam opozice (liberálové, umělci)
    \item[30. léta] hladomor na Ukrajině
    \item[$-$] bělomořsko-baltský kanál: pokus spojit Bílé a Baltské moře, nefunkční, byl málo hluboký
\end{itemize}

\subsection*{Politika}
\begin{itemize}
    \vspace{-0.5em}
    \setlength\itemsep{0.15em}
    \item[1922] V. I. Lenin vyřazen z politiky, místo něj nastupuje \textbf{Josif Vissarionovič Stalin}
    \item[24.1.1924] Lenin umírá jako lidská troska
    \item[$-$] boj o moc kolem Stalina se pohybují Zinověv a Kameněv, na druhé straně Trockij, všichni chtějí být generálním tajemníkem, moc na sebe strhl Stalin a postupně se ostatních zbavil
    \item[(1928)] vyloučení Zinověva a Kameněva z komunistické strany
    \item[1936] Zinověv a Kameněv popraveni, Trockij vypovězen, skončil v Mexiku
    \item[$-$] Frida Kahlo, jedna z nejznámějších mexických umělkyň, s Trockým měla jakýsi románek
    \item[(1940)] zavražděn Trockij
    \item[30. léta] čistky, stály stát obrovské síly, protože třeba za druhé světové války potom neměli odborníky
    \item[$-$] ČEKA, NKVD, KGB: tajné policie, stalinův noschled \textbf{Berija} spravoval gulagy nebo Katyňský masakr (1940)
    \item[$-$] zahraniční politika: sovětské Rusko v mezinárodní izolaci, vytáhli se z ní díky:
    \item[4.-5.1922] \textit{Konference v Janově}, měla se tu řešit otázka Německých reparací, byli tam i zástupci Němců a Rusů a ti se pak sešli v blízkém Rapaldu, kde vznikla
    \item[16.4.1922] \textit{Rapallská smlouva}: navzájem si odpustili reparace, tím se dostali z politické izolace
    \item[1934] vstup Ruska do Společenství národů, protože se evropské mocnosti bály Hitlera, jenže po útoku na Finsko jsou zase vyhozeni
\end{itemize}


\section*{Výmarská republika}

\begin{itemize}
  \vspace{-0.5em}
  \setlength\itemsep{0.15em}
  \item[9.11.1918] zrušeno císařství, vyhlášena demokratická republika
  \item[leden 1919] volby do parlamentu (do této doby prozatímní vláda s předs. Friedrichem Ebertem /SD/)
  \item[únor 1919] parlament se sejde ve Výmaru, zde jde sepsána i ústava
  \item[srpen 1919] ústava, je prezident (první Fr. Ebert), kancléř (ekvivalentní s naším předsedou parlamentu) s vládou, dvoukomorový parlament (Říšský sněm -- volení, Říšská rada -- jmenování zemskými vládami podle místní příslušnosti), hlavním městem Berlín
  \item[$-$] četná povstání (Bavorská republika rad, \dots)
  \item[březen 1920] \textsc{Kappův puč}  (Wolfgang Kapp) v Berlíně, pokus o pravicový puč, nesouhlas s rozpuštěním armády
  \item[1921] stanovena částka reparací na 132 mld. marek, což je astronomické číslo
  \item[$-$] vláda s tím nesouhlasí, volí metodu tzv. \textit{pasivní resistence}:  nic nevyráběli, aby jim to Dohoda nezabavila apod., Dohoda obsadí Porúří (1923), aby si reparace vydobyli
  \item[$-$] i z těchto důvodu německá marka čelí hyperinflaci (1923 litr mléka stojí 4,2 milionů marek)
  \item[8./9.11.1923] \textsc{mnichovský pivní puč}: v Mnichově v nějaké pivnici byly \uv{bavorské politické špičky}, Hitler tam nakráčel s SA, chtěl po nich něco, oni mu nic nedali, SA s Hitlerem šli pochodovat do ulic, pak je všechny zatkli
  \item[$-$] Hitler je vězněn, v této době sepisuje \textbf{Mein Kampf}
  \item[září 1923] odstoupeno od taktiky pasivní resistence, nový kancléř
  \item[listopad 1923] měnová reforma
  \item[1924] \textit{Dawsův plán}: Američané chtěli z Němců reparace dostat, pokud německá ekonomika nebude fungovat tak nikdy nic nezaplatí, čili platby se zmírnily, formou různých půjček, úvěrů nastartovali německou ekonomiku
  \item[1925] nový prezident \textbf{Paul von Hindenburg}
  \item[1925] \textit{Locarnský garanční pakt}: Německo slibuje, že jeho západní hranice jsou neměnné (ale ne ty východní)
  \item[1929] \textit{Youngův plán}: reparace se snižují na čtvrtinu
  \item[1929-33] \textit{Velká hospodářská krize}: krachla newyorská burza, je to v krachu
  \item[1932] \textit{Laussanská konference}: reparace odpuštěny, němci musí zaplatit aspoň 4 mld. (ani to se nestane)
  \item[$-$] vrcholem meziválečných mírových snah je \textit{Briand-Kellogův pakt}  -- 60 států se zapojilo, v případě porušení měly být třeba sankce, ale těžce to nefunguje
\end{itemize}


\section*{První republika}
\begin{itemize}
    \vspace{-0.5em}
    \setlength\itemsep{0.15em}
    \item[28.10.1918] vyhlášena republika
    \item[$-$] ČSR vzešla na základě toho, že zaniká Rakousko-Uhersko, je pro nás zásadní, aby byl Versailleský mírový systém dodržován, dostali jsme Hlučínsko a nějaká území od nového Rakouska
    \item[$-$] \textit{Trianonská smlouva} s Maďarskem: získání Zakarpatské Rusi a Slovenska
    \item[$-$] Malá dohoda: ČSR, Rumunsko a budoucí Jugoslávie, orientace na Francii
    \item[$-$] president Masaryk, ministr zahraničí Edvard Beneš
    \item[$-$] i po vyhlášední státu trvalo dva roky, než se Československ árepublika konstituovala
    \item[11.11.1918] problém po vyhlášení státu Poláků, kteří obsadili značnou část Těšínska, argumentovali tím, že se tam hovoří většinově polsky, ale už dlouho toto území patřilo k Českému království, Poláci chtěli, aby tamní obyvatelé volili do parlamentu, měli být i povoláni do polské armády; na to zareagovali Češi, především legionáři tzv.
    \item[(23.-30.1.1919)] \textsc{sedmidenní válkou} $\rightarrow$ zabrání Těšínska, velmoci donutily se ČSR stáhnout
    \item[1920] \textit{arbitráž ve Spa}, dohoda Dohody, kudy povede hranice mezi ČSR a Polskem $\rightarrow$ neurčila to válka, ale velmoci; Bohumínsko-košická dráha je na našem území, a zůstaly tam i doly, zbytek mají Poláci, došlo i k rozpůlení Těšínska na český a polský
    \item[$-$] Poláci to nikdy nezkousli a po podepsání Mnichovské dohody si Těšín hned nárokovali
    \item[$-$] další problém je se Slovenskem, Maďaři odtamtud nechtěli
    \item[$-$] bylo vytvořeno ministerstvo pro Slovensko (ministrem Vavro Šrobár)
    \item[leden 1919] Češi museli vojensky obsadit Slovensko
    \item[$-$] na území Maďarska se vytvořila bolševická vláda Maďarská republika rad, Čechoslováci překročili hranice Maďarska s cílem porazit tuto vládu, to se nepodařilo a Maďaři šli do protiútoku a obsadili dokonce kus Slovenska, v části Slovenska zase vznikla Slovenská republika rad
    \item[$-$]  na popud Francie Maďaři vyhnáni a Slovenská republika rad byla zrušena
    \item[1920] situace vyřešena \textit{Červnovou mírovou smlouvou} v Trianonu, kde byly stanoveny hranice mezi Slovenskem a Maďarskem
    \item[$-$] to se nelíbí Maďarům, protože ztratili Slovensko, zůstává tam maďarské obyvatelstvo
    \item[$-$] za Mnichovské dohody Maďaři část Slovenska získali a obsadili
    \item[říjen 1918] sešel se Masaryk s Rusíny ve Spojených státech, kde se dohodli na  tom, že Rusíni chtějí být součástí Českoslovesnké republiky, musí to ještě potvrdit oficiální politická reprezentace
    \item[8.5.1919] centrem Podkarpatské Rusi je Užhorod, tam se potvrdilo, že to bude součást ČSR
    \item[$-$] pro ČSR spíše nevýhodné, území je velmi zaostalé, na tomto území prochází maďarizace, žádné vzdělání
\end{itemize}

\subsection*{České pohraničí}
\begin{itemize}
    \vspace{-0.5em}
    \setlength\itemsep{0.15em}
    \item[$-$] už před první světovou válkou se používá termín \textit{sudety}
    \item[$-$] nějakých 25 \% obyvatelstva ČSR jsou menšiny
    \item[$-$] po vyhlášení republiky byly vytvořeny čtyři německé provincie (Jihlava, Olomouc, Brno byly také silné německé enklávy)
  \item Československo muselo obsadit pohraničí vojenskou silou, sudety, němci
  \item národností složení -- víc němců než slováků, myšlenka čechoslovakismu -- češi a slováci dvě větve jednoho národa
  \item české země nejindustriálnější částí RU monarchie, sklářství, těžší průmysl, potravinářství, pivovarnictví
  \item slovensko už bylo agrární a podkarpatská rus ta byla úplně ztracená
  \item jsou důležité měnové reformy -- muselo dojít k rychlé měnové odluce od rakouské koruny, za toto všechno děkujeme legendárnímu českému ekonomovi dru. Aloisi Rašínovi, měl obrovské znalosti, až chodící ekonomická encyklopedie, snažil se aby české země neměly vysokou inflaci (i na úkor zaměstanosti), že se musí uskromnit lidé aby se nastartovala ekonomika -- to mu udělalo pár nepřátel, tedy zejména anarchokomunistů -- nakonec ho zastřelili
  \item čtvrtinu rakouských bankovek stáhl z oběhu -- měna posílila, byla zavedena československá koruna v poměru 1:1, povedlo se to rychle, do r. 1922 vyřešil inflaci, položil základy státních rezerv (pokladů)
  \item další důležitou reformou byla ta pozemková -- poválečný parlament zrušil šlechtické tituly, také byly omezeny statky církví a velkostatkářů -- nad 150 ha museli půdu prodat, lesy připadly především státu, tato půda která byla rozprodána (asi 28\% rozlohy ČSR), zásluha Antonína Švehly
  \item sociální situace nicméně ideální nebyla, byli sirotci, vdovy, byla nezaměstanost, nedostatek výrobků, potravin -- typická poválečná krisička
  \item 14.11.1918 v Thunovském paláci na Revolučním národním shromáždění byla zvolena vláda Všenárodní koalice v čele s Karlem Kramářem, bylo asi 20 stran v obecném povědomí: agrárníci (Švehla -- vnitro), sociální demokraté (Fr. Soukup -- spravedlnost), národní demokracie (mladočeši, pravice, Kramář -- předseda, Rašín -- finance), národní socialisté (Václav Klofáč -- obrana), čsl strana lidová (Jan Šrámek -- bez portfeje), slováci (v. šrobár -- ministr slovenska), původně i MRŠ -- min. války (ale atentát z dílny Edvarda Beneše ho utnul (vtip(?)))
  \item potom proběhly volby do obecních zastupitelstev, zvítězila socdem, za nimi agrárníci, v důsledku tohoto Kramář podává demisi, vzniká vláda tzv. rudozelené koalice v čele Vlastimil Tusar
  \item 28.1.1919 založena MUNI, 27.6.1919 Komenského univerzita v Bratislavě (nojo slováci mají výšku fakt super)
  \item 29.2.1920 schválena Ústavní listina republiky Československé -- \uv{Masarykova ústava}, tři šložky, dvoukom. parl., jako dnes ale jiné počty posl. a sen. a jiné věkové limity a délky mandátů
  \item u presidenta už se počítalo s tím, že by mohl nastoupit dr. Edvard Beneš, takže tam byl limit od 35 let
  \item zřízeny nezávislé soudy, ústavní soud (toto trochu výjimka myslim)
  \item v dubnu 1920 probíhají první regulerní parlamentní volby, ústávají socdem, takže druhá Tusarova vláda
  \item joo Masaryk, byly volby presidentské regulerní, v roce 1920 bezkonkurenčně vítězí Tomáš Garrigue Masaryk
\end{itemize}



\end{document}
