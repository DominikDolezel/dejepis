\documentclass{article}
\usepackage{fullpage}
\usepackage[czech]{babel}
\usepackage{amsfonts}

\title{\vspace{-2cm}Český stát\vspace{-1.7cm}}
\date{}
\author{}

\begin{document}
\maketitle
\section*{Přírodní podmínky}
\begin{itemize}
    \vspace{-0.5em}
    \setlength\itemsep{0.15em}
    \item[$-$] nížiny, husté a nepropustné lesy $\rightarrow$ přirozená hranice
    \item[$-$] \textit{trojpolní systém} -- lada, ozim, jař
    \item[$-$] keramika, textilie, směnný obchod
    \item[$-$] půdu vlastní šlechta, lidé si ji pronajímají za \textit{rentu}
    \item[$-$] \textit{rustikál} = půda, kterou si vesničané pronajímají od šlechty
    \item[$-$] \textit{dominikál} = půda, kterou vlastní šlechta (zámky, statky krále)
    \item[$-$] kolem vesnic pastviny ve společném majetku obce
    \item[$-$] obchodníci v podhradí
\end{itemize}

\section*{Osídlení}
\begin{itemize}
    \vspace{-0.5em}
    \setlength\itemsep{0.15em}
    \item[$-$] Charváti (V Čechy) -- Libice nad Cidlinou, vládnou Slavníkovci
    \item[$-$] Čechové (střední Čechy) -- Budeč, Levý Hradec, vládnou Přemyslovci
    \item[$-$] Přemyslovci název od Přemysla Oráče
    \item[$-$] postupně sjednotí ostatní kmeny
\end{itemize}

\begin{itemize}
    \vspace{-0.5em}
    \setlength\itemsep{0.15em}
    \item[poč. 9. st.] \textbf{Karel Veliký} vtrhne do Čech, vynutí si tribut = poplatek za to, že na ně nebude útočit
    \item[845] v Řezně 14 českých kmenových knížato bylo pokřtěno
    \item[2. pol. 9. st.] sjendocovací proces: prostor Čech ovládl kmen Čechů, vládnou Přemyslovci
    \item[$-$] legendární knížata: Přemysl, Nezamysl, Mnata, Vojen, Vnislav, Křesomysl, Neklan, Hostivít, \dots
\end{itemize}

\section*{Bořivoj (867 -- 894)}
\begin{itemize}
    \vspace{-0.5em}
    \setlength\itemsep{0.15em}
    \item[$-$] manželka Ludmila
    \item[$-$] zástupce Svatupluka -- knížete VM v Čechách
    \item[883] křest
    \item[$-$] stavba rotundy sv. Klimenta -- první křesťanský kostel v Čechách
    \item[$-$] začátek stavby Pražského hradu, kostelák Panny Marie
    \item[$-$] synové: Spytihněv a Vratislav
\end{itemize}

\section*{Spytihněv (894 -- 915)}
\begin{itemize}
    \vspace{-0.5em}
    \setlength\itemsep{0.15em}
    \item[$-$] dostává se, když Svatopluk umírá
    \item[$-$] odtrhne Čechy od Velké Moravy
    rotunda na Budči ssv. Petra a Pavla
    \item[$-$] budování Pražského hradu
    \item[$-$] základy státní správy
\end{itemize}

\section*{Vratislav (915 -- 921)}
\begin{itemize}
    \vspace{-0.5em}
    \setlength\itemsep{0.15em}
    \item[$-$] manželka Drahomíra (nechala zavraždit Ludmilu)
    \item[$-$] dva synové: Václav a Boleslav
    \item[$-$] úspěšné boje s Maďary 
\end{itemize}




\end{document}
