\documentclass{article}
\usepackage{fullpage}
\usepackage[czech]{babel}
\usepackage{amsfonts}

\title{\vspace{-2cm}Druhá světová válka\vspace{-1.7cm}}
\date{}
\author{}

\begin{document}
\maketitle

\begin{itemize}
    \vspace{-0.5em}
    \setlength\itemsep{0.15em}
    \item[$-$] začátek 1. září 1939, konec 2. září 1945
\end{itemize}
\subsection*{Situace před válkou}
\begin{itemize}
    \vspace{-0.5em}
    \setlength\itemsep{0.15em}
    \item[$-$] agresivní země: Německo, Itálie, Japonsko; VB pokračuje v politice izolacionismu
    \item[$-$] OSN je k ničemu, ničenu nezabránila
    \item[březen 1939] zánik Československa, vyhlášen protektorát, Německo získalo v Litvě přístav Memel (= Kleipeda) $\rightarrow$ Litva nemá přístup k moři
    \item[$-$] Hitler má další požadavky, tentokrát směřují k Polsku, chce spojit Německo s Východním Pruskem přes Polské území, taky chce přístav Gdaňsk (polosamostatný městský stát), tentokrát Hitler narazil -- Anglie i Francie pochopily, že ústupky nezabraly, začínají tedy Poslku garantovat neměnnost hranic
    \item[duben 1939] Italové anektovali Albánii, ve Španělsku se ujímá moci generál Franco, taky Hitler slavil svoje padesátiny
    \item[23.8.1939] rozebíhají se jednání mezi Brity, Francouzi a Sověty s Němci, výsledek je \textit{pakt Ribbentrop-Molotov} o neútočení mezi Němci a Rusy, měl dodatky, které veřejnost neznala, v nich šlo o rozdělení sfér vlivu v Evropě mezi němci a Sovětským svazem, ty dodatky byly zvěřejněny až v roce 1989
    \item[$-$] Sověti chtějí zpět to, co ztratili po první světové válce (Finsko, Polsko, kus Běloruska, Ukrajiny, Besarabii)
    \item[1.9.1939] Mussolini Hitlerovi oznamuje, že itálie není připravena naválku a preferoval by něco jako Mnichov (konferenci), ale hitler chtěl válčit, a tak zinscenoval útok na německou vysílačku Gleiwitz (Gliwice), útočí na polský poloostrov Westerplatte, kde taky měli sklad munice, vpád je bleskový (\textit{blitzkrieg}), tanky jsou podporovány Luftwaffe, Poláci nejsou vyzbrojeni, od začátku jasné, že Poláci prohrají, k útoku se připojila jako jediná země Slovensko
    \item[$-$] reakce velmocí: Francie a británie vyhlašují Němcům válku, ale nic nedělají (\textit{Sitzkrieg}), chtějí blokovat dovoz do Německa (neutrální země mohly obchodovat), nebyla však moc úspěšná
    \item[$-$] co udělali Němeci ze západu, to udělal Sovětský svaz z východu
    \item[17.9.] vpád Sovětského svazu do Polska, oficiálně zdůvodňují tím, že chtějí udělat předsunuté hranice, aby chránili svoje obyvatelstvo před vpádem Němců
    \item[$-$] v roce 1940 došlo k vyvraždění polské elity u Katyni, Stalin dlouho tvrdil, že to udělali Němeci, ale udělali to Sověti
    \item[$-$] vznik nové německo-sovětské hranice, Polsko zase není (někdy se mluví o čtvrtém dělení Polska)
    \item[$-$] ošetřeno hospodářskou smlouvou (únor 1940)
    \item[$-$] vytvoření území Generálního gouvernementu, pod německým vlivem, cílem postupně odtud vystrnadit kohokoliv (Poláky a židy) a udělat z toho jakési německé území, v jeho čele je Hanse Frank

\end{itemize}

\subsection*{Periodisace války}
\begin{enumerate}
    \vspace{-0.5em}
    \setlength\itemsep{0.15em}
    \item 1.9.1939 až 21.6.1941
    \item 22.6.1941 (útok na Sovětský svaz) až konec roku 1942 (bitva u Stalingradu)
    \item 2.2.1943 až léto 1944
    \item 6.6.1944 (vylodění v Normandii) až 8.5.1945 (konec války v Evropě)
    \item květen 1945 až 2.9.1945
\end{enumerate}

\subsection*{Charakter války}

\begin{itemize}
    \vspace{-0.5em}
    \setlength\itemsep{0.15em}
    \item[$-$] padlo 110 milionu vojáků
    \item[$-$] dalších 58 dalších civilistů zemřelo
    \item[$-$] 35 mil. zraněných
    \item[$-$] druhá průmyslová revoluce $\rightarrow$ tanky, letectvo, ponorky, atomové bomby, penicilin
    \item[$-$] Němci měli navrch do roku 1941, kdy vstoupily do války Spojené státy
    \item[$-$] fašistické státy: Německo, Itálie, Slovensko, Maďarsko, Rumunsko, Bulharsko, Chrovatsko, Japonsko, Thajsko (umožňovalo Japonsku operovat ze svého území), Finsko
    \item[$-$] neutrální státy: Švédsko, Irsko, Švýcarsko, Portugalsko, Španělsko, Turecko
\end{itemize}

\subsection*{Zisky SSSR do roku 1940}
\begin{itemize}
    \vspace{-0.5em}
    \setlength\itemsep{0.15em}
    \item[podzim 1939] Sověti do Pobaltí přesunuli jednotky, pak tam proběhly volby, vznikly prosovětské vlády, výsledkem byl vstup Pobalstkých zemí do SOvětského svazu v červenci 1940
    \item[červen 1940] SSSR vnutil Rumunsku smlouvu o vstupu do SSSR: Besarabie, Bukovina
    \item[$-$] pokus o úpravu hranic s Finskem nevyšel, chtěli poloostrov Hanko, Finsko odmítlo, SSSR napadl Finsko
    \item[30.11.1939-12.3.1940] \textsc{zimní válka}: Finové se brání, mají postavenu Mannerheimovu linii (Carl Gustav Emil Mannerheim, tehdejší Finský president, považován za zakladatele Finska), válka vyvolala nevoli především Británie a Francie, kteří finančně Finsko podporovaly, SSSR vyřazeno ze SPolečenství národů, končí \textit{Moskevským mírem} (12.3.1940), získávají jenom 10 \% území Finska
    \item[$-$] Mannerheimova linie nebyla proražena (jen malinko)
    \item[$-$] když byli Sověti napadnuti Němci v roce 1941, Finové se postavili na stranu Němců
\end{itemize}

\subsection*{Útok na Skandinávii}
\begin{itemize}
    \vspace{-0.5em}
    \setlength\itemsep{0.15em}
    \item[$-$] aktivita Německa
    \item[9.4.-10.6.1940] \textsc{válka o Skandinávii}: jen o Dánsko a Norsko, operace Weserübung, Dánsko dobyto hned, D8nové se hezky zachovali ke svým Židovským obyvatelům, kteří  byli na lodích přepraveni do Švédska
    \item[$-$] Britové a Francouzi chtějí zastavit dodávky železa ze Švédska do Německa, což vyvolalo útok na Skandinávii
    \item[$-$] Vidkun Quisling -- vznik kolaborantské vlády v Norsku, státní převrat
    \item[$-$] na straně Norů od začátku Britové a Francouzi
    \item[$-$] na sklonku války od 10. května 1940 začíná útok na západní Evropu (Benelux a Francii), jakmile Němci zaútočili, Francie i Anglie stáhly svá vojska z války o Skandinávii a Němci dobyli Narwick, čímž končí válka o Skandinávii
\end{itemize}

\subsection*{Útok na západní Evropu}
\begin{itemize}
    \vspace{-0.5em}
    \setlength\itemsep{0.15em}
    \item[10.5.-22.6.1940] operace Gelb: útok na Benelux
    \item[12.5.1940] útok na Francii, vpadli tam do severu, na jihu je totiž Maginotova linie, Francouzům pomáhají Británie a benelux, ale neúspěšně, někteří vojáci po porážce u Calais přepraveni do Británie z přístavu Dunkerque
    \item[10.6.1940] Mussolini útočí na Francii a získává drobné území
    \item[$-$] premiér Philippe Pétain, který válčil už v první světové válce, byl donucen, aby tu kapitulaci podepsal on: \textit{Kapitulace v Compiegne}
    \item[22.6.1940] Francie kapituluje
    \item[$-$] Rotterdam krom jednoho chrámu srovnán se zemí
    \item[$-$] Němci okupují Francii na Z u moře, v centrální části je tzv. Vichistická Francie (kolaborační režim), vydžel jen do srpna 1944, kdy začíná osvobozování Francie z jihu
    \item[$-$] mezi spojenci se objevily i jednotky Charlese de Gaulla, který utekl do zahraničí a poté organizoval jednotky svobodné Francie
\end{itemize}

\subsection*{Bitva o Británii}
\begin{itemize}
    \vspace{-0.5em}
    \setlength\itemsep{0.15em}
    \item bitva o británii -- němci by chtěli dobýt británii, vylodění pomocí operace lvoun
    \item adlertag 13. srpna
    \item do řijna nejintenzivnější vzdušná válka o britský prostor
    \item první německá prohra
    \item od roku 1942 to už budou britové kteří bombardují německo
    \item Luftwaffe x RAF
    \item[říjen 1940] na španělsko-francouzských hranicích probíhá jednání Hitlera a Franca, pokud se Franco přidá do války a dobude Gibraltar, tak si ho Španělsko může nechat a dostane k tomu nějakou tu kolonii, nicméně Franco to nepřijal, Španělsko bude neutrální
    \item[listopad 1940] Hitler nabízí Stalinovi, jestli se nepřidá do války proti kapitalistům
    \item v RAF i českoslovenští dobrovolníci
    \item pojďme na Balkán
    \item[duben 1939] Itálie anektuje Albánii, plánuje co dál
    \item[říjen 1940] Mussolini útočí na Řecko, je to obří fiasko, Řekové částečně vstoupili do Albánie
    \item[březen 1941] Hitler si je nakláněl, v březnu se Hitlerovi povedlo dostat Jugoslávii do aliance, proběhne tam ale do týdne puč, Hitler reaguje a Jugoslávie je v rámci operace Marika během dubna obsazena, pak je dobyto i Řecko, trošku je okousají, osamostatní se Chorvatsko v čele s ustašovci, v čele s Ante Paveličem, dále Srbsko v čele s gen. Nedičem
    \item Řecko je připojeno k Itálii, Makedonii získává Bulharsko
    \item v Jugoslávii je silná partyzánská aktivita v čele s Josipem Brozem Titem (this is a surprise tool that will help us later)
    \item[květen -- červen 1941] operace Merkur -- invaze na Krétu
    \item válka v africe
    \item Italové drží Libyi, Ethiopii, Somálsko a Eritreu, chtějí dobýt britský Egypt
    \item[$\Rightarrow$] otevíráme libyjsko-egyptskou frontu (září 1940)
    \item[únor 1941] Italové jsou zatlačeni
    \item[březen 1941] vložte postavu německého tankového velitele Erwina Rommela, nicméně přestal být v popředí kvůli Barbarosse, opevní Tobruk, který je pak pracně dobýván
    \item[květen 1941] habešsko-somálská fronta: Italové vedou neúspěšný útok na britské Somálsko, Briti zareagovali tím, že dobyli celou Italskou východní Afriku
    \item[duben 1941] Japonsko a SSSR uzavírají smlouvu o neútočení
    \item[březen 1941] zákon o půjčce a pronájmu -- USA \uv{půjčí} Britům a ostatním spojencům (potom i SSSR) vybavení za peníze, které zaplatí po válce (pak se to promlčelo)
    \item jedeme na východní frontu
    \item[22. června 1941] začíná útok Německa na SSSR, okolo 70\% německé vojenské síly se přesune na východní frontu, Němci mají srovnatelný počet ale mnohem vyšší kvalitu vybavení, Stalin taky před válkou zlikvidoval většinu schopných generálů v rámci čistek (Tuchačevsky), sovětské vedení je tedy nezkušené
    \item ze začátku padaly celé sovětské armády do německého zajetí
    \item Stalina částečně zachránilo to, že se mu povedlo přesunout část průmysl za Ural
    \item německý útok veden třemi směry, na sever do Leningradu, v centru do Moskvy, na jih do Stalingradu, na Kavkaz (ropa)
    \item německá invaze ale nemá před zimou moc času, na podzim začíná pršet, pak je krutá zima, na kterou není Německo plně připraveno
    \item[září 1941 - leden 1944] Leningrad je obklíčen, začíná dlouhé obléhání Leningradu, Leningrad zásobován pouze po jezeře
    \item na podzim Hitler prohlašuje, že už vyhrál, v říjnu začíná operace Tajfun -- dobytí Moskvy, Německo je na 20 km od Moskvy
    \item začíná ale zima a je to v pytli, v prosinci začíná protiútok generála Žukova, který frontu posouvá v místech až 200 km na západ
    \item[srpen 1941] Atlantická charta -- Chruchill a FDR prohlašují, že musí porazit fašistické země, že uchází především o osvobození fašisty podrobených zemí
    \item[$-$] do Británie se stahují vlády okupovaných zemí
    \item[$-$] Britové mají výhody: velké námořnictvo, letectvo, impérium, radar
    \item[květen 1940] k moci se dostávají
    konzervativci v čele s Churchillem, odmítají politiku appeasementu
    \item[$-$] strategie měla být taková, že británii budou nejprve bombardovat a pak se tam vylodí: operace Seelöwe
    \item[10.7.1940] začíná bombardování britských ostrovů
    \item[13.8.1940] den orlů -- den, kdy bombardování bylo nejintenzivnější
    \item[$-$] střet Geringovy Luftwaffe s RAF
    \item[$-$] Churchill chodil mezi lidi, dodával jim odvahu
    \item[$-$] v letectvu také českoslovenští letci
    \item[$-$] podařilo se Británii ubránit -- první německý neúspěch, nezískali britské ostrovy
\end{itemize}

\subsection*{Válka v pacifiku}
\begin{itemize}
  \item[1937] čínsko-japonská válka
  \item[1940] Japonsko obsazuje francouzskou Indočínu
  \item USA uvaluje na Japonsko ropné embargo, což vede k
  \item[7.12.1941] útok na Pearl Harbor, potopili zepár lodí, letadel, den na to FDR předstupuje před Kongres a USA vyhlašuje válku Japonsku (do pár dnů i jejich spojencům -- Německu, Itálii)
  \item v letech 1940-42 vobdobí rychlé japonské expanze, obsazují evropské kolonie v regionu, prvně Japonci vnímáni jako osvoboditelé od kolonizátorů, pozěji místním obyvatelům slibují za pomoc nezávislost po válce
  \item[3.-7.6.1942] bitva u Midway -- kombinovaná námořní a pozemní bitva o ostrov Midway, Američani rozluštili japonský kód, odrazili jejich útok -- obrat ve válce
  \item[VIII. 1942 - II. 1943] bitva o Guadalcanal (Šalamounovy o-vy), USA donutí Japonce se stáhnout, velká pozemní výhra USA
  \item[1.1.1942] Washingtonská deklarace -- viz Atlantická charta: Němci dostanou nabudku, z tohoto pak vyplývá OSN
  \item 1943 celkově zlomový rok v druhé sv. v.
  \item koncem roku 1942 rozluštěna německá šifra Enigma (Turing)
\end{itemize}

-- hodně od hanky --

\subsection*{Apeninský poloostrov 1945}
\begin{itemize}
    \vspace{-0.5em}
    \setlength\itemsep{0.15em}
    \item[9.4.1945] dobyto Miláno
    \item[$-$] Spojenci v Pádské nížině, Němci zatlačeni k Pádu
    \item[$-$] Mussolini chtěl utéct do Švýcarska, ale byl chycen partyzány a zabit, ostatky jeho milenky byly vyvěšeny v Miláně
    \item[2.5.] německá kapitulace
\end{itemize}

\subsection*{Západní fronta 1945 - Hamburk}
\begin{itemize}
    \vspace{-0.5em}
    \setlength\itemsep{0.15em}
    \item[1944] Němci se tu pokusili o poslední ofenzivu, boje směřovali k přístavuv v Antverpách, výběžek zlikvidován do ledna 1945, důvodem bylo špatné zásobování a zimní počasí
    \item[únor] Spojencům se podařilo prolomit Siegfriedovu linii podíl řeky Maasy dlouhých asi 5 km, snažili se dál pronikat do Třetí říše, zlomové bylo to, že se dostali přes řeku
    \item[březen 1945] dostávají se přes řeku a postupují dál do TŘETÍ ŘÍŠE, PŘESTOŽE nĚMCI MOSTY NIČILI, poslední z nich -- most u remagennu, kterého využili spojenci
    \item[$-$] pravidelné nálety na TŘETÍ říši už od roku 1943
    \item[12.4.] smrt F. D. Roosevelta, prezidentem se stává jeho viceprezident Harry S. Truman, ten pokračuje v téže zahraniční politice
    \item[25.4.] Američané a Sověti se setkali na Labi (u Torgau), dosáhli tak demarkační linie
    \item[$-$] rozbombardovaný Hamburg, Drážďany\\
    -- nsection porážka německa
    \item[30.4.] \textsc{bitva o Berlín}: na Berlín táhnout dva fronty: Ukrajinský a Běloruský, Berlín dobyl Žukov, bránit se mohlo jen asi milion Němců, zatímco Sovětů bylo dva a půl milionu, Hitler pobýva, v bunkru i s ostatními, byl tam i s jeho milenkou Evou Braunovou, kterou si den předtím, než spáchali sebevraždu, si ji vzal za ženu (ampulkou, Hitler se prostřelil hlavu)
    \item[2.5.] oficiální kapitulace Berlína
    \item[7.5.] \textit{Kapitulace v Remeši} Němců, za ně podepisoval Alfred Jodl, s platností od 8.5.
    \item[8.5.] Stalinovi se to zdálo málo, došlo k další kapitulaci v Berlíně, kterou podepsal Wilhelm Kettel
\end{itemize}

\magore{\documentclass{article}
\usepackage{fullpage}
\usepackage[czech]{babel}
\usepackage{amsfonts}

\title{\vspace{-2cm}Druhá světová válka\vspace{-1.7cm}}
\date{}
\author{}

\begin{document}
\maketitle

\begin{itemize}
    \vspace{-0.5em}
    \setlength\itemsep{0.15em}
    \item[$-$] začátek 1. září 1939, konec 2. září 1945
\end{itemize}
\subsection*{Situace před válkou}
\begin{itemize}
    \vspace{-0.5em}
    \setlength\itemsep{0.15em}
    \item[$-$] agresivní země: Německo, Itálie, Japonsko; VB pokračuje v politice izolacionismu
    \item[$-$] OSN je k ničemu, ničenu nezabránila
    \item[březen 1939] zánik Československa, vyhlášen protektorát, Německo získalo v Litvě přístav Memel (= Kleipeda) $\rightarrow$ Litva nemá přístup k moři
    \item[$-$] Hitler má další požadavky, tentokrát směřují k Polsku, chce spojit Německo s Východním Pruskem přes Polské území, taky chce přístav Gdaňsk (polosamostatný městský stát), tentokrát Hitler narazil -- Anglie i Francie pochopily, že ústupky nezabraly, začínají tedy Poslku garantovat neměnnost hranic
    \item[duben 1939] Italové anektovali Albánii, ve Španělsku se ujímá moci generál Franco, taky Hitler slavil svoje padesátiny
    \item[23.8.1939] rozebíhají se jednání mezi Brity, Francouzi a Sověty s Němci, výsledek je \textit{pakt Ribbentrop-Molotov} o neútočení mezi Němci a Rusy, měl dodatky, které veřejnost neznala, v nich šlo o rozdělení sfér vlivu v Evropě mezi němci a Sovětským svazem, ty dodatky byly zvěřejněny až v roce 1989
    \item[$-$] Sověti chtějí zpět to, co ztratili po první světové válce (Finsko, Polsko, kus Běloruska, Ukrajiny, Besarabii)
    \item[1.9.1939] Mussolini Hitlerovi oznamuje, že itálie není připravena naválku a preferoval by něco jako Mnichov (konferenci), ale hitler chtěl válčit, a tak zinscenoval útok na německou vysílačku Gleiwitz (Gliwice), útočí na polský poloostrov Westerplatte, kde taky měli sklad munice, vpád je bleskový (\textit{blitzkrieg}), tanky jsou podporovány Luftwaffe, Poláci nejsou vyzbrojeni, od začátku jasné, že Poláci prohrají, k útoku se připojila jako jediná země Slovensko
    \item[$-$] reakce velmocí: Francie a británie vyhlašují Němcům válku, ale nic nedělají (\textit{Sitzkrieg}), chtějí blokovat dovoz do Německa (neutrální země mohly obchodovat), nebyla však moc úspěšná
    \item[$-$] co udělali Němeci ze západu, to udělal Sovětský svaz z východu
    \item[17.9.] vpád Sovětského svazu do Polska, oficiálně zdůvodňují tím, že chtějí udělat předsunuté hranice, aby chránili svoje obyvatelstvo před vpádem Němců
    \item[$-$] v roce 1940 došlo k vyvraždění polské elity u Katyni, Stalin dlouho tvrdil, že to udělali Němeci, ale udělali to Sověti
    \item[$-$] vznik nové německo-sovětské hranice, Polsko zase není (někdy se mluví o čtvrtém dělení Polska)
    \item[$-$] ošetřeno hospodářskou smlouvou (únor 1940)
    \item[$-$] vytvoření území Generálního gouvernementu, pod německým vlivem, cílem postupně odtud vystrnadit kohokoliv (Poláky a židy) a udělat z toho jakési německé území, v jeho čele je Hanse Frank

\end{itemize}

\subsection*{Periodisace války}
\begin{enumerate}
    \vspace{-0.5em}
    \setlength\itemsep{0.15em}
    \item 1.9.1939 až 21.6.1941
    \item 22.6.1941 (útok na Sovětský svaz) až konec roku 1942 (bitva u Stalingradu)
    \item 2.2.1943 až léto 1944
    \item 6.6.1944 (vylodění v Normandii) až 8.5.1945 (konec války v Evropě)
    \item květen 1945 až 2.9.1945
\end{enumerate}

\subsection*{Charakter války}

\begin{itemize}
    \vspace{-0.5em}
    \setlength\itemsep{0.15em}
    \item[$-$] padlo 110 milionu vojáků
    \item[$-$] dalších 58 dalších civilistů zemřelo
    \item[$-$] 35 mil. zraněných
    \item[$-$] druhá průmyslová revoluce $\rightarrow$ tanky, letectvo, ponorky, atomové bomby, penicilin
    \item[$-$] Němci měli navrch do roku 1941, kdy vstoupily do války Spojené státy
    \item[$-$] fašistické státy: Německo, Itálie, Slovensko, Maďarsko, Rumunsko, Bulharsko, Chrovatsko, Japonsko, Thajsko (umožňovalo Japonsku operovat ze svého území), Finsko
    \item[$-$] neutrální státy: Švédsko, Irsko, Švýcarsko, Portugalsko, Španělsko, Turecko
\end{itemize}

\subsection*{Zisky SSSR do roku 1940}
\begin{itemize}
    \vspace{-0.5em}
    \setlength\itemsep{0.15em}
    \item[podzim 1939] Sověti do Pobaltí přesunuli jednotky, pak tam proběhly volby, vznikly prosovětské vlády, výsledkem byl vstup Pobalstkých zemí do SOvětského svazu v červenci 1940
    \item[červen 1940] SSSR vnutil Rumunsku smlouvu o vstupu do SSSR: Besarabie, Bukovina
    \item[$-$] pokus o úpravu hranic s Finskem nevyšel, chtěli poloostrov Hanko, Finsko odmítlo, SSSR napadl Finsko
    \item[30.11.1939-12.3.1940] \textsc{zimní válka}: Finové se brání, mají postavenu Mannerheimovu linii (Carl Gustav Emil Mannerheim, tehdejší Finský president, považován za zakladatele Finska), válka vyvolala nevoli především Británie a Francie, kteří finančně Finsko podporovaly, SSSR vyřazeno ze SPolečenství národů, končí \textit{Moskevským mírem} (12.3.1940), získávají jenom 10 \% území Finska
    \item[$-$] Mannerheimova linie nebyla proražena (jen malinko)
    \item[$-$] když byli Sověti napadnuti Němci v roce 1941, Finové se postavili na stranu Němců
\end{itemize}

\subsection*{Útok na Skandinávii}
\begin{itemize}
    \vspace{-0.5em}
    \setlength\itemsep{0.15em}
    \item[$-$] aktivita Německa
    \item[9.4.-10.6.1940] \textsc{válka o Skandinávii}: jen o Dánsko a Norsko, operace Weserübung, Dánsko dobyto hned, D8nové se hezky zachovali ke svým Židovským obyvatelům, kteří  byli na lodích přepraveni do Švédska
    \item[$-$] Britové a Francouzi chtějí zastavit dodávky železa ze Švédska do Německa, což vyvolalo útok na Skandinávii
    \item[$-$] Vidkun Quisling -- vznik kolaborantské vlády v Norsku, státní převrat
    \item[$-$] na straně Norů od začátku Britové a Francouzi
    \item[$-$] na sklonku války od 10. května 1940 začíná útok na západní Evropu (Benelux a Francii), jakmile Němci zaútočili, Francie i Anglie stáhly svá vojska z války o Skandinávii a Němci dobyli Narwick, čímž končí válka o Skandinávii
\end{itemize}

\subsection*{Útok na západní Evropu}
\begin{itemize}
    \vspace{-0.5em}
    \setlength\itemsep{0.15em}
    \item[10.5.-22.6.1940] operace Gelb: útok na Benelux
    \item[12.5.1940] útok na Francii, vpadli tam do severu, na jihu je totiž Maginotova linie, Francouzům pomáhají Británie a benelux, ale neúspěšně, někteří vojáci po porážce u Calais přepraveni do Británie z přístavu Dunkerque
    \item[10.6.1940] Mussolini útočí na Francii a získává drobné území
    \item[$-$] premiér Philippe Pétain, který válčil už v první světové válce, byl donucen, aby tu kapitulaci podepsal on: \textit{Kapitulace v Compiegne}
    \item[22.6.1940] Francie kapituluje
    \item[$-$] Rotterdam krom jednoho chrámu srovnán se zemí
    \item[$-$] Němci okupují Francii na Z u moře, v centrální části je tzv. Vichistická Francie (kolaborační režim), vydžel jen do srpna 1944, kdy začíná osvobozování Francie z jihu
    \item[$-$] mezi spojenci se objevily i jednotky Charlese de Gaulla, který utekl do zahraničí a poté organizoval jednotky svobodné Francie
\end{itemize}

\subsection*{Bitva o Británii}
\begin{itemize}
    \vspace{-0.5em}
    \setlength\itemsep{0.15em}
    \item bitva o británii -- němci by chtěli dobýt británii, vylodění pomocí operace lvoun
    \item adlertag 13. srpna
    \item do řijna nejintenzivnější vzdušná válka o britský prostor
    \item první německá prohra
    \item od roku 1942 to už budou britové kteří bombardují německo
    \item Luftwaffe x RAF
    \item[říjen 1940] na španělsko-francouzských hranicích probíhá jednání Hitlera a Franca, pokud se Franco přidá do války a dobude Gibraltar, tak si ho Španělsko může nechat a dostane k tomu nějakou tu kolonii, nicméně Franco to nepřijal, Španělsko bude neutrální
    \item[listopad 1940] Hitler nabízí Stalinovi, jestli se nepřidá do války proti kapitalistům
    \item v RAF i českoslovenští dobrovolníci
    \item pojďme na Balkán
    \item[duben 1939] Itálie anektuje Albánii, plánuje co dál
    \item[říjen 1940] Mussolini útočí na Řecko, je to obří fiasko, Řekové částečně vstoupili do Albánie
    \item[březen 1941] Hitler si je nakláněl, v březnu se Hitlerovi povedlo dostat Jugoslávii do aliance, proběhne tam ale do týdne puč, Hitler reaguje a Jugoslávie je v rámci operace Marika během dubna obsazena, pak je dobyto i Řecko, trošku je okousají, osamostatní se Chorvatsko v čele s ustašovci, v čele s Ante Paveličem, dále Srbsko v čele s gen. Nedičem
    \item Řecko je připojeno k Itálii, Makedonii získává Bulharsko
    \item v Jugoslávii je silná partyzánská aktivita v čele s Josipem Brozem Titem (this is a surprise tool that will help us later)
    \item[květen -- červen 1941] operace Merkur -- invaze na Krétu
    \item válka v africe
    \item Italové drží Libyi, Ethiopii, Somálsko a Eritreu, chtějí dobýt britský Egypt
    \item[$\Rightarrow$] otevíráme libyjsko-egyptskou frontu (září 1940)
    \item[únor 1941] Italové jsou zatlačeni
    \item[březen 1941] vložte postavu německého tankového velitele Erwina Rommela, nicméně přestal být v popředí kvůli Barbarosse, opevní Tobruk, který je pak pracně dobýván
    \item[květen 1941] habešsko-somálská fronta: Italové vedou neúspěšný útok na britské Somálsko, Briti zareagovali tím, že dobyli celou Italskou východní Afriku
    \item[duben 1941] Japonsko a SSSR uzavírají smlouvu o neútočení
    \item[březen 1941] zákon o půjčce a pronájmu -- USA \uv{půjčí} Britům a ostatním spojencům (potom i SSSR) vybavení za peníze, které zaplatí po válce (pak se to promlčelo)
    \item jedeme na východní frontu
    \item[22. června 1941] začíná útok Německa na SSSR, okolo 70\% německé vojenské síly se přesune na východní frontu, Němci mají srovnatelný počet ale mnohem vyšší kvalitu vybavení, Stalin taky před válkou zlikvidoval většinu schopných generálů v rámci čistek (Tuchačevsky), sovětské vedení je tedy nezkušené
    \item ze začátku padaly celé sovětské armády do německého zajetí
    \item Stalina částečně zachránilo to, že se mu povedlo přesunout část průmysl za Ural
    \item německý útok veden třemi směry, na sever do Leningradu, v centru do Moskvy, na jih do Stalingradu, na Kavkaz (ropa)
    \item německá invaze ale nemá před zimou moc času, na podzim začíná pršet, pak je krutá zima, na kterou není Německo plně připraveno
    \item[září 1941 - leden 1944] Leningrad je obklíčen, začíná dlouhé obléhání Leningradu, Leningrad zásobován pouze po jezeře
    \item na podzim Hitler prohlašuje, že už vyhrál, v říjnu začíná operace Tajfun -- dobytí Moskvy, Německo je na 20 km od Moskvy
    \item začíná ale zima a je to v pytli, v prosinci začíná protiútok generála Žukova, který frontu posouvá v místech až 200 km na západ
    \item[srpen 1941] Atlantická charta -- Chruchill a FDR prohlašují, že musí porazit fašistické země, že uchází především o osvobození fašisty podrobených zemí
    \item[$-$] do Británie se stahují vlády okupovaných zemí
    \item[$-$] Britové mají výhody: velké námořnictvo, letectvo, impérium, radar
    \item[květen 1940] k moci se dostávají
    konzervativci v čele s Churchillem, odmítají politiku appeasementu
    \item[$-$] strategie měla být taková, že británii budou nejprve bombardovat a pak se tam vylodí: operace Seelöwe
    \item[10.7.1940] začíná bombardování britských ostrovů
    \item[13.8.1940] den orlů -- den, kdy bombardování bylo nejintenzivnější
    \item[$-$] střet Geringovy Luftwaffe s RAF
    \item[$-$] Churchill chodil mezi lidi, dodával jim odvahu
    \item[$-$] v letectvu také českoslovenští letci
    \item[$-$] podařilo se Británii ubránit -- první německý neúspěch, nezískali britské ostrovy
\end{itemize}

\subsection*{Válka v pacifiku}
\begin{itemize}
  \item[1937] čínsko-japonská válka
  \item[1940] Japonsko obsazuje francouzskou Indočínu
  \item USA uvaluje na Japonsko ropné embargo, což vede k
  \item[7.12.1941] útok na Pearl Harbor, potopili zepár lodí, letadel, den na to FDR předstupuje před Kongres a USA vyhlašuje válku Japonsku (do pár dnů i jejich spojencům -- Německu, Itálii)
  \item v letech 1940-42 vobdobí rychlé japonské expanze, obsazují evropské kolonie v regionu, prvně Japonci vnímáni jako osvoboditelé od kolonizátorů, pozěji místním obyvatelům slibují za pomoc nezávislost po válce
  \item[3.-7.6.1942] bitva u Midway -- kombinovaná námořní a pozemní bitva o ostrov Midway, Američani rozluštili japonský kód, odrazili jejich útok -- obrat ve válce
  \item[VIII. 1942 - II. 1943] bitva o Guadalcanal (Šalamounovy o-vy), USA donutí Japonce se stáhnout, velká pozemní výhra USA
  \item[1.1.1942] Washingtonská deklarace -- viz Atlantická charta: Němci dostanou nabudku, z tohoto pak vyplývá OSN
  \item 1943 celkově zlomový rok v druhé sv. v.
  \item koncem roku 1942 rozluštěna německá šifra Enigma (Turing)
\end{itemize}

\section{IV. fáze}
\begin{itemize}
  \item začíná vyloděním v Normandii 6.6.1944 -- operace Overlord
  \item pláže označeny krycími jmény, USA se vyloďuje na plážích Utah, Omaha, Briti na plážích Gold a Sword, Kanaďani na pláži Juno, a zepár Frantíků (de Gaulle) na pláži Sword
  \item považováno za největší vylodění v historii
  \item Spojenci utrpěli poměrně veké ztráty, ale už nebyli vytlačeni
  \item postupně ovládnou úsek okolo Cherbourgu, Caen asi 30 km do vnitrozemí
  \item potom se vylodí u na jihu, u Tolouse, Nice
  \item 25. 8. osvobozena Paříž
  \item pokus o atentát na Hitlera - operace Valkýra, generálové von Stauffenberg, von Moltke
  \item postupně byli Němci dobití z východu i od západu
  \item Varšavské povstání -- Rudá armáda dostala příkaz, aby se zastavila, povstání organizoval západní odboj, takže se Stalin akorát pěkně zbavil pro-západní inteligence
  \item pak to postupně osvobozují, mrtě datumů
  \item tři námořní bitvy v zátoce Leyte -- Spojenci získali Filipíny
\end{itemize}

\end{document}
}

\end{document}
