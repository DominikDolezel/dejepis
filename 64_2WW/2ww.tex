\documentclass{article}
\usepackage{fullpage}
\usepackage[czech]{babel}
\usepackage{amsfonts}

\title{\vspace{-2cm}Druhá světová válka\vspace{-1.7cm}}
\date{}
\author{}

\begin{document}
\maketitle

\begin{itemize}
    \vspace{-0.5em}
    \setlength\itemsep{0.15em}
    \item[$-$] začátek 1. září 1939, konec 2. září 1945
\end{itemize}
\subsection*{Situace před válkou}
\begin{itemize}
    \vspace{-0.5em}
    \setlength\itemsep{0.15em}
    \item[$-$] agresivní země: Německo, Itálie, Japonsko; VB pokračuje v politice izolacionismu
    \item[$-$] OSN je k ničemu, ničenu nezabránila
    \item[březen 1939] zánik Československa, vyhlášen protektorát, Německo získalo v Litvě přístav Memel (= Kleiped) $\rightarrow$ Litva nemá přístup k moři
    \item[$-$] Hitler má další požadavky, tentokrát směřují k Polsku, chce spojit Německo s Východním Pruskem přes Polské území, taky chce přístav Gdaňsk (polosamostatný městský stát), tentokrát Hitler narazil -- Anglie i Francie pochopily, že ústupky nezabraly, začínají tedy Poslku garantovat neměnnost hranic
    \item[duben 1939] Italové anektovali Albánii, ve Španělsku se ujímá moci generál Franco, taky Hitler slavil svoje padesátiny
    \item[23.8.1939] rozebíhají se jednání mezi Brity, Francouzi a Sověty s Němci, výsledek je \textit{pakt Ribbentrop-Molotov} o neútočení mezi Němci a Rusy, měl dodatky, které veřejnost neznala, v nich šlo o rozdělení sfér vlivu v Evropě mezi němci a Sovětským svazem, ty dodatky byly zvěřejněny až v roce 1989
    \item[$-$] Sověti chtějí zpět to, co ztratili po první světové válce (Finsko, Polsko, kus Běloruska, Ukrajiny, Besarabii)
    \item[1.9.1939] Mussolini Hitlerovi oznamuje, že itálie není připravena naválku a preferoval by něco jako Mnichov (konferenci), ale hitler chtěl válčit, a tak zinscenoval útok na německou vysílačku Gleiwitz (Gliwice), útočí na polský poloostrov Westerplatte, kde taky měli sklad munice, vpád je bleskový (\textit{blitzkrieg}), tanky jsou podporovány Luftwaffe, Poláci nejsou vyzbrojeni, od začátku jasné, že Poláci prohrají, k útoku se připojila jako jediná země Slovensko
    \item[$-$] reakce velmocí: Francie a británie vyhlašují Němcům válku, ale nic nedělají (\textit{Sitzkrieg}), chtějí blokovat dovoz do Německa (neutrální země mohly obchodovat), nebyla však moc úspěšná
    \item[$-$] co udělali Němeci ze západu, to udělal Sovětský svaz z východu
    \item[17.9.] vpád Sovětského svazu do Polska, oficiálně zdůvodňují tím, že chtějí udělat předsunuté hranice, aby chránili svoje obyvatelstvo před vpádem Němců
    \item[$-$] v roce 1940 došlo k vyvraždění polské elity u Katyni, Stalin dlouho tvrdil, že to udělali Němeci, ale udělali to Sověti
    \item[$-$] vznik nové německo-sovětské hranice, Polsko zase není (někdy se mluví o čtvrtém dělení Polska)
    \item[$-$] ošetřeno hospodářskou smlouvou (únor 1940)
    \item[$-$] vytvoření území Generálního gouvernementu, pod německým vlivem, cílem postupně odtud vystrnadit kohokoliv (Poláky a židy) a udělat z toho jakési německé území, v jeho čele je Hanse Frank

\end{itemize}

\subsection*{Periodisace války}
\begin{enumerate}
    \vspace{-0.5em}
    \setlength\itemsep{0.15em}
    \item 1.9.1939 až 21.6.1941
    \item 22.6.1941 (útok na Sovětský svaz) až konec roku 1942 (bitva u Stalingradu)
    \item 2.2.1943 až léto 1944
    \item 6.6.1944 (vylodění v Normandii) až 8.5.1945 (konec války v Evropě)
    \item květen 1945 až 2.9.1945
\end{enumerate}

\subsection*{Charakter války}

\begin{itemize}
    \vspace{-0.5em}
    \setlength\itemsep{0.15em}
    \item[$-$] padlo 110 milionu vojáků
    \item[$-$] dalších 58 dalších civilistů zemřelo
    \item[$-$] 35 mil. zraněných
    \item[$-$] druhá průmyslová revoluce $\rightarrow$ tanky, letectvo, ponorky, atomové bomby, penicilin
    \item[$-$] Němci měli navrch do roku 1941, kdy vstoupily do války Spojené státy
    \item[$-$] fašistické státy: Německo, Itálie, Slovensko, Maďarsko, Rumunsko, Bulharsko, Chrovatsko, Japonsko, Thajsko (umožňovalo Japonsku operovat ze svého území), Finsko
    \item[$-$] neutrální státy: Švédsko, Irsko, Švýcarsko, Portugalsko, Španělsko, Turecko
\end{itemize}

\subsection*{Zisky SSSR do roku 1940}
\begin{itemize}
    \vspace{-0.5em}
    \setlength\itemsep{0.15em}
    \item[] 
\end{itemize}



\end{document}
