\documentclass{article}
\usepackage{fullpage}
\usepackage[czech]{babel}
\usepackage{amsfonts}

\title{\vspace{-2cm}Druhá světová válka\vspace{-1.7cm}}
\date{}
\author{}

\begin{document}
\maketitle

\begin{itemize}
    \item začátek \textbf{1. září 1939}, konec \textbf{2. září 1945}
\end{itemize}

\subsection*{Situace před válkou}
\begin{itemize}
    \item agresivní země: Německo, Itálie, Japonsko; VB pokračuje v politice izolacionismu
    \item OSN je k ničemu, ničenu nezabránila
    \item[březen 1939] \textbf{zánik Československa}, vyhlášen protektorát, Německo získalo v Litvě přístav Memel (= Kleipeda) $\rightarrow$ Litva nemá přístup k moři
    \item Hitler má další požadavky, tentokrát směřují k Polsku, chce spojit Německo s Východním Pruskem přes Polské území, taky chce přístav Gdaňsk (polosamostatný městský stát), tentokrát Hitler narazil -- Anglie i Francie pochopily, že ústupky nezabraly, začínají tedy Polsku garantovat neměnnost hranic
    \item[duben 1939] \textsc{Italové anektovali Albánii}, ve Španělsku se ujímá moci \textbf{generál Franco}, taky Hitler slavil svoje padesátiny
    \item[23.8.1939] rozebíhají se jednání mezi Brity, Francouzi a Sověty s Němci, výsledek je \textit{pakt Ribbentrop-Molotov} o neútočení \textbf{mezi Němci a Rusy}, měl \textbf{dodatky}, které veřejnost neznala, v nich šlo o rozdělení sfér vlivu v Evropě mezi němci a Sovětským svazem, ty dodatky byly zvěřejněny až v roce 1989
    \item Sověti chtějí zpět to, co ztratili po první světové válce (Finsko, Polsko, kus Běloruska, Ukrajiny, Besarabii)
    \item[1.9.1939] Mussolini Hitlerovi oznamuje, že Itálie není připravena na válku a preferoval by něco jako Mnichov (konferenci), ale Hitler chtěl válčit, a tak \textbf{zinscenoval útok na německou vysílačku Gleiwitz} (Gliwice), útočí na polský poloostrov Westerplatte, kde taky měli sklad munice, vpád je bleskový (\textit{blitzkrieg}), tanky jsou podporovány Luftwaffe, Poláci nejsou vyzbrojeni, od začátku jasné, že Poláci prohrají, k útoku se připojila jako jediná země Slovensko
    \item reakce velmocí: Francie a Británie vyhlašují Němcům válku, ale nic nedělají (\textit{Sitzkrieg}), chtějí blokovat dovoz do Německa (neutrální země mohly obchodovat), nebyla však moc úspěšná
    \item co udělali Němeci ze západu, to udělal Sovětský svaz z východu
    \item[17.9.1939] \textsc{vpád Sovětského svazu do Polska}, oficiálně zdůvodňují tím, že chtějí udělat předsunuté hranice, aby chránili svoje obyvatelstvo před vpádem Němců
    \item[1940] \textsc{vyvraždění polské elity u Katyni}, Stalin dlouho tvrdil, že to udělali Němeci, ale udělali to Sověti
    \item vznik nové německo-sovětské hranice, Polsko zase není (někdy se mluví o čtvrtém dělení Polska)
    \item[(únor 1940)] ošetřeno hospodářskou smlouvou
    \item vytvoření \textbf{území Generálního gouvernementu}, pod německým vlivem, cílem postupně odtud vystrnadit kohokoliv (Poláky a židy) a udělat z toho jakési německé území, v jeho čele je \textbf{Hanse Frank}
\end{itemize}

\subsection*{Periodisace války}
\begin{enumerate}
    \item 1.9.1939 až 21.6.1941
    \item 22.6.1941 (útok na Sovětský svaz) až konec roku 1942 (bitva u Stalingradu)
    \item 2.2.1943 až léto 1944
    \item 6.6.1944 (vylodění v Normandii) až 8.5.1945 (konec války v Evropě)
    \item květen 1945 až 2.9.1945
\end{enumerate}

\subsection*{Charakter války}
\begin{itemize}
    \item padlo 110 milionu vojáků
    \item dalších 58 dalších civilistů zemřelo
    \item 35 milionů zraněných
    \item druhá průmyslová revoluce $\rightarrow$ tanky, letectvo, ponorky, atomové bomby, penicilin
    \item Němci měli navrch do roku 1941, kdy vstoupily do války Spojené státy
    \item fašistické státy: Německo, Itálie, Slovensko, Maďarsko, Rumunsko, Bulharsko, Chrovatsko, Japonsko, Thajsko (umožňovalo Japonsku operovat ze svého území), Finsko
    \item neutrální státy: Švédsko, Irsko, Švýcarsko, Portugalsko, Španělsko, Turecko
\end{itemize}

\subsection*{Zisky SSSR do roku 1940}
\begin{itemize}
    \item[podzim 1939] Sověti do Pobaltí přesunuli jednotky, pak tam proběhly volby, vznikly \textbf{prosovětské vlády}, výsledkem byl vstup Pobalstkých zemí do Sovětského svazu v červenci 1940
    \item[červen 1940] SSSR vnutil Rumunsku smlouvu o vstupu do SSSR: Besarabie, Bukovina
    \item pokus o úpravu hranic s Finskem nevyšel, chtěli poloostrov Hanko, Finsko odmítlo, \textsc{SSSR napadl Finsko}
    \item[30.11.1939-12.3.1940] \textsc{zimní válka}: Finové se brání, mají postavenu Mannerheimovu linii (\textbf{Carl Gustav Emil Mannerheim}, tehdejší Finský president, považován za zakladatele Finska), válka vyvolala nevoli především Británie a Francie, které finančně Finsko podporovaly, SSSR vyřazeno ze Společenství národů, končí \textit{Moskevským mírem} (12.3.1940), získávají jenom 10 \% území Finska
    \item Mannerheimova linie nebyla proražena (jen malinko)
    \item když byli Sověti napadnuti Němci v roce 1941, Finové se postavili na stranu Němců
\end{itemize}

\subsection*{Útok na Skandinávii}
\begin{itemize}
    \item aktivita Německa
    \item[9.4.-10.6.1940] \textsc{válka o Skandinávii}: jen o Dánsko a Norsko, operace \textbf{Weserübung}, Dánsko dobyto hned, Dánové se hezky zachovali ke svým Židovským obyvatelům, kteří byli na lodích přepraveni do Švédska
    \item Britové a Francouzi chtějí zastavit dodávky železa ze Švédska do Německa, což vyvolalo \textbf{útok na Skandinávii}
    \item \textbf{Vidkun Quisling} -- vznik kolaborantské vlády v Norsku, státní převrat
    \item na straně Norů od začátku Britové a Francouzi
    \item na sklonku války od 10. května 1940 začíná \textbf{útok na západní Evropu} (Benelux a Francii), jakmile Němci zaútočili, Francie i Anglie stáhly svá vojska z války o Skandinávii a Němci dobyli Narwick, čímž končí válka o Skandinávii
\end{itemize}

\subsection*{Útok na západní Evropu}
\begin{itemize}
    \item[10.5.-22.6.1940] \textsc{operace Gelb}: útok na Benelux
    \item[12.5.1940] \textsc{útok na Francii}, vpadli tam do severu, na jihu je totiž Maginotova linie, Francouzům pomáhají Británie a Benelux, ale neúspěšně, někteří vojáci po porážce u Calais přepraveni do Británie z přístavu Dunkerque
    \item[10.6.1940] \textsc{Mussolini útočí na Francii} a získává drobné území
    \item premiér \textbf{Philippe Pétain}, který válčil už v první světové válce, byl donucen, aby tu kapitulaci podepsal on: \textit{Kapitulace v Compiegne}
    \item[22.6.1940] \textbf{Francie kapituluje}
    \item Rotterdam krom jednoho chrámu srovnán se zemí
    \item Němci okupují Francii na západě u moře, v centrální části je tzv. \textbf{Vichistická Francie} (kolaborační režim), vydžel jen do srpna 1944, kdy začíná osvobozování Francie z jihu
    \item mezi Spojenci se objevily i jednotky \textbf{Charlese de Gaulla}, který utekl do zahraničí a poté organizoval jednotky svobodné Francie
\end{itemize}

\subsection*{Balkán}
\begin{itemize}
    \item[duben 1939] \textsc{Itálie anektuje Albánii}, plánuje co dál
    \item[říjen 1940] \textsc{Mussolini útočí na Řecko}, je to obří fiasko, Řekové částečně vstoupili do Albánie
    \item[březen 1941] Hitler si je nakláněl, v březnu se Hitlerovi povedlo dostat Jugoslávii do aliance, proběhne tam ale do týdne puč, Hitler reaguje a Jugoslávie je v rámci \textbf{operace Marika} během dubna obsazena, pak je dobyto i Řecko, trošku je okousají, osamostatní se Chorvatsko v čele s ustašovci, v čele s \textbf{Ante Paveličem}, dále Srbsko v čele s \textbf{gen. Nedičem}
    \item Řecko je připojeno k Itálii, Makedonii získává Bulharsko
    \item v Jugoslávii je silná partyzánská aktivita v čele s \textbf{Josipem Brozem Titem}
    \item[5.-6. 1941] \textbf{operace Merkur} -- invaze na Krétu
    \item válka v Africe
    \item Italové drží Libyi, Ethiopii, Somálsko a Eritreu, chtějí dobýt britský Egypt
    \item[$\Rightarrow$] otevíráme libyjsko-egyptskou frontu (září 1940)
    \item[únor 1941] Italové jsou zatlačeni
    \item[březen 1941] německý tankový velitel Erwina Rommela přestal být v popředí kvůli Barbarosse, opevní Tobruk, který je pak pracně dobýván
    \item[květen 1941] habešsko-somálská fronta: Italové vedou neúspěšný útok na britské Somálsko, Briti zareagovali tím, že dobyli celou Italskou východní Afriku
    \item[duben 1941] Japonsko a SSSR uzavírají \textbf{smlouvu o neútočení}
    \item[březen 1941] \textbf{zákon o půjčce a pronájmu} -- USA \uv{půjčí} Britům a ostatním spojencům (potom i SSSR) vybavení za peníze, které zaplatí po válce (pak se to promlčelo)
\end{itemize}

\subsection*{Válka v Africe}
\begin{itemize}
    \item[1941] \textbf{operace Crusader}: vyproštění Tobrúku (město Libyi poblíž hranice s Egyptem) z německého obléhání
    \item[květen 1942] německý útok, dostali se až do Egypta
    \item[23. říjen 1942] \textsc{El Alamein}: zlomová bitva, Britové nezvítězili, ale definitivně zastavili Němce a Italy, gen. Bernard Law Montgomery -- dále Britové zatlačují směrem na západ
    \item[listopad 1942] \textbf{operace Pochodeň}: v Maroku a Alžírsku, v přístavech se vylodili spojenci a britské armády, Generál Eisenhower se setkal s Brity z Egypta, které vedl Montgomery, Němce a Italy uzavřeli v Tunisku
    \item[květen 1943] \textbf{definitivně vypuzeno Německo}, Afrika očištěna od německých a italských armád
    \item[$\Rightarrow$] předpoklad k vylodění spojenců na Sicílii a jižní Itálii
\end{itemize}

\subsection*{Východní fronta}
\begin{itemize}
    \item[22. června 1941] začíná \textsc{útok Německa na SSSR}, okolo 70 \% německé vojenské síly se přesune na východní frontu, Němci mají srovnatelný počet, ale mnohem vyšší kvalitu vybavení, Stalin taky před válkou zlikvidoval většinu schopných generálů v rámci čistek (Tuchačevsky), sovětské vedení je tedy nezkušené
    \item \textsc{bitva u Stalingradu}
    \begin{itemize}
      \item[květen 1942] začátek německé ofenzivy \textbf{operací Blau}
      \item nejprve dorazili na Kavkaz, rychlý postup
      \item dobyvání Stalingradu, \textbf{Friedrich Paulus}, plán: zaútočit na Stalingrad a dobýt ho
      \item velitelé obrany Stalingradu: Vasilij Čujkov, Georgij Žukov
      \item Stalingrad na řece Volze, Sověti tudy zásobují a dodávají vojáky
      \item[podzim 1943] bojuje se o každý dům
      \item \textbf{Němci se dostali do obklíčení}, Paulus žádal po Hitlerovi, aby se mohli stáhnout, Hitler fascinován dobytím Stalingradu, rozhodl, že je budou zásobovat letecky, zásobování nestačilo, obklíčení
      \item[2. 2. 1943] armáda se dostává do zajetí
      \item [31. 1. 1943] \textbf{kapitulace Němců}
      \item ve Stalingradu bylo cca 90 000 Němců
      \item neúspěch Hitlera, jedna ze zlomových bitev východní fronty
    \end{itemize}
    \item ze začátku padaly celé sovětské armády do německého zajetí
    \item Stalina částečně zachránilo to, že se mu povedlo přesunout část průmysl za Ural
    \item německý útok veden třemi směry, na sever do Leningradu, v centru do Moskvy, na jih do Stalingradu, na Kavkaz (ropa)
    \item německá invaze ale nemá před zimou moc času, na podzim začíná pršet, \textbf{pak je krutá zima}, na kterou \textbf{není Německo plně připraveno}
    \item[9. 1941-1. 1944] Leningrad je obklíčen, začíná dlouhé obléhání Leningradu, Leningrad zásobován pouze po jezeře
    \item na podzim Hitler prohlašuje, že už vyhrál, v říjnu začíná \textbf{operace Tajfun} -- dobytí Moskvy, Německo je na 20 km od Moskvy
    \item začíná ale zima a je to v pytli, v prosinci začíná \textbf{protiútok generála Žukova}, který frontu posouvá v místech až 200 km na západ
    \item[srpen 1941] \textbf{Atlantická charta}: Chruchill a FDR prohlašují, že musí porazit fašistické země, že uchází především o osvobození fašisty podrobených zemí
\end{itemize}

\subsection*{Bitva o Británii}
\begin{itemize}
    \item do Británie se stahují vlády okupovaných zemí
    \item Britové mají výhody: velké námořnictvo, letectvo, impérium, radar
    \item[květen 1940] k moci se dostávají
    \textbf{konzervativci} v čele s \textbf{Churchillem}, odmítají politiku appeasementu
    \item Němci by chtěli dobýt Británii, plánovali vyloddění tzv. \textbf{operace Lvoun} (Seelöwe), která se nakonec neuskutečnila
    \item[10.7.1940] začíná bombardování Britských ostrovů
    \item[13.8.1940] \textbf{Adlertag} (Den orlů): den, kdy bombardování bylo nejintenzivnější
    \item do řijna nejintenzivnější vzdušná válka o britský prostor
    \item podařilo se Británii ubránit -- první německý neúspěch, nezískali Britské ostrovy
    \item od roku 1942 to už budou Britové, kteří bombardují Německo
    \item střet Geringovy Luftwaffe s RAF
    \item[říjen 1940] na španělsko-francouzských hranicích probíhá \textbf{jednání Hitlera a Franca}, pokud se Franco přidá do války a dobude Gibraltar, tak si ho Španělsko může nechat a dostane k tomu nějakou tu kolonii, nicméně Franco to nepřijal, Španělsko bude neutrální
    \item[listopad 1940] Hitler nabízí Stalinovi, jestli se nepřidá do války proti kapitalistům
    \item v RAF i českoslovenští dobrovolníci
    \item Churchill chodil mezi lidi, dodával jim odvahu
\end{itemize}

\subsection*{Válka v Pacifiku}
\begin{itemize}
  \item[7.7.1937-2.9.1945] \textsc{čínsko-japonská válka}
  \item[1940] Japonsko obsazuje francouzskou Indočínu
  \item USA uvaluje na Japonsko ropné embargo, což vede k
  \item[7.12.1941] \textbf{útok na Pearl Harbor}, potopili zepár lodí, letadel, den na to Roosevelt předstupuje před Kongres a USA vyhlašuje válku Japonsku (do pár dnů i jejich spojencům -- Německu, Itálii)
  \item[1940-1942] období rychlé japonské expanze, obsazují evropské kolonie v regionu, prvně Japonci vnímáni jako osvoboditelé od kolonizátorů, později místním obyvatelům slibují za pomoc nezávislost po válce
  \item[3.-7.6.1942] \textsc{bitva u Midway}: kombinovaná námořní a pozemní bitva o ostrov Midway, Američani rozluštili japonský kód, odrazili jejich útok -- obrat ve válce
  \item[8. 1942-2. 1943] \textsc{bitva o Guadalcanal} (Šalamounovy ostrovy), USA donutí Japonce se stáhnout, velká pozemní výhra USA
  \item[1.1.1942] \textit{Washingtonská deklarace}: viz Atlantická charta: Němci dostanou nabudku, z tohoto pak vyplývá OSN
  \item[1943] celkově zlomový rok v druhé světové válce
  \item koncem roku 1942 rozluštěna německá šifra Enigma (Turing)
\end{itemize}

\subsection*{Apeninský poloostrov 1945}
\begin{itemize}
    \item[9.4.1945] dobyto Miláno
    \item Spojenci v Pádské nížině, Němci zatlačeni k Pádu
    \item Mussolini chtěl utéct do Švýcarska, ale byl chycen partyzány a zabit, ostatky jeho milenky byly vyvěšeny v Miláně
    \item[2.5.1945] německá kapitulace
\end{itemize}

\subsection*{Západní fronta 1945}
\begin{itemize}
    \item[1944] Němci se tu pokusili o poslední ofenzivu, boje směřovali k přístavuv v Antverpách, výběžek zlikvidován do ledna 1945, důvodem bylo špatné zásobování a zimní počasí
    \item[únor] Spojencům se podařilo prolomit Siegfriedovu linii podél řeky Maasy dlouhých asi 5 km, snažili se dál pronikat do Třetí říše, zlomové bylo to, že se dostali přes řeku
    \item[březen 1945] dostávají se přes řeku a postupují dál do třetí říše, protože Němci mosty ničili, poslední z nich -- most u Remagennu, kterého využili spojenci
    \item pravidelné nálety na Třetí říši už od roku 1943
    \item[12.4.1945] smrt F. D. Roosevelta, prezidentem se stává jeho viceprezident \textbf{Harry S. Truman}, ten pokračuje v téže zahraniční politice
    \item[25.4.1945] Američané a Sověti se setkali na Labi (u Torgau), dosáhli tak demarkační linie
    \item rozbombardovaný Hamburg, Drážďany
\end{itemize}

\subsection*{Porážka Německa}
\begin{itemize}
    \item[30.401945.] \textsc{bitva o Berlín}: na Berlín táhnout dva fronty: Ukrajinský a Běloruský, Berlín dobyl Žukov, bránit se mohlo jen asi milion Němců, zatímco Sovětů bylo dva a půl milionu, Hitler pobývá v bunkru i s ostatními, byl tam i s jeho milenkou \textbf{Evou Braunovou}, kterou si den předtím, než spáchali sebevraždu, vzal za ženu (ampulkou, Hitler si i prostřelil hlavu)
    \item[2.5.1945] \textbf{oficiální kapitulace Berlína}
    \item[7.5.1945] \textit{Kapitulace v Remeši} Němců, za ně podepisoval \textbf{Alfred Jodl}, s platností od 8.5.
    \item[8.5.1945] Stalinovi se to zdálo málo, došlo k další \textit{kapitulaci v Berlíně}, kterou podepsal \textbf{Wilhelm Kettel}
\end{itemize}

\subsection*{Situace v Československu}
\begin{itemize}
  \item[27. září 1941] Neurath byl měkký vůči Čechům a Moravanům, odvolán oficiálně ze zdravotních důvodů
  \item příjezd \textbf{Reinharda Heydricha}
  \item [28. 9.] vyhlásil stanné právo, zákazy nočního vycházení, zostření kontrol dokladů, pracovní knížky, v noci se nesmělo svítit, zostřený režim
  \item Londýnské vedení připravuje \textbf{atentát na Heydricha} -- významná akce, někdy se vyčítá
  \item \textbf{Gabčík a Kubiš}, \textbf{Čurda} (prozradil kontakty, zradil je), zasekl se samopal, házeli granát
  \item \textbf{Heydrichiáda} -- \textbf{Hermann Frank} to řídil, podílel se na potrestání českého národa
\end{itemize}

\subsection*{Třetí fáze}
\begin{itemize}
    \item[5.-7. 1943] začíná \textsc{bitvou u Kurského oblouku} -- největší tanková bitva vůbec, výběžek fronty, nasazeno největší množství tanků v dějinách -- Kursk, Orel, Prochorovka; Sověti vyhráli, přišli na zranitelné místo německých tanků, navíc odhalili útok. Společně se Stalingradem je to jedna ze zlomových bitev. Sověti posunuli linii za řeku Dněpr. Goebbels mluvil o strategickém zkracování fronty, nemohli přiznat neúspěch, propaganda.
    \item[1943] Němci objevili u Smolenska hromadné hroby. V roce 1940 Sověti zlikvidovali polskou inteligenci, vojenské špičky, masový hrob ve Smolensku, území Polska. Navzájem se obviňovali se Sověty. Nakonec vyšlo najevo (až v roce 1990, otevření archivů, Gorbačov), že to byli Sověti.
    \item[leden 1943] \textbf{konference v Casablance}: Churchill, Roosevelt. Společný postup a tak dále. Začala se řešit otázka otevření fronty na Apeninském poloostrově, předpoklad obsazení, vyčištění Afriky.
    \item[květen 1943] osvobozena Afrika, Britové s Montgomerym, v Tunisku se střetli. Naplnil se předpoklad.
    \item[9. 7. 1943] \textbf{Operace Husky} invaze na Sicílii, o měsíc později leden 1943 3. 9. 1943 -- Apeninský poloostrov. V Itálii se děly věci, Mussolini nevedl dobře ve válce, byl sesazen a dokonce uvězněn. Byl vybrán nový premiér - \textbf{Pietro Badoglio}, do pozice jmenován tehdejším králem. Spojenci rychle postupovali, vylodili se a měli to, válčilo se, ale postupovali rychle, hrozilo, že dobyjí Řím. Co udělal Hitler? Nemohl to dopustit, Wehrmacht obsadil sever a střed Apeninského poloostrova včetně Říma. Dokázali letecky dramaticky osvobodit Mussoliniho, dosadit ho jako loutku - vyhlásili na severu Republica di Salò. Pro spojence náročné, intenzivní boje, dále osvobozují, teprve \textbf{2. 5. 1945 Wehrmacht v Itálii kapituloval}; boje až do konce války v Evropě. Salerno, velitel G. S. \textbf{Patton} (osvobodil pak Plzeň).
    \item[11.-12. 1943] \textbf{Teherán}, první konference velké trojky (Churchill, Roosevelt, Stalin) -- plánují vylodění v Normandii, otevření západní fronty.
\end{itemize}

\section*{Čtvrtá fáze}
\begin{itemize}
  \item[6.6.1944] začíná \textsc{vyloděním v Normandii}, \textbf{operace Overlord}
  \item pláže označeny krycími jmény, USA se vyloďuje na plážích Utah, Omaha, Briti na plážích Gold a Sword, Kanaďani na pláži Juno, a Francouzi (de Gaulle) na pláži Sword
  \item považováno za největší vylodění v historii
  \item Spojenci utrpěli poměrně veké ztráty, ale už nebyli vytlačeni
  \item postupně ovládnou úsek okolo Cherbourgu, Caen asi 30 km do vnitrozemí
  \item potom se vylodí u na jihu, u Tolouse, Nice
  \item[25. 8.] osvobozena Paříž
  \item pokus o atentát na Hitlera -- \textbf{operace Valkýra}, generálové von Stauffenberg, von Moltke
  \item postupně byli Němci dobití z východu i od západu
  \item \textit{Varšavské povstání} -- Rudá armáda dostala příkaz, aby se zastavila, povstání organizoval západní odboj, takže se Stalin akorát pěkně zbavil prozápadní inteligence
  \item pak to postupně osvobozují, mrtě datumů
  \item tři námořní bitvy v zátoce Leyte -- Spojenci získali Filipíny
\end{itemize}

\subsection*{Pražské povstání}
\begin{itemize}
    \item do konce dubna 1945 byla většina Slovenska osvobozena, bylo tu spoustu partyzánských jednotek
    \item[duben 1945] Brno osvobozeno v rámci Bratislavsko-Brněnské operace 26.4., Bratislava 4.4.
    \item naše území osvobozováno třemi směry, Spojenci ze západu, ale mohli dosáhnout jen demarkační linie (Plzeň)
    \item[7.5.] Spojenci dosáhli demarkační linie, den předtím osvobozena Plzeň, americké armádě velel gen. Patton
    \item zbylé dva proudy: v rámci Varšavsko-Berlínské operace první ukrajinský proud maršála Koněva, čtvrtý ukrajinský proud maršála Petrova, pak Jeremeňkova -- ti šli ze severu, z jihu šel maršál Malinovský
    \item na Slovensku se mohla vytvořit naše první poválečná vláda, protože Slovensko bylo osvobozeno dřív, díky politice appeasementu Beneš 1943 podepsal se Stalinem smlouvu o spolupráci, přátelství
    \item[březen 1945] když se chýlila válka ke konci, spojil se Beneš s komunistickými reprezentanty odboje domluvil na tom, jak bude vypadat první vláda -- bylo tam šest stran, ale ne agrárnící, Hlinkova fašistická strana, byla tedy vytvořena vláda národní fronty ČEchů a Slováku (čeští komunisté, čeští sociální demkoraté, čeští lidovci, národní socialisté, slovenští komunisté, slovenská strana demokratická), v čele vlády Zdeněk Fiellinger, představitel sociální demokratické strany, ale silně levicově orientován, vytvořena na území osvobozených Košic 4.4., o den později přijali tzv \textit{Košický vládní program}, zaznívá tam zahraniční orientace na SSSR, potrestání zrádců a kolaborantů apod.
    \item Němci se pořád drželi v protektorátu a chtěli se postupně posouvat na západ a postupně se vzdát, ale to jak postupovali, postupně ničili vše, na co přišli; v posledních válečných dnech popravovali partyzány a vyhlazovali spolupracující vesnice, taky využívali náš ekonomický potenciál
    \item ve středních Čechách operovala armáda německých vojáků -- skupina \textit{Mitte}, navíc tu byly jednotky SS -- protektorát pořád držen němci
    \item když se posune Rudá armáda z východu, hrozí nebezpečí, že nastane konfliktu
    \item vzniká další odbojová organizace Česká národní rada, která chce ozbrojeným povstáním v protektorátu urychlit odsun Němců, v čele byl Albert Pražák, ale taky komunisté (Smrkovský), mělo vypuknout někdy 7.5., jenže od 1.5. začala propukat povstání po celé republice, nakonec propuklo v Praze 5.5.
    \item velitelé povstání začali spolupracovat s Českým rozhlasem, po Praze vyrostly barikády, západní spojenci by tam byli mohli dorazit, ale museli se zastavit na demarkační linii
    \item povstalci neměli dost zbraní, němci používali lidi jako lidské štíty, byli ozbrojení
    \item v nejhorších dnech 6. a 7. května povstalcům pomohli \textit{Vlasovci} (jednotky generála Vlasova) -- zajatí Rusové a Ukrajinci Němci, kteří vytvořili vojenskou jednotku a bojovali za války na straně Němců, báli se toho, co s nimi Stalin udělá, jejich cílem tedy bylo projít prahou a stáhnout se do západního zajetí, pomohli v Praze zásadním způsobem, i když pražští velitelé odmítali jejich pomoc, západní spojenci je potom vydali Stalinovi
    \item[7.5.] bombardování Prahy Němci, podepsána kapitulace v Remeši, o den později (8.5.) podepsána i v Praze s tím, že jim Češi umožní se stáhnout na západ
    \item fanatičtí členové SS válčili dál
    \item na základě Pražské operace první ukrajinský front vyslal část vojáků na Prahum, ti v ranních hodinách 9.5. tanková jednotka Ribalkova vyjíždějí do Prahy a postupně likvidují zbytky SSáků v Praze
    \item opravdu do posledního dne váoky v Evropě se bojovalo i na našem pzemí
    \item[10.5.] bylo jasné, že se Fillingerova vláda může přesunout do Prahy
    \item protože vnímali českou národní radu jako konkurenta, byla rozpuštěna vládou
    \item
\end{itemize}

\subsection*{Postupim}
\begin{itemize}
    \item na konci války v Evropě se V USA jednalo o vzniku nové mezinárodní organisace, která se bude primárně snažit udržet mír
    \item[26.6.] podepsána \textbf{Charta OSN} -- založení OSN, za Československo podepisoval ministr zahraničí Jan Masaryk
    \item[červenec-srpen] uspořádat Evropu po druhé světové válce měla Postupimská konference ve složení Stalin, Churchill a Clement Attlee, který vyhrál volby, Truman
    \item okpuační zony Německa, Rakouska, berlín a Vídeň rozdělena na 4 zóny, politika 4D: demokratizace, dekartelizace (zborjní firmy přestanou chrlit zbraně), denacifikace (zákaz NSDAP), demilitarizace (odzbrojení), reeparace pomocí surovin z německa, Rusové demontovali zařízení podniků a odváželi si je do SSSR; východní Prusko bude mít částečně jižní část Polsko, ale oblast Kaliningradu získá SSSR, stanovena západní hranice Polska na řekách Odra-Nisa, preventivně odsunuty německé menšiny z Polska, Československa a Maďarska, připraví se mírové smlouvy s německými satelity
    \item Američanům se už podařilo zrealizovat pokusy s jadernou zbraní
    \item pro nás důležité, protože jsme mohli \uv{vylikvidovat} Němce -- citován dr. Beneš
\end{itemize}

\subsection*{Pátá fáze 8.5.-2.9.}
\begin{itemize}
    \item císař Hirohito odmítá kapitulovat, boje o Ivodžimu, Okinama (ostrovy)
    \item Američané předpokládali, že válka bude dál pokračovat, do bojů se zapojili i Sověti
    \item[6.8.] atomová bomba Little Boy na Hirošimu
    \item[6.8.] vstup SSSR do války
    \item[9.8.] druhá bomba Fat Man na Nagasaki
    \item[14.8.] císař Hirohito kapituluje, ale někteří fanatici válčí dál
    \item[2.9.] vylodění křižníku Missouri, na němž byla podepsána kapitulace Japonska -- definitivní konec druhé světové války
    \item[16.12.1944] operace Noc a mlha, poslední pokus Němců zvrátit vývoj na západní frontě, poslední německá operace, přes Ardeny, směrem k Antverpám, cílem je obsadit významný zásobovací přístav spojenců. Když se tam blížili, vytvořil se výběžek, trojúhelník, Němci vpadli na území ovládané spojenci. Tzv. 2. Stalingrad, do ledna 1945 spojenci výběžek zlikvidovali, spoustu ztrát na obou stranách, u Němců vázlo zásobování.
    \item[červenec 1944] Bretton Woods – v rámci OSN vznikla Světová banka a Mezinárodní měnový fond; cílem bylo, aby se po válce hned mohly obnovit ekonomiky válkou zničené.
    \item[srpen 1944]  Roosevelt a Churchill v Quebecku, řešili rozdělení Německa (prý nevýznamné).
    \item[říjen 1944]  Moskva, Churchill a Stalin, jakoby si porcují Evropu, říkají si kdo kde bude mít větší vliv; Bulharsko a Maďarsko - Stalin si vymínil vliv z 80\%.
    \item poválečná Jugoslávie - nechali 50/50 že se uvidí, jak se to vyvede.
    \item Řecko a Egejské ostrovy osvobozované Brity, Británie 90\% vlivu.
    \item[únor 1945] 2. konference velké trojky - letovisko Jalta na Krymu; významná; koordinace operací, stanovení demarkační linie; Vostok, Terst, u nás západní Čechy Američani (Plzeň, KV, ...), zbytek Sověti. Západní spojenci se snažili Sověty vtáhnout do bojů v Pacifiku, když skončí válka v Evropě, nejpozději do 3 měsíců se Stalin zapojí do bojů v Tichomoří, (za Kurylské ostrovy a jižní Sachalin). Německo - 4 okupační zóny - Britové, Američané, Francouzi, Sovětský svaz; na poslední konferenci říkali, že je důležité, aby bylo Německo po dostání se z krize obnoveno. Polovina reparací Sovětskému svazu, stanovena výše. Hranice Polska nevyřešena. Začíná vznikat OSN, deklarace Spojených národů (do té doby Společnost národů).
\end{itemize}

\subsection*{Východní fronta 1945}
\begin{itemize}
    \item Budapešť - fašistický stát, dlouho se bránili.
    \item Varšava - povstání, nechali ho vykrvácet, povstání západních odborů.
    \item Bratislava - v rámci Bratislavsko-brněnské operace, 4.4., Brno 26.4.
    \item Vídeň.
\end{itemize}

\subsection*{Slovenské národní povstání (SNP)}
\begin{itemize}
    \item Víme, že Tiso odtrhl Slovensko, slovenský štát vyhlášen 1939 14.3., od léta 39 oficiální název republika slovenská, měli ústavu, totalitní charakter státu, prezidentem Josef Tiso, člen Hlinkovy slovenské lidové strany; po válce se snažil utéct, 47 chycen a popraven, Slováci k němu mají ambivalentní vztah, založil samostatný slovenský stát, ale zároveň klerofašista; v čele Slovenska vlády, 1. Vojtěch Tuka z SLS, pak Štefan Tiso; fašistický stát, německý satelit, mladí vychováváni v rámci Hlinkovy mládeže, ostří vůči Čechům a Maďarům a Židům (obchod na korze).
    \item Když vznikl slovenský stát, těžili z toho, že založili s Německem smlouvu, vyráběli pro 3. říši, konjunktura, snižuje se nezaměstnanost; pak to uvadá, 44 Německo prohrává, i na Slovensku nespokojenost, partyzánské oddíly, nechtějí být na straně fašistů, režim Tisův v krizi.
    \item Na konci roku 1943 vzniká Slovenská národní rada, spojila komunistický a nekomunistický odboj, vzniká vánoční dohodou 21.12.
    \item Na jaře 1944 vojenské ústředí, v čele Jan Golian, začali řešit se slovenskou národní radou přípravu povstání. 2 varianty: vyhlásí povstání, když se přiblíží Rudá armáda v rámci Slovenska (společně osvobodí Slovensko), nebo Slovensko začnou obsazovat Němci; partyzáni, nepokoje, Němci chtějí, aby Tiso vyhlásil stanné právo, to neuklidnilo situaci, pak začali Němci Slovensko obsazovat, 29.8., v podstatě se spojila jedna z podmínek, 29.8. Jan Golian vyhlásil povstání.
    \item Vypuklo povstání, sami by to nezvládli bez pomoci Rudé armády, povstalci vydali deklaraci že chtějí obnovit ČS. stát; ale odsouzeno k záhubě, neměli dost lidí ani zbraní; zvládli by to jen s pomocí Rudé armády, Sověti zahájili novou operaci kvůli povstání - Karpatsko-dukelská operace 8.9., skončila 28.10., změnili směr postupu, šli pomoct Slovákům; nicméně dobývání Karpat náročné, obrovské množství techniky Němcům, nakonec se podařilo prolomit dukelský průsmyk, 6.10. začíná osvobozování bývalého Československa, se Sověty 1. ČS. armádní sbor, v čele Ludvík Svoboda - pod Uralem se vytvořil, jak postupovali nabalovali se další, společně s armádou procházeli územím bývalého Československa.
    \item Povstalci se pak stáhli do hor a působili jako partyzáni, do dubna bylo Slovensko osvobozeno, ještě půl roku se válčilo. Museli se stáhnout do hor, hodně z nich přes zimu zmrzlo v horách. 60 vesnic, osad vypáleno 15k Slováků do koncentračních táborů, 4k tisícě obětí, Hlinkovy gardy likvidovali ty, co spolupracovali s partyzány, povstalci, vyhladili, vyhlazení desítek obcí kde byly nalezeny kontakty na partyzány (Tiso to dělal). Hodně krvavé, umíralii civilisté. Slováci se vymanili z nacismu.
\end{itemize}

\subsection*{Apeninský poloostrov 1945}
\begin{itemize}
    \item Nadějné osvobozování podzim 44 se zvrhlo v dobývání celého poloostrova spojenci, klíčové momenty Monte Casino, Řím, … Němci zatlačili k řece Pádu. 9.4. v dubnu dobyto Miláno, sídlo Mussoliniho, chtěl utéct do Švýcarska, nakonec chycen partyzány a zabit 28.4. Jeho ostatky a jeho ženy byly vyvěšeny v Miláně na čerpací stanici. Wehrmacht kapituluje až 2.5. - německá kapitulace.
\end{itemize}

\subsection*{Západní fronta 1945}
\begin{itemize}
    \item Cíl - Berlín, na konci dubna se tam setkali s Rudou armádou; rychlejší přes Rýn, Němci už od ledna vyklízejí Francii, Belgii, Holandsko, vojáci i civilisté.
\end{itemize}

\subsection*{Berlínská operace}
\begin{itemize}
    \item Začíná 16.4., končí 8.5. kapitulací; cílem dobytí Berlína, Němci velký odpor, 2 mil. vojáků, RA 2,5 mil. vojáků; Berlín obklíčen, boje o jednotlivé ulice a budovy, těžké boje, kruté; sovětští vojáci pod tlakem; pak probíhaly kapitulace německých armád, nějaká armáda měla kapitulovat, když se dozvěděli že mají kapitulovat před RA, jeli radši na západní frontu, kde se vzdali západním Spojencům.
\end{itemize}


\subsection*{Proces dekolonizace}
\begin{itemize}
    \item Mnoho zemí pomáhalo kolonizátorům s příslibem následné samostatnosti, což se postupně plnilo.
\end{itemize}

\subsection*{Velké sjezdy NSDAP v Norimberku}
\begin{itemize}
    \item Tribunál v Norimberku: počet obžalovaných 24, většina špiček utekla do Jižní Ameriky, kde byli postupně hledáni.
    \item Eichmann: Izraelský Mosad ho našel a unesl do Jeruzaléma, kde byl vyslýchán a odsouzen k trestu smrti.
    \item Pouze hrstka byla dopadena: Goering - spáchal sebevraždu ve vězení před vykonáním trestu smrti.
    \item Následně probíhaly další tribunály, ale bylo pochytáno málo zločinců kvůli změně dokladů.
    \item Simon Wiesenthal: rakouský Žid, který přežil několik koncentračních táborů a po válce se stal lovcem nacistů, pátral po ukrývaných nacistech.
    \item Nacistické organizace (SA, SS, Gestapo) byly definovány jako zločinecké.
    \item Nové zločiny byly definovány: zločiny proti lidskosti, proti míru, válečné zločiny.
\end{itemize}

\subsection*{Tribunál Tokijský}
\begin{itemize}
    \item Vyrovnání se s japonskými fašisty: 28 obžalovaných, většina ale utekla nebo spáchala sebevraždu.
\end{itemize}

\subsection*{Proces denacifikace v Německu}
\begin{itemize}
    \item Komplikovaný proces zahrnující i Hitlerjugend.
    \item Většina Němců byla členy NSDAP, zatýkání všech by znamenalo, že nikdo nezůstane na svobodě, což bylo náročné.
\end{itemize}

\subsection*{Pařížské mírové smlouvy}
\begin{itemize}
    \item Cíl: potrestat fašistické státy, podepsány 10.2.1947 s Finskem, Maďarskem, Rumunskem, Bulharskem a Itálií.
\end{itemize}

\subsection*{Okupační zóny}
\begin{itemize}
    \item Německo, Rakousko a Japonsko byly rozděleny na okupační zóny.
    \item Území byla navrácena do stavu před válkou: Itálie musela vrátit Istrii a Rijeku.
\end{itemize}

\subsection*{Okupace Japonska USA}
\begin{itemize}
    \item Pod vedením generála MacArthura s cílem demokratizovat japonskou společnost.
    \item Vyhnání fašistů ze státní správy, nastolení parlamentní demokracie.
    \item Sanfranciská mírová smlouva s Japonskem 1951: hranice dle roku 1854, Japonsko přišlo o mnoho území.
\end{itemize}

\subsection*{Rakouská státní smlouva 1955}
\begin{itemize}
    \item Rakousko se zavázalo k neutralitě, okupační armády se stáhly, dodnes neutralitu drží, nejsou členy NATO.
    \item Rakousko bylo rozděleno na okupační zóny.
\end{itemize}

\subsection*{Dekolonizace}
\begin{itemize}
    \item[1945-1956] \textbf{1. fáze} – týká se jihovýchodní Asie (Francouzská Indočína, Indie, nizozemské kolonie, Indonésie) a nově vznikajících států jako Izrael.
    \item[1956-1965] \textbf{2. fáze} – především Afrika.
    \item[1965-] zbylé kolonie (Mosambik, Angola, od Portugalska).
    \item Rozvojové země a třetí svět.
    \item Hnutí nezúčastněných 1961: mezinárodní organizace více než 100 států, které nejsou zapojeny do žádných mocenských bloků, vzniklo v roce 1961.
\end{itemize}

\end{document}
