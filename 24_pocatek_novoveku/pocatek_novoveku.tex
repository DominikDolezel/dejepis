\documentclass{article}
\usepackage{fullpage}
\usepackage[czech]{babel}
\usepackage{amsfonts}

\title{\vspace{-2cm}Počátek novověku\vspace{-1.7cm}}
\date{}
\author{}

\begin{document}
\maketitle

\begin{itemize}
    \vspace{-0.5em}
    \setlength\itemsep{0.15em}
    \item[1492] mezník mezi středověkem a novověkem
\end{itemize}

\section*{Příčiny}
\begin{itemize}
    \vspace{-0.5em}
    \setlength\itemsep{0.15em}
    \item[$-$] Osmané brání v cestě na Orient
    \item[$-$] do Evropy se dováží koření, bavlna, porcelán, čaj, hedvábí, které najednou nemají
    \item[$-$] v Evropě došly zdroje zlata a stříbra
    \item[$-$] touha po zisku území za účelem zlepšení pozice, odbytiště pro výrobky
\end{itemize}

\section*{Předpoklady}
\begin{itemize}
    \vspace{-0.5em}
    \setlength\itemsep{0.15em}
    \item[$-$] posloužila mapa Eratosthena z Kyrémy z doby helénistické
    \item[$-$] mapa \textbf{Toscanelliho}, též oživil představu toho, že Země je kulatá
    \item[$-$] znalost kompasu od Arabů, \textit{astroláb} = určuje polohu hvězd podle polohy Slunce
    \item[$-$] nové typy lodí, \textit{karaky} či \textit{karabely}: mají hlubší ponor $\rightarrow$ stabilnější, hlubší podpalubí
    \item[$-$] \textbf{Martin Boheim}, autor prvního globu
    \item[$-$] na Pyrenejském poloostrově skončila reconquista
\end{itemize}

\section*{Koncepce}
\begin{itemize}
    \vspace{-0.5em}
    \setlength\itemsep{0.15em}
    \item[$-$] jak se dostat do Indie?
    \item[a.] plout kolem Afriky
    \item[b.] plout pořád na západ a prostě na ni \uv{narazit}
    \item[$-$] vítězí první možnost obeplouvání Afriky
    \item[1487] \textbf{Bartolomeo Díaz} (Portugalec) dorazí k Mysu Dobré naděje
    \item[1497] \textbf{Vasco da Gama} (Portugalec) se dostane až na západní pobřeží Indie, opěrný bod Kalikat
    \item[$-$] podél trasy zakládány obchodní stanice, dováží koření
    \item[1500] \textbf{Pablo Álvárez Cabral} (Portugalec) doplouvá do Brazílie, považován za objevitele
    \item[$-$] od 16. století pronikají na území dnešní Číny, Japonska
\end{itemize}

\section*{Španělsko}
\begin{itemize}
    \vspace{-0.5em}
    \setlength\itemsep{0.15em}
    \item[$-$] strategie: plují na západ a doufají, že do Indie dorazí
    \item[$-$] Isabela Kastilská a ferdinand Aragonský podporují Kryštofa Kolumba
    \item[$-$] záchytný bod Kanárské ostrovy
    \item[1492] Kryštof Kolumbus vyplouvá z přístavu Palos se třemi loďmi, po nějaké době doplouvají ke Karibiku, pak k Hispaniole
    \item[$-$] Kolumbus umírá v chudobě zapomenutý
    \item[$-$] \textbf{Amerigo Vespuci} začal popisovat, co na výpravě s Kolumbem vidí, začíná se mluvit o zemi Amerigově $\rightarrow$ Amerika, termín už od roku 1507
    \item[$-$] Mercatorova mapa
    \item[1500] \textbf{Vicente Pinzón} objevuje ústí Amazonky
    \item[1513] \textbf{Vasco de Balboa} překračuje Panamskou šíji -- potvrzení objevu nového kontinentu
    \item[1519] \textbf{Fernao Magalhaes} obeplul celou zeměkouli, Portugalec působící ve španělských službách, čímž doložil, že je kulatá
\end{itemize}


\section*{Koloniální válka}
\begin{itemize}
    \vspace{-0.5em}
    \setlength\itemsep{0.15em}
    \item[$-$] o nově objevená území mezi Španěli a Portugalci
    \item[$-$] 2000 km západně od Kapverdských ostrovů linie, na západ od ní Španělské a na východ Portugalské
    \item[$-$] od 16. století začínají pronikat mimo Evropu nová koloniální vlastníci: Nizozemci, Britové
\end{itemize}

\section*{Důsledky}
\begin{itemize}
    \vspace{-0.5em}
    \setlength\itemsep{0.15em}
    \item[$-$] do Evropy příliv zlata a stříbra
    \item[$-$] vytvoření světového obchodu, zánik tradičních obchodních center na Ap. pol. a přesouvají se do Evropy
    \item[$-$] přivezení nových nemocí, třeba Syfilis
\end{itemize}



\end{document}
