\documentclass{article}
\usepackage{fullpage}
\usepackage[czech]{babel}
\usepackage{amsfonts}

\title{\vspace{-2cm}Polské království\vspace{-1.7cm}}
\date{}
\author{}

\begin{document}
\maketitle
\section*{Obecně.}
\begin{itemize}
    \vspace{-0.5em}
    \setlength\itemsep{0.15em}
    \item[$-$] Slezané, Vislané (Krakov), Pomořané, Polané (SZ, Hnězdno)
    \item[$-$] tyto kmeny sjednotil \textbf{Měšek} z Piastovců (963 -- 992), manželka Doubravka (dcera Boleslava I.) -- šiřitelka křesťanství
    \item[$-$] sídelní město: Hnězdno
    \item[$-$] oráč Piast
\end{itemize}

\section*{Boleslav Chrabrý (992 -- 1025)}
\begin{itemize}
    \vspace{-0.5em}
    \setlength\itemsep{0.15em}
    \item[$-$] zahájil expanzi:
    \begin{itemize}
        \vspace{-0.5em}
        \setlength\itemsep{0.15em}
        \item[$-$] na východ, až do Kijeva
        \item[$-$] na západ, Lužice
    \end{itemize}
    \item[(1018)] \textsc{Budyšínský mír} s Jindřichem II.
    \item[$-$] nadstandartní vztahy se Silvestrem II.
    \item[(1000)] v Hnězdně zřízeno arcibiskupství při hrobu sv. Vojtěcha, arcibiskup \textbf{Radim} (nevlastní bratr Vojtěcha)
    \item[(1025)] na konci života královský titul $\rightarrow$ první polský král
    \item[$-$] soustava hradů
\end{itemize}

\section*{11. st.}
\begin{itemize}
    \vspace{-0.5em}
    \setlength\itemsep{0.15em}
    \item[1038] sídelním místem Krakov
    \item[1039] zisk Krakovska, Hnězdna Břetislavem I.
\end{itemize}

\section*{12. st.}
\begin{itemize}
    \vspace{-0.5em}
    \setlength\itemsep{0.15em}
    \item[(1076)] \textbf{Boleslav II. Smělý}: titul krále polského
    \item[$-$] \textbf{Boleslav III. Křivoústý}: zánik jednotného Polského státu
    \item[(1138)] nástupnický řád, kdo bude kde vládnout
\end{itemize}

\section*{13. st.}

\subsection*{Konrád Mazovský}
\begin{itemize}
    \vspace{-0.5em}
    \setlength\itemsep{0.15em}
    \item[1226] pozval do země řád německých rytířů, aby šířili křesťanství $\rightarrow$ ovládli Prusko
    \item[$-$] porážka až v \textsc{bitvě u Grunwaldu}
\end{itemize}

\subsection*{Jindřich II. Pobožný}
\begin{itemize}
    \vspace{-0.5em}
    \setlength\itemsep{0.15em}
    \item[1241] porážka Mongoly v \textsc{bitvě u Legnice}
\end{itemize}






\end{document}
