\documentclass{article}
\usepackage{fullpage}
\usepackage[czech]{babel}
\usepackage{amsfonts}

\title{\vspace{-2cm}Vznik USA, velká francouzská revoluce, Napoleon\vspace{-1.7cm}}
\date{}
\author{}

\begin{document}
\maketitle

\section*{Vznik USA}
\subsection*{Anglické kolonie}
\begin{itemize}
    \vspace{-0.5em}
    \setlength\itemsep{0.15em}
    \item[$-$] kolonizují především Angličané, Francouzi a Nizozemci
    \item[1624-1752] 13 anglických kolonií
    \item[1584-1587] Vitginie, anglická osada na ostrově Roanoke (mořeplavci Francis Drake, Walter Ralleigh) $\rightarrow$ akce neúspěšná, osada se neudržela
    \item[1607] až za Startovce Jakuba I. se zakládá \textit{Virginia} jako první anglická osada, \textbf{Jamestown}
    \item[$-$] \textbf{Mayflower} -- otcové poutníci, zakládají osady: 1620 z Plymouthu, 1630 založení Bostonu
    \item[$-$] Den díkuvzdání: děkuji indiánům, že je naučili pěstovat kukuřici a poskytli jim potravu pro přežití
    \item[$-$] jsou tu i Nizozemci $\rightarrow$ \textsc{anglo-nizozemská válka}, nizozemské město \textbf{Nový Amsterdam}, vyhrávají Angličané $\rightarrow$ \textbf{New York}
    \item[$-$] anglo-francouzská válka = \textsc{sedmiletá válka}, Angličané získávají Kanadu
    \item[$-$] každá kolonie má vlastní samosprávu, krále zastupuje guvernér, nemají zastoupení v anglickém parlamentu
    \item[$-$] kolonie postupně zabírají další území, Angličané však tato území považují za svá
    \item[$-$] utlačování a likvidace indiánů
\end{itemize}

\subsection*{Jiří III. z Hannoverovců (1760-1820)}
\begin{itemize}
    \vspace{-0.5em}
    \setlength\itemsep{0.15em}
    \item[$-$] po sedmileté válce je i navzdory vítězství Anglie ekonomicky zdecimována
    \item[$-$] konec vstřícné politiky vůči koloniím
    \item[pol. 18. st.] osady se chtějí dál rozvíjet (průmysl, podnikání, obchod), to se Britům v duchu merkantilismu nelíbí
    \item[$-$]  na severu spíše drobní farmáři, \textit{puritáni}, ze severu se vyváží železná ruda, dřevo, kožešiny, vznikají nejstarší a nejznámější americká města; na jihu velkostatkáři a plantáže (bavlna, rýže), využívání práce černochů dovážených z Afriky
    \item[1765] zavádí \textit{kolkovné}: kolek je jakási známka, která se lepí na oficiální tiskoviny, dokumenty, $\rightarrow$ Anglie z toho má peníze, protože jim je prodává, osady odporují
    \item[1767] dovozní cla na papír, sklo, \dots
    \item[1770] \textsc{Bostonský masakr}, osadníci obklíčili anglické vojáky v mostu a vojáci začali střílet
    \item[1773] \textsc{Bostonské pití čaje}: osadníci z lodí vyházeli krabice s čajem (protest proti dovozním clům) $\rightarrow$ Britové uzavírají bostonský přístav a posílají tam další vojsko
    \item[1774] první Kontinentální kongres ve Filadelfii, píší Jiřímu petice a stížnosti, nulová reakce, připravují se na boj s Angličany
    \item[1775] \textsc{incident v Lexingtonu a Concordu}: Anglické vojsko chce odzbrojit armádu osadníků
    \item[1775] vytváření armády osadníků, ta je však vytížená např. sklizní, postupně se však na jejich stranu začínají připojovat i ostatní země, které mají konflikty s Anglií (Francie, Nizozemsko), vede George Washington
    \item[1776] Britové vyhnáni z Bostonu

  \end{itemize}

  \subsection*{Získání nezávislosti}
  \begin{itemize}
      \vspace{-0.5em}
      \setlength\itemsep{0.15em}
    \item[1775-1781] druhý Kontonentální kongres, přibyl k nim Thomas Paine, požaduje \uv{zdravý rozum} -- nezávislost osad
    \item[4.7.1776] sepsání \textit{Prohlášení o nezávislosti}, jako hlavní písař považován \textbf{Thomas Jefferson}, vychází z osvícenských myšlenek
    \item[$\Rightarrow$] zahájení \textsc{války o nezávislost} mezi osadníky a Brity (1776-1783)
    \item[1777] přijetí první ústavy = \textit{Články Konfederace}, každý stát stejný hlas v jednokomorovém parlamentu, už od začátku nefungovalo
    \item[1777] \textsc{bitva u Saratogy} mezi osadníky a Angličany, vítězství osadníků $\Rightarrow$  Francie slibuje pomoc osadníkům, což se opravdu stalo, v čele amerických vojáků
    \item[1781] \textsc{bitva u Yorktownu}, američní osadníci definitivně vítězí
    \item[1783] mírovou smlouvou v Paříži Anglie uznává ztrátu svých kolonií v Americe
    \item[$-$] nový stát má hranice podél řeky Mississippi, Florida ještě španělská
    \item[1787] nová ústava, nově federace, dodnes, flexibilní (dá se jednoduše měnit)
    \item[1791] Bill of Rights: 10 dodatků do ústavy ohledně základních lidských práv
    \item[$-$] moc výkonná v rukou prezidenta, též hlavou kabinetu (vlády), prezidentská republika
    \item[$-$] moc zákonodárná tvořena dvoukomorovým kongresem (Kapitol): sněmovna reprezentantů (podle lidnatosti), senát (dva zástupci z každého státu), komory jsou rovnocenné
    \item[$-$] moc soudní tvoří nejvyšší soud a další nezávislé soudy
    \item[$-$] první prezident \textbf{George Washington}, velel armádě osadníků, po válce dostal obrovské pozemky, prý největším statkářem v USA
    \item[$-$] po něm \textbf{John Adams}, třetí \textbf{Thomas Jefferson}
    \item[$-$] koncepce vývoje státu:
    \begin{itemize}
        \vspace{-0.5em}
        \setlength\itemsep{0.15em}
        \item[$-$] Thomas Jefferson chce pořád stát zemědělský, přerušení vztahů s Británií, orientace na Francii, slabá ústřední moc
        \item[$-$] Alexander Hamilton chce silnou ústřední moc, orientaci na Británii, průmyslovou zemi
    \end{itemize}

\end{itemize}

\section*{Velká francouzská revoluce}

\subsection*{Příčiny, situace ve Francii}

\begin{itemize}
    \vspace{-0.5em}
    \setlength\itemsep{0.15em}
    \item[$-$] po smrti Ludvíka XIV. zdevastovaná říše, tisíce žebráků, Ludvík XV. (jeho pravnuk), ve válkách slezských proti Marii Terezii, ve válce sedmileté už na straně Marie Terezie, ztráta území v Kanadě a Indii, Ludvík XVI. (jeho vnuk), za něj vypukne VFR
    \item[$-$] absolutismus, cenzora, katolicismus, žádná svoboda projevu, nesnášenlivost vůči jiným než katolickému náboženství (po zrušení ediktu nantského)
    \item[$-$] Marie Antoinetta (manželka Ludvíka XVI.): Francouzi ji výrazně nesnáší (\uv{pyšná Rakušanka}), nakupuje šperky, pořádá párty, mastí hazardní hry $\rightarrow$ neoblíbená
    \item[$-$] tehdy nejilnější a nejlidnatější stát v Evropě
    \item[$-$] tři stavové: duchovní (privilegované), aristokracie (privilegované), zbytek
    \item[$-$] ekonomická situace: daně, desátky, odpustky, velkostatky nemají peníze na rozvoj
    \item[$-$] v Anglii začíná průmyslová revoluce, tady ne, podnikání brzdí cechy a manufaktury
    \item[$-$] hrozba finančního kolapsu, nákladný život ve Versailles, zvyšování státního dluhu, pomáhání americkým osadám (rozpočtově nefiskální)
    \item[$-$] neúroda, konkurenční levné zboží z Anglie
    \item[$-$] střídání finančních ministrů, nestabilita
    \item[5.5.1789] svolání generálních stavů do Versailles (duchovní a aristokracie), pověřeni králem východisko této jízlivé situace, ideální zdanit bohaté stavy, což nechtějí, král se zase nechce vzdát absolutismu,
    \item[17.6.1789] nakonec se třetí stav prohlásil za národní shromáždění, cálem je zrušit absolutismus a vypracovat ústavu omezující krále, v duchu osvícenských myšlenek se musí rozdělit státní moc na tři složky, lidé si jsou rovni předs zákonem, král samozřejmě nesouhlasí, nechce, aby dál pokračovali
    \item[$-$] král se tajně připravuje na vojenský zásah
    \item[14.7.1789] \textsc{dobytí Bastily}, počátek VFR, královské vězení, symbol absolutismu, děla na pevnosti byla natočena na chudinské čtvrti, to se lidem nelíbí, proto na ni zaútočili, dobyli ji, bylo tam však jenom asi sedm vězňů, důležité spíš jako symbol, rovněž ovládají pařížskou radnici
    \item[4.8.1789] \textsc{zrušena privilegia}: rovnost před zákonem, zrušení šlechtických privilegií, zrušení poddanství
    \item[26.8.1789] \textit{Deklarace práv člověka a občana}: formulovány základní lidská práva
    \item[5.10.1789] rozzuřený dav (nemají na jídlo) vtrhnul do královského sídla ve Versailles a donutil ho, aby se přestěhovali do Paříže
\end{itemize}

\subsection*{Konstituční monarchie}
\begin{itemize}
    \vspace{-0.5em}
    \setlength\itemsep{0.15em}
    \item[$-$] útěk krále z Francie k Belgii (Varennes), chce obnovit svoji pozici ve Francii a absolutistickou monarchii, byl však poznán a vrácen zpět do Paříže
    \item[(3.9.) 1791] zákonodárné shromáždění pracuje na ústavě, která vyšla v platnost
    \item[$\rightarrow$] konec absolutismu, konstituční monarchie = moc panovníka je omezena ústavou
    \item[$-$] dělba státní moci na tři složky:
    \begin{itemize}
        \vspace{-0.5em}
        \setlength\itemsep{0.15em}
        \item[$-$] moc zákonodárná: Zákonodárné národní shromáždění, vzešlé z voleb
        \item[$-$] moc výkonná: vláda + král, jmenuje ministry, vláda se zodpovídá Zákonod. shromáždění (= parlamentu)
        \item[$-$] moc soudní, soudci též voleni
    \end{itemize}
    \item[$-$] volby výrazně omezeny: \textit{majetkový cenzus}, smí volit jen bohatí (asi čtyři miliony voličů)
    \item[$-$] decentralizace státu, některé kompetence přeneseny na departementy (= kraje)
    \item[$-$] působení tzv. politických klubů, zárodky politických stran:
    \begin{itemize}
        \vspace{-0.5em}
        \setlength\itemsep{0.15em}
        \item[$-$] \textit{girondisté}: umírnění stoupenci republiky, tehdejší levice, reprezentují spíše bohatší vrstvy
        \item[$-$] \textit{cordelliéři}: nejradikálnější republikáni, levice (Danton, Marat) a časem k nim přibudou i \item[$-$] \textit{jakobíni}, ze začátku \uv{klub přátel monarchie}, po příchodu Robesierra se zradikalizovali, stoupenci monarchie
        \item[$-$] \textit{feullanti}: stoupenci monarchie (La Fayette)
    \end{itemize}
    \item[$-$] \textit{sanculotti}: nejchudší pasivní občané, nemohli volit
    \item[$-$] stát nemá peníze, zrušeily se dávky, ale nestačily se stvořit nové, stát nemí peníze $\Rightarrow$ obracejí se na církev $\Rightarrow$ nesouhlas některé části občanů
\end{itemize}

\subsection*{Vznik republiky}
\begin{itemize}
    \vspace{-0.5em}
    \setlength\itemsep{0.15em}
    \item[$-$] tehdy v rakouských zemích vládne bratr marie Antoinetty Leopold II., obrací se na jeho pomoc, Rakousko (a také Prusko a další německy hovořící státy) souhlasily s tím, že francouzskému králi pomohou obnovit monarchii za pomocí vojenské intervence
    \item[duben 1792] impulz k tomu, aby gerondisté vyhlásili Rakousku \textsc{preventivní válku}
    \item[$-$] nebyli shcopni válčit kvůli některým monarchistickým generálům, cizí armáda se blíží k francouzským hranicím, dobrovolníci z Marseilles si zpívají Marseillaisu a podařilo se jim cizí armádu zastavit $\rightarrow$ symbol revoluce, později hymna
    \item[9./10.8.1792] \textsc{povstání v Paříži}, vyhlášena pařížská Komuna = společnost rovných, král zatčen a uvězněn
    \item[$-$] vyhlášeny volby do Národního Konventu, většinu získávají gerondisté, volební právo už mají všichni
    \item[21.9.1792] hned na prvním konventu vyhlášena republika, sesazen král, Francouzi den před tím zásadně porážejí zahraniční armády
    \item[$-$] král obviněn z velezrady
    \item[21.1.1793] král popraven, jejich syn Ludvík později umírá
\end{itemize}
\subsection*{Jakobínská diktatura (2.6.1793-27./28.7.1794)}

\begin{itemize}
    \vspace{-0.5em}
    \setlength\itemsep{0.15em}


    \item[$\Rightarrow$] rozruch ve Evropě, všichni monarchové se přidávají na stranu intervenčních armád (Británie, Španělsko, Nizozemsko), složitá situace
    \item[$-$] v departmantech protirevolucionářské vzpoury, rolnické nepokoje ve Vendée
    \item[$-$] jakobíni prosazují tzv. \textit{malé maximum}: maximální ceny obilí a mouky
    \item[2.6.1793] nepřehledné a složité situace využívají jakobíni, zorganizují nbové \textsc{povstání v Paříži}, popravili girondisty, ujímají se moci, v čele \textbf{Maxmilien Robespierre} -- diktatura jakobínů
    \item[$-$] vypracována nová jakobínská ústava, nevešla nikdy v platnost (čekalo se, než skončí válka, co se nestalo) je plánována nulová dělba státní moci, všechna moc je v Konventu
    \item[$-$] v rámci Konventu vytvořen výbor pro veřejné dobro vedenýž Robespierrem, všichzni nepřátelé jakobínů likvidováni, mněl Francii chránit před zahgraniční intervencí
    \item[$-$] zavedení branné poovinnosti s cílem účinně bojovat se zahraničními vojsky, řada velitelů nerozvážně posílá nováčky do války možná díky tomu se intervenční armádě daří a dostává se i za území Francie
    \item[$-$] sťata i Marie Antoinetta
    \item[$-$] kult Nejvyšší bytosti: zaveden místo katolictví, deismus (bůh stvořil svět, ale dál už nezasahuje)
    \item[$-$] nový kalendář se začátkem za vzniku republiky (ne za Krista)
    \item[$-$] zavedeny metrické jednotky
    \item[$-$] \textit{velké maximum}: maximální ceny vztahující se na základní zboří (potraviny), stanovily maximální mzdy
    \item[1793] válčené úspěchy \textbf{Napolena Bonaparteho}, vyhání Angličany z přístavu Toulon
    \item[červenec 1793]  Jean Paul Marat (cordellier), zavražděn ve vaně Charlottou Corday, pochybnosti okolo jeho smrti
    \item[$-$] Dekrety podezřelých: zkrácené procesy
    \item[$-$] Danton (umírnění): \uv{dost už bylo poprav} $\rightarrow$ popraven x zběsilí (héberisté)
    \item[$-$] odhalování nepřátel revoluce, ne vždy prokazatelně
    \item[červen 1794] \textit{velký teror}: všichni, kdo nesouhlasí s Robespierrem skončí na popravišti, i jeho vlastní zastánci v parlamentu si nebyli jisti svým krkem
    \item[$-$] Robespierrovi nabízí titul diktátora, odmítá a chystá se zakročit proti svým spojencům ,kteří zneužili situace a jakkoliv se obohatili
    \item[$\rightarrow$ ]  poslední kapka, spiknutí proti Robespierrovi, svržen, dle nového kalendáře 9. thermidoru \textit{thermidorský převrat}
    \item[28.7.1794]  \textsc{poprava Robespierra}
\end{itemize}

\subsection*{Thermidorská reakce (období direktoria) 1794-1799}
\begin{itemize}
    \vspace{-0.5em}
    \setlength\itemsep{0.15em}
    \item[$-$] republika zůstává zachována, moc v rukou bohatých
    \item[$-$] konec řízeného hospodářství: maxima a minima zrušena $\rightarrow$ nárůst cen
    \item[$-$] teror proti jakobínům, jako vyřízení účtů
    \item[$-$] ve společnosti panuje nespokojenost, nikdo není spokojen
    \item[$-$] Francie úspěšná ve válkách, místo obranných už útočné, mír s Pruskem, Nizozemím a Španělskem
    \item[srpen 1795] nová ústava = \textit{direktorium}, konec Konventu, moc výkonnou má pět direktorů,  moc zákonodárná v rukou Radě 500 (dolní komora) a Radě starších (horní komora)
    \item[$-$] pokusy o změnu systému: \textsc{Spiknutí rovných} v čele s Babeufem, snažili se nastolit primitivní komunismus, po odhalení spiknutí popraven
    \item[$-$] royalisté (stoupenci monarchie) též chystali státní převrat, potlačil je Napoleon Bonaparte
    \item[$-$] je zapotřebí někoho, kdo stabilizuje situaci $\rightarrow$ tím je Napoleon Bonaparte
    \item[9.11.1799] státní převrat vedený Nepoleonem Bonapartem, definitivní konec revoluce
    \item[$-$] moc na sebe strhlo kolegium tří konzulů $\rightarrow$ období konzulátu, Napoleon prvním konzulem
    \item[$-$] platí do roku 1804, kdy se z Napoleona stává císař
    \item[2.8.1802] Napoleon v referendu získal titul doživotního konzula
    \item[2.12.1804] stává se císařem
\end{itemize}

\section*{Napoleonská doba}
\begin{itemize}
    \vspace{-0.5em}
    \setlength\itemsep{0.15em}
    \item[$-$] z Korsiky, údajně měl rád matematiku, první manželka Josefina Beauharnais, žádný syn
\end{itemize}

\subsection*{Domácí politika}
\begin{itemize}
    \vspace{-0.5em}
    \setlength\itemsep{0.15em}
    \item[$-$] potřeba nalezení silné osoby, kterou byl Napoleon
    \item[$-$] opírá se o vzdělanou vrstvu byrokratů
    \item[$-$] u moci jsou bohatí podnikatelé a obchodníci
    \item[$-$] vybízel emigranty, aby se vrátili zpátky
    \item[$-$] uděloval ocenění za zástluhy o stát, s tím cílem zaveden nový řád \textit{Čestné legie}
    \item[(1804)] občanský zákoník \textit{Code Civil}, považován za jeden z prvních moderních občanských zákoníků, zaručuje rovnost před zákonem, dědit mohou všichni synové (do té doby funguje tzv. \textit{majorát}, tedy dědí nejstarší syn), povolení manželského rozvodu
    \item[(1801)] konkordát s papežem, katolická církev ztratila spoustu majetků, tímto se tedy narovnaly vztahy mezi státem a papežem
    \item[$-$] doba provázena cenzurou, trestání protistátních aktivit
    \item[$-$] založení \textit{Banque de France}, francouzské banky, centrální banka Francie
    \item[$-$] zlikvidování inflace, modernizace školství, cílem je produkovat vzdělané občany oddané státu
    \item[$-$] pravidelně vybírány daně, vytvoření Napoleonovy velké armády \textit{Grande Armée}
    \item[$-$] \uv{chudým dal, co potřebovali, bohatým, co chtěli}
\end{itemize}

\subsection*{Zahraniční politika}
\begin{itemize}
    \vspace{-0.5em}
    \setlength\itemsep{0.15em}
    \item[$-$] excelentní stratég a vojevůdce
    \item[1792-1797] \textsc{první koaliční válka}, vpád do Belgie, mír s Pruskem
    \item[1796-1797] tažení na Apeninský poloostrov, kde jsou všechna území ovládána Habsburky a Bourbony, je tam tedy vítán, taktika bleskového úderu, postupně dobyl území, ze kterých vytvořil tzv. \textit{sesterské republiky}
    \item[$-$] Helvétská (dnešní Švýcarsko) a Batavský (dnešní Nizozemí) sesterká republika
    \item[(1797)] \textsc{mír v Campo Formio} s Rakouskem, které tím uznává, že ztrácí Habsburské území v Itálii a Belgie
    \item[$-$] velkým konkurentem je Velká Británie, zahajuje tedy tažení do Egypta, chce přerušit jejich cesty do Indie
    \item[1798-1799] úspěšné pozemní bitvy
    \item[$-$] neúspěšné námořní bitvy u Akry a u Abukiru, protože narazil na skvělého britského vojevůdce Horatia Nelsona
    \item[$-$] nalezení Rosetské desky u Nilu, díky ní mohl jean Francois Champollion doluštit hieroglyfy
    \item[$-$] poté se vrací do Francie, kde provádí převrat, nastoluje éru konzulátu, ve kterém se stává prvním konzulem
\end{itemize}

\subsection*{Druhá koalice (Anglie, Rusko, Rakousko a spol.)}
\begin{itemize}
    \vspace{-0.5em}
    \setlength\itemsep{0.15em}
    \item[1798] Rusové vpadli do severní Itálie pod vedením A. V. Suvorova a obsazují Milán, jednu ze sesterských republik
    \item[$-$] Rusové se nakonec z koalice stáhli, protože Britové bez konzultace obsadili Maltu
    \item[1800] tím se otevřela cesta Napoleonovi přes Alpy do Itálie
    \item[1800] v \textsc{bitvě u Marenga} poráží Rakousko a znovu získává své území
    \item[1801] \textsc{mír v Lunéville} Rakousko znovu potvrzuje ztrátu Lombardie a Belgie
    \item[$-$] tři roky \uv{ozbrojeného míru} s Anglií, chystá se, ale neválčí
    \item[1804] získává titul císaře, Francie se stává císařstvím
\end{itemize}

\subsection*{Třetí koalice (Anglie, Rakousko, Rusko, Švédsko)}
\begin{itemize}
    \vspace{-0.5em}
    \setlength\itemsep{0.15em}
    \item[$-$] příprava invaze na britské ostrovy, pozornost Francouzů se obrátila na Rakousko
    \item[21.10.1805] \textsc{bitva u Trafalgaru}, v této námořní bitvě proti francouzsko-španělské flotile se postavili Britové v šele s Horatiem Nelsonem, Britové vyhráli, Nelson však byl smrtelně zraněn
    \item[(15.-20.10.1805)] \textsc{bitva u Ulmu}, Francie opět vítězí nad Habsburky vedené vévodou Karlem, bratrem Františka II., též jim pomáhali Rusové pod vedením Kutuzova
    \item[3.11.] vstup Napoleona do Vídně
    \item[2.12.] \textsc{bitva u Slavkova} = bitva tří císařů
    : car ALexandr I., František II., vítězí Napoleon a donutí je podepsat tzv.
    \item[6.12.] \textit{Bratislavský mír}: Rakousko-Uhersko ztrácí značná území
\end{itemize}


\subsection*{Čtvrtá koalice (Prusko, Velká Británie, Rusko, Švédsko)}
\begin{itemize}
    \vspace{-0.5em}
    \setlength\itemsep{0.15em}
    \item[$-$] Napoleonovi se podařilo získat celá Apeninský poloostrov
    \item[(1806)] rozbití SŘŘ Napoleonem, místo ní stvořil \textit{Rýnský spolek}, jehož byl protektor, František II. (tehdejší císař SŘŘ) už se nemůže titulovat jako císař, protože SŘŘ už neexistuje
    \item[$-$] porážka Pruska, z pruské části Polska (získané při trojím dělení Polska) vytvořil \textit{Velkovévodství varšavské}
    \item[$-$] po dobytí Berlína vyhlásil \textit{kontinentální blokádu}: zákaz obchodu s Velkou Británií, Francouzi však nebyli schopni nahradit britské zboží, dohoda byla porušována
    \item[1807] Rusové šli na pomoc Prusku proti Napoleonovi: \textsc{bitva u Jílového} mezi ruskem a Napoleonem, která dopadla nerozhodně
    \item[1807] \textsc{bitva u Friendlandu} porážka Rusů, Napoleon získává královec a vytlačují Rusy
    \item[červenec 1807] \textit{mír v Tylži} s Rusy (Rusové mají dodržovat kontinentální blokádu a bylo jim slíbeno Finsko) a Prusy (Prusové uznávali scé ztráty a vznik knížectví varšavského, vznik Ránského spolku atd.)
    \item[1807] Napoleon obsazuje Portugalsko, protože se odsud do Evropy vyváží britské obilí
    \item[1808] Napoleon útočí na Španělsko, díky tomu sesazeni Bourboni, králem se stal Nepoleonův bratr Josef I., ve Španělsku však nespokojenost, přišli jim na pomoc Angličané, nakonec se jim podařilo Francouze vyhnat
\end{itemize}

\subsection*{Pátá koalice (Velká Británie, Rakousko)}

\begin{itemize}
    \vspace{-0.5em}
    \setlength\itemsep{0.15em}
    \item[1809] \textsc{bitva u Aspern} proti Rakousko, Napoleon poražen, \textsc{bitva a Wagramu} a \textsc{bitva u Znojma} už však prohrává Rakousko
    \item[$\rightarrow$] \textit{Vídeňský mír}, ztrácí další území
    \item[$-$] k uzavření nové spojenecké smlouvy mezi Francií a Rakouskem přispěl ministerský předseda Metternich, kterému se podařilo s nimi navázat spolupráci a uzavřít spojeneckou smluvu, mimojiné zorganizoval svatbu Ffrantiška I. s Marií Luisou
    \item[1810] Napoleon obsadil Holandsko, kde vládne další Napoleonův bratr Ludvík
    \item[$\rightarrow$] Napoleon ovládá většinu Evropy
\end{itemize}



\end{document}
