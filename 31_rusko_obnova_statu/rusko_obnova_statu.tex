\documentclass{article}
\usepackage{fullpage}
\usepackage[czech]{babel}
\usepackage{amsfonts}

\title{\vspace{-2cm}Rusko -- obnova státu \vspace{-1.7cm}}
\date{}
\author{}

\begin{document}
\maketitle

\section*{Kyjevská Rus (882-1125)}
\begin{itemize}
    \vspace{-0.5em}
    \setlength\itemsep{0.15em}
    \item[$-$] vládnou Rurikovci, na počátku 12. století se drobí na jednotlivá knížectví
    \item[$-$] konec těchto knížectví přivodili Mongolové = Tataři
    \item[1223] porážka knížat Mongoly \textsc{na řece Kalce}
    \item[1240] dobývají Kyjev, hlavní město Kyjevské Rusi
    \item[$-$] v Novgorodském knížectví vládne kníže Alexandr
    \item[1240] \textsc{Alexandr na řece Něvě poráží Švédy}
    \item[1242]  \textsc{ALexandr poráží na zamrzlém jezeře řád německých rytířů}
    \item[$-$] většinu knížectví ovládají Mongolové, vytváří svůj stát, tzv. \textit{Zlatou hordu}
    \item[$-$] nutí je, aby jim platili daně, Zlatá horda se postupně rozpadá na menší území
    \item[$-$] novým hlavním městem obnoveného slovanského státu se stává Moskva, chtějí se zbavit nadvlády Mongolů
\end{itemize}


\section*{Ivan I. Kalita}
\begin{itemize}
    \vspace{-0.5em}
    \setlength\itemsep{0.15em}
    \item[$-$] snažil se zabránit, any Mongolové najížděli do Ruska
    \item[$-$] nabídl se, že bude vybírat daň pro Mongoly místo nich
\end{itemize}


\section*{Dmitrij Donský}
\begin{itemize}
    \vspace{-0.5em}
    \setlength\itemsep{0.15em}
    \item[$-$] vnuk Ivana
    \item[1380] poprvé poráží Mongoly \textsc{na Kulikovském poli}, ovšem jen nakrátko
\end{itemize}


\section*{Ivan III. (1462-1505)}
\begin{itemize}
    \vspace{-0.5em}
    \setlength\itemsep{0.15em}
    \item[$-$] vnuk Dmitrije
    \item[$-$] definitivně poráží Mongoly, nemusí jim platit daň, zakládá Moskevskou Rus, dobývají zpět své území
    \item[$-$] buduje Moskvu jako tzv. \textit{třetí Řím}, cítí se jako nový Konstantin Veliký, oporou je mu pravoslavné náboženství, titul \textit{velkovévoda}
    \item[$-$] budování administrativy, právní normy = \textit{suděbnik}, jednotné velené armády, národním jazykem ruština
    \item[$-$] budování sídla moskevských vládců = \textit{Kreml}
    \item[$-$] žena neteř posledního byzantského císaře Konstantina XI., šíření vlivu byzantské kultury
    \item[$-$] dvojí typy půdy: \textit{votčina} = dědičný majetek, dobrá půda, kterou získávají \textit{bojaři} = šlechta, členové \textit{bojarské dumy} = poradní sbor panovníka; \textit{poměstí} = nedědičná půda, za služby vladaři ji dostávali poddaní, jen pro sebe, nedědí se
\end{itemize}

\section*{Ivan IV. Hrozný (1533-1584)}
\begin{itemize}
    \vspace{-0.5em}
    \setlength\itemsep{0.15em}
    \item[1547] prohlásil se za prvního ruského cara, chce asertovat dominanci, \textit{samoděržaví} = absolutismus
    \item[$-$] řada reforem, dobře si vede ve válkách se zbytky Mongolských chanátů, odhání je od řeky Volhy a dostává se tak ke Kaspickému moři
    \item[$-$] založení (zamrzajícího) přístavu Archangelsk, obchod s Anglií (současnice královna Alžběta), Nizozemím
    \item[$-$] chrám Vasila Blaženého v centru Moskvy
    \item[$-$] nejbohatší šlechtický rod: Stroganovovi, sponzorují války v oblasti Sibiře
    \item[$-$] \textsc{Livonská válka}, chce přístup k Baltskému moři (dnešní Estonsko a Lotyšsko), byl však neúspěšný, neměl podporu bojarů
    \item[$-$] Ivan za to bojary potrestal reformou půdy = \textit{opričnina}, dobrou dědičnou půdu bojarům konfiskuje a rozdělil ji mezi jemu oddanou šlechtu = \textit{opričnici}, horší půdu dal bojarům = \textit{zemština}, reforma krvavá, musel od toho odstoupit
\end{itemize}


\section*{Smuta (1584-1613)}
\begin{itemize}
    \vspace{-0.5em}
    \setlength\itemsep{0.15em}
    \item[$-$] po Ivanově smrti zůstává syn Fjodor, ten je slabomyslný, má ještě bratra Dimitrije, kterého však zavraždil, místo něj tedy vládne Fjodorův švagr \textbf{Boris Godunov}, stabilizuje Rusko, nakonec záhadně umírá, carem se prohlašuje \textbf{Lžidimitrij I.}, který se vydává za zavražděného Dimitrije, později předák bojarů \textbf{Vasilij Šuskyj}, poté další \textbf{Lžidimitrij II.}, poté Moskvu dobývají Poláci a dosazují \textbf{Vladislava Polského}
    \item[=] nestabilní období střídání panovníků, které konsoliduje až nový panovnický rod \textbf{Romanovci}, nastupuje Michail Fjodorovič Romanov, zakládá moderní ruský stát
\end{itemize}



\end{document}
