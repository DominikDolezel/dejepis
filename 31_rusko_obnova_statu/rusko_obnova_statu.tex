\documentclass{article}
\usepackage{fullpage}
\usepackage[czech]{babel}
\usepackage{amsfonts}

\title{\vspace{-2cm}Rusko -- obnova státu \vspace{-1.7cm}}
\date{}
\author{}

\begin{document}
\maketitle

\section*{Kyjevská Rus (882-1125)}
\begin{itemize}
    \vspace{-0.5em}
    \setlength\itemsep{0.15em}
    \item[$-$] vládnou Rurikovci, na počátku 12. století se drobí na jednotlivá knížectví
    \item[$-$] konec těchto knížectví přivodili Mongolové = Tataři
    \item[1223] porážka knížat Mongoly \textsc{na řece Kalce}
    \item[1240] dobývají Kyjev, hlavní město Kyjevské Rusi
    \item[$-$] v Novgorodském knížectví vládne kníže Alexandr
    \item[1240] \textsc{Alexandr na řece Něvě poráží Švédy}
    \item[1242]  \textsc{ALexandr poráží na zamrzlém jezeře řád německých rytířů}
    \item[$-$] většinu knížectví ovládají Mongolové, vytváří svůj stát, tzv. \textit{Zlatou hordu}
    \item[$-$] nutí je, aby jim platili daně, Zlatá horda se postupně rozpadá na menší území
    \item[$-$] novým hlavním městem obnoveného slovanského státu se stává Moskva, chtějí se zbavit nadvlády Mongolů
\end{itemize}


\section*{Ivan I. Kalita}
\begin{itemize}
    \vspace{-0.5em}
    \setlength\itemsep{0.15em}
    \item[$-$] snažil se zabránit, any Mongolové najížděli do Ruska
    \item[$-$] nabídl se, že bude vybírat daň pro Mongoly místo nich
\end{itemize}


\section*{Dmitrij Donský}
\begin{itemize}
    \vspace{-0.5em}
    \setlength\itemsep{0.15em}
    \item[$-$] vnuk Ivana
    \item[1380] poprvé poráží Mongoly \textsc{na Kulikovském poli}
\end{itemize}


\section*{Ivan III.}
\begin{itemize}
    \vspace{-0.5em}
    \setlength\itemsep{0.15em}
    \item[$-$] vnuk Dmitrije
    \item[$-$] definitivně poráží Mongoly, nemusí jim platit daň
    \item[$-$]
\end{itemize}



\end{document}
