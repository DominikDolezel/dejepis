\documentclass{article}
\usepackage{fullpage}
\usepackage[czech]{babel}
\usepackage{amsfonts}

\title{\vspace{-2cm}Vznik Československé republiky\vspace{-1.7cm}}
\date{}
\author{}

\begin{document}
\maketitle

\begin{itemize}
    \vspace{-0.5em}
    \setlength\itemsep{0.15em}
    \item[$-$] Česko, Slovensko, Podkarpatská Rus
    \item[28.7.1914] František Josef I. v manifestu \textit{Mým národům} ohlašuje, že monarhcie je ve válce
    \item[25.7.1914]uzavřena Říšská rada (parlament)
    \item[$-$] změny v ekonomice, orientace na militarismus, zaveden přídělový systém
    \item[$-$] Češi musí bojovat za monarchii na východní frontě
    \item[$-$] rekace českých politiků:
    \begin{itemize}
        \vspace{-0.5em}
        \setlength\itemsep{0.15em}
        \item[$-$] ti, co chodili do Říšské rady nebyli pro to, aby se měla monarchie rozbít, ještě po válce vydali stanovisko, že se má zůstat v monarchii
        \item[$-$] skupina kolem Karla Kramáře: začal začal spojovat vznik Československa s Rusy, chtěl mít Československo jako monarchii
        \item[$-$] skupina kole Tomáše Garriguea Masaryka
    \end{itemize}
\end{itemize}

\subsection*{Tomáš Garrigue Masaryk}
\begin{itemize}
    \vspace{-0.5em}
    \setlength\itemsep{0.15em}
    \item[$-$] do politiky se zapojil po první světové válce, kdy mu bylo přes šedesát
    \item[$-$] vystudoval sociologii, studoval gymnasium v Brně, ve Vídni filosofickou fakultu
    \item[$-$] po studiích začíná přednášet na české části pražské Karlovy university
    \item[$-$] habilitoval se prací \textit{Sebevražda}
    \item[$-$] při svém pobytu v Lipsku se seznámil s Američankou Charlotte Garrigou, s níž se v USA oženil a poté se vrátili do Čech
    \item[$-$] založil stranu lidovou, následně přejmenovanou na pokrokovou
    \item[$-$] už před válkou se zapsal do povědomí: přiklonil se k vědcům, kteří považovali RKZ za falsa
    \item[$-$] zapojil se do \textit{Hilsneriády}, kde upozorňoval proti antisemitismu
    \item[$-$] po vypouknutí války odchází do Itálie a následně do Švýcarska
    \item[4.7.1915] v Curychu vystupuje při příležitosti 500. výročí upálení Mistra Jana Husa
    \item[6.7.1915] přesouvá se do Ženevy, kde říká, že se má Československo odtrhnout od monarchie a že Češi a Slováci mají mít svůj vlastní stát
    \item[září 1915] do Ženevy odchází jeho žák, Edvard Beneš, aby mohl Beneš nastoupit do funkce presidenta, prosadil Masaryk, že funkce presidenta je už od 35 let
    \item[$-$] zakládají Československý zahraniční komitet
    \item[únor 1916] poté se přesouvají do Francie a zakládají Českou národní radu (ČNR), to je politický ilegální orgán, tři hlavní působící: Masaryk, Beneš, Štefánik
    \item[$-$] úkolem ČNR bylo přesvědčit dohodové země, aby souhlasily s tím, že se Rakousko-Uhersko rozpadne na nástupnické státy
    \item[říjen 1916] \textit{Cleavelandská dohoda}, byla podepsána mezi Čechy a Slováky, znamená, že chceme vlastní federativní stát
\end{itemize}

\subsection*{České země během války}
\begin{itemize}
    \vspace{-0.5em}
    \setlength\itemsep{0.15em}
    \item[$-$] zaveden ostrý protičeský kurs, zrušena svoboda slova, censura
    \item[$-$] Habsburkové se potřebují soustředit na válku a ne řešit zlobivé Čechy -- bylo zatknuto několik českých politiků, někteří dokonce za protirakouskou politiku byli odsouuzeni k trestu smrti, zachránila je amnestie
    \item[březen 1915] vznik \textit{Maffie}: snaží se mezi občany rozšiřovat povědomí o tom, že by bylo dobré mít svůj vlastní stát a udávat informace Dohodě o dění na území Rakouska-Uherska, spolupracovali s nimi i Slováci
    \item[$-$] proti Maffii stáli čeští politici, kteří dřív docházeli do Říšského sněmu, sami si vytvořili \textit{Český svaz poslanců ŘR}: prohalšují, že budou poslušní Habsburkům
\end{itemize}

\subsection*{Ruské legie}
\begin{itemize}
    \vspace{-0.5em}
    \setlength\itemsep{0.15em}
    \item[$-$] jedním z nástrojů, jak Čechy v rámci monarhcie zviditelnit, byly tzv. ruské legie
    \item[$-$] na území cizích států působí legionáři: české jednotky, které bojovaly na straně Dohody
    \item[$-$] měly i propagační význam -- propagovaly myšlenky samostatného Československého státu
    \item[(12.8.1914)] car Mikuláš povolil vznik české družiny v Kyjevě
    \item[$-$] postupné vytváření jednotky československé
    \item[$-$] do Ruska přijel i Masaryk s cílem jednotku zorganizovat, což se mu podařilo
    \item[(2.7.)1917] \textsc{bitva u Zborova}, první úspěch českých legií na úkor Rakousko-Uherské armády;
    \item[(7.11.1917)] Bolševický převrat: Masaryk nařizuje, aby se Čechoslováci do situace v Rusku nevměšovali (\textit{ozbrojená neutralita}), mají bojovat maximálně v zájmu Čechoslováků
    \item[$-$] Čechoslováci se mají přes tzv. trassibiřskou magistrálu (vlak) přesunout na západní frontu, kde by bojovali na straně Dohody, těžká operace (Rusáci je chtěli odzbrojit, vylodění Japonců)
    \item[$-$] Rusové vnímají Čechoslováky velmi negativně, protože obsadili transsibiřskou magistrálu, čímž jim zabránili rozšiřovat revoluci na Sibiř
    \item[listopad 1918] doba, kdy existuje československý stát -- legionáře navštívil Rastislav Štefánik a vyřídil jim, aby pokračovali v bojích v Rusku na straně intervenčních armád, ti však nechtěli
    \item[$-$] ukořistění části ruského pokladu, který ukradl Kolčak
    \item[prosinec 1919] první transport legionářů z Vladivostoku
    \item[listopad 1920] dokončení evakuace -- tomuto návratu se říká \textit{transibiřská anabáze}
    \item[$-$] generálové Syrový, Krejčí, Čeček
    \item[$-$] v Rusku bojovalo více než 10 000 českých vojáků
\end{itemize}

\subsection*{Československé legie}
\begin{itemize}
    \vspace{-0.5em}
    \setlength\itemsep{0.15em}
    \item[$-$] vznikla v rámci francouzské cizinecké legie až na konci války
    \item[$-$] říkali si \textit{rota Nazdar}
    \item[$-$] zapsali se v bitvách u Terronu, Arrasu, Vousieres, Chemin des Dames
    \item[$-$] Itálie: boje na Piavě, Doss Alto
    \item[$-$] Srbsko
    \item[$-$] přispělo k myšlence vzniku samostatného československého státu
\end{itemize}

\subsection*{Rakousko-Uhersko 1916-1917}
\begin{itemize}
    \vspace{-0.5em}
    \setlength\itemsep{0.15em}
    \item[21.11.1916] prasynovec Františka Josefa I. stanovil novou politiku, široká politická amnestie (zachánení Kramáře, Rašína)
    \item[$-$] vedl tajná jednání s Dohodou s cílem uzavřít mír, byla prozrazena
    \item[$-$] Český svaz: sdružuje politiky, kteří chtjěí být součástí monarchie
    \item[$-$] proti setrvání v Habsburské monarchii protestuje česká inteligence, sepisuje \textit{manifest českých spisovatelů}, autorem je Jaroslav Kvapil, podepsalo jej 222 osob, obracejí na české poslance, aby hájili zájmy českého národa, jinak ať odejdou
    \item[17.5.1917] poslanci se nejspíše pod vlivem manifestu začali hlásit k získáni autonomie a připojení Slovenska k Čechu
    \item[6.1.1918] \textit{Tříkrálová deklarace}: habsurkové ji označili za velezrádný dokument, čeští politici se v ní hlásí k vytvoření samostatného státu
    \item[13.4.] v Praze v Obecním domě se sešli významní představitelé a Alois jirásek tu předčítal slavnostní přísahu v tom smyslu, aby vznikl samostatný československý stát
\end{itemize}

\subsection*{Vojenské vzpoury 1918}
\begin{itemize}
    \vspace{-0.5em}
    \setlength\itemsep{0.15em}
    \item[$-$] Boka Kotorská (v Černé Hoře) bouřili se tam čechoslováci, u ž nechtjěí válčit, Kragujevac (Srbsko), Piava
    \item[30.5.] \textit{Pitsburská dohoda}: podepisuje ji masaryk v USA, už se nemluví federaci, ale o autonomii Slovenska, což nebylo dodrženo
    \item[$-$] postupně Dohodové velmoci uznaly, že má vzniknout budoucí Československo, uznaly československou národní radu za oficiální politický orgán
\end{itemize}

\section{Československo}
\begin{itemize}
  \item[13. 7.] se poskládal Národní výbor dle posledních voleb (Kramář, Švehla, Rašín, Klofáč, Soukup) -- připravovali se, že převezmou moc až skončí válka, pod nimi funguje
  \item[od srpna] Zemská hospodářská rada -- otázka hospodářství, měnové odluky, železnic
  \item[14. 10.] uspořádána generální stávka nějakou socialistickou radou (taky pod N výborem) proti vývozu potravin z čs
  \item současně vznikla 14.10. čs. prozatimní vláda, tedy že se ČNR (TGM, EB, MRŠ) pasuje na vládu
  \item když je konec války na spadnutí přichází 16.10. Karel I. s manifestem -- Mým věrným národům rakouským -- snažil se nabídnout národům v rámci monarchie federalizaci, nicméně už je pozdě
  \item[18. 10.] Washingtonská deklarace, od Masaryka Wilsonovi, tohoto dne zveřejněna v Paříži -- na jakých základech bude čs. stát vybudován (republika, demokracie), stejného dne přichází tzv. Wilsonova nóta, že souhlasí s rozpadem rakouska a vznikne čsr
  \item[24. 10.] zrušena smlouva s německem, 25.10. zhroucení italské fronty, 26.10. karel žádá o separátní mír, 27.10. posílá Gyula Andrassy nótu že přijímají mírové podmínky, 28.10. zákon o zřízení samostatného československého státu (z pera Aloise Rašína), krátce po poledni tedy vyhlášena republika, všichni důležití byli v ženevě, tam jednali s politiky v emigraci kdo bude president, jaká bude vláda atd., známí jsou ti kteří zůstali u nás -- muži 28. října --Alois Rašín, Jiří Stříbrný, Antonín Švehla, František Soukup, Vavro Šrobár
  \item[30. 10.] sešli se zástupci slovenských politiků ve svätem Martine -- Martinská deklarace -- Slovenská národní rada, vyjádřili se, že společně s Čechy chtějí vytvořit čs. stát
  \item[13. 11.] Prozatimní ústava, národní výbor zvětšili, rozšířili o slováky -- Revoluční národní shromáždění -- všechno se to odehrálo v Obecním domě
  \item[14. 11.] se v Thunovském paláci sešlo RNS, prohlásili že čs bude republika, prezidentem zvolili v nepritomnosti masaryka, sesadili Habsburky a jmenovlai vládu v čele s karlem Kramářem
  \item[21. 12. 1918] TGM se dostává do ČSR, tady by končila písemka
\end{itemize}


\end{document}
