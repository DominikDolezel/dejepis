\documentclass{article}
\usepackage{fullpage}
\usepackage[czech]{babel}
\usepackage{amsfonts}

\title{\vspace{-2cm}Vznik Československé republiky\vspace{-1.7cm}}
\date{}
\author{}

\begin{document}
\maketitle

\begin{itemize}
    \vspace{-0.5em}
    \setlength\itemsep{0.15em}
    \item[$-$] Česko, Slovensko, Podkarpatská Rus
    \item[28.7.1914] František Josef I. v manifestu \textit{Mým národům} ohlašuje, že monarhcie je ve válce
    \item[25.7.1914]uzavřena Říšská rada (parlament)
    \item[$-$] změny v ekonomice, orientace na militarismus, zaveden přídělový systém
    \item[$-$] Češi musí bojovat za monarchii na východní frontě
    \item[$-$] rekace českých politiků:
    \begin{itemize}
        \vspace{-0.5em}
        \setlength\itemsep{0.15em}
        \item[$-$] ti, co chodili do Říšské rady nebyli pro to, aby se měla monarchie rozbít, ještě po válce vydali stanovisko, že se má zůstat v monarchii
        \item[$-$] skupina kolem Karla Kramáře: začal začal spojovat vznik Československa s Rusy, chtěl mít Československo jako monarchii
        \item[$-$] skupina kole Tomáše Garriguea Masaryka
    \end{itemize}
\end{itemize}

\subsection*{Tomáš Garrigue Masaryk}
\begin{itemize}
    \vspace{-0.5em}
    \setlength\itemsep{0.15em}
    \item[$-$] do politiky se zapojil po první světové válce, kdy mu bylo přes šedesát
    \item[$-$] vystudoval sociologii, studoval gymnasium v Brně, ve Vídni filosofickou fakultu
    \item[$-$] po studiích začíná přednášet na české části pražské Karlovy university
    \item[$-$] habilitoval se prací \textit{Sebevražda}
    \item[$-$] při svém pobytu v Lipsku se seznámil s Američankou Charlotte Garrigou, s níž se v USA oženil a poté se vrátili do Čech
    \item[$-$] založil stranu lidovou, následně přejmenovanou na pokrokovou
    \item[$-$] už před válkou se zapsal do povědomí: přiklonil se k vědcům, kteří považovali RKZ za falsa
    \item[$-$] zapojil se do \textit{Hilsneriády}, kde upozorňoval proti antisemitismu
    \item[$-$] po vypouknutí války odchází do Itálie a následně do Švýcarska
    \item[4.7.1915] v Curychu vystupuje při příležitosti 500. výročí upálení Mistra Jana Husa
    \item[6.7.1915] přesouvá se do Ženevy, kde říká, že se má Československo odtrhnout od monarchie a že Češi a Slováci mají mít svůj vlastní stát
    \item[září 1915] do Ženevy odchází jeho žák, Edvard Beneš, aby mohl Beneš nastoupit do funkce president, prosadil Masaryk, že funkce presidenta je už od 35 let
    \item[$-$] zakládají Československý zahraniční komitet
    \item[únor 1916] poté se přesouvají do Francie a zakládají Českou národní radu (ČNR), to je politický ilegální orgán, tři hlavní působící: Masaryk, Beneš, Štefánik
    \item[$-$] úkolem ČNR bylo přesvědčit dohodové země, aby souhlasily s tím, že se Rakousko-Uhersko rozpadne na nástupnické státy
    \item[říjen 1916] \textit{Cleavelandská dohoda}, byla podepsána mezi Čechy a Slováky, znamená, že chceme vlastní federativní stát
\end{itemize}

\subsection*{České země během války}
\begin{itemize}
    \vspace{-0.5em}
    \setlength\itemsep{0.15em}
    \item[$-$] zaveden ostrý protičeský kurs, zrušena svoboda slova, censura
    \item[$-$] Habsburkové se potřebují soustředit na válku a ne řešit zlobivé Čechy -- bylo zatknuto několik českých politiků, někteří dokonce za protirakouskou politiku byli odsouuzeni k trestu smrti, zachránila je amnestie
    \item[březen 1915] vznik \textit{Maffie}: snaží se mezi občany rozšiřovat povědomí o tom, že by bylo dobré mít svůj vlastní stát a udávat informace Dohodě o dění na území Rakouska-Uherska, spolupracovali s nimi i Slováci
    \item[$-$] proti Maffii stáli čeští politici, kteří dřív docházeli do Říšského sněmu, sami si vytvořili \textit{Český svaz poslanců ŘR}: prohalšují, že budou poslušní Habsburkům
\end{itemize}

\subsection*{Ruské legie}
\begin{itemize}
    \vspace{-0.5em}
    \setlength\itemsep{0.15em}
    \item[$-$] jedním z nástrojů, jak Čechy v rámci monarhcie zviditelnit, byly tzv. ruské legie
    \item[$-$] na území cizích států působí legionáři: české jednotky, které bojovaly na straně Dohody
    \item[$-$] měly i propagační význam -- propagovaly myšlenky samostatného Československého státu
    \item[$-$] 
\end{itemize}



\end{document}
