\documentclass{article}
\usepackage{fullpage}
\usepackage[czech]{babel}
\usepackage{amsfonts}

\title{\vspace{-2cm}Francie po stoleté válce, nizozemská revoluce, Tudorovci, obnova Ruska\vspace{-1.7cm}}
\date{}
\author{}



\begin{document}
\maketitle


\section*{Francie po stoleté válce (1337-1453)}
\begin{itemize}
    \vspace{-0.5em}
    \setlength\itemsep{0.15em}
    \item[$-$] vládnou panovníci z rodu Valois
    \item[$-$] jediní, kdo ve Francii pracovali, byli měšťané
    \item[$-$] po stabilizaci Francie přichází expanzivní politika, cílem je Apeninský poloostrov
\end{itemize}

\subsection*{Karel VIII. (1483-1498)}
\begin{itemize}
    \vspace{-0.5em}
    \setlength\itemsep{0.15em}
    \item[$-$] pokusil se získat Neapolské království, kde kdysi vládla fran. dynastie z Anjou
    \item[$-$] vyfoukl mu ho \textbf{Ferdinand Aragonský} (zakladatel Španělska), na jeho stranu se totiž přiklonila tzv. \textit{benátská liga} (Maxmilián Habsburský, papež, Benátky, Milán)
\end{itemize}

\subsection*{Ludvík XII. (1498-1515)}
\begin{itemize}
    \vspace{-0.5em}
    \setlength\itemsep{0.15em}
    \item[$-$] dočasně získal Milánské vévodství
    \item[$-$] poté se musí stáhnout pryč, získají to Habsburkové
    \item[$\Rightarrow$] války mezi Francií a Habsburky pokračují
\end{itemize}

\subsection*{František I. (1515-1547)}
\begin{itemize}
    \vspace{-0.5em}
    \setlength\itemsep{0.15em}
    \item[$-$] válčil se španělským králem \textbf{Karlem V.}
    \item[$-$] \textit{kníže renesance}, na jeho dvoře pobývali třeba Leonardo, Michelangelo, Rafael, Rabelais, Tizian
    \item[$-$] úředním jazykem ve Francii se stává francouzština, rozšiřuje svoji knihovnu
    \item[$-$] Habsburkové v této době drží obrovská území, položí základy Habsburské mnohonárodnostní monarchie, \uv{Habsburkové vládnou říši, které slunce nezapadá}
    \item[$(1525)$]\textsc{bitva u Pávie}, František I. proti Karlu V., vyhrává Karel, František byl zajat
\end{itemize}


\subsection*{Náboženské války (1562-1589)}
\begin{itemize}
    \vspace{-0.5em}
    \setlength\itemsep{0.15em}
    \item[$-$] mezi katolickou církví a protestanty = \textit{hugenoti}
    \item[$-$] vyvrcholí za tzv. tří Jindřichů, proot se jim říka války tří Jindřichů
    \item[1562] zmasakrování hugenotů ve Wassy, počátek válek
    \item[$-$] u moci je \textbf{Jindřich III.}, v čele katolíků \textbf{Jindřich de Guise}, šel Jindřichovi III. po krku, sám se chce stát králem,
    \item[$\Rightarrow$] 1588 Jindřichem III. zavražděn
    \item[$-$] vůdce hugenotů \textbf{Jindřich de Bourbon}, princ navarský
    \item[23. srepn 1572] busieness nápad: oženit Jindřicha Bourbonského s \textbf{Markétou z Valois} (princezna) = \textit{bartolomějská noc}, ale války dál pokračují
    \item[1589] zlomový rok, kdy byl král Jindřich III. zavražděn $\Rightarrow$ konec rogu Valois, nastupuje dynastie \textbf{Bourbon}
    \item[1593] konvertoval Jindřich Bourbonský ke katolicismu, 1594 korunován na Jindřicha IV.
\end{itemize}


\subsection*{Jindřich IV. de Bourbon (1594-1610)}
\begin{itemize}
    \vspace{-0.5em}
    \setlength\itemsep{0.15em}
    \item[1598] \textsc{edikt nantský} zrovnoprávňuje katolíky a hugenoty, mohou do státní správy
    \item[$-$] budování infrastruktury, kladení důrazu na venkov, proniknutí Francouzů do Kanady, dočasné snížení daní
    \item[$-$] i po konvertování ke katolicismu podporuje protestanty
    \item[1610] zavražděn katolickým mnichem
    \item[$-$] po Markétě (bezdětné manželství) další sňatek s \textbf{Marií Medicejskou}, syn \textbf{Ludvík XIII.}, ten je při jeho smrti ještě nezletilý, nějakou dobu tedy vládne Marie jako regentka
\end{itemize}


\subsection*{Ludvík XIII.}
\begin{itemize}
    \vspace{-0.5em}
    \setlength\itemsep{0.15em}
    \item[$-$] dobytí pevnosti La Rochelle, hugenoti tam byli vyhladověni
    \item[$-$] manželka Anna Rakouská
    \item[$-$] kardinál \textbf{Richelieu}, nejdříve na straně Marie, když se však ujal moci Ludvík, byl plně oddán jemu, působil jako první ministr, po přiklonění k Ludvíkovi se distancoval od Habsburků
    \item[1635] \textsc{třicetiletá válka} proti Habsburkům, do ní Francii zavedl právě Richelieu
\end{itemize}

\section*{Španělsko (15.-16. století)}

\begin{itemize}
    \vspace{-0.5em}
    \setlength\itemsep{0.15em}
    \item[$-$] u vzniku stojí Ferdinand Aragonský a Isabela Kastilská, personální unie, každý si vládne na svém písečku
    \item[1492] konec \textit{reconquisty}, působení inkvizice, plavba Kryštofa Kolumba
    \item[$-$] proti zakladatelům Španělska opozice velkých feudálů = \textit{grandové}, ti získali sílu z bojů s Araby
    \item[$-$] Ferdinand se tedy přiklonil na stranu měst, která se spojila do tzv. \textit{Sv. hermanandy}, s jeho pomocí města porazila šlechtice, \textit{grandy}
    \item[$-$] \textbf{Johana Šílená} (Jana Kastilská), dcera Ferdinanda a Isabely se provdala sa Filipa Habsburského
    \item[$-$] právě Filip Habsburský sňatkem přinesl Nizozemí
\end{itemize}

\subsection*{Karel I. Habsburský (1516-1556)}
\begin{itemize}
    \vspace{-0.5em}
    \setlength\itemsep{0.15em}
    \item[$-$] od roku 1516 králem Španělským, 1519 císař SŘŘ, za něj probíhá reformace
    \item[$-$] Augsburský mír, on sám rezignuje na obojí (jak krále Španělska tak císaře SŘŘ), protože neudržel katolické náboženství
    \item[$-$] opíral se o vysokou šlechtu $\Rightarrow$
    \item[1520] \textsc{povstání komunerů} = povstání měst v čele s Toledem, Karel tvrdě potlačil, města byla nejen poražena, ale i zatížena daněmi $\Rightarrow$ ekonomické zatížení, postupně se do čela Evropy začíná dostávat spíše Anglie a Francie
    \item[$-$] Španělsko má sice hodně bohatství z kolonií, ale celé je to díky nařízení daní promrháno
\end{itemize}


\subsection*{Filip II. Habsburský (1556-1598)}
\begin{itemize}
    \vspace{-0.5em}
    \setlength\itemsep{0.15em}
    \item[$-$] syn Karla I., manželka Marie Tudorovna
    \item[$-$] tvrdě prosazuje katolicismus, oporou jeho politiky je tedy církev a střední šlechta
    \item[$-$] hospodářská politika nemá žádnou koncepci $\Rightarrow$ státní bankrot, který se snaží vyřešit vysokými daněmi
    \item[1588] \textsc{porážka španělské Armady} (loďstvo) Alžbětou Tudorovnou, i když se prezentovala jako neporazitelná
    \item[$-$] výstavba královského paláce Escorialu
    \item[$-$] dočasně připojil Portugalsko
\end{itemize}


\section*{Nizozemská revoluce}
\begin{itemize}
    \vspace{-0.5em}
    \setlength\itemsep{0.15em}
    \item[$-$] nizozemská provincie: nejbohatší španělská provincie, tvořeno 17 provinciemi, téměř nezávislá země na Španělsku
    \item[$-$] provincie mají vlastní sněm, prodlouženou rukou Filipa II. byl \textit{generální místodržitel}, tehdy \textbf{Markéta Parmská}, jeho nevlastní dcera
    \item[$-$] pestré náboženské složení: katolíci, ale především stoupenci Martina Luthera = lutherání, stoupenci Jana Kalvína = kalvinisté, novokřtěnci
    \item[$-$] po státním bankrotu FIlip zvyšuje daně, ... kde jinde než (v ČEZ) v Nizozemí, snaží se tam prosadit absolutismus
    \item[$\Rightarrow 1566$] \textit{obrazoborecké hnutí} = hnutí zaměřené proti Filipu II., především nižší vrstvy, které ničily katolické kostely
    \item[$-$] šlechta se snaží spíš o diplomatické vyjednávání s Filipemjsi říkal že máš
    \item[$\Rightarrow$] záminka pro Filipa, aby do Nizozemí poslal armádu, v čele je \textbf{vévoda z Alby} (Fernand Álvarez de Toledo), který nastolil krutovládu, teror, zničil Nizozemskou opozici
    \item[$-$] někteří z Nizozemí utíkají, třeba \textbf{Vilém Oranžský}, který později stanul v čele povstání proti Španělům
    \item[$-$] \uv{řádění Španělů} vyvolalo celonárodní revoluci, v jejím čele je \textbf{Vilém Oranžský}, je to první buržoazní revoluce
    \item[$-$] \textit{gézové} = žebráci, i oni bojovali po boku Nizozemských stavů proti Španělům, dělí se na jižní (říkají si lesní) a severní (mořští)
    \item[1572] mořští gézové napadli jeden z přístavů ovládaných Španělskem, počátek revoluce
    \item[$-$] po odchodu vévody z Alby nastupuje \textbf{Juan d'Austria}
    \item[$-$] \textit{gentská pacifikace} = po vyplenění Antverp Španěli se stavové domluvili na společném postupu
    \item[1579] nizozemské provincie se rozdělují:
    \begin{itemize}
        \vspace{-0.5em}
        \setlength\itemsep{0.15em}
        \item[$-$] na jihu se utvoří tzv. \textbf{Arraská unie}, tyto provincie už nechtějí válku a chtějí se se Španěli domluvit o stažení vojsk a oni jim za odměnu zůstanou věrní, součástí Španělska
        \item[$-$] na severu se utvoří tzv. \textbf{Utrechtská unie}, tyto provincie chtějí bojovat do té doby, než Španěle nevyženou, což se nakonec podařilo
    \end{itemize}
    \item[1581] severní provincie vyhlašují \textit{Spojené nizozemské provincie}, tedy nezávislost na Španělsku, v jejím čele stanul \textbf{Vilém I. Oranžský}, moc dlouho nevydrží, po něm nastupuje jeho syn \textbf{Mořic Oranžský}
    \item[1609] příměří se Španěli, mají uznávát Nizozemsko, platí však jen do 1621 (30letá válka), museli se tedy přidat na stranu Habsburků, po jejím konci v roce 1648 (Vestfalský mír) bylo Nizozemsko uznáno \textit{de iure} uznáno za svobodný stát
    \item[$-$] jižní Nizozemí, které zůstalo věrné Španělsku, je budoucí Belgie
\end{itemize}

\section*{Tudorovci}

\begin{itemize}
    \vspace{-0.5em}
    \setlength\itemsep{0.15em}
    \item[$-$] Tudorovci nastupují \textbf{Jindřichem VII.}, který spojil rody Lancaster a York po válce růží, vymírají \textbf{Alžbětou I.}
\end{itemize}

\subsection*{Válka růží (1455-1485\footnote{zdroj: Šleza -- podle ní při zkoušení prý správně} či 1487\footnote{zdroj: Wikipedie} či 1488\footnote{zdroj: Šlezina prezentace}}
\begin{itemize}
    \vspace{-0.5em}
    \setlength\itemsep{0.15em}
    \item[$-$] po stoleté válce v Anglii válka růží (podle erbů)
    \item[1485] finální bitva \textsc{u Bosworthu}, Jindřich Tudor vs. Richard III., Richard zemře a Jindřich nastupuje do čela Anglie z rodu Lancasterů
    \item[$-$] po válce stabilizace státu, zakládání (hlavně textilních) manufaktur, námořního obchodu v důsledku \textit{ohrazování} = šlechta zabírá půdu drobným zemědělcům, na nich začíná chovat ovce, nově začínají vlnu zpracovávat a vyvážet sukno
    \item[$-$] vyhnaní lidé jsou pracovní síla, která je zaměstnána v manufakturách
    \item[$-$] východoindická a moskevská obchodní společnost, obchod s otroky z Afriky
\end{itemize}

\subsection*{Jindřich VII. Tudor (1485-1509)}
\begin{itemize}
    \vspace{-0.5em}
    \setlength\itemsep{0.15em}
    \item[$-$] manželka \textbf{Alžběta z Yorku}, aby obrousil neshody po válkách růží
    \item[$-$] většina šlechty pobita $\Rightarrow$ téměř žádná opozice, vznik nové šlechty
    \item[$-$] významnou oporou arcibiskup \textbf{John Morton}, též kancléřem
    \item[$-$] \textit{hvězdná komora} = soudní dvůr, pojmenován podle výzdoby stropu, zřízen pro souzení lidí, kteří byli v opozici vůči králi, potírání odboje
    \item[$-$]  \textit{yeomani} = svobodní vlastníci půdy s právem nosit zbraň, rekrutuje se z nich pěchota, mají ekonomický a vojenský význam
    \item[$-$] \textit{gentry} = nižší, venkovská šlechta a obchodníci
    \item[$-$] vyšší územní správní jednotka: \textit{hrabství}, základní správní jednotka: \textit{farnost}
    \item[$-$] zavedení pravidelných daní $\Rightarrow$ obnova státu
    \item[$-$] má po něm nastoupit jeho syn \textbf{Arthur}, ten ale zemřel
    \item[$-$] první suchý dok v Anglii: dok v Postmorthu
\end{itemize}

\subsection*{Jindřich VIII. Tudor (1509-1547)}
\begin{itemize}
    \vspace{-0.5em}
    \setlength\itemsep{0.15em}
    \item[$-$] renesanční vzdělanec: ovládá latinu, francouzštinu, španělštinu, činný v oblasti umění
    \item[$-$] položeny základy kapitalismu
    \item[$-$] kardinál, rádce a zároveň arcibiskup \textbf{Thomas Wolsey}, později rádce \textbf{Thomas Moore}
    \item[$-$] první manželka \textbf{Kateřina Aragonská}, dcera Marie, kterou zdědil po svém starším bratrovi Arthurovi, který se měl stát králem, ale nakonec nenastoupil, nedala mu syna, v tom to všechno vězí $\Rightarrow$ rozvod, papež Kliment VII. odmítá, Jindřich se obrátí na anglické duchovenstvo, odříká poslušnost papeži, arcibiskup Canterburský \textbf{Thomas Cranmer} prohlásil rozvod za platný $\Rightarrow$ 1533 odtrhnutí od Římského papeže, vytvoření anglikánského náboženství
    \item[$-$] druhá manželka \textbf{Anna Boleynová}, později popravena za cizoložství, dcera Alžběta
    \item[$-$] královské loďstvo, nové mince
    \item[$-$] další manželky J. Seymourová (syn, zemřela přirozenou smrtí), A. Klévská (manželství 6 měsíců, politický sňatek, získal Klévsko), K. Howardová (nevěrná), K. Parrová (přežila Jindřicha, vychovávala Alžbětu, do té se zamiloval její manžel)
\end{itemize}

\subsection*{Eduard VI. (1547-1553)}
\begin{itemize}
    \vspace{-0.5em}
    \setlength\itemsep{0.15em}
    \item[$-$] syn Jindřicha a jeho třetí manželky Jany Seymourové
    \item[$-$] když nsatoupil, byl mladý, dočasně vládla regentská rada
    \item[$-$] pokračoval v šlépějích svého otce, avšak doba byla nestabilní, jako mladý umírá na tuberkulozu
    \item[$-$] zanechává po sobě závěť, korunu přenechává \textbf{Janě Greyové} (praneteř), ta vládla devět dnů, než ji sesadila \textbf{Marie I. Tudorovna}, dcera Kateřiny Aragonské, jeho nevlastní sestra

\end{itemize}


\subsection*{Marie I. Tudorovna (1553-1558)}
\begin{itemize}
    \vspace{-0.5em}
    \setlength\itemsep{0.15em}
    \item[$-$] dcera Kateřiny Aragonské a Jindřicha
    \item[$-$] katolička, likvidovala anglikány, \textit{první želízko Říma na anglickém trůnu}
    \item[$-$] manžel Filip II., též propagaátor katolicismu
    \item[$-$] válka proti Francii na straně Španělska, Anglie ztrácí přístav Calais
\end{itemize}

\subsection*{Alžběta I. (1558-1603)}
\begin{itemize}
    \vspace{-0.5em}
    \setlength\itemsep{0.15em}
    \item[$-$] nikdy se neprovdala, neměla žádné děti, \textit{alžbětínská Anglie}
    \item[$-$] dcera Jindřicha VIII. a Jany Boleynové
    \item[$-$] pokračuje v politice svého otce, podporuje anglikánskou církev, ale je i nakloněna kompromisům, díky tomu období stability a prosperity
    \item[$-$] William Cecil -- její rádce, Robert Dudley -- její nejznámější milenec
    \item[$-$] podnikající šlechta je oporou královny
    \item[$-$] \textbf{Francis Drake} jako druhý obeplul svět na lodi Zlatá laň, byl to pirát, Alžběta jim dávala vojenský doprovod, protože přepadali španělské lodě, další mořeplavec Walter Raleigh
    \item[$-$] do Severní Ameriky se dostávají až po Alžbětině smrti
    \item[1588] \textsc{Filip zaútočil v La Manšském průlivu}, Španělé se chtějí vylodit na Britských ostrovech, Britům však velí piráti, mají lepší techniku, Španělé poraženi, počátek konce Španělů na moři, na jejich úkor mají dominantní pozici Angličané a Nizozemci
    \item[$-$] William Shakespeare (divadlo Globe), Christopher Marlowe, Erasmus Rotterdamský
\end{itemize}


\subsection*{Marie Stuartovna}
\begin{itemize}
    \vspace{-0.5em}
    \setlength\itemsep{0.15em}
    \item[$-$] v době Alžběty skotská královna, pravnučka Jindřicha VII.
    \item[$-$] manžel František II., poté lord Darnley (zavražděn), poté hrabě Bothwell
    \item[$-$] prosazovala katolicismus, ale většina Skotska jsou stoupenci Jana Kalvína
    \item[1568] donucena šlechtou k abdikaci ve prospěch svého syna, \textbf{Jakuba VI.}, Marie odchází do Londýna, kde byla de facto zajatcem Alžběty I.
    \item[$-$] ukázalo se, že připravovala spiknutí proti Alžbětě
    \item[1587] popravena
\end{itemize}

\begin{itemize}
    \vspace{-0.5em}
    \setlength\itemsep{0.15em}
    \item[1603] umírá Alžběta I., králem se stává \textbf{Jakub I. Stuartovec}, personální unie
    \item[1707] Spojené království Velké Británie, zrušení personální unie
  \end{itemize}

\section*{Obnova Ruska}

  \subsection*{Kyjevská Rus (882-1125)}
  \begin{itemize}
      \vspace{-0.5em}
      \setlength\itemsep{0.15em}
      \item[$-$] vládnou Rurikovci, na počátku 12. století se drobí na jednotlivá knížectví
      \item[$-$] konec těchto knížectví přivodili Mongolové = Tataři
      \item[1223] porážka knížat Mongoly \textsc{na řece Kalce}
      \item[1240] dobývají Kyjev, hlavní město Kyjevské Rusi
      \item[$-$] v Novgorodském knížectví vládne kníže Alexandr
      \item[1240] \textsc{Alexandr na řece Něvě poráží Švédy}
      \item[1242]  \textsc{Alexandr poráží na zamrzlém jezeře řád německých rytířů}
      \item[$-$] většinu knížectví ovládají Mongolové, vytváří svůj stát, tzv. \textit{Zlatou hordu}
      \item[$-$] nutí je, aby jim platili daně, Zlatá horda se postupně rozpadá na menší území
      \item[$-$] novým hlavním městem obnoveného slovanského státu se stává Moskva, chtějí se zbavit nadvlády Mongolů
  \end{itemize}


  \subsection*{Ivan I. Kalita}
  \begin{itemize}
      \vspace{-0.5em}
      \setlength\itemsep{0.15em}
      \item[$-$] snažil se zabránit, any Mongolové najížděli do Ruska
      \item[$-$] nabídl se, že bude vybírat daň pro Mongoly místo nich
  \end{itemize}


  \subsection*{Dmitrij Donský}
  \begin{itemize}
      \vspace{-0.5em}
      \setlength\itemsep{0.15em}
      \item[$-$] vnuk Ivana
      \item[1380] poprvé poráží Mongoly \textsc{na Kulikovském poli}, ovšem jen nakrátko
  \end{itemize}


  \subsection*{Ivan III. (1462-1505)}
  \begin{itemize}
      \vspace{-0.5em}
      \setlength\itemsep{0.15em}
      \item[$-$] vnuk Dmitrije
      \item[$-$] definitivně poráží Mongoly, nemusí jim platit daň, zakládá Moskevskou Rus, dobývají zpět své území
      \item[$-$] buduje Moskvu jako tzv. \textit{třetí Řím}, cítí se jako nový Konstantin Veliký, oporou je mu pravoslavné náboženství, titul \textit{velkovévoda}
      \item[$-$] budování administrativy, právní normy = \textit{suděbnik}, jednotné velené armády, národním jazykem ruština
      \item[$-$] budování sídla moskevských vládců = \textit{Kreml}
      \item[$-$] žena neteř posledního byzantského císaře Konstantina XI., šíření vlivu byzantské kultury
      \item[$-$] dvojí typy půdy: \textit{votčina} = dědičný majetek, dobrá půda, kterou získávají \textit{bojaři} = šlechta, členové \textit{bojarské dumy} = poradní sbor panovníka; \textit{poměstí} = nedědičná půda, za služby vladaři ji dostávali poddaní, jen pro sebe, nedědí se
  \end{itemize}

  \subsection*{Ivan IV. Hrozný (1533-1584)}
  \begin{itemize}
      \vspace{-0.5em}
      \setlength\itemsep{0.15em}
      \item[1547] prohlásil se za prvního ruského cara, chce asertovat dominanci, \textit{samoděržaví} = absolutismus
      \item[$-$] řada reforem, dobře si vede ve válkách se zbytky Mongolských chanátů, odhání je od řeky Volhy a dostává se tak ke Kaspickému moři
      \item[$-$] založení (zamrzajícího) přístavu Archangelsk, obchod s Anglií (současnice královna Alžběta), Nizozemím
      \item[$-$] chrám Vasila Blaženého v centru Moskvy
      \item[$-$] nejbohatší šlechtický rod: Stroganovovi, sponzorují války v oblasti Sibiře
      \item[$-$] \textsc{Livonská válka}, chce přístup k Baltskému moři (dnešní Estonsko a Lotyšsko), byl však neúspěšný, neměl podporu bojarů
      \item[$-$] Ivan za to bojary potrestal reformou půdy = \textit{opričnina}, dobrou dědičnou půdu bojarům konfiskuje a rozdělil ji mezi jemu oddanou šlechtu = \textit{opričnici}, horší půdu dal bojarům = \textit{zemština}, reforma krvavá, musel od toho odstoupit
  \end{itemize}


  \subsection*{Smuta (1584-1613)}
  \begin{itemize}
      \vspace{-0.5em}
      \setlength\itemsep{0.15em}
      \item[$-$] po Ivanově smrti zůstává syn Fjodor, ten je slabomyslný, má ještě bratra Dimitrije, kterého však zavraždil, místo něj tedy vládne Fjodorův švagr \textbf{Boris Godunov}, stabilizuje Rusko, nakonec záhadně umírá, carem se prohlašuje \textbf{Lžidimitrij I.}, který se vydává za zavražděného Dimitrije, později předák bojarů \textbf{Vasilij Šuskyj}, poté další \textbf{Lžidimitrij II.}, poté Moskvu dobývají Poláci a dosazují \textbf{Vladislava Polského}
      \item[=] nestabilní období střídání panovníků, které konsoliduje až nový panovnický rod \textbf{Romanovci}, nastupuje Michail Fjodorovič Romanov, zakládá moderní ruský stát
  \end{itemize}





\end{document}
