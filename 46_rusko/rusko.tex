\documentclass{article}
\usepackage{fullpage}
\usepackage[czech]{babel}
\usepackage{amsfonts}

\title{\vspace{-2cm}Rusko\vspace{-1.7cm}}
\date{}
\author{}

\begin{document}
\maketitle

\begin{itemize}
    \vspace{-0.5em}
    \setlength\itemsep{0.15em}
    \item[$-$] vládnou zde Romanovci, ALexandr I. (vnuk Kateřiny II.)
    \item[$-$] zisk Finska po Vídeňském kongresu
    \item[$-$] při nástupu Mikuláše proti němu mladí důstojníci zorganizovali \textit{povstání Děkabristů}, neúspěšné
    \item[$-$] Rusko je v této době agrární zemí, vyváží obilí, velké množství nevzdělaných nevolníků
    \item[$-$] samoděržaví, pravoslavné náboženství, většinu majetků drží venkovská šlechta
    \item[$-$] Mikuláš I. se snaží podporovat vysokoškolské vzdělání, dosahuje na něj však jenom vysoká šlechta, většina lidí je však pořád nevzdělaná
    \item[$-$] kontrola názorů poddaných, k čemuž sloužila tajná policie $\rightarrow$ jen málo lidí v opozici, mezi nimi byli třeba \textit{slavjanofilové} (kritizovali absolutismus, Rusko se má vrátit ke svým slovanským kořenům a hodnotám, mělo by jít cestou něco mezi absolutismem a liberalismem), \textit{západníci} (vidí vzor na západě, kritizují zaostalost Ruska, jednoznačně preferují liberalismus a demokracii, nejradikálnější ze západníků se potom dostali i do emigrace)


\end{itemize}

\subsection*{Zahraniční politika}
\begin{itemize}
    \vspace{-0.5em}
    \setlength\itemsep{0.15em}
    \item[$-$] potlačil povstání v Polsku, po porážce nastává rusifikace Polska a emigrace Poláků
    \item[1849] pomáhá Habsburkům porazit Uhry u \textsc{Világoše}
    \item[$-$] východní otázka: Rusové chtějí ovládnout přes Černé moře úžiny Bospor a Dardanely a dostat ke Středozemnímu moři, jenže tam jsou Osmané
    \item[$-$] Rusové vědí, že Osmané jsou pro ně samotné moc velký oříšek, proto navrhli Britům, jestli se k nim nepřipojí, nechtějí však
    \item[$-$] proto Rusové začají podnikat sami, obsazují Moldavské a Valašské knížectví

\end{itemize}


\end{document}
