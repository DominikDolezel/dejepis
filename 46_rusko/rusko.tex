\documentclass{article}
\usepackage{fullpage}
\usepackage[czech]{babel}
\usepackage{amsfonts}

\title{\vspace{-2cm}Rusko\vspace{-1.7cm}}
\date{}
\author{}

\begin{document}
\maketitle

\begin{itemize}
    \vspace{-0.5em}
    \setlength\itemsep{0.15em}
    \item[$-$] vládnou zde Romanovci, ALexandr I. (vnuk Kateřiny II.)
    \item[$-$] zisk Finska po Vídeňském kongresu
    \item[$-$] při nástupu Mikuláše proti němu mladí důstojníci zorganizovali \textit{povstání Děkabristů}, neúspěšné
    \item[$-$] Rusko je v této době agrární zemí, vyváží obilí, velké množství nevzdělaných nevolníků
    \item[$-$] samoděržaví, pravoslavné náboženství, většinu majetků drží venkovská šlechta
    \item[$-$] Mikuláš I. se snaží podporovat vysokoškolské vzdělání, dosahuje na něj však jenom vysoká šlechta, většina lidí je však pořád nevzdělaná
    \item[$-$] kontrola názorů poddaných, k čemuž sloužila tajná policie $\rightarrow$ jen málo lidí v opozici, mezi nimi byli třeba \textit{slavjanofilové} (kritizovali absolutismus, Rusko se má vrátit ke svým slovanským kořenům a hodnotám, mělo by jít cestou něco mezi absolutismem a liberalismem), \textit{západníci} (vidí vzor na západě, kritizují zaostalost Ruska, jednoznačně preferují liberalismus a demokracii, nejradikálnější ze západníků se potom dostali i do emigrace)


\end{itemize}

\subsection*{Zahraniční politika}
\begin{itemize}
    \vspace{-0.5em}
    \setlength\itemsep{0.15em}
    \item[$-$] potlačil povstání v Polsku, po porážce nastává rusifikace Polska a emigrace Poláků
    \item[1849] pomáhá Habsburkům porazit Uhry u \textsc{Világoše}
    \item[$-$] východní otázka: Rusové chtějí ovládnout přes Černé moře úžiny Bospor a Dardanely a dostat ke Středozemnímu moři, jenže tam jsou Osmané
    \item[$-$] Rusové vědí, že Osmané jsou pro ně samotné moc velký oříšek, proto navrhli Britům, jestli se k nim nepřipojí, nechtějí však
    \item[1853] proto Rusové začají podnikat sami, obsazují Moldavské a Valašské knížectví
    \item[$-$] ruský admirál Nachimov zničil téměř celou tureckou flotilu, jsou ohroženy zájmy Britů -- třeba dovoz obilí
    \item[$-$] do války se na stranu Osmanů přidávají Britové, Francie, Sardinské království
    \item[$-$] drtivá část bojů se odehrála na Krymu
    \item[1856] konec války, mír není pro Rusy pozitivní, protože ztrácí část Besarabie a kontrolu nad Dunajskou plavbou, Bospor a Dardanely byly prohlášeny za neutrální
    \item[$-$] car Mikuláš I. umírá
    \item[$-$] ruská armáda je neschopná, mají zaostalé zbraně, špatné velení, zprávy se přenášejí velmi pomalu, vázne zásobování
\end{itemize}


\subsection*{Alexandr II. (1855-1881)}
\begin{itemize}
    \vspace{-0.5em}
    \setlength\itemsep{0.15em}
    \item[$-$] nový car, pochopil, že tudy cesta nevede
    \item[1861] ruší nevolnictví, mohou se vykoupit z pozice nevolníků nebo si to odpracovat
    \item[$-$] různé reformy školství, soudnictví, správy, armády
    \item[$-$] \textsc{povstání Poláků}, které Rusové opět potlačili
    \item[$-$] vznik nové opozie, říkají si \textit{národníci}, chtějí vyprovokovat masové povstání, které smete cara, hlavním nositelem revoluce má být venkovský lid (národ), mezi nimi i teroristické organisace (Narodnaja volja)
    \item[1881] atentáty na Alexandra, celkem 7, zasáhl ho až ten poslední, kdy umírá v kočáře, na který dopadla bomba
    \item[1867] prodej Aljašky Spojeným státům americkým
\end{itemize}

\subsection*{Alexandr III. (1881-1894)}
\begin{itemize}
    \vspace{-0.5em}
    \setlength\itemsep{0.15em}
    \item[$-$] protože byl Alexandr zavražděn v důsledku atentátu, utužují nevolnictví
    \item[$-$] jinak vede mírovou politiku
\end{itemize}

\subsection*{Mikuláš II. }
\begin{itemize}
    \vspace{-0.5em}
    \setlength\itemsep{0.15em}
    \item[$-$] Rusko expanduje: upevnění pozice v Kavkazu, další zájem o Afghanistán a o Persii (dnešní Írán), jenže tato území chtějí i Britové
    \item[$-$] Britové vyhlašují protektorát v Afghanistánů, nezávislá zůstává část Persie, jednu třetinu získávají Britové a tu druhou Rusové
    \item[$-$] za francouzské peníze probíhá industrializace Ruska (investoři)
    \item[$-$] neúspěšný zájem o Čínu
    \item[1904-1905] \textsc{rusko-japonská válka}, Rusové poraženi u Mukdenu, Cušimě a ztrácí přístav Port Arthur
\end{itemize}

\subsubsection*{Domácí politika}
\begin{itemize}
    \vspace{-0.5em}
    \setlength\itemsep{0.15em}
    \item[$-$] zprůmyslnění Ruska pomáhá zahraniční kapitál, především Moskva a Petrohrad, jinak pořád vyváží zemědělské produkty
    \item[$-$] politické strany v znikají v souvislosti s tím, že roste počet dělníků, sociální demokraté:
    \begin{itemize}
        \vspace{-0.5em}
        \setlength\itemsep{0.15em}
        \item[$-$] \textit{bolševici}, radikálové vedení Leninem, preferují proletářskou revoluci
        \item[$-$] \textit{menševici} standardní západní pojetí marxismu-leninismu
    \end{itemize}
    \item[$-$] dále kadeti a eseři
    \item[22.1.1905] \textsc{krvavá neděle} G. Gapon v Zimnín paláci, strážci začínají střílet do davu, to vyprovokovalo řadu dalších nepokojů a stávek
    \item[$-$] \textit{Říjnovým manifestem} je Rusům poprvé zajištěno volební právo a zřízen zástupní sbor = \textit{Duma}, ekonomické reformy umožňující podnikání
    \item[$-$] po zklidnění situace další utužování absolutismu
    \item[1917] \textsc{únorová revoluce} sesazen, exportován na Sibiř do Tobolska, jakmile se bolševici dostali k moci, převezli celou jeho ordinu do Jekatěrinburgu, kde je bolševici do jednoho vyvraždili
\end{itemize}



\end{document}
