\documentclass{article}
\usepackage{fullpage}
\usepackage[czech]{babel}
\usepackage{amsfonts}

\title{\vspace{-2cm}Francie za Ludvíka XIV.\vspace{-1.7cm}}
\date{}
\author{}

\begin{document}
\maketitle

\begin{itemize}
    \vspace{-0.5em}
    \setlength\itemsep{0.15em}
    \item[1594] Jindřichem IV. nastupuje dynastie Bourbonů, vymírá dynastie z Valois
    \item[$-$] Ludvík též patří k této dynastii
    \item[$-$] vrchol absolutismus ve Francii, \uv{stát jsem já}
    \item[$-$] znám také jako \uv{král slunce}, podporoval významné umělce, v mládí provozoval balet
    \item[$-$] \textbf{Ludvík XIII.}, syn Jindřicha IV. a otec Ludvíka XIV., měl manželku \textbf{Annu Rakouskou}, která byla regentkou, než Ludvík dospěl po otcově smrti
    \item[$-$] první ministr kardinál \textbf{Jules Mazarin}, fakticky Ludvík začíná vládnout až po Mazarinově smrti
    \item[$-$] Mazarin se snaží protlačovat absolutismus, problémy se stavovksým odbojem, proti němu se bouří šlechta a duchovenstvo = \textit{fronda princů a parlamentu}, stavovský odboj nakonec zlomen
    \item[$-$] žena \textbf{Marie Tereza}, děti nejen s ní ale i se svými milenkami
    \item[$-$] tehdy měla v Evropě největší a nejmodernější armádu, nejlidnatější stát v evropě
    \item[$-$] snaží se oslabit svoji konkurenci (vysokou šlechtu), zbavil je moci politické a připoutal si je ke svému dvoru $\rightarrow$ Versailles
    \item[$-$] \textit{privilegované stavy} = osvobozeni od některých povinností, především nemusí platit daně (šlechta, církev)
    \item[$-$] policejní síť, centralizace, \textit{intendanti} = královští úředníci, kteří mají na vše dohlížet v jednotlivých částech Francie (placení daní, soudní kompetence)
    \item[$-$] růst daní, odstraňování cel, budování infrastruktury
    \item[$-$] na konci jeho vlády ve Francii hodně žebráků
\end{itemize}

\section*{Zahraniční politika}
\begin{itemize}
    \vspace{-0.5em}
    \setlength\itemsep{0.15em}
    \item[$-$] Habsburkové zaneprázdněni Turky $\Rightarrow$ francie si může řídit v Evropě a zabírat si územíčka
    \item[$-$] kolonie: Kanada, Louisiana, Východoindická a Západoindická společnost
    \item[$-$] \textit{požadavek přirozených hranic}: podél oceánů, moře, Pyrenejí, řeky Rýn, Alpy $\Rightarrow$ začal válčit a zabírat některá území
    \item[1701-1714] \textsc{válka o dědictví španělské}: ve Španělsku vymřela dynastie Habsburků, o dědictví se uchází Habsburkové -- Leopold I. chce prosadit svého syna Karla a spojenci (Anglie, Nizozemí, ...) proti Ludvíkovi, který chtěl prosadit svého vnuka Filipa z Anjou
    \item[1713] \textsc{mír v Utrechtu} mezi Francií a Anglií
    \item[1714] \textsc{mír v Rastattu} mezi Francií a SŘŘ
    \item[$-$] z válek vítězně vychází Anglie, Ludvík ale dosadil Filipa na španělský trůn pod podmínkou, že se Španělsko nespojí s Francií
    \item[$-$] Rakouští Habsburkové získávají španělské Nizozemí (budoucí Belgii) a území na Apeninském poloostrově, která však později ztratí
    \item[$-$] Britové získávají Newfoundland, Hudsonův záliv, Gibraltar, Menorcu a monopol na obchod s africkými otroky (dováželi otroky do Jižní Ameriky)
\end{itemize}


\section*{Náboženská politika}
\begin{itemize}
    \vspace{-0.5em}
    \setlength\itemsep{0.15em}
    \item[$-$] prosazování katolicismu
    \item[1685] zrušení nantského ediktu $\Rightarrow$ odchod hugenotů, ekonomicky nevýhodné
    \item[$-$] \textit{dragonády} = proces převracování hugenotů na katolíky, dragouni (vojáci) jsou povinně ubytování v hugenotských domácnostech
\end{itemize}

\section*{Kulturní politika}
\begin{itemize}
    \vspace{-0.5em}
    \setlength\itemsep{0.15em}
    \item[$-$] francouzská móda, popularita francouzštiny, \uv{zlatý věk Francie}
    \item[$-$] Versailles
    \item[$-$] ministr financí \textbf{Jean-Baptiste Colbert} zavádí politiku \textit{merkantilismu} = více vyvážíme než dovážíme, navíc podporoval rozvoj manufaktur, orientoval se na luxusní zboží (parfémy, krajky, drahé látky), dovoz jen z kolonií nebo pod vysokými cly = \textit{protekcionismus}
\end{itemize}

\begin{itemize}
    \vspace{-0.5em}
    \setlength\itemsep{0.15em}
    \item[$-$] po něm nastupuje Ludvík XV. ve dvou letech
\end{itemize}

\end{document}
