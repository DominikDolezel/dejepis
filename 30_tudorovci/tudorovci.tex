\documentclass{article}
\usepackage{fullpage}
\usepackage[czech]{babel}
\usepackage{amsfonts}

\title{\vspace{-2cm}Tudorovci (1485-1603)\vspace{-1.7cm}}
\date{}
\author{}

\begin{document}
\maketitle

\begin{itemize}
    \vspace{-0.5em}
    \setlength\itemsep{0.15em}
    \item[$-$] Tudorovci nastupují \textbf{Jindřichem VII.}, který spojil rody Lancaster a York po válce růží, vymírají \textbf{Alžbětou I.}
\end{itemize}

\section*{Válka růží (1455-1485\footnote{zdroj: Šleza} či 1487\footnote{zdroj: Wikipedie} či 1488\footnote{zdroj: Šlezina prezentace}}
\begin{itemize}
    \vspace{-0.5em}
    \setlength\itemsep{0.15em}
    \item[$-$] po stoleté válce v Anglii válka růží (podle erbů)
    \item[1485] finální bitva \textsc{u Bosworthu}, Jindřich Tudor vs. Richard III., Richard zemře a Jindřich nastupuje do čela Anglie z rodu Lancasterů
    \item[$-$] po válce stabilizace státu, zakládání (hlavně textilních) manufaktur, námořního obchodu v důsledku \textit{ohrazování} = šlechta zabírá půdu drobným zemědělcům, na nich začíná chovat ovce, nově začínají vlnu zpracovávat a vyvážet sukno
    \item[$-$] vyhnaní lidé jsou pracovní síla, která je zaměstnána v manufakturách
    \item[$-$] východoindická a moskevská obchodní společnost, obchod s otroky z Afriky
\end{itemize}

\section*{Jindřich VII. Tudor (1485-1509)}
\begin{itemize}
    \vspace{-0.5em}
    \setlength\itemsep{0.15em}
    \item[$-$] manželka \textbf{Alžběta z Yorku}, aby obrousil neshody po válkách růží
    \item[$-$] většina šlechty pobita $\Rightarrow$ téměř žádná opozice, vznik nové šlechty
    \item[$-$] významnou oporou arcibiskup \textbf{John Morton}, též kancléřem
    \item[$-$] \textit{hvězdná komora} = soudní dvůr, pojmenován podle výzdoby stropu, zřízen pro souzení lidí, kteří byli v opozici vůči králi, potírání odboje
    \item[$-$]  \textit{yeomani} = svobodní vlastníci půdy s právem nosit zbraň, rekrutuje se z nich pěchota, mají ekonomický a vojenský význam
    \item[$-$] \textit{gentry} = nižší, venkovská šlechta a obchodníci
    \item[$-$] vyšší územní správní jednotka: \textit{hrabství}, základní správní jednotka: \textit{farnost}
    \item[$-$] zavedení pravidelných daní $\Rightarrow$ obnova státu
    \item[$-$] má po něm nastoupit jeho syn \textbf{Arthur}, ten ale zemřel
    \item[$-$] první suchý dok v Anglii: dok v Postmorthu
\end{itemize}

\section*{Jindřich VIII. Tudor (1509-1547)}
\begin{itemize}
    \vspace{-0.5em}
    \setlength\itemsep{0.15em}
    \item[$-$] renesanční vzdělanec: ovládá latinu, francouzštinu, španělštinu, činný v oblasti umění
    \item[$-$] položeny základy kapitalismu
    \item[$-$] kardinál, rádce a zároveň arcibiskup \textbf{Thomas Wolsey}, později rádce \textbf{Thomas Moore}
    \item[$-$] první manželka \textbf{Kateřina Aragonská}, dcera Marie, kterou zdědil po svém starším bratrovi Arthurovi, který se měl stát králem, ale nakonec nenastoupil, nedala mu syna, v tom to všechno vězí $\Rightarrow$ rozvod, papež Kliment VII. odmítá, Jindřich se obrátí na anglické duchovenstvo, odříká poslušnost papeži, arcibiskup Canterburský \textbf{Thomas Cranmer} prohlásil rozvod za platný $\Rightarrow$ 1533 odtrhnutí od Římského papeže, vytvoření anglikánského náboženství
    \item[$-$] druhá manželka \textbf{Anna Boleynová}, později popravena za cizoložství, dcera Alžběta
    \item[$-$] královské loďstvo, nové mince
    \item[$-$] další manželky J. Seymourová (syn, zemřela přirozenou smrtí), A. Klévská (manželství 6 měsíců, politický sňatek, získal Klévsko), K. Howardová (nevěrná), K. Parrová (přežila Jindřicha, vychovávala Alžbětu, do té se zamiloval její manžel)
\end{itemize}

\section*{Eduard VI. (1547-1553)}
\begin{itemize}
    \vspace{-0.5em}
    \setlength\itemsep{0.15em}
    \item[$-$] syn Jindřicha a jeho třetí manželky Jany Seymourové
    \item[$-$] když nsatoupil, byl mladý, dočasně vládla regentská rada
    \item[$-$] pokračoval v šlépějích svého otce, avšak doba byla nestabilní, jako mladý umírá na tuberkulozu
    \item[$-$] zanechává po sobě závěť, korunu přenechává \textbf{Janě Greyové} (praneteř), ta vládla devět dnů, než ji sesadila \textbf{Marie I. Tudorovna}, dcera Kateřiny Aragonské, jeho nevlastní sestra

\end{itemize}


\section*{Marie I. Tudorovna (1553-1558)}
\begin{itemize}
    \vspace{-0.5em}
    \setlength\itemsep{0.15em}
    \item[$-$] dcera Kateřiny Aragonské a Jindřicha
    \item[$-$] katolička, likvidovala anglikány, \textit{první želízko Říma na anglickém trůnu}
    \item[$-$] manžel Filip II., též propagaátor katolicismu
    \item[$-$] válka proti Francii na straně Španělska, Anglie ztrácí přístav Calais
\end{itemize}

\section*{Alžběta I. (1558-1603)}
\begin{itemize}
    \vspace{-0.5em}
    \setlength\itemsep{0.15em}
    \item[$-$] nikdy se neprovdala, neměla žádné děti, \textit{alžbětínská Anglie}
    \item[$-$] dcera Jindřicha VIII. a Jany Boleynové
    \item[$-$] pokračuje v politice svého otce, podporuje anglikánskou církev, ale je i nakloněna kompromisům, díky tomu období stability a prosperity
    \item[$-$] William Cecil -- její rádce, Robert Dudley -- její nejznámější milenec
    \item[$-$] podnikající šlechta je oporou královny
    \item[$-$] \textbf{Francis Drake} jako druhý obeplul svět na lodi Zlatá laň, byl to pirát, Alžběta jim dávala vojenský doprovod, protože přepadali španělské lodě, další mořeplavec Walter Raleigh
    \item[$-$] do Severní Ameriky se dostávají až po Alžbětině smrti
    \item[1588] \textsc{Filip zaútočil v La Manšském průlivu}, Španělé se chtějí vylodit na Britských ostrovech, Britům však velí piráti, mají lepší techniku, Španělé poraženi, počátek konce Španělů na moři, na jejich úkor mají dominantní pozici Angličané a Nizozemci
    \item[$-$] William Shakespeare (divadlo Globe), Christopher Marlowe, Erasmus Rotterdamský
\end{itemize}


\section*{Marie Stuartovna}
\begin{itemize}
    \vspace{-0.5em}
    \setlength\itemsep{0.15em}
    \item[$-$] v době Alžběty skotská královna, pravnučka Jindřicha VII.
    \item[$-$] manžel František II., poté lord Darnley (zavražděn), poté hrabě Bothwell
    \item[$-$] prosazovala katolicismus, ale většina Skotska jsou stoupenci Jana Kalvína
    \item[1568] donucena šlechtou k abdikaci ve prospěch svého syna, \textbf{Jakuba VI.}, Marie odchází do Londýna, kde byla de facto zajatcem Alžběty I.
    \item[$-$] ukázalo se, že připravovala spiknutí proti Alžbětě
    \item[1587] popravena
\end{itemize}

\begin{itemize}
    \vspace{-0.5em}
    \setlength\itemsep{0.15em}
    \item[1603] umírá Alžběta I., králem se stává \textbf{Jakub I. Stuartovec}, personální unie
    \item[1707] Spojené království Velké Británie, zrušení personální unie
  \end{itemize}

\end{document}
