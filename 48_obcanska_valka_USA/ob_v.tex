\documentclass{article}
\usepackage{fullpage}
\usepackage[czech]{babel}
\usepackage{amsfonts}

\title{\vspace{-2cm}Občanská válka v USA\vspace{-1.7cm}}
\date{}
\author{}

\begin{document}
\maketitle
\begin{itemize}
    \vspace{-0.5em}
    \setlength\itemsep{0.15em}
    \item[4.7.1776] 13 anglických osad: Vyhlášení nezávislosti
    \item[1783] získání nezávislosti, západní hranice Mississippi
    \item[1787] ústava
    \item[1800] Washington
    \item[1823] Monroeova doktrína, James Monroe jeden z prvních presidentů, snaží se Evropanům zabránit intervenci v Americe
\end{itemize}

\subsection*{Územní růst}
\begin{itemize}
    \vspace{-0.5em}
    \setlength\itemsep{0.15em}
    \item[$-$] během půl století dosáhly Spojené státy na severu k Velikým jezerům, na jihu k Mexiku
    \item[pol. 19. st.] \textsc{válka s Mexikem}, Američané získávají značná území
    \item[$-$] Alexandr II. prodává Aljašku
    \item[$-$] postupně západní hranice až u Tichého oceánu
    \item[$-$] když se rozšiřuje území, startuje boj o půdu, střety s indiány, vyvražďování
    \item[$-$] v polovině 19. století žije v USA asi 31 milionů obyvatel
    \item[$-$] dvě politické strany:
    \begin{itemize}
        \vspace{-0.5em}
        \setlength\itemsep{0.15em}
        \item[(1828)] \textit{demokratická}, jih, farmáři, připouští otroky
        \item[(1854)] \textit{republikánská}, sever, průmyslníci, nechce otroky
    \end{itemize}
    \item[$-$] \textit{abolicionismus} = hnutí za zrušení otroctví
    \item[$-$] John Brown: vůdce, který si myslel, že když získá zbraně, vyprovovukuje povstání, neúspěšné
\end{itemize}

\subsection*{Ekonomické rozdíly}
\begin{itemize}
    \vspace{-0.5em}
    \setlength\itemsep{0.15em}
    \item[$-$] severovýchod (Nová Anglie): koncentrace průmyslu, dovoz bavlny z jihu
    \item[$-$] středozápad: obilnice Spojených států, Chicago, problém: kde sehnat sezónní pracovníky
    \item[$-$] západ: farmy, zemědělství
    \item[$-$] jih: \uv{království bavlny}, 4 miliony otroků
    \item[$-$] otázka vlivu v nových státech
    \item[1860] prezidentem Abraham Lincoln, republikán, tvrdě proti otrokářství
    \item[1861] na to reaguje 11 jižanských států, které se odtrhnou, zakládají vlastní útvar: Konfederaci s hlavním městem (Richmond) a presidentem (Jeffersen Davis)
    \item[12./14.4.1861] napadení Unijní pevnosti Fort Sumter u Charlestonu: začíná válka mezi severem a jihem, využívá se vymožeností, jež přinesla industriální revoluce
    \item[$-$] sever chce blokovat území jihu
    \item[$-$] sever je ekonomicky silnější, je jich víc, ale mají málo generálů, počáteční neúspěchy
    \item[$-$] jih nadšeně bojuje, ze začátku jsou úspěšní, jako první zavedli povinné odvody, vojsko vede Edvard Lee
    \item[1861] vítězství jižanů u \textsc{Bull Run}
    \item[1862] seveřané vydávají \textit{Emancipační proklamaci}, která se stává účinnou až od dalšího roku, ruší otroctví v Unii, černošské obyvatelstvo na oplátku začalo vytvářet armádní oddíly proti jižanům
    \item[$-$] sever vydává \textit{Zákon o bezplatných přídělech půdy}: každý Američan mohl za pakatel získat půdu, po pěti letech se stal jejím vlastníkem
    \item[$-$] díky těmto krokům se seveřanům podařilo získat převahu
    \item[červenec 1863] \textsc{bitva u Gettysburgu}, seveřané defnitivně vyhrávají
    \item[1864] Lincoln podruhé zvolen presidentem
    \item[9.4.1865] Konfederace kapituluje, generál Lee podepsal kapitulační listinu, dobyt Richmond
    \item[14.4.1865] fanatický stoupenec jižanů spáchá atentát ve Fordově divadle na Lincolna, který je zastřelen
    \item[$-$] válka si vyžádala celkem 600000 padlých
    \item[$-$] nastává obodbí rekonstrukce 1865-1867
    \item[$-$] Ku-Klux-Klan: urtranacionalisti, kteří se snažili zastrašovat a likvidovat černošské obyvatelstvo, \uv{bílé kápě}, za zakladatele se považuje bývalý voják Konfederace, hnutí existovalo i přes zákaz
\end{itemize}

\subsection*{Období rekonstrukce 1865-1877}
\begin{itemize}
    \vspace{-0.5em}
    \setlength\itemsep{0.15em}
    \item[$-$] levná pracovní síla bývalých otroků, černošské obyvatelstvo získává občanská práva
    \item[$-$] v praxi pořád zůstává rasismus, v dnešní době je opět na vzestupu, proti němu bojuje \textbf{Martin Luther King}
    \item[1866] Francie jako dar posílá Sochu svobody
    \item[$-$] po válce není většina plantážníků půdu obdělávat, a tak ji pronajímají bývalým otrokům ži bělochům, ti za část úrody, kterou odevzdávali majitelům, plantáže obstarávali
    \item[$-$] průmyslové výrobky si ceny udrželi $\rightarrow$ ceny obilí klesaly, plantážnící nemohou splácet, zadlužují se a musí prodávat pozemky $\rightarrow$ vznik monopolů
    \item[$-$] USA se z krize dostaly a zahájily ekonomický vzestup, na začátku 20. století dohnávají Velkou Británii a Německo
    \item[$-$] obrovské množství cizinců, příliv obyvatelstva

\end{itemize}


\subsection*{Zahraniční politika}
\begin{itemize}
    \vspace{-0.5em}
    \setlength\itemsep{0.15em}
    \item[$-$] anekce Savojských ostrovů, Havajských ostrovů
    \item[1898] \textsc{španělsko-americká válka}, získávají značná území
    \item[$-$] expandují obchodně do Číny
    \item[$-$] výstavba Panamského průplavu převzána po Francouzích, jeoh existence umožnila intensivnější obchod mezi USA a zeměmi na Dálném východě
\end{itemize}


\end{document}
