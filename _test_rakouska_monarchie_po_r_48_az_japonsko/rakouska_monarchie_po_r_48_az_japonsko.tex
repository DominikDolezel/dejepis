\documentclass{article}
\usepackage{fullpage}
\usepackage[czech]{babel}
\usepackage{amsfonts}

\title{\vspace{-2cm}Občanská válka ve Spojených státech amerických, Rakouská monarchie po roce 1848, České země, konec 19. století, kolonie \vspace{-1.7cm}}
\date{}
\author{}

\begin{document}
\maketitle

\section*{Občanská válka v USA}
\begin{itemize}
    \vspace{-0.5em}
    \setlength\itemsep{0.15em}
    \item[4.7.1776] 13 anglických osad: Vyhlášení nezávislosti
    \item[1783] získání nezávislosti, západní hranice Mississippi
    \item[1787] ústava
    \item[1800] Washington
    \item[1823] Monroeova doktrína, James Monroe jeden z prvních presidentů, snaží se Evropanům zabránit intervenci v Americe
\end{itemize}

\subsection*{Územní růst}
\begin{itemize}
    \vspace{-0.5em}
    \setlength\itemsep{0.15em}
    \item[$-$] během půl století dosáhly Spojené státy na severu k Velikým jezerům, na jihu k Mexiku
    \item[pol. 19. st.] \textsc{válka s Mexikem}, Američané získávají značná území
    \item[$-$] Alexandr II. prodává Aljašku
    \item[$-$] postupně západní hranice až u Tichého oceánu
    \item[$-$] když se rozšiřuje území, startuje boj o půdu, střety s indiány, vyvražďování
    \item[$-$] v polovině 19. století žije v USA asi 31 milionů obyvatel
    \item[$-$] dvě politické strany:
    \begin{itemize}
        \vspace{-0.5em}
        \setlength\itemsep{0.15em}
        \item[(1828)] \textit{demokratická}, jih, farmáři, připouští otroky
        \item[(1854)] \textit{republikánská}, sever, průmyslníci, nechce otroky
    \end{itemize}
    \item[$-$] \textit{abolicionismus} = hnutí za zrušení otroctví
    \item[$-$] John Brown: vůdce, který si myslel, že když získá zbraně, vyprovovukuje povstání, neúspěšné
\end{itemize}

\subsection*{Ekonomické rozdíly}
\begin{itemize}
    \vspace{-0.5em}
    \setlength\itemsep{0.15em}
    \item[$-$] severovýchod (Nová Anglie): koncentrace průmyslu, dovoz bavlny z jihu
    \item[$-$] středozápad: obilnice Spojených států, Chicago, problém: kde sehnat sezónní pracovníky
    \item[$-$] západ: farmy, zemědělství
    \item[$-$] jih: \uv{království bavlny}, 4 miliony otroků
    \item[$-$] otázka vlivu v nových státech
    \item[1860] prezidentem Abraham Lincoln, republikán, tvrdě proti otrokářství
    \item[1861] na to reaguje 11 jižanských států, které se odtrhnou, zakládají vlastní útvar: Konfederaci s hlavním městem (Richmond) a presidentem (Jeffersen Davis)
    \item[12./14.4.1861] napadení Unijní pevnosti Fort Sumter u Charlestonu: začíná válka mezi severem a jihem, využívá se vymožeností, jež přinesla industriální revoluce
    \item[$-$] sever chce blokovat území jihu
    \item[$-$] sever je ekonomicky silnější, je jich víc, ale mají málo generálů, počáteční neúspěchy
    \item[$-$] jih nadšeně bojuje, ze začátku jsou úspěšní, jako první zavedli povinné odvody, vojsko vede Edvard Lee
    \item[1861] vítězství jižanů u \textsc{Bull Run}
    \item[1862] seveřané vydávají \textit{Emancipační proklamaci}, která se stává účinnou až od dalšího roku, ruší otroctví v Unii, černošské obyvatelstvo na oplátku začalo vytvářet armádní oddíly proti jižanům
    \item[$-$] sever vydává \textit{Zákon o bezplatných přídělech půdy}: každý Američan mohl za pakatel získat půdu, po pěti letech se stal jejím vlastníkem
    \item[$-$] díky těmto krokům se seveřanům podařilo získat převahu
    \item[červenec 1863] \textsc{bitva u Gettysburgu}, seveřané defnitivně vyhrávají
    \item[1864] Lincoln podruhé zvolen presidentem
    \item[9.4.1865] Konfederace kapituluje, generál Lee podepsal kapitulační listinu, dobyt Richmond
    \item[14.4.1865] fanatický stoupenec jižanů spáchá atentát ve Fordově divadle na Lincolna, který je zastřelen
    \item[$-$] válka si vyžádala celkem 600000 padlých
    \item[$-$] nastává obodbí rekonstrukce 1865-1867
    \item[$-$] Ku-Klux-Klan: urtranacionalisti, kteří se snažili zastrašovat a likvidovat černošské obyvatelstvo, \uv{bílé kápě}, za zakladatele se považuje bývalý voják Konfederace, hnutí existovalo i přes zákaz
\end{itemize}

\subsection*{Období rekonstrukce 1865-1877}
\begin{itemize}
    \vspace{-0.5em}
    \setlength\itemsep{0.15em}
    \item[$-$] levná pracovní síla bývalých otroků, černošské obyvatelstvo získává občanská práva
    \item[$-$] v praxi pořád zůstává rasismus, v dnešní době je opět na vzestupu, proti němu bojuje \textbf{Martin Luther King}
    \item[1866] Francie jako dar posílá Sochu svobody
    \item[$-$] po válce není většina plantážníků půdu obdělávat, a tak ji pronajímají bývalým otrokům ži bělochům, ti za část úrody, kterou odevzdávali majitelům, plantáže obstarávali
    \item[$-$] průmyslové výrobky si ceny udrželi $\rightarrow$ ceny obilí klesaly, plantážnící nemohou splácet, zadlužují se a musí prodávat pozemky $\rightarrow$ vznik monopolů
    \item[$-$] USA se z krize dostaly a zahájily ekonomický vzestup, na začátku 20. století dohnávají Velkou Británii a Německo
    \item[$-$] obrovské množství cizinců, příliv obyvatelstva
\end{itemize}

\subsection*{Zahraniční politika}
\begin{itemize}
    \vspace{-0.5em}
    \setlength\itemsep{0.15em}
    \item[$-$] anekce Savojských ostrovů, Havajských ostrovů
    \item[1898] \textsc{španělsko-americká válka}, získávají značná území
    \item[$-$] expandují obchodně do Číny
    \item[$-$] výstavba Panamského průplavu převzána po Francouzích, jeoh existence umožnila intensivnější obchod mezi USA a zeměmi na Dálném východě
\end{itemize}

\section*{Habsburská monarchie po roce 1848}

\begin{itemize}
    \vspace{-0.5em}
    \setlength\itemsep{0.15em}
    \item[$-$] revoluce přinesla změnu císaře a zrušení poddanství, všichni si mají být rovni
    \item[4. 3. 1849] revoluci ukončuje oktrojovaná \textit{Stadionova ústava}, která má respektovat zrušení poddanství a občanská práva
    \item[31. 12. 1851] je však k ničemu, protože během krátké chvíle je \textit{Silvestrovskými patenty} obnoven absolutismus $\rightarrow$ \textit{bachovský absolutismus}
    \item[$-$] národnostní útlak, zejména těch, kteří se účastnili revoluce
\end{itemize}

\subsection*{Domácí politika}
\begin{itemize}
    \vspace{-0.5em}
    \setlength\itemsep{0.15em}
    \item[$-$] občané vyloučeni ze správy státu
    \item[$-$] dokončení průmyslové revoluce
    \item[$-$] \textit{Živnostenský řád} umožňuje svobodu podnikání
    \item[$-$] vznik nových bank, burz, textilní, zemědělský, potravinářský průmysl
    \item[1873] hospodářská krize, kterou přežili jen ti silnější
    \item[$-$] zrovnoprávnění dosud jen tolerovaných náboženství (\textit{Tolerančním patentem}): lutheránství, kalvinismus a pravoslaví
\end{itemize}

\subsection*{Zahraniční politika}
\begin{itemize}
    \vspace{-0.5em}
    \setlength\itemsep{0.15em}
    \item[(1853-1856)] \textsc{krymská válka} otevřeně nevystupuje, zachovává neutralitu, dostává se do mezinárodní isolace
    \item[1859] \textsc{bitva u Solferina a Magenty}, kdy se Italové snaží vyhnat Habsburky z jejich území, Rakousko je poraženo, odvoláni významní politici
    \item[20.10.1860] tím končí absolutismus, Bach je odvolán, František vydává \textit{Říjnový diplom}, slibuje ústavnost a liberalismus, zřízení parlamentu
    \item[26.2.1861] vydává \textit{Únorovou} = \textit{Schmerlingovu} ústavu, konstituční monarchie, zakládá říšskou radu o dvou komorách (horní -- panská, členy jmenuje císař, dolní -- poslanecká sněmovna, do které se poslanci volí skrze zemské sněmy), kurijní volební systém (volí se v rámci čtyř kurijí, později přibyla ještě pátá), pro muže všeobecné volební právo, ale ne rovné (zezačátku volili jen ti majetní)
    \item[3.7.1866] \textsc{bitva u Sadové}, konec \textsc{prusko-rakouské války}, na straně Prusů je Itálie, Italové od poraženého rakouská získávají Benátsko
    \item[$-$] díky těmto fiaskům musí Rakušáci vyslyšet požadavky Maďarů $\rightarrow$ přistupují na \textit{dualismus}, tzv. Rakousko-Uherské vyrovnání
    \item[17.2.1867] G. Ardassy uherským ministerským předsedou, vytváří si vlastní ústavu, vzniká personální unie Rakousko-Uhersko (Předlitavsko a Zalitavsko)
    \item[$-$] František Josef I. korunován na uherského krále
\end{itemize}

\subsection*{Rakousko-Uhersko}
\begin{itemize}
    \vspace{-0.5em}
    \setlength\itemsep{0.15em}
    \item[$-$] císař František Josef I., společná tři ministerstva (financí, zahraničí, války), delegace jsou společný zákonodárný orgán, kde jsou zastoupeni jak Rakušáci tak Uhři
\end{itemize}

\subsection*{50.-60. léta 19. století}
\begin{itemize}
    \vspace{-0.5em}
    \setlength\itemsep{0.15em}
    \item[$-$] pronásledováni: František Palacký (stáhl se z politiky), Karel Havlíček Borovský (skončil v Brixenu), Josef Kajetán Tyl (vyhnán z Prahy, musel žít jako kočovný herec na venkově)
    \item[1863] Češi se přidali k politice \textit{pasivní resistence}, to jest protestování po Vídeňském centralismu, protest spočíval v tom, že nedocházeli do Říšské rady
    \item[$-$] i přes politickou censuru vzniká řada uměleckých děl: Babička, Kytice
    \item[1861] \textit{Únorová ústava} říká, že jsme konstituční monarchií, to jsou momenty, kdy začíná ožívat politika a společenský život v českých zemích
    \item[$-$] malíři Josef Mánes (výzdoba Staré radnice), Karel Purkyně, Jaroslav Čermák, Max Švabinský, Josef Hlávka (podporoval studenty)
    \item[$-$] spolek Hlahol (pěvecký): Smetana, Sokol (tělovýchovný): Tyrš, Fügner, Prozatímní divadlo (odpolední česká představení), Umělecká beseda (sdružuje umělce nejrůznějších oborů), Spolek českých lékařů, májovci
\end{itemize}

\subsection*{Dualismus}
\begin{itemize}
    \vspace{-0.5em}
    \setlength\itemsep{0.15em}
    \item[1866] porážka Rakušanů v \textsc{bitvě u Sadové}, došlo k narovnání s Maďary, ti si vypracovali svoji ústavu
    \item[$-$] tři stejná ministerstva, různé vlády
    \item[$-$] \textit{Prosincová ústava}, ta byla vypracována pro Předlitavsko
    \item[$-$] Čechům se to nelíbí, začínají protestovat
    \item[1867] delegace na národopisnou výstavu do Moskvy, přijetí u cara Alexandra II.
    \item[1867] při přesunu českých korunovačních klenot z Vídně do Prahy demonstrace
    \item[16.5.1868] položen základní kámen Národního divadla (architekti Josef Zítek, Josef Šulc)
    \item[1868] tábory lidu Čechů na důležitých historických místech, odmítají platit daně
\end{itemize}

\section*{České země ve druhé polovině 19. století}

\subsection*{Česko-německé vyrovnání}
\begin{itemize}
    \vspace{-0.5em}
    \setlength\itemsep{0.15em}
    \item[$-$] \textit{fundamentální články}: pokus o česko-rakouské vyrovnání
    \item[1871]  vyhlášeno německé císařství ve Versailles, Habsburkové se báli války s Německem, chtěli se domluvit s Čechy -- české vyrovnání
    \item[$-$] \textit{fundamentálky}  = samospráva Českých zemí, clastní zemská vláda, navýšené pravomoci
    \item[$-$] František Josef I. slíbil, že se korunuje na čského krále $\rightarrow$ obrovská vina nevole, protože by Maďaři přišli o exklusivitu, protestují Maďaři, Němci $\rightarrow$
    \item[$-$] intervence Bismarcka proti posílení Slovanů kvůli obavám, že by se přiklonili k Rusku
    \item[$\rightarrow$] vyrovnání neproběhlo
\end{itemize}

\subsection*{Česká politika}
\begin{itemize}
    \vspace{-0.5em}
    \setlength\itemsep{0.15em}
    \item[$-$] vyostření vztahů mezi Čechy a Němci (nakupují velkostatky a mají tak volební právo) $\rightarrow$ podvody
    \item[1862] vygradování situace za voleb do Zemského sněmu (tzv \textit{Chabruskové volby})
    \item[$-$] v čele českých politiků \textit{staročeši}, to se nelíbí \textit{mladočechům} (kritisují pasivní resistenci staročechů -- od r. 1863 na protest přestali docházet do říšské rady) $\rightarrow$ dosud jednotná politická scéna se rozestupuje
    \begin{itemize}
        \vspace{-0.5em}
        \setlength\itemsep{0.15em}
        \item[$-$] národní strana staročechů
        \item[$-$] národní strana svobodomyslná (mladočeši, Národní listy bratrů Gregorových)
    \end{itemize}
    \item[1879] konec pasivní resistence, státoprávní ohrazení, vstup zpět do politiky, v čele stále staročeši
    \item[$-$] aktivní podpora vídeňské vlády, jsou loajální, le něco málo za to získají jako ústupek (\textit{drobečková politika}), byo toho však příliš málo $\rightarrow$
    \item[$-$] po příchodu české žádosti na úřad musí úředníci odpovědět
    \item[$-$] Karloferdinandova universite rozdělena na českou a německou část
\end{itemize}

\subsection*{Punktace}
\begin{itemize}
    \vspace{-0.5em}
    \setlength\itemsep{0.15em}
    \item[1890] plán snažící se o česko-německé vyrovnání o 11 bodech
    \item[$-$] domluva s Němci v monarchii, nová úprava územněsprávních jednotekv českém prostoru podle národnostních kriterií
    \item[$-$] mladočeši jsou nespokoejní, staročeši prohrávají ve volbách a odcházejí z politiky
\end{itemize}


\subsection*{Vlády}
\begin{itemize}
    \vspace{-0.5em}
    \setlength\itemsep{0.15em}
    \item[$-$] Eduard Taaffe: snížil povinnost pracovní doby, volebního censu, zavedl nemocenské a úrazové pojištění, měnová reofrma, zlatky nahrazeny korunami
    \item[$-$] Kaimír Badeni: zřízena 5. kurie, všeobecné ale ne rovné volby (různé kurie mají různý hlas), nevolí ženy
    \item[1906] \textit{Max von Beckova reforma} všeobecné reovné volební právo, konec kuriového systému
    \item[1907] volby do říšské rady
    \item[$-$] české politické strany \begin{itemize}
        \vspace{-0.5em}
        \setlength\itemsep{0.15em}
        \item[$-$] národní strana svobodomyslná
        \item[$-$] českoslovanská strana sociálně-demokratická dělnická (dnes SOCDEM, nejstarší strana)
        \item[$-$] československá strana agrární (význam za První republiky, Antonín Švehla)
        \item[$-$] křesťansko-sociální strana (v čele Jan Šrámek)
        \item[$-$] strana národně-sociální (v čele Klofáč)
        \item[$-$] česká strana lidová (inteligence, pokrok, prosazují humanismus, vzdělanost, v čele Tomáš Garrigue Masaryk)
    \end{itemize}
\end{itemize}

\section*{Vzdělanost a kultura v 19. století}

\begin{itemize}
    \vspace{-0.5em}
    \setlength\itemsep{0.15em}
    \item[$-$] České země -- gramotnost až 97 \%, v rámci monarchie velmi vzdělané
    \item[$-$] hustá síť základních škol, 5 let, pak 3 roky měšťanka, pak se budovala síť gymnasií, těch jsou dva typy:
    \begin{itemize}
        \vspace{-0.5em}
        \setlength\itemsep{0.15em}
        \item[$-$] \textit{klasická}:  řečtina, latina, humanitní předměty
        \item[$-$] \textit{reálná}: matematika, moderní jazyky, přírodní vědy, rýsování, kreslení
    \end{itemize}
    \item[$-$] obchodní akademie, uměleckoprůmyslové školy, vysoké školy (rozštěpení KFU 1882 na českou a německou část), polytechnický ústav $\rightarrow$  česká a německá technika
    \item[$-$] 1. dívčí gymnasium, o jeho vznik se zasloužila Eliška Krásnohorská
    \item[$-$] Jan Otto (největší české nakladatelství), Mánes (umělecký spolek)
    \item[1891] Jubilejní zemská výstava: 1. elektrická dráha, výstaviště -- Letná, Petřínská rozhledna, Stromovka, František Křižík (oblouková lampa, tramvaje, fontána)
\end{itemize}

\subsection*{Národní divadlo}
\begin{itemize}
    \vspace{-0.5em}
    \setlength\itemsep{0.15em}
    \item[$-$] heslo \uv{národ sobě} (nad jevištěm)
    \item[1868] položení základního kamene
    \item[1881] dokončení, korunní princ Rudolf, opera Libuše, v tomtéž roce požár
    \item[1883] znovuotevření, Libuše
    \item[$-$] \textit{triga} = trojspřeží
    \item[$-$] \textit{generace Národního divadla} (umělci spjati s ND): Aleš, Ženíšek (původní opona), Mařák, Hynais (dnešní opona), první shořela, Schnirch (socha trojspřežení = \textit{triga})
\end{itemize}

\subsection*{Další tvůrci (mimo generaci ND)}
\begin{itemize}
    \vspace{-0.5em}
    \setlength\itemsep{0.15em}
    \item[$-$] Ladislav Šaloun -- socha Jana Husa na Straoměstském náměstí
    \item[$-$] Antonín Wiehl -- hrobka Slavín na Vyšehradě
    \item[$-$] Josef Václav Myslbek -- socha svatý Václav na koni, Václavské náměstí
    \item[$-$] Ema Destinnová -- úspěchy Berlín, New York
    \item[$-$] historismus: Hluboká, Bouzov
    \item[$-$] secese: Benešova ulice v Brně, Tivoli, Obecní dům v Praze
    \item[$-$] lidová tvorba v architektuře: Jurkovičova vila, Libušín
    \item[$-$] České země speciální, objevuje se u nás kubismus v architektuře -- kubistické stavby: Dům U Černé Datky Boží, Josef Chochol, Otto Gutfreund
    \item[$-$] impresionismus: Antonín Slavíček,
    \item[$-$] Josef Uprka: krojované ženy, slovácký, lidové
    \item[$-$] secese: Mucha
    \item[$-$] česká kultura, vzdělanost dosahuje světových parametrů, úrovně
\end{itemize}

\section*{Konec 19. století, kolonie}

\subsection*{Koloniální expanse}
\begin{itemize}
    \vspace{-0.5em}
    \setlength\itemsep{0.15em}
    \item[$-$] \textit{kolonie}: strategický, vojenský, ekonomický význam
    \item[1891] všeněmecký svaz: Němci žijící všude, Mitteleuropa -- představa ovládnutí prostoru střední Evropy Němci
    \item[$-$] sinusoidy antisemitismu -- vzrůstá, další vlna $\rightarrow$  \textit{sionismus}: snaha vytvořit vlastní stát (Theodor Herzel)
    \item[1800-1898] Afriku si rozdělí Italové (Libye, Somálsko, kus Erithrei), Německo (Togo, Cameron, Německá východní  západní Afrika), Francie (sever), Británie (jih)
    \item[$-$] střetnutí v Súdánu (Fašodě): francouzský sbor se vrátil, později symbol porážky
\end{itemize}

\subsection*{Blízký a Střední východ}
\begin{itemize}
    \vspace{-0.5em}
    \setlength\itemsep{0.15em}
    \item[$-$] střet různých koloniálních velmocí, především kvůli ropnému bohatství, je tu osmanské císařství
    \item[$-$] německý císař nabízí Osmanům vystavění Bagdádské dráhy, Osmané souhlasí, nelíbí se to Britům, jižněji vyhlásili protektorát a zastavili tak pronikání Němců směrem k Perskému zálivu
    \item[1907] rusko-britská smlouva, později povede k vytvoření bloku válčících v první světové válce
    \item[$-$] Němci začínají stavět námořní flotilu, Britové se cítí ohrožení
    \item[$-$] konfilkty na Korejském poloostrově
    \item[1894-1895] \textsc{čínsko-japonská válka}, vítězí Japonci, získávají vliv a Taiwan
    \item[1904-1905] \textsc{rusko-japonská válka}, vítězí opět Japonci, poté krvavá neděle v Rusku, Říjnový diplom Mikuláše II.
    \item[$-$] Japonci získávají třeba přístav Port Arthur, Jižní Sachalin, kus Mandžuska (SV provincie Číny)
    \item[$-$] Velká Británie: Indie, Kašmír, Nepál, Barma, protektorát nad Afganistánem, kus Persie, Šalamounovy ostrovy, Papua
    \item[$-$] Siam (dnešní Thajsko) si udrželo nezávislost
    \item[$-$] Francie: drží Indočínu (Vietnam, Laos, Kambodža), postupně proniká do Číny
    \item[$-$] Nizozemí drží Indonesii
    \item[$-$] Němci: část Nové Guinee a Bismarckovy ostrovy
    \item[$-$] USA: 1898 \textsc{americko-španělská válka}, získávají Filipíny, havajské ostrovy, ostrov Guam
    \item[$-$] Čína: zatím císařstvím, ale kolabule
    \item[$-$] \textit{Povstání boxerů} v Číně proti pronikání Evropanů do Číny
    \item[1912] v Číně vyhlášena republika, dvě strany (\textit{Kuomintang} vedené Čankajšekem proti komunistické straně vedené Mao-Ce-tungem), po nějaké době začíná občanská válka po vytvoření vlády, nechtějí komunisty
    \item[1937] konec občanské války vpádem Japonska do Číny, opět se spojili proti Japonsku během druhé světové války
\end{itemize}

\subsection*{Evropa}
\begin{itemize}
    \vspace{-0.5em}
    \setlength\itemsep{0.15em}
    \item[$1873$] \textbf{Spolek tří císařů} X tzv. \textit{východní otázka}
\end{itemize}

\subsection*{Balkán}
\begin{itemize}
  \item[$-$]nezávislé státy: Řecko; Srbské, Valašské a Moldavská knížectví
  \item[$1778$] \textbf{San Stefano} Rusko donutilo Osmany uznat Rumunsko, Srbsko, Černou Horu
  %Císaři se snaží pomáhat Osmanům -> blokují rusko; Rusko podporuje knizectvi -> proti osmanum
  \item[$1878$]\textbf{Berlínský kongres} %nemam tuseni, neda se to stihat,
  \item[$-$] po něm nastupuje Fridirchův szn jako Vilém II.
  \item[$-$] někdy se o tomto roce mluvá jako o roce tří císařů
  \item[$-$] Vilém má jiné smýšlení než Bismarck
  \item[$-$] ještě za Viléma I.: do Afriky, do Asie, vznik německého koloniálního spolku , budování námořní flotily
  \item[$-$]  \textit{Spolek tří císařů}, \textit{Všeněmecký spolek}: snaha o německou světovládu i v zemích, v nichž se nachází Němci, dobytá Evropy
  \item[$-$] stavba Bagdádské dráhy
  \item[$-$] militarismus, velmocenská politika, zbrojařská horečka
  \item[$-$] nově chemický, elektrotechnický, dopravní průmysl (AEG, Siemens)
  \item[$-$] francouzské reparace splaceny v roce 1872 v podobě množství akciových společností
  \item[1873] krize  na Vídeňské burse, jež zlikvidovala spoustu společností, zejména železářské společnosti, hlavní problém: nevrátily se peníze na nějakou technickou výstavu, po které zůstal velký dluh
\end{itemize}

\subsection*{?}
\begin{itemize}
    \vspace{-0.5em}
    \setlength\itemsep{0.15em}
    \item[1879] vznik dvojspolku Němci--Rakousko-Uhersko
    \item[1883] s podmínkou, že získají území na úkor Rakouska-Uherska se připojuje i Itálie $\rightarrow$ trojspolek; Rakušané se území vzdát nechtějí
    \item[1914] propuknutí války, Itálie se prohlásila za neutrální, později do války vstoupili na základě dohody (Londýnská smlouva), na základě které jí tato území byla slíbena
    \item[$-$] na straně Německa bojují centrální mocnosti: Německo, Rakousko-Uhersko, Osmani, Bulharsko
    \item[$-$] druhý blok -- Dohoda: Britové, Francouzi a Rusové
\end{itemize}

\subsection*{Francie}
\begin{itemize}
    \vspace{-0.5em}
    \setlength\itemsep{0.15em}
    \item[$-$] 3. republika, 1875 nová ústava
    \item[$-$] po Britech druhá největší koloniální velmoc
    \item[$-$] ekonomika: reparace, ztráta Alsaska a Lotrinska, spíše střední průmysl, investice do domácí výroby
    \item[$-$] vnímáni jako bankéři světa
    \item[$-$] vkládají kapitál do cenných papírů, vyvážejí kapitál do méně vyspělých zemí
    \item[$-$] Dreyfusova aféra, Blériot, který přeletěl la-Manche
    \item[1893] francouzsko-ruská smlouva
    \item[1898] Fašoda
    \item[1904] Srdečná dohoda po napravení vztahů po fašodě v Africe
\end{itemize}

\subsection*{Velká Británie}
\begin{itemize}
    \vspace{-0.5em}
    \setlength\itemsep{0.15em}
    \item[$-$] Viktorie I., pojejí smrti končí Hannoverská dynastie
    \item[$-$] syn Eduarda VII., Jiří V. (dědeček Alžbety II.) zakládá dynastii Windsoru, to je ta současná
    \item[$-$] Jiří V. má dva syny: Edvard VIII. se zamiloval do americké dámy a díky tomu neustál posici britského krále, v témže roce, kdy nastoupil, abdikuje, po něm nastupuje druhý syn, Jiří VI., v roce 1952 umírá a nastupuje Alžběta II. (dcera Jiřího VI.)
    \item[$-$] Britové ztrácí posici nejvyspělejší země na úkor Ameriky a Němců
    \item[1876] Viktorie císařovna indická
    \item[$-$] korunní kolonie, v této době už pouze Indie a Irsko, všechny ostatní jsou už dominia (mají vlastní samosprávu)
    \item[$-$] snaží se vytvořit silné loďstvo, proto jim vadila válečná flotila budovaná Němci
    \item[1904] srdečná smlouva s Francií, narovnání vztahů pošramocených fašodou
    \item[1902] smlouva s Japonskem, porušují dosavadní politiku isolace, smlouva proti Rusku
    \item[1907] britsko-ruská smlouva, vymezení sfér vlivu
\end{itemize}


\subsection*{USA}
\begin{itemize}
    \vspace{-0.5em}
    \setlength\itemsep{0.15em}
    \item[$-$] mluvíme o tzv. \textit{pozlaceném věku}: ekonomický rozvoj, nárůst obyvatelstva (Spojené státy jsou nejrozvinutější ekonomikou, za nimi Německo i Nelká Nritánie)
    \item[$-$] bráno jako země zaslíbená, země svobody; “selfmademan” člověk, který se sám vypracoval (USA země velkých možností)
    \item[$-$] monopoly, trusty, koncerny (jakoby velké akciovky) x stát
    \item[$-$] největším producentem a vývozcem železa, uhlí, ropy, mědi, stříbra, elektrifikace
    \item[$-$] značky:  \begin{itemize}
        \vspace{-0.5em}
        \setlength\itemsep{0.15em}
        \item[$-$] John Rockefeller, Standard Oil: nafta
        \item[$-$] Andrew Carnegie, Steel Corporation: ocel
        \item[$-$] Henry Ford, Ford Motor Company (první běžící pás)
        \item[$-$] pomáhali zakládat různé instituce, podporovali kulturu atd.
    \end{itemize}
    \item[$-$] vyšší mzdy vůči Evropě $\rightarrow$ větší koupěschopnost, poptávka
    \item[$-$] velkoměsta, NY burza na Wall Street, Pátá avenue
    \item[$-$] olitické strany: republikáni x demokraté
    \item[$-$] republikáni zastupují: průmyslníky, finančníky, obchodníky; hlavní týpek: Theodor Roosevelt
    \item[$-$] demokraté: zemědělce
    \item[$-$] zavedena prohibice (alkohol), jsou proti monopolům, volební právo pro ženy
    \item[$-$] zahraniční politika: úspěšná (\uv{skvělá malá válka}) 1898 americko-španělská válka (Guam, Havajské ostrovy, Filipíny, Portoriko, Kuba)
    \item[(1901-1914)] budování Panamského průplavu  + smlouva o užívání
    \item[(1910)] Panamerická unie, ekonomický a finanční protektorát
    \item[$-$] zlepšení vztahů s VB, zhoršení vztahů s Německem

\end{itemize}

\subsection*{Japonsko}
\begin{itemize}
    \vspace{-0.5em}
    \setlength\itemsep{0.15em}
    \item[$-$] \textit{šógunát}  (u moci je šógun = vojenský velitel, drží výkonnou moc), mají i císaře, ale ten je mimo tu moc (je uctíván jako náboženská autorita)
    \item[$-$] izolace vůči pronikání velmocí (izolacionismus)
    \item[1854] otevření Japonska velmocím: Matthew C. Perry (USA) tam doplul, Japonci museli podepsat Ansejské dohody $\rightarrow$  konec izolace
    \item[1867]  abdikuje poslední šógun
    \item[$-$] císař Mucuhito (Meidži), stěhuje se do Tokia
    \item[$-$] osvícená vláda, reformy: branná povinnost, vojenská flotila, samurajové přišli o moc, přijetí ústavy (Japonsko konstituční monarchií)
    \item[$-$] budování velkých rodinných firem (obchody, banky), zprůmyslnění
    \item[1875]  Japonci dali Rusům jižní Sachalin výměnou za Kurilské ostrovy (pak si to Japonci vezmou zpět)
    \item[1894/95]  čínsko-japonská válka, získali i Taiwan
    \item[$-$] Rusové chtějí Korejský poloostrov
    \item[1901]  anekce Korei
    \item[1902]  smlouva s Velkou Británií x Rusko
    \item[1907]  smlouva s Ruskem a Francií
\end{itemize}




\end{document}
