\documentclass{article}
\usepackage{fullpage}
\usepackage[czech]{babel}
\usepackage{amsfonts}

\title{\vspace{-2cm}Fancie za Ludvíka XIV., Habsburská monarchie po třicetileté válce, Rusko v 17. a 18. století\vspace{-1.7cm}}
\date{}
\author{}

\begin{document}
\maketitle

\section*{Francie za Ludvíka XIV.}

\begin{itemize}
    \vspace{-0.5em}
    \setlength\itemsep{0.15em}
    \item[1594] Jindřichem IV. nastupuje dynastie Bourbonů, vymírá dynastie z Valois
    \item[$-$] Ludvík též patří k této dynastii
    \item[$-$] vrchol absolutismus ve Francii, \uv{stát jsem já}
    \item[$-$] znám také jako \uv{král slunce}, podporoval významné umělce, v mládí provozoval balet
    \item[$-$] \textbf{Ludvík XIII.}, syn Jindřicha IV. a otec Ludvíka XIV., měl manželku \textbf{Annu Rakouskou}, která byla regentkou, než Ludvík dospěl po otcově smrti
    \item[$-$] první ministr kardinál \textbf{Jules Mazarin}, fakticky Ludvík začíná vládnout až po Mazarinově smrti
    \item[$-$] Mazarin se snaží protlačovat absolutismus, problémy se stavovksým odbojem, proti němu se bouří šlechta a duchovenstvo = \textit{fronda princů a parlamentu}, stavovský odboj nakonec zlomen
    \item[$-$] žena \textbf{Marie Tereza}, děti nejen s ní ale i se svými milenkami
    \item[$-$] tehdy měla v Evropě největší a nejmodernější armádu, nejlidnatější stát v evropě
    \item[$-$] snaží se oslabit svoji konkurenci (vysokou šlechtu), zbavil je moci politické a připoutal si je ke svému dvoru $\rightarrow$ Versailles
    \item[$-$] \textit{privilegované stavy} = osvobozeni od některých povinností, především nemusí platit daně (šlechta, církev)
    \item[$-$] policejní síť, centralizace, \textit{intendanti} = královští úředníci, kteří mají na vše dohlížet v jednotlivých částech Francie (placení daní, soudní kompetence)
    \item[$-$] růst daní, odstraňování cel, budování infrastruktury
    \item[$-$] na konci jeho vlády ve Francii hodně žebráků
\end{itemize}

\subsection*{Zahraniční politika}
\begin{itemize}
    \vspace{-0.5em}
    \setlength\itemsep{0.15em}
    \item[$-$] Habsburkové zaneprázdněni Turky $\Rightarrow$ francie si může řídit v Evropě a zabírat si územíčka
    \item[$-$] kolonie: Kanada, Louisiana, Východoindická a Západoindická společnost
    \item[$-$] \textit{požadavek přirozených hranic}: podél oceánů, moře, Pyrenejí, řeky Rýn, Alpy $\Rightarrow$ začal válčit a zabírat některá území
    \item[1701-1714] \textsc{válka o dědictví španělské}: ve Španělsku vymřela dynastie Habsburků, o dědictví se uchází Habsburkové -- Leopold I. chce prosadit svého syna Karla a spojenci (Anglie, Nizozemí, ...) proti Ludvíkovi, který chtěl prosadit svého vnuka Filipa z Anjou
    \item[1713] \textsc{mír v Utrechtu} mezi Francií a Anglií
    \item[1714] \textsc{mír v Rastattu} mezi Francií a SŘŘ
    \item[$-$] z válek vítězně vychází Anglie, Ludvík ale dosadil Filipa na španělský trůn pod podmínkou, že se Španělsko nespojí s Francií
    \item[$-$] Rakouští Habsburkové získávají španělské Nizozemí (budoucí Belgii) a území na Apeninském poloostrově, která však později ztratí
    \item[$-$] Britové získávají Newfoundland, Hudsonův záliv, Gibraltar, Menorcu a monopol na obchod s africkými otroky (dováželi otroky do Jižní Ameriky)
\end{itemize}


\subsection*{Náboženská politika}
\begin{itemize}
    \vspace{-0.5em}
    \setlength\itemsep{0.15em}
    \item[$-$] prosazování katolicismu
    \item[1685] zrušení nantského ediktu $\Rightarrow$ odchod hugenotů, ekonomicky nevýhodné
    \item[$-$] \textit{dragonády} = proces převracování hugenotů na katolíky, dragouni (vojáci) jsou povinně ubytování v hugenotských domácnostech
\end{itemize}

\subsection*{Kulturní politika}
\begin{itemize}
    \vspace{-0.5em}
    \setlength\itemsep{0.15em}
    \item[$-$] francouzská móda, popularita francouzštiny, \uv{zlatý věk Francie}
    \item[$-$] Versailles
    \item[$-$] ministr financí \textbf{Jean-Baptiste Colbert} zavádí politiku \textit{merkantilismu} = více vyvážíme než dovážíme, navíc podporoval rozvoj manufaktur, orientoval se na luxusní zboží (parfémy, krajky, drahé látky), dovoz jen z kolonií nebo pod vysokými cly = \textit{protekcionismus}
\end{itemize}

\begin{itemize}
    \vspace{-0.5em}
    \setlength\itemsep{0.15em}
    \item[$-$] po něm nastupuje Ludvík XV. ve dvou letech
\end{itemize}

\section*{Habsburská monarchie po třicetileté válce}

\subsection*{Ferdinand III. (1637-1657)}

\begin{itemize}
    \vspace{-0.5em}
    \setlength\itemsep{0.15em}
    \item[$-$] nastupuje po Ferdinandu II. Štýrském, za třicetileté války
    \item[$-$] za svého života nechal korunovat svého syna (Ferdinan IV.), ten však umírá, takže vládne dál
    \item[$-$] vzdělaný, ovládá několik jazyků
    \item[$-$] nakonec po něm nastupuje jeho čtvrtý syn Leopold I.
    \item[1651] \textit{soupis poddaných} po třicetileté válce
    \item[1654] \textit{berní rula} = soupis obcí, katastr $\rightarrow$ lepší výměra daní
\end{itemize}

\subsection*{Leopold I. (1657-1705)}
\begin{itemize}
    \vspace{-0.5em}
    \setlength\itemsep{0.15em}
    \item[$-$] současník Ludvíka XIV., zároveň soupeř, ale i obdivovatel Ludvíka a krásy, kterou ve Versailles vybudoval
    \item[$-$] v čele Habsburské monarchie, císařem SŘŘ
    \item[$-$] vládne dlouze $\Rightarrow$ povede se mu upevnit absolutismus po třicetileté válce
    \item[$-$] po úbytku obyvatelstva rostou robotnostní povinnosti poddaných
    \item[1680] \textit{robotní patent}, poddaní mohou robotovat maximálně tři dny v týdnu
    \item[(1692-1695)] \textsc{povstání Chodů} (strážci českých jižních hranic, osvobozeni od poddanských povinností) , po existenci monarchie však již žádná hranice neexistuje a nemají smydl, privilegia jsou jim upíráná $\Rightarrow$ povstání za znovuzískání privilegií, v čele \textbf{Jan Sladký Kozina} proti \textbf{Lamingerovi} (Lomikar), povstání končí popravou Koziny
    \item[$-$] čarodějnické procesy
    \item[1663/4] \textsc{války s Turky}, úspěšné, ale Habsburkům dochází peníze $\Rightarrow$ mír
    \item[1683]  \textsc{dvouměsíční obléhání Vídně} Turky, Osmany se podařilo porazit díky Karlu Lotrinskému
    \item[1699] Osmany vyhání \textbf{Evžen Savojský}, \textsc{Karlovický mír} = Osmané se vzdávají Uher
    \item[1701] \textsc{války o dědictví Španělské}
    \item[$-$] dva synové, Josef a Karel

\end{itemize}


\subsection*{Josef I. (1705-1711)}
\begin{itemize}
    \vspace{-0.5em}
    \setlength\itemsep{0.15em}
    \item[$-$] pokračování \textsc{válek o dědictví Španělské}
    \item[1703] \textsc{povstání uherských stavů} proti Habsburské nadvládě
    \item[1711] \textsc{Szátmárský mír} mezi Habsburky a uherskými stavy, uherským stavům jsou zachováan privilegie a stavy uznají právo Habsburků na uherský trůn
\end{itemize}

\subsection*{Karel VI. (1711-1740)}
\begin{itemize}
    \vspace{-0.5em}
    \setlength\itemsep{0.15em}
    \item[$-$] konec války o dědictví španělské
    \item[$-$] války s Turky, neúspěchy
    \item[17.4.1713] \textsc{Pragmatická sankce}: chce zajistit nástupnost i ženským potomkům (vydána před narozením Marie Terezie)
    \item[$-$] Juro Jánošík, přepadal bohaté a dával chudým
\end{itemize}

\subsection*{Marie Terezie (1740-1780)}
\begin{itemize}
    \vspace{-0.5em}
    \setlength\itemsep{0.15em}
    \item[$-$] manžel \textbf{František Štěpán Lotrinský}, vnuk Karla Lotrinského (porážky Osmanů u Vídně), 16 dětí (patří do Habsbursko-Lotrinské dynastie)
    \item[1740] arcivévodkyně rakouská
    \item[1741] královna uherská
    \item[$-$] pragmatická sankce nebyla ze začátku akceptována $\Rightarrow$
    \item[1740-1748] \textsc{války o dědictví rakouské} = \textsc{války slezské}, odboj vede pruský král \textbf{Fridrich II. Veliký}, Marie Terezie nesouhlasila s korunovací Albrechta českým králem při dobytí Prahy Bavorskem, Saskem a Francií, nakonec vyhnáni $\Rightarrow$ \textsc{berlínský mír}: Habsburkové ztrácejí Slezsko, ale České země jsou osvobozeny a 1743 Marie Terezie se stává českou královnou
    \item[$-$] Francie je ve válkách na straně proti habsburkům
    \item[1745] \textsc{Drážďanský mír}: Pragmatická sankce potvrzena, František I. Štěpán I. je císařem
    \item[1748] definitivní konec válek \textsc{mírem v Cáchách}
    \item[1756-1763] \textsc{sedmiletá válka}: Francie x Anglie, Prusko x Rakousko, přenese se do kolonií, v Indii a Kanadě vítězí Angličané
    \item[$-$] baron Trenk bojoval na straně Marie Terezie, jeho jednotky = \textit{panduři}, loupí
    \item[$-$] \textit{osvícenský absolutismus}, doba reforem, převládá názor, že člověk je schopen udělat stát dokonalejším, reformy
    \item[$-$] centralizace (ve Vídni), snaha zemi ekonomicky povznést
    \item[$-$] \textbf{Václav Antonín Kounic} v čele zahraniční politiky, hrabě \textbf{Leopold Josef Daun} v čele armády, \textbf{Bedřich VIlém Haugwitz} se stará o vnitrostátní politiku
\end{itemize}

\subsubsection*{Reformy}
\begin{itemize}
    \vspace{-0.5em}
    \setlength\itemsep{0.15em}
    \item[(1748)] 1. tereziánský katastr (soupis půdy poddanské), rustikál $\Rightarrow$ lepší vybírání daní
    \item[(1757)] 2. tereziánský katastr (soupis půdy vrchnostenské), dominikál
    \item[(1754)] 1. sčítání lidu, asi 18,8 milionů, Čechy 2,26 mil.
    \item[$-$] číslování domů, povinná příjmení
    \item[$-$] společné úřady pro celou monarchii: \textit{direktorium} pro finanční politiku a veřejné záležitosti (v čele Haugwitz), poradní státní rada (poradní sbor direktoria)
    \item[$-$] třístupňový státní aparát: ústřední (Vídeň), \textit{gubernia} (v jednotlivých zemích), magistráty (kraje, města)
    \item[(1759)] zřízen \textit{Nejvyšší soudní dvůr} se sídlem ve Vídni
    \item[$-$] \textit{Tereziánský trestní zákoník} mírnější, ale pořád zachovává mučení, \textit{jednací soudní řád}
    \item[$-$] \textit{manufaktury}, v Čechách výroba textilu, skla, hedvábí
    \item[$-$] celní unie mezi Čechami a Rakouskem
    \item[$-$] podpora podnikání, \textit{protekcionalismus} = stát se snaží zamezit dovozu ze zahraničí $\Rightarrow$ vysoké clo
    \item[$-$] budování silnic, poštovního spojení, tolarová a zlatková měna, \textit{bankocetle} = první bankovky na našem území, jednotné míry a váhy
    \item[$-$] pícniny jako potrava pro dobytek $\Rightarrow$ \textsc{jetelová revoluce}, dobytek se ustájí a nezabíjí na zimu
    \item[$-$] dostávají se k nám brambory z Braniborska
    \item[(1767)] \textit{Urbariální patent} = upřesnění vztahů, práv a povinností mezi vrchností a poddanými
    \item[(1775)] \textit{Robotní patent} = nevolnické povstání ve Rtyni na Náchodsku, poraženi u Chlumce nad Cidlinou, rychtář Nývlt, práce max. 3 dny v týdnu
    \item[$-$] \textit{raabizace} (dle R. A. Raaba: dvorský rada), část roboty lze nahradit peněžními dávkami $\Rightarrow$ šlechta nechce, protože by přišla oi pracovní sílu
    \item[$-$] vojenská reforma (důsledek porážky Pruskem), modernizace armády, vojenská škola, nové oblečení, stavby pevností
    \item[$-$] \textit{povinná školní docházka} od 6 do 12 let (\textit{školy triviální}), ale někteří pořád nedodržují
    \item[$-$] movitější děti v krajských městech: školy hlavní (druhý stupeň), školy normální (pajdák)
    \item[$-$] jezuitům odňata univerzita a střední školství, zesvětštění školství, němčina nahrazuje latinu
    \item[1777] biskupství v Olomouci povýšeno na arcibiskupství, biskupství v Brně
    \item[$-$] Uhry zůstávají agrární, Rakousko a Česko: rozvoj řemesla, později zprůmyslnění
\end{itemize}

\subsection*{Josef II.}
\begin{itemize}
    \vspace{-0.5em}
    \setlength\itemsep{0.15em}
    \item[1764] německým králem, o rok později (po smrti otce) císařem SŘŘ, spoluvladařem Marie Terezie
    \item[$-$] typický osvícenský panovník, oblíbený $\Rightarrow$ \textit{selský císař}
    \item[$-$] chce získat informace o tom, jak funguje monarchie pod jménem hrabě Falkenštejn
    \item[$-$] chce monarchii, kde si občané byli rovni, to však v této době nebylo možné, často naráží na konzervativce, mnoho reforem zrušeno
    \item[$-$] dokončení reformy armády, vojáci mají právo na osobní život, doživotní služba nahrazena, penze
    \item[$-$] stavba terezína, opevnění Josefova, barokní opevnění Hradce Králové
    \item[1783] spojení Slezska a Moravy, správním centrem se stává Brno
    \item[$-$] vznik magistrátů
    \item[1784] spojením čtyř měst pražských položen základ novodobé Prahy
    \item[$-$] Občanský zákoník: snaha o rovnost občanů před zákonem
    \item[$-$] zrušen trest smrti, zrušeno čarodějnictví, zrušení mučení, odstranění středověkých přežitků
    \item[$-$] zrušeno několik klášterů, ponechal jen ty, které vykazovaly nějakou činnost, rušení kostelů, zřizování nemocnic, porodnic, dražba ruolfinských sbírek, otevření Nosticova (= Stavovského) divadla, uvedení hry Don Giovanni
    \item[13.10.1781] \textsc{Toleranční patent}: tolerována kromě katolictví i lutheránství, kalvinismus a pravoslaví, exulanti se mohou vrátit
    \item[$-$] nařídil farářům povinnost vést matriky
    \item[$-$] napravení postavení Židů: musí mít německé jméno, ale mohou vykonávat většinu povolání, mají přístup k univerzitám
    \item[1.11.1781] \textsc{Patent o zrušení nevolnictví}: z nevolníků se stali poddaní, nevolník je přímo připoután půdě, poddaný musí robotovat, odvádět desátky, ale mohou svobodně odejít z panství, mají víc práv
    \item[$-$] omezení cenzury
\end{itemize}


\subsection*{Leopold II.}
\begin{itemize}
    \vspace{-0.5em}
    \setlength\itemsep{0.15em}
    \item[1790] císař, o rok později český král
    \item[$-$] Leopoldův syn nastupuje jako František II., ale Napoleon zničí SŘŘ, takže se tituluje jako císař Rakouský František I.
\end{itemize}


\section*{Rusko v 17. a 18. století}

\begin{itemize}
    \vspace{-0.5em}
    \setlength\itemsep{0.15em}
    \item[$-$] Ivan IV. se prohlásil za prvního cara, Fjodor Ivanovič (vymírá dynastie Rurikovců), Lžidimitrijové, období krize = \textit{smuta} (1587-1613)
    \item[1613] zvolen carem \textbf{Michail Romanov}, jím nastupuje dynastie Romanovců
\end{itemize}

\subsection*{Michail Romanov, Alexej Romanov (1645-1676)}
\begin{itemize}
    \vspace{-0.5em}
    \setlength\itemsep{0.15em}
    \item[$-$] dlouho pod nadvládou Mongolů, vývoj zastaven, teprve teď obnovují
    \item[$-$] absolutistická vláda = \textit{samoděržaví}
    \item[$-$] velkou moc má pravoslavná církev
    \item[$-$] drtivá většina společnosti nevolníci, zemědělci, zaostalé za Evropou, vše dováženo
    \item[$-$] nízká kulturní, vzdělanostní úroveň obyvatelstva
    \item[$\Rightarrow$] nutné reformy, přístav Archangelsk v zimě zamrzá, chce získat přístup k Baltu nebo Černému moři k otevření obchodu
    \item[$-$] postupně pronikají k Tichému oceánu
    \item[$-$] území Kozáků (obyvatelé ruských stepí) je ovládáno Polsko-litevským státem, pomáhali proti Mongolům $\Rightarrow$
    \item[1648-1654] \textsc{povstání Bohdana Chmelnického}, levobřežní Ukrajina připojena k Rusku
    \item[(1670-1671)] \textsc{selské války} v čele s Stěnkou Razinem, protože poddaní byli nespokojeni
\end{itemize}

\subsection*{Petr Veliký (1689-1725)}
\begin{itemize}
    \vspace{-0.5em}
    \setlength\itemsep{0.15em}
    \item[$-$] zakladatel moderního ruského státu, obrovské množství reforem
    \item[$-$] nechal mučit svého syna, později ho nechal popravit
    \item[$-$] obdivovatel západu, uznává nové techniky
    \item[1682] dostává se k moci s bratrem Ivanem V., ale vládne jejich sestra Sofie, v sedmnícti letech ji sesadí a vsadí do kláštera $\Rightarrow$ dostává se do čela Ruska, o chvíli později umírá i jeho bratr
    \item[1699] úspěšný ve válkách s Turky, \textsc{dobýjí pevnost Azov}
    \item[1697]  inkognito vyjíždí do západní Evropy, zjišťuje situaci a inspiruje se
    \item[1698] \textsc{povstání střeleckých pluků}, zatímco je v Evropě, chtějí dosadit zpět jeho sestru Sofii
    \item[1700-1721] \textsc{severní válka} vytvoření prtišvédské koalice: Rusko, Dánsko, Polsko, Sasko
    \begin{itemize}
        \vspace{-0.5em}
        \setlength\itemsep{0.15em}
        \item[1700] \textsc{bitva u Narvy}, Švédové porážejí Rusy
        \item[1703] zakládá nové sídelní město Petrohrad, \uv{okno do Evropy}, (1712) hlavní město
        \item[1709]  \textsc{bitva u Poltavy}, švédská vojska vtažena na jih Ruska, kde vojska Petra Velikého vítězí, boje ale pokračují dál
        \item[1721] \textit{Nystadský mír}, Rusko získává přístup k Baltu,  Švédové ztrácí značná území
    \end{itemize}
    \item[$-$] zavedení pravidelné armády, budování válečného námořnictva, Petropavlovská pevnost, školství, budování infrastruktury, zakládání manufaktur, kde pracují nevolníci, sjednocená měna \textit{rubl}
    \item[$-$] rozdělení území na části = \textit{gubernie}
    \item[$-$] zjednodušení písma = \textit{graždanka}
\end{itemize}

\subsection*{Kateřina II. Velká (1762-1796)}
\begin{itemize}
    \vspace{-0.5em}
    \setlength\itemsep{0.15em}
    \item[$-$] sesadila svého manžela Petra III. (vnuk Petra Velikého), nechala ho uvěznit, kde byl zavražděn, tzv. \uv{dámská revoluce}
    \item[$-$] doba palácových převratů (6 vladařů za 27 let)
    \item[$-$] lutheránský původ, po příjezdu do Ruska však konvertovala k pravoslaví
    \item[$-$] vládne v duchu osvícenského absolutismu, ale krutá a sexuálně aktivní, literárně činná, milenci z řad člechty
    \item[$-$] \uv{zlatý věk Ruska}, založení Lomonosovy univerzity v Moskvě
    \item[$-$] propagovala očkování proti neštovicím, sama se nechala očkovat
    \item[$-$] centralizace, utužení nevolnictví (\textit{mužici})
    \item[$-$] porážka osmanských Turků, dosáhnutí Černého moře, Krymu: vznik Sevastopolu, Oděsy $\Rightarrow$ kolonizace Ukrajiny, Černomoří, Čukotky, Sachalin, Aljaška
    \item[$-$] \textit{Potěmkinovy vesnice}, měl na starosti kolonizaci, kulisy, aby při návštěvě Kateřiny \uv{bylo na co se dívat}, jeden z jejích milenců
    \item[$-$] \textsc{trojí dělení Polska}
\end{itemize}

\subsection*{Zánik Polska}
\begin{itemize}
    \vspace{-0.5em}
    \setlength\itemsep{0.15em}
    \item[$-$] \uv{zlatá svoboda} šlechty, dělá si co chce, sleduje jenom svoje zájmy
    \item[1386] dědička Hedvika se provdala za litevského knížete Jagiella, spojí se do polsko-litevské personální unie, vládnou Jagellonci
    \item[(1569)] personální unie se ruší, vzniká lublisnká unie, o tři roky později vymírají Jagellonci
    \item[$-$] král je volen šlechtou, \textit{sejm} = zastupitelský orgán (šlechtici, každý z nich má právo veta)
    \item[$-$] drtivá část společnosti znevolněna, národnostně pestré
    \item[$-$] krize využívají sousedé, třeba Švédsko prosazuje svoji dynastii
    \item[$-$] hluboká krize nastává v 18. století za polského krále Augusta III., toho využívá Kateřina, která vojensky prosadila do čela Polska jednoho ze svých stoupenců, nastupuje jako král \textbf{Stanislav II. August}, toho nepřijímá šlechta $\Rightarrow$ sesazen $\Rightarrow$ Rusko se spojí s Rakouskem a Pruskem, domluví se na \textsc{trojím dělením Polska}, každý si vezme asi třetinu, šlechta musí přísahat věrnost
    \item[$-$] šlechta tuší, že je něco špatně, mění své zákony, pokouší se vytvořit novou ústavu (1791), taškařice $\Rightarrow$ druhé dělení
    \item[$-$] po abdikaci Stanislava II. proběhne třetí dělení
    \item[$-$] dělění v letech: 1772 (Rusko, Prusko, Habsburkové), 1793 (Rusko a Prusko), 1795 (Rusko, Prusko, Habsburkové)
    \item[$-$] někdy se též mluví o čtvrtém dělení Polska, což je počátek druhé světové války
\end{itemize}

\subsection*{Vznik Pruska}
\begin{itemize}
    \vspace{-0.5em}
    \setlength\itemsep{0.15em}
    \item[$-$] základem Braniborské markrabství, kde vládne dynastie Hohenzollernové od počátku 15. st.
    \item[$-$] postupně se Braniborské markrabství zvětšuje, připojeno vévodství Pruské, po třicetileté válce patří mezi nejvýznamnější části SŘŘ
    \item[2. pol. 17. st.] kurfiřt Fridrich Vilém: reformy, zakládá manufaktury, posiluje území
    \item[1701] jeho syn, Fridrich III., se v Königsbergu prohlásil za krále (vládne jako Fridrich I.), mluvíme o Pruském království = Braniborsko, Západní Pomořany, Východní Prusko
    \item[$-$] buduje Berlín jako sídelní město, využívá práci nevolníků, většina lutheránu, ale nábožensky tolerantní
\end{itemize}

\subsubsection*{Friedrich I. (1701-1713)}
\begin{itemize}
    \vspace{-0.5em}
    \setlength\itemsep{0.15em}
    \item[$-$] buduje Berlín jako sídelní město, vznik typických barokních zámků
    \item[$-$] velkostatkáři = \textit{junkeři}, většina společnosti, nevolníci
    \item[$-$] zdrojem ekonomiky je vývoz obilí
    \item[$-$] založení pruské univerzity, uchycení lutheránství, ale jsou nábožensky tolerantní vůči ostatním
\end{itemize}

\subsubsection*{Fridrich Vilém I. (1713-1740)}
\begin{itemize}
    \vspace{-0.5em}
    \setlength\itemsep{0.15em}
    \item[$-$] syn Fridricha I., \uv{kaprál na trůně} (celý život chodil v uniformě), militarzoval Prusko
    \item[$-$] branná povinnost pro nevolníky, vojenský drill, poslušnost
    \item[$-$] \textsc{severní válka} mezi Švédkem a protišvédské koalici, Prusko získává malá území
    \item[$-$] rozvoj \textit{merkantilismu}, školství, ale prioritná je poříd militarizace a armáda
\end{itemize}

\subsubsection*{Friedrich II. Veliký (1740-1786)}
\begin{itemize}
    \vspace{-0.5em}
    \setlength\itemsep{0.15em}
    \item[$-$] současník Marie Terezie, za něj zahájeno trojí dělení Polska
    \item[$-$] typicky osvícenský panovník, na jeho dvoře pobýval Voltaire, I. Kant
    \item[$-$] \textit{Edikt} o náboženské svobodě
    \item[$-$] rozvoj obchodu, stavění infrastruktury, zakládání manufaktur, uve do Pruska odborníky ze zahraničí
    \item[$-$] \uv{filosof ze Sanssouci} (rokokový zámek), věnolal se prakticky filosofii, má sloužit tomu státu (ne jako Ludvík XIV.)
    \item[$-$] účastnil se \textsc{slezských válek}, \textsc{sedmileté války}
\end{itemize}





\end{document}
