\documentclass{article}
\usepackage{fullpage}
\usepackage[czech]{babel}
\usepackage{amsfonts}

\title{\vspace{-2cm}Pařížský mírový systém\vspace{-1.7cm}}
\date{}
\author{}

\begin{document}
\maketitle


\subsection*{Versailleská smlouva}
\begin{itemize}
  \item nedaleko Paříže se v palácích Versailleského komplexu domlouvají mírové podmínky, oficiálně začínají jednání 18. 1. 1919 (tedy přesně 48 let potom, co zde bylo vyhlášeno Německé císařství)
  \item účastní se především tzv. \textit{velká trojka} -- David Lloyd George (prem. UK), Woodrow Wilson (pres. USA), Georges Clemenceau (prem. Fr.), dále Vittorio Orlando (prem. It.) a Nobuaki Makino (prem. Jap.), popř. se scházeli v deseti, kde byli tito a jejich ministři zahraničí, bez pozvání zástupci poražených zemí a sovětů
  \item[7. 5. 1919] předložili body mírové smlouvy, Němci z toho byli zničení, něm. vláda to odmítla a podala demisi, Němci to nakonec stejně 28. 6. 1919 podepsali
  \item podmínky pro Německo
  \begin{itemize}
    \item museli vrátit Francii Alsasko-Lotrinsko
    \item museli postoupit Belgii Eupén-Malmedy
    \item levý břeh Rýnu bude 15 let okupován, 50 km na pravém břehu je demilitarizovaná zóna
    \item Němci postupují Polsku Poznaňsko, polský koridor
    \item postupují Memel mandátu, potom Litvě, Gdaňsk postupují mandátu
    \item postupují Hlučínsko Československu
    \item Německo ztrácí všechny kolonie
    \item zrušení všeobecné branné povinnosti, prof. armáda max 100k mužů, redukce válečného loďstva, zákaz letectva, rozpuštění generálního štábu
    \item 10 let dodávat uhlí Francii, Belgii, Itálii
    \item museli postoupit část Šlesvicka Dánsku
    \item peněžní reparace
    \item Sársko jako mandát
    \item dále referendum v jižním Prusku a v Horním Slezsku, ty ale vyhrává Německo
  \end{itemize}
\end{itemize}

\subsection*{Saint-Germainská smlouva }
\begin{itemize}
  \item[10. 9. 1919] smlouva s Rakouskem, musí to být republika, muselo uznat nástupnické státy
  \item postupují Valticko a Vitorazsko ČSR, Jižní Tyrolsko, Istrii, Terst Itálii, Halič Polsku a Bukovinu Rumunsku
  \item zákaz sjednocení s Německem
  \item reparace, armáda do 30k vojáků
\end{itemize}

\subsection*{Trianonská mlouva s Maďarskem}
\begin{itemize}
    \vspace{-0.5em}
    \setlength\itemsep{0.15em}
    \item[(4. 6.) 1920] podepsána v Trianonnu s Maďarskem
    \item[$-$] ztráta 70 \% území a 60 \% obyvatelstva
    \item[$-$] Maďaři ztratili: Slovensko, Podkarpatskou Rus, museli uznat existenci násupnických států
    \item[$-$] po konferenci plebiscit
    \item[$-$] smlouva vyvolala obrovskou nespokojenost
\end{itemize}

\subsection*{Neuillyská smlouva s Bulharskem}
\begin{itemize}
    \vspace{-0.5em}
    \setlength\itemsep{0.15em}
    \item[27. 11. 1919] územní ztráty ve prospěch sousedů (jižní část Řecku -- ztracení přístupu k Egejskému moři)
\end{itemize}

\subsection*{Sévreská smlouva s Tureckem}
\begin{itemize}
    \vspace{-0.5em}
    \setlength\itemsep{0.15em}
    \item[10. 8. 1920] podepsána, tráta 4 / 5 území, naprosto nepřijatelná
    \item[$-$] Turci dosájli revize díky generálovi Mustafu Kemalovi
    \item[$-$] ze sultanátu se stává republika v čele s Mustafou Kemalem
    \item[$-$] vznik sekulárního státu, ženy mají volební právo
\end{itemize}

\subsection*{Blízký východ}
\begin{itemize}
    \vspace{-0.5em}
    \setlength\itemsep{0.15em}
    \item[$-$] vytvoření dvou mandátních území: britský a francouzský mandát
    \item[$-$] jednání o tom, kdy tu vzniknou nezávislá území
    \item[$-$] eskalace napětí mezi židy a araby
\end{itemize}

\subsection*{Společenství národů}
\begin{itemize}
    \vspace{-0.5em}
    \setlength\itemsep{0.15em}
    \item[10. 1. 1920] vznik na Pařížské konferenci
    \item[$-$] hlavní cíl: zabránit dalším válkám, to se nepovedlo, poté rozpuštěna
    \item[$-$] sídlem byla Ženeva, dnes sídlo OSN
    \item[$-$] otec myšlenky americký president Woodrow Wilson
    \item[$-$] na půdě Společenstí národůaktivní tehdy ministr zahraničí Edvard Beneš
\end{itemize}

\subsection*{Malá Dohoda}
\begin{itemize}
    \vspace{-0.5em}
    \setlength\itemsep{0.15em}
    \item[$-$] političtí reprezentanti Československa, Jugoslávie a Rumunska sbližuje strach z Maďarska, Karla I. (možnost Velkých Uher, o což se pokusí) se sblížili už na Pařížské konferenci
    \item[$-$] organisace společného postupu těchto zemí
\end{itemize}

\subsection*{Washingtonská mírová konference}
\begin{itemize}
    \vspace{-0.5em}
    \setlength\itemsep{0.15em}
    \item[$-$] měla uspořádat poměry na Dálném východě
    \item[$-$] potvrzení držav v Tichomoří
    \item[$-$] Číně byl vnucen princip otevřených dvěří (cíl Ameriky)
    \item[$-$] poměr tonáží válečných lodí v Tichomoří
    \item[$-$] říká se mu \textit{Versaillesko-Washingtonský systém}
    \item[$-$] problémy: absence organizace, která by pořádky udržovávala, příliš přísná k poraženým, vznik mnohonárodnostních států (problém s menšinami), Němci (Hitler) to později využije k tomu, aby napravil křivdu z Versailles
\end{itemize}



\end{document}
