\documentclass{article}
\usepackage{fullpage}
\usepackage[czech]{babel}
\usepackage{amsfonts}

\title{\vspace{-2cm}Velká francouzská revoluce (1789-1794/1799)\vspace{-1.7cm}}
\date{}
\author{}

\begin{document}
\maketitle

\section*{Příčiny, situace ve Francii}

\begin{itemize}
    \vspace{-0.5em}
    \setlength\itemsep{0.15em}
    \item[$-$] po smrti Ludvíka XIV. zdevastovaná říše, tisíce žebráků, Ludvík XV. (jeho pravnuk), ve válkách slezských proti Marii Terezii, ve válce sedmileté už na straně Marie Terezie, ztráta území v Kanadě a Indii, Ludvík XVI. (jeho vnuk), za něj vypukne VFR
    \item[$-$] absolutismus, cenzora, katolicismus, žádná svoboda projevu, nesnášenlivost vůči jiným než katolickému náboženství (po zrušení ediktu nantského)
    \item[$-$] Marie Antoinetta (manželka Ludvíka XVI.): Francouzi ji výrazně nesnáší (\uv{pyšná Rakušanka}), nakupuje šperky, pořádá párty, mastí hazardní hry $\rightarrow$ neoblíbená
    \item[$-$] tehdy nejilnější a nejlidnatější stát v Evropě
    \item[$-$] tři stavové: duchovní (privilegované), aristokracie (privilegované), zbytek
    \item[$-$] ekonomická situace: daně, desátky, odpustky, velkostatky nemají peníze na rozvoj
    \item[$-$] v Anglii začíná průmyslová revoluce, tady ne, podnikání brzdí cechy a manufaktury
    \item[$-$] hrozba finančního kolapsu, nákladný život ve Versailles, zvyšování státního dluhu, pomáhání americkým osadám (rozpočtově nefiskální)
    \item[$-$] neúroda, konkurenční levné zboží z Anglie
    \item[$-$] střídání finančních ministrů, nestabilita
    \item[5.5.1789] svolání generálních stavů do Versailles (duchovní a aristokracie), pověřeni králem východisko této jízlivé situace, ideální zdanit bohaté stavy, což nechtějí, král se zase nechce vzdát absolutismu,
    \item[17.6.1789] nakonec se třetí stav prohlásil za národní shromáždění, cálem je zrušit absolutismus a vypracovat ústavu omezující krále, v duchu osvícenských myšlenek se musí rozdělit státní moc na tři složky, lidé si jsou rovni předs zákonem, král samozřejmě nesouhlasí, nechce, aby dál pokračovali
    \item[$-$] král se tajně připravuje na vojenský zásah
    \item[14.7.1789] \textsc{dobytí Bastily}, počátek VFR, královské vězení, symbol absolutismu, děla na pevnosti byla natočena na chudinské smrti, to se lidem nelíbí, proto na ni zaútočili, dobyli ji, bylo tam však jenom asi sedm vězňů, důležité spíš jako symbol, rovněž ovládají pařížskou radnici
    \item[$-$] 


\end{itemize}

\end{document}
