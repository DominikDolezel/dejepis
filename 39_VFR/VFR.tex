\documentclass{article}
\usepackage{fullpage}
\usepackage[czech]{babel}
\usepackage{amsfonts}

\title{\vspace{-2cm}Velká francouzská revoluce (1789-1794/1799)\vspace{-1.7cm}}
\date{}
\author{}

\begin{document}
\maketitle

\section*{Příčiny, situace ve Francii}

\begin{itemize}
    \vspace{-0.5em}
    \setlength\itemsep{0.15em}
    \item[$-$] po smrti Ludvíka XIV. zdevastovaná říše, tisíce žebráků, Ludvík XV. (jeho pravnuk), ve válkách slezských proti Marii Terezii, ve válce sedmileté už na straně Marie Terezie, ztráta území v Kanadě a Indii, Ludvík XVI. (jeho vnuk), za něj vypukne VFR
    \item[$-$] absolutismus, cenzora, katolicismus, žádná svoboda projevu, nesnášenlivost vůči jiným než katolickému náboženství (po zrušení ediktu nantského)
    \item[$-$] Marie Antoinetta (manželka Ludvíka XVI.): Francouzi ji výrazně nesnáší (\uv{pyšná Rakušanka}), nakupuje šperky, pořádá párty, mastí hazardní hry $\rightarrow$ neoblíbená
    \item[$-$] tehdy nejilnější a nejlidnatější stát v Evropě
    \item[$-$] tři stavové: duchovní (privilegované), aristokracie (privilegované), zbytek
    \item[$-$] ekonomická situace: daně, desátky, odpustky, velkostatky nemají peníze na rozvoj
    \item[$-$] v Anglii začíná průmyslová revoluce, tady ne, podnikání brzdí cechy a manufaktury
    \item[$-$] hrozba finančního kolapsu, nákladný život ve Versailles, zvyšování státního dluhu, pomáhání americkým osadám (rozpočtově nefiskální)
    \item[$-$] neúroda, konkurenční levné zboží z Anglie
    \item[$-$] střídání finančních ministrů, nestabilita
    \item[5.5.1789] svolání generálních stavů do Versailles (duchovní a aristokracie), pověřeni králem východisko této jízlivé situace, ideální zdanit bohaté stavy, což nechtějí, král se zase nechce vzdát absolutismu,
    \item[17.6.1789] nakonec se třetí stav prohlásil za národní shromáždění, cálem je zrušit absolutismus a vypracovat ústavu omezující krále, v duchu osvícenských myšlenek se musí rozdělit státní moc na tři složky, lidé si jsou rovni předs zákonem, král samozřejmě nesouhlasí, nechce, aby dál pokračovali
    \item[$-$] král se tajně připravuje na vojenský zásah
    \item[14.7.1789] \textsc{dobytí Bastily}, počátek VFR, královské vězení, symbol absolutismu, děla na pevnosti byla natočena na chudinské čtvrti, to se lidem nelíbí, proto na ni zaútočili, dobyli ji, bylo tam však jenom asi sedm vězňů, důležité spíš jako symbol, rovněž ovládají pařížskou radnici
    \item[4.8.1789] \textsc{zrušena privilegia}: rovnost před zákonem, zrušení šlechtických privilegií, zrušení poddanství
    \item[26.8.1789] \textit{Deklarace práv člověka a občana}: formulovány základní lidská práva
    \item[5.10.1789] rozzuřený dav (nemají na jídlo) vtrhnul do královského sídla ve Versailles a donutil ho, aby se přestěhovali do Paříže
\end{itemize}

\section*{Konstituční monarchie}
\begin{itemize}
    \vspace{-0.5em}
    \setlength\itemsep{0.15em}
    \item[$-$] útěk krále z Francie k Belgii (Varennes), chce obnovit svoji pozici ve Francii a absolutistickou monarchii, byl však poznán a vrácen zpět do Paříže
    \item[(3.9.) 1791] zákonodárné shromáždění pracuje na ústavě, která vyšla v platnost
    \item[$\rightarrow$] konec absolutismu, konstituční monarchie = moc panovníka je omezena ústavou
    \item[$-$] dělba státní moci na tři složky:
    \begin{itemize}
        \vspace{-0.5em}
        \setlength\itemsep{0.15em}
        \item[$-$] moc zákonodárná: Zákonodárné národní shromáždění, vzešlé z voleb
        \item[$-$] moc výkonná: vláda + král, jmenuje ministry, vláda se zodpovídá Zákonod. shromáždění (= parlamentu)
        \item[$-$] moc soudní, soudci též voleni
    \end{itemize}
    \item[$-$] volby výrazně omezeny: \textit{majetkový cenzus}, smí volit jen bohatí (asi čtyři miliony voličů)
    \item[$-$] decentralizace státu, některé kompetence přeneseny na departementy (= kraje)
    \item[$-$] působení tzv. politických klubů, zárodky politických stran:
    \begin{itemize}
        \vspace{-0.5em}
        \setlength\itemsep{0.15em}
        \item[$-$] \textit{girondisté}: umírnění stoupenci republiky, tehdejší levice, reprezentují spíše bohatší vrstvy
        \item[$-$] \textit{cordelliéři}: nejradikálnější republikáni, levice (Danton, Marat) a časem k nim přibudou i \item[$-$] \textit{jakobíni}, ze začátku \uv{klub přátel monarchie}, po příchodu Robesierra se zradikalizovali, stoupenci monarchie
        \item[$-$] \textit{feullanti}: stoupenci monarchie (La Fayette)
    \end{itemize}
    \item[$-$] \textit{sanculotti}: nejchudší pasivní občané, nemohli volit
    \item[$-$] stát nemá peníze, zrušeily se dávky, ale nestačily se stvořit nové, stát nemí peníze $\Rightarrow$ obracejí se na církev $\Rightarrow$ nesouhlas některé části občanů
\end{itemize}

\section*{Vznik republiky}
\begin{itemize}
    \vspace{-0.5em}
    \setlength\itemsep{0.15em}
    \item[$-$] tehdy v rakouských zemích vládne bratr marie Antoinetty Leopold II., obrací se na jeho pomoc, Rakousko (a také Prusko a další německy hovořící státy) souhlasily s tím, že francouzskému králi pomohou obnovit monarchii za pomocí vojenské intervence
    \item[duben 1792] impulz k tomu, aby gerondisté vyhlásili Rakousku \textsc{preventivní válku}
    \item[$-$] nebyli shcopni válčit kvůli některým monarchistickým generálům, cizí armáda se blíží k francouzským hranicím, dobrovolníci z Marseilles si zpívají Marseillaisu a podařilo se jim cizí armádu zastavit $\rightarrow$ symbol revoluce, později hymna
    \item[9./10.8.1792] \textsc{povstání v Paříži}, vyhlášena pařížská Komuna = společnost rovných, král zatčen a uvězněn
    \item[$-$] vyhlášeny volby do Národního Konventu, většinu získávají gerondisté, volební právo už mají všichni
    \item[21.9.1792] hned na prvním konventu vyhlášena republika, sesazen král, Francouzi den před tím zásadně porážejí zahraniční armády
    \item[$-$] král obviněn z velezrady
    \item[21.1.1793] král popraven, jejich syn Ludvík později umírá
\end{itemize}
\section*{Jakobínská diktatura (2.6.1793-27./28.7.1794)}

\begin{itemize}
    \vspace{-0.5em}
    \setlength\itemsep{0.15em}


    \item[$\Rightarrow$] rozruch ve Evropě, všichni monarchové se přidávají na stranu intervenčních armád (Británie, Španělsko, Nizozemsko), složitá situace
    \item[$-$] v departmantech protirevolucionářské vzpoury, rolnické nepokoje ve Vendée
    \item[$-$] jakobíni prosazují tzv. \textit{malé maximum}: maximální ceny obilí a mouky
    \item[2.6.1793] nepřehledné a složité situace využívají jakobíni, zorganizují nbové \textsc{povstání v Paříži}, popravili girondisty, ujímají se moci, v čele \textbf{Maxmilien Robespierre} -- diktatura jakobínů
    \item[$-$] vypracována nová jakobínská ústava, nevešla nikdy v platnost (čekalo se, než skončí válka, co se nestalo) je plánována nulová dělba státní moci, všechna moc je v Konventu
    \item[$-$] v rámci Konventu vytvořen výbor pro veřejné dobro vedenýž Robespierrem, všichzni nepřátelé jakobínů likvidováni, mněl Francii chránit před zahgraniční intervencí
    \item[$-$] zavedení branné poovinnosti s cílem účinně bojovat se zahraničními vojsky, řada velitelů nerozvážně posílá nováčky do války možná díky tomu se intervenční armádě daří a dostává se i za území Francie
    \item[$-$] sťata i Marie Antoinetta
    \item[$-$] kult Nejvyšší bytosti: zaveden místo katolictví, deismus (bůh stvořil svět, ale dál už nezasahuje)
    \item[$-$] nový kalendář se začátkem za vzniku republiky (ne za Krista)
    \item[$-$] zavedeny metrické jednotky
    \item[$-$] \textit{velké maximum}: maximální ceny vztahující se na základní zboří (potraviny), stanovily maximální mzdy
    \item[1793] válčené úspěchy \textbf{Napolena Bonaparteho}, vyhání Angličany z přístavu Toulon
    \item[červenec 1793]  Jean Paul Marat (cordellier), zavražděn ve vaně Charlottou Corday, pochybnosti okolo jeho smrti
    \item[$-$] Dekrety podezřelých: zkrácené procesy
    \item[$-$] Danton (umírnění): \uv{dost už bylo poprav} $\rightarrow$ popraven x zběsilí (héberisté)
    \item[$-$] odhalování nepřátel revoluce, ne vždy prokazatelně
    \item[červen 1794] \textit{velký teror}: všichni, kdo nesouhlasí s Robespierrem skončí na popravišti, i jeho vlastní zastánci v parlamentu si nebyli jisti svým krkem
    \item[$-$] Robespierrovi nabízí titul diktátora, odmítá a chystá se zakročit proti svým spojencům ,kteří zneužili situace a jakkoliv se obohatili
    \item[$\rightarrow$ ]  poslední kapka, spiknutí proti Robespierrovi, svržen, dle nového kalendáře 9. thermidoru \textit{thermidorský převrat}
    \item[28.7.1794]  \textsc{poprava Robespierra}
\end{itemize}

\section*{Thermidorská reakce (období direktoria) 1794-1799}
\begin{itemize}
    \vspace{-0.5em}
    \setlength\itemsep{0.15em}
    \item[$-$] republika zůstává zachována, moc v rukou bohatých
    \item[$-$] konec řízeného hospodářství: maxima a minima zrušena $\rightarrow$ nárůst cen
    \item[$-$] teror proti jakobínům, jako vyřízení účtů
    \item[$-$] ve společnosti panuje nespokojenost, nikdo není spokojen
    \item[$-$] Francie úspěšná ve válkách, místo obranných už útočné, mír s Pruskem, Nizozemím a Španělskem
    \item[srpen 1795] nová ústava = \textit{direktorium}, konec Konventu, moc výkonnou má pět direktorů,  moc zákonodárná v rukou Radě 500 (dolní komora) a Radě starších (horní komora)
    \item[$-$] pokusy o změnu systému: \textsc{SPiknutí rovných} v čele s Babeufem, snažili se nastolit primitivní komunismus, po odhalení spiknutí popraven
    \item[$-$] royalisté (stoupenci monarchie) též chystali státní převrat, potlačil je Napoleon Bonaparte
    \item[$-$] je zapotřebí někoho, kdo stabilizuje situaci $\rightarrow$ tím je Napoleon Bonaparte
    \item[9.11.1799] státní převrat vedený Nepoleonem Bonapartem, definitivní konec revoluce
    \item[$-$] moc na sebe strhlo kolegium tří konzulů $\rightarrow$ období konzulátu, Napoleon prvním konzulem
    \item[$-$] platí do roku 1804, kdy se z Napoleona stává císař
    \item[2.8.1802] Napoleon v referendu získal titul doživotního konzula
    \item[2.12.1804] stává se císařem
\end{itemize}

\section*{Napoleonská doba}
\begin{itemize}
    \vspace{-0.5em}
    \setlength\itemsep{0.15em}
    \item[$-$] z Korsiky, údajně měl rád matematiku, první manželka Josefina Beauharnais, žádný syn
    \item[$-$] 
\end{itemize}



\end{document}
