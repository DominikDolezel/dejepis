\documentclass{article}
\usepackage{fullpage}
\usepackage[czech]{babel}
\usepackage{amsfonts}

\title{\vspace{-2cm}Předkolumbovské civilizace\vspace{-1.7cm}}
\date{}
\author{}

\begin{document}
\maketitle

\section*{Inkové}
\begin{itemize}
    \vspace{-0.5em}
    \setlength\itemsep{0.15em}
    \item[$-$] Latinská Amerika, Andy, údolí Cuzco; nejstarší osídlení v této oblasti: 1. tis. př. n. l.
    \item[$-$] v 15. st. mají minimálně 10 000 000 obyvatel
    \item[$-$] nejznámější místo: Machu Picchu
    \item[$-$] Titicaca (jezero), goeglyfy na náhorní plošině Nazca
    \item[$-$] skvělí stavitelé -- bez pojiva, jen kvádry -- chrámy, silnice, terasovitá pole
    \item[$-$] kalendář 365,25 dne, uzlíkové písmo -- Kipu, kovy, ale ne kolo, chovají lamy a taky alpaky, kult boha slunce -- zlato
    \item[1531 -- 1535] zánik: Francisco Pizarro, Diego Almagro, založili Limu

\end{itemize}

\section*{Mayové}
\begin{itemize}
    \vspace{-0.5em}
    \setlength\itemsep{0.15em}
    \item[$-$] poloostrov Yucatán, Guatemala, Salvádor
    \item[$-$] počátek letopočtu -- městské státy
    \item[$-$] pyramidy Chichén Itza, Palanqe, kalendář $18 \times 20 + 5$, znali 0, Mayské číslice, neznali kovy, lidské oběti, hieroglyfy

\end{itemize}

\section*{Aztékové}
\begin{itemize}
    \vspace{-0.5em}
    \setlength\itemsep{0.15em}
    \item[$-$] Mexiko, hl. m. Tenochtitlan, až 10 000 000 ob.
    \item[$-$] hieroglyfy, akao i jako platidlo
    \item[1519 -- 1521] zánik: Hernán Cortés

\end{itemize}




\end{document}
