\documentclass{article}
\usepackage{fullpage}
\usepackage[czech]{babel}
\usepackage{amsfonts}

\title{\vspace{-2cm}Mezinárodní situace před druhou světovou válkou\vspace{-1.7cm}}
\date{}
\author{}

\begin{document}
\maketitle


JAPONSKO
císař Hirohito - stoupencem expanzivní politiky, jeho prvním cílem bylo Mandžusko (bohaté na přírodní zdroje, místo odbytu japonské výroby), 1931 Mukdenský incident (zinscenování něčeho simono), samostatný stát Mandžusko (císař Pchu I.)
Japonsko-čínská válka - začala v červenci 1937 u mostu Marca Pola; masové vraždění civilistů - 300k obyvatel, skončila kapitulací Japonska v září 1945 (považuje se za součást 2. WW)
pakt proti Kominterně - uzavřeno mezi Japonskem a Německem v listopadu 1936, týká se společného zacílení proti SSSR (proti komunismu), v roce 1937 se k paktu připojila Itálie; “tajný” dodatek o tom, že pokud by jeden stát zaútočil na Rusko, tak ostatní státy nesmí být na straně Ruska (WTF);; počátek linie Berlín-Řím-Tokio

ITÁLIE
válka o Etiopii v prosinci 1934 nějaké brikule a spojené národy nic neudělaly
říjen 1935 zahájení války, postup do vnitrozemí Etiopie, v Etiopii císař Haile Salassie, požádal o pomoc společnost národů (měl full ass projev), zas nic moc neudělali
francouzský premiér Pierre Laval slíbil Itálii kus Etiopie → pád francouzské vlády (?)
Etiopie se stala součástí Italské východní Afriky (italské kolonie v Eritrei a Somálsku)
ocelový pakt - květen 1939; mezi Itálií a Německem, zavázali se k vzájemné pomoci v případě války, v roce 1940 se připojilo Rumunsko

politika appeasementu (=politika ústupků, zmírňování a tolerance ve snaze předejít vojenskému či politickému konfliktu








nahrávka slyš cca 19. minuta

OD POLITIKY APPEASEMENTU K MNICHOVSKÉ DOHODĚ
posílení Sudetoněmecké strany (působí pro československo, ale chtějí ho rozbít)
1935 parlamentní volby - sudetoněmci nejvíc hlasů, nebyli přizvání k tvorbě vlády, to dělali agrárníci (nemam tušení o čem je řeč a stydím se), premiér Milan Hodža
1935 prezident Beneš
1937 program národnostní politiky
duben 1938 sjezd SdP, z toho vznikly tzv. Karlovarské požadavky - požadují samosprávu pohraničí
plán Fall Grün - tajný plán vojenského útoku na Československo
květen 1938 částečná mobilizace
německá strategie: požadovat nesplnitelné
září 1938 projev Hitlera na sjezdu NSDAP, sudetští němci se bouří,...→
ultimátní nóta ČSR aby předala sudety Německu (v rámci zachování míru tzv.), v praze demonstrace asi 250k lidí
chtěli i území od polska (těšínsko?) a maďarska (jižní slovensko)

OD MNICHOVSKÉ KONFERENCE K DRUHÉ REPUBLICE
29./30. září 1938
zástupci států: Adolf Hitler, Neville Chamberlain, Benito Mussolini, Édouard Daladier
podepsání Mnichovské dohody→ vyklidit pohraničí do 10. října 1938
1. října 1938 vznik druhé republiky (Česko-Slovensko)

konec markéty ingrové


\end{document}
