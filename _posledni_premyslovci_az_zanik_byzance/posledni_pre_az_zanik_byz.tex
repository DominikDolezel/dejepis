\documentclass{article}
\usepackage{fullpage}
\usepackage[czech]{babel}
\usepackage{amsfonts}

\title{\vspace{-2cm}Poslední Přemyslovci, Uhry, Polsko, vznik Španělska a zánik Byzance\vspace{-1.7cm}}
\date{}
\author{}

\begin{document}
\maketitle

\section*{Poslední Přemyslovci}
\subsection*{Přemysl Otakar I.}
\begin{itemize}
    \vspace{-0.5em}
    \setlength\itemsep{0.15em}
    \item[1198] zisk královského titulu
    \item[26. 9. 1212] \textsc{Zlatá bula sicilská} (Fridrich II.) = titul českého krále je dědičný, svobodná volba, navrácena Morava, možnost jmenování biskupů $\rightarrow$ stabilizace
    \item[1216] \textit{primogenitura} = první syn zdědí tituly otce (dříve \textit{seniorát})
    \item[$-$] manželky \textbf{Adléta Míšeňská, Konstancie Uherská}
    \item[$-$] nová mince \textit{brakteát} (denár stále v oběhu), na vlajce nově český lev (nahrazuje černou orlici)
    \item[$-$] sídla Porta coeli, Křivoklát, Zvíkov
\end{itemize}

\subsection*{Václav I.}
\begin{itemize}
    \vspace{-0.5em}
    \setlength\itemsep{0.15em}
    \item[1241] nájezdy Mongolů v Evropě
    \item[1243] udělení městských privilegií Brnu
    \item[1247] povstání vedené šlechtou a jeho synem \textbf{Přemyslem Otakarem II.}
    \item[1248] \textsc{bitva u Mostu}, Václav I. vítězí a Přemysla uvězní, nakonec se stejně stane králem
    \item[1249] Jihlavský horní zákoník
\end{itemize}

\subsection*{Přemysl Otakar II., \uv{král železný a zlatý}}
\begin{itemize}
    \vspace{-0.5em}
    \setlength\itemsep{0.15em}
    \item[$-$] manželky \textbf{Markéta Babenberská, Kunhuta Uherská}
    \item[$-$] ovládaná území: Čechy, Morava, Chebsko, Rakousy, Korutany, Kraňsko, Štýrsko, Furlansko
    \item[1254/55] \textsc{Křížová výprava do Pruska} $\rightarrow$ Královec
    \item[$-$] vznik zemského soudu, Českých desek zemských (úřední dokumenty); založil 50 měst
    \item[1273] císařem SŘŘ \textbf{Rudolf Habsburský} $\rightarrow$ nepřátelství s Přemyslem
    \item[1276] Přemysl Otakar II. ztrácí v bitvách rakouské země
    \item[26. 8. 1278] \textsc{bitva na Moravském poli u Suchých Krut}, Přemysl Otakar II. umírá
\end{itemize}

\subsection*{Václav II.}
\begin{itemize}
    \vspace{-0.5em}
    \setlength\itemsep{0.15em}
    \item[$-$] v době smrti Přemysla Otakara II. mu bylo 7 let $\rightarrow$ vládne \textbf{Ota Braniborský}
    \item[$-$] vězněn, poté vykoupen šlechtou
    \item[$-$] kutání stříbra v Kutné Hoře $\rightarrow$  pražský groš, mincovna Vlašský dvůr
    \item[1300] královské právo horníků
    \item[1300] manželkou \textbf{Eliška Rejčka} (z Polska), zisk titulu polského krále
    \item[1301] zisk titulu uherského krále
    \item[1304] \textbf{Albrecht Habsburský} táhne do Čech, odražen
    \item[$-$] zakládá Zbraslavský klášter
    \item[1305] umírá na tuberkulózu
\end{itemize}

\subsection*{Václav III.}
\begin{itemize}
    \vspace{-0.5em}
    \setlength\itemsep{0.15em}
    \item[$-$] vzdal se uherské koruny; táhl do Polska, aby potlačil povstání Ladislava Lokýtka, v Olomouci zabit $\rightarrow$ Přemyslovci vymírají po meči
\end{itemize}

\subsection*{Boje o českou korunu}
Jindřich Korutanský, Rudolf Habsburský, 1310 Jan Lucemburský

\section*{Uhry}

\subsection*{13. st.}
\begin{itemize}
    \vspace{-0.5em}
    \setlength\itemsep{0.15em}
    \item[1222] \textit{Zlatá bula Ondřeje II.}
\end{itemize}

\subsection*{Ondřej III.}
\begin{itemize}
    \vspace{-0.5em}
    \setlength\itemsep{0.15em}
    \item[$-$] poslední Arpádovec, jeho dcera Alžběta zasnoubena s Václavem III.
\end{itemize}

\subsection*{Václav III.}
\begin{itemize}
    \vspace{-0.5em}
    \setlength\itemsep{0.15em}
    \item[$-$] vládne pod jménem Ladislav V.
    \item[$-$] Omodějovci, Matúš Čák Trenčianský se odmítají podrobit královské moci
    \item[1305] vzdává se uherské koruny
\end{itemize}


\subsection*{Dynastie z Anjou}
\subsubsection*{Karel Robert}
\begin{itemize}
    \vspace{-0.5em}
    \setlength\itemsep{0.15em}
    \item[$-$] posílil královskou pozici
\end{itemize}

\subsubsection*{Ludvík I. Veliký}
\begin{itemize}
    \vspace{-0.5em}
    \setlength\itemsep{0.15em}
    \item[$-$] doba rozmachu, rozvoje, rozšiřování území, jeho matka byla sestrou Kazimíra
    \item[1370] uherským i polským králem $\rightarrow$ sjednocení -- polsko-uherská personální unie
    \item[$-$] dvě dcery: \textbf{Marie} (dědička uherské koruny), \textbf{Hedvika} (dědička polské koruny)
\end{itemize}

\subsubsection*{Zikmund Lucemburský}
\begin{itemize}
    \vspace{-0.5em}
    \setlength\itemsep{0.15em}
    \item[$-$] syn Karla IV., oženil se s Marií Uherskou
\end{itemize}

\section*{Polsko}
\subsection*{Přemysl II. Velkopolský}
\begin{itemize}
    \vspace{-0.5em}
    \setlength\itemsep{0.15em}
    \item[$-$] sjednotil Polsko, 1296 umírá, po něm Václav II. a Václav III.
\end{itemize}

\subsection*{Vladislav I. Lokýtek z Piastovců}
\begin{itemize}
    \vspace{-0.5em}
    \setlength\itemsep{0.15em}
    \item[$-$] opět sjednotil Polsko, králem
\end{itemize}

\subsection*{Kazimír III. Veliký}
\begin{itemize}
    \vspace{-0.5em}
    \setlength\itemsep{0.15em}
    \item[$-$] dokončil sjednocování Polska
    \item[$-$] poslední Piastovec, po jeho smrti problém s obsazováním polské koruny
\end{itemize}

\subsection*{Dynastie Anjou}
\subsubsection*{Ludvík I. Veliký -- viz dříve}

\subsubsection*{Hedvika}
\begin{itemize}
    \vspace{-0.5em}
    \setlength\itemsep{0.15em}
    \item[$-$] vzala si litevského knížete \textbf{Jogalia} $\rightarrow$ dynastie \textbf{Jagellonců}
    \item[(1386)] polsko-litevská personální unie
\end{itemize}

\begin{itemize}
    \vspace{-0.5em}
    \setlength\itemsep{0.15em}
    \item[1410] \textsc{bitva u Grunwaldu} proti řádu německých rytířů, výhra polsko-litevské unie
    \item[(1569 -- 1795)] Lublinská unie: spojení států jako takových
\end{itemize}

\section*{Vznik Španělska}
\begin{itemize}
    \vspace{-0.5em}
    \setlength\itemsep{0.15em}
    \item[$-$] výsledek \textit{reconquisty} -- křesťané chtějí znovu dobýt území
\end{itemize}

\subsection*{Situace před vznikem}
\begin{itemize}
    \vspace{-0.5em}
    \setlength\itemsep{0.15em}
    \item[$-$] \textbf{Vizigóti} Tolosánská říše, později Vizigótská říše, centrum v Toledu
    \item[711] území si podrobili Arabové, centrum \textbf{Córdoba}
    \item[10. st.] vznik arabského chalífátu; tolerance vůči původním obyvatelům, vzkvétá kultura a vzdělanost
    \item[$-$] významná centra: Granada, Sevilla, Toledo
    \item[$-$] \textit{Al-Andalus}: arabský název Pyrenejského poloostrova v době, kdy tu byli muslimové
\end{itemize}

\subsection*{Reconquista}
\begin{itemize}
    \vspace{-0.5em}
    \setlength\itemsep{0.15em}
    \item[8. st.] (Pelayo), dobývá Asturii a Galicii
    \item[$-$] \textbf{Karel Veliký} vybuduje \textit{španělskou marku} = oblast kolem Pyrenejí, kterou ovládl, Píseň o Rolandovi
\end{itemize}

\subsection*{11. -- 12. století}
\begin{itemize}
    \vspace{-0.5em}
    \setlength\itemsep{0.15em}
    \item[$-$] vznik Kastilie a Leonu, Aragonu a Katalánska a pak Portugalska; křesťané zatlačili muslimy na jih
    \item[$-$] \textbf{Ferdinand I.}, kastilský král a jeho syn \textbf{Alfonso VI.} -- dobývá Toledo, Cordobu, Sevillu a Zaragozu
    \item[$-$] \textit{hidalgové} = vojáci ze středních a vyšších vrstev, \textbf{Rodrigo Díaz de Vívar} = Cid, poté králem Valencie
    \item[$-$] \textit{berbeři} = obyvatelé severní Afriky, \textit{maurové} = obyvatelé severní Afriky a Arabové
    \item[1212] \textsc{bitva u Las Navas de Tolosa}, zásadní porážka Arabů, od té doby jsou vytlačováni
    \item[1492] \textsc{dobytí Granady}
    \item[(1469)] Kastilské království dědí \textbf{Isabela Kastilská}, dědicem Aragonu \textbf{Ferdinand II. Aragonský} $\rightarrow$ sňatek
    \item[1479] personální unie $\rightarrow$ základ Španělského království
    \item[$-$] \textit{inkvizice}: cíl vyhnat obyvatele jiného než katolického náboženství $\rightarrow$ hospodářský úpadek
    \item[$-$] \textit{autodafé} = veřejné vyhlášení někoho za kacíře a provedení rozsudku, většinou upálení
    \item[$-$] Isabela podporuje \textbf{Kryštofa Kolumba}
    \item[$-$] jejich dcera \textbf{Johana Šílená}, žena Filipa Habsburského
    \item[$-$] jejich syn \textbf{Karel V.} = \textbf{Carlos I.}, později císař SŘŘ
    \item[$-$] jeho bratr \textbf{Ferdinand I. Habsburský}, po vymření Jagellonců položil základy Habsburské monarchie, jejíž jsme též byli součástí, později císař SŘŘ
\end{itemize}

\section*{Zánik Byzance}
\begin{itemize}
    \vspace{-0.5em}
    \setlength\itemsep{0.15em}
    \item[$-$] první podlomení za 4. křížové výpravy
    \item[$-$] Latinské císařství, poté Nikájské císařství
    \item[(1261)] \textsc{dobytí Konstantinopole} $\rightarrow$ obnovení Byzance, vládnou \textbf{Palailogovci}
\end{itemize}

\subsection*{Palailogovci}
\begin{itemize}
    \vspace{-0.5em}
    \setlength\itemsep{0.15em}
    \item[$-$] nepřátelé: italští obchodníci (konkurence), sousedi; za nich sociální nepokoje
    \item[(1274)] \textit{Lyonská církevní unie} = spojení křesťanství V a Z církve, hlavou má být papež
\end{itemize}


\subsection*{Turecká expanze}
\begin{itemize}
    \vspace{-0.5em}
    \setlength\itemsep{0.15em}
    \item[$-$] příchod Seldžúků, Mantzikert (1071), Jeruzalém (1076)
    \item[13. / 14. st.] příchod osmanských Turků $\rightarrow$ krize Seldžuků
\end{itemize}


\subsection*{Osmanská říše 1299 -- 1922}
\begin{itemize}
    \vspace{-0.5em}
    \setlength\itemsep{0.15em}
    \item[$-$] v SZ dnešního Turecka; teokratický stát, opírá se o armádu
    \item[1299] \textbf{Osman I. Gází} se prohlásil za vládce
    \item[$-$] jeho syn \textbf{Orchan} dobyl Nikáiu, Nikomédii
    \item[$-$] \textit{janičáři} = elitní pěchota turecké armády, rekrutováni z řad křesťanských kluků, \textit{siphaiové} = jízda
\end{itemize}

\subsubsection*{Murad I. (1359 -- 1389)}
\begin{itemize}
    \vspace{-0.5em}
    \setlength\itemsep{0.15em}
    \item[$-$] ovládl Thrákii, dobyl Adrianopol $\rightarrow$ hlavní město
    \item[1389] \textsc{bitva na Kosově poli}, podrobení Srbů
\end{itemize}

\subsubsection*{Bajezid I.}
\begin{itemize}
    \vspace{-0.5em}
    \setlength\itemsep{0.15em}
    \item[$-$] dobytí Serdiky (dnešní Sofie)
    \item[1393] vyvrácení Bulharska
    \item[1396] \textsc{bitva u Nikopole} (vedl Zikmund Lucemburský), Osmané vyhráli, Evropané neúspěšní
    \item[$-$] vznikla \textit{Florentská unie}, aby se mohla uskutečnit křížová výprava proti Osmanům
\end{itemize}

\subsubsection*{Murad II.}
\begin{itemize}
    \vspace{-0.5em}
    \setlength\itemsep{0.15em}
    \item[1444] \textsc{bitva u Varny} proti křižákům, Osmané opět vítězí
\end{itemize}

\begin{itemize}
    \vspace{-0.5em}
    \setlength\itemsep{0.15em}
    \item[1453] dobytí Cařihradu, smrt Konstantina XI.
    \item[$\Rightarrow$] definitivní konec Byzance, Konstantinopol = Istanbul jako hlavní město Osmanů
    \item[$-$] Evropané se nafrní a nechtějí s nimi obchodovat, hledají nové obchodní cesty na Orient
\end{itemize}

\subsubsection*{Sulejman I. Nádherný}
\begin{itemize}
    \vspace{-0.5em}
    \setlength\itemsep{0.15em}
    \item[1426] \textsc{bitva u Moháče}, kde bylo poraženo uherské vojsko, Osmani získávají drtivou většinu Uher
    \item[(1429)] Osmané obléhají Vídeň
    \item[(1541)] dobytí Budína
    \item[1683] konečně vítězství Evropanů \textsc{u Vídně}
\end{itemize}




\end{document}
