\documentclass{article}
\usepackage{fullpage}
\usepackage[czech]{babel}
\usepackage{amsfonts}

\title{\vspace{-2cm}\vspace{-1.7cm}}
\date{}
\author{}

\begin{document}
\maketitle
\begin{itemize}
  \item bitva o británii -- němci by chtěli dobýt británii, vylodění pomocí operace lvoun
  \item adlertag 13. srpna
  \item do řijna nejintenzivnější vzdušná válka o britský prostor
  \item první německá prohra
  \item od roku 1942 to už budou britové kteří bombardují německo
  \item Luftwaffe x RAF
  \item[říjen 1940] na španělsko-francouzských hranicích probíhá jednání Hitlera a Franca, pokud se Franco přidá do války a dobude Gibraltar, tak si ho Španělsko může nechat a dostane k tomu nějakou tu kolonii, nicméně Franco to nepřijal, Španělsko bude neutrální
  \item[listopad 1940] Hitler nabízí Stalinovi, jestli se nepřidá do války proti kapitalistům
  \item v RAF i českoslovenští dobrovolníci
  \item pojďme na Balkán
  \item[duben 1939] Itálie anektuje Albánii, plánuje co dál
  \item[říjen 1940] Mussolini útočí na Řecko, je to obří fiasko, Řekové částečně vstoupili do Albánie
  \item[březen 1941] Hitler si je nakláněl, v březnu se Hitlerovi povedlo dostat Jugoslávii do aliance, proběhne tam ale do týdne puč, Hitler reaguje a Jugoslávie je v rámci operace Marika během dubna obsazena, pak je dobyto i Řecko, trošku je okousají, osamostatní se Chorvatsko v čele s ustašovci, v čele s Ante Paveličem, dále Srbsko v čele s gen. Nedičem
  \item Řecko je připojeno k Itálii, Makedonii získává Bulharsko
  \item v Jugoslávii je silná partyzánská aktivita v čele s Josipem Brozem Titem (this is a surprise tool that will help us later)
  \item[květen -- červen 1941] operace Merkur -- invaze na Krétu
  \item válka v africe
  \item Italové drží Libyi, Ethiopii, Somálsko a Eritreu, chtějí dobýt britský Egypt
  \item[$\Rightarrow$] otevíráme libyjsko-egyptskou frontu (září 1940)
  \item[únor 1941] Italové jsou zatlačeni
  \item[březen 1941] vložte postavu německého tankového velitele Erwina Rommela, nicméně přestal být v popředí kvůli Barbarosse, opevní Tobruk, který je pak pracně dobýván
  \item[květen 1941] habešsko-somálská fronta: Italové vedou neúspěšný útok na britské Somálsko, Briti zareagovali tím, že dobyli celou Italskou východní Afriku
  \item[duben 1941] Japonsko a SSSR uzavírají smlouvu o neútočení
  \item[březen 1941] zákon o půjčce a pronájmu -- USA \uv{půjčí} Britům a ostatním spojencům (potom i SSSR) vybavení za peníze, které zaplatí po válce (pak se to promlčelo)
  \item jedeme na východní frontu
  \item[22. června 1941] začíná útok Německa na SSSR, okolo 70\% německé vojenské síly se přesune na východní frontu, Němci mají srovnatelný počet ale mnohem vyšší kvalitu vybavení, Stalin taky před válkou zlikvidoval většinu schopných generálů v rámci čistek (Tuchačevsky), sovětské vedení je tedy nezkušené
  \item ze začátku padaly celé sovětské armády do německého zajetí
  \item Stalina částečně zachránilo to, že se mu povedlo přesunout část průmysl za Ural
  \item německý útok veden třemi směry, na sever do Leningradu, v centru do Moskvy, na jih do Stalingradu, na Kavkaz (ropa)
  \item německá invaze ale nemá před zimou moc času, na podzim začíná pršet, pak je krutá zima, na kterou není Německo plně připraveno
  \item[září 1941 - leden 1944] Leningrad je obklíčen, začíná dlouhé obléhání Leningradu, Leningrad zásobován pouze po jezeře
  \item na podzim Hitler prohlašuje, že už vyhrál, v říjnu začíná operace Tajfun -- dobytí Moskvy, Německo je na 20 km od Moskvy
  \item začíná ale zima a je to v pytli, v prosinci začíná protiútok generála Žukova, který frontu posouvá v místech až 200 km na západ
  \item[srpen 1941] Atlantická charta -- Chruchill a FDR prohlašují, že musí porazit fašistické země, že uchází především o osvobození fašisty podrobených zemí
\end{itemize}
\end{document}
