\documentclass{article}
\usepackage{fullpage}
\usepackage[czech]{babel}
\usepackage{amsfonts}

\title{\vspace{-2cm}Husitství\vspace{-1.7cm}}
\date{}
\author{}

\begin{document}
\maketitle

\section*{Příčiny}
\begin{itemize}
    \vspace{-0.5em}
    \setlength\itemsep{0.15em}
    \item[$-$] círekv silná, světská moc ji neovládá, pronikla do každé vesničky
    \item[$-$] kněží nemají požadovanou úroveň ani morální ani vzdělání
    \item[od 1139] \textit{celibát}, ten je však *často* porušován
    \item[$-$] nárůst majetků církve -- za Lucemburků vlastnila třetinu veškeré půdy
    \item[$-$] \textit{desátky}, \textit{odpustky} -- vyhlášeny byzantským papežem, který potřeboval finanční prostředky na boj s druhým papežem
    \item[$-$] kupčení s církevními úřady, kumulace funkcí
    \item[$-$] objevují se kritici, reformátoři: \uv{církev se má vrátit do čistoty, má pečovat jen o duchovní stránku věřících}
    \item[$-$] hnutí na UK: nová zbožnost, \uv{devotio moderna}: církev má jen napomáhat věřícím, ti si mají najít sami cestu k bohu, za Karla IV. příchod \textbf{Konrada Waldhausera} (řeholník, augustinián), kritizoval faráře, že mají ze své práce byznys, vystupuje proti svatokupectví
    \item[$-$] spousta následovníků, třeba \textbf{Jan Milíč z Kroměříže}: značnou část majetku rozdal chudým, ze zbytku vybudoval Novopražskou kapli, vychovává nové kazatele
    \item[$-$] ostatní popuzeni, stěžují si u byzantského papeže, musí odjet na obhajobu do Avignonu, umírá
    \item[$-$] další následovníci: \textbf{Matěj z Janova}, významný teolog, mistr UK, působil i na Sorboně, kritizoval poměry v církvi, požaduje svobodu slova kazatelů
    \item[1414] \textbf{Jakoubek ze Stříbra}, univerzitní mistr, jako kazatel zavádí přijímání pod obojí, ostatní se potom přidávají
    \item[$-$] laik \textbf{Tomáš Štítný ze Štítného}, nemá vzdělání, ale kritizoval církev a usiloval o mravní nápravu
    \item[$-$] \textbf{Jeroným Pražský}, vypravil se do Kostnice hájit Jana Husa, neúspěšně, protože rok po něm byl též upálen; mistr na 4 univerzitách, studoval na Oxfordu
    \item[$-$] ovlivňování \textbf{Johnem Wycliffem}, profesor Oxfordské univerzity, přeložil bibli do angličtiny, sepsal nejrůznější spisy, kterými se inspirovali ostatní, \uv{náprava je možná pouze světskou mocí} = s pomocí panovníka
\end{itemize}


\section*{Dekret kutnohorský}
\begin{itemize}
    \vspace{-0.5em}
    \setlength\itemsep{0.15em}
    \item[$-$] na Karlově univerzitě studovali zahraniční studenti a působili zahraniční profesoři
    \item[$-$] 4 základní národy: český, bavorský, polský, saský; český v menšině (jen $2/5$)
    \item[$-$] Hus využil situace, že Václav IV. potřeboval jeho pomoc, který inicioval svolání Pisánského koncilu -- zástupci z KU tam měli jít
    \item[1409] vydání dekretu Václavem IV., došlo k počeštění univerzity, Češi mají tři hlasy a ostatní národnosti jen jeden
    \item[$-$] cizinci však odchází (většinou do nově vznikající univerzity v Lipsku), univerzita ztrácí prestiž
\end{itemize}


\subsection*{Mistr Jan Hus}
\begin{itemize}
    \vspace{-0.5em}
    \setlength\itemsep{0.15em}
    \item[$-$] vyučoval na Karlově univerzitě, 1401--2 děkanem artistické fakulty
    \item[1402] kázá v Betlémské kapli
    \item[1410] papež vydává nařízení, že se může kázat jen ve farních zařízeních, to nesplňuje Betlémská kaple, tehdejší pražský arcibiskup (Zbyněk Zajíc z Hazmbarka) tedy zakazuje kázání v této kapli, též přikázal spálit Wycleffovy spisy
    \item[1412] Pisánský papež \textbf{Jan XXIII.} vydává bulu o nařízení prodávání odpustků, proti ní Hus neprotestuje, ostatní si stěžují a zesměšňují prodej odpustků
    \item[$\rightarrow$] Václav IV. pro výstrahu tři účastníky protestů popravil, otřes v kritizující komunitě, ale Hus dále pokračuje v kázání
    \item[$-$] později však Hus odpustky kritizuje, je dán do \textit{klatby}, nad Prahou vyhlášen \textit{interdikt} = zákaz provádění církevních obřadů, proto odchází na venkov
    \item[$-$] pobývá na Kozím hrádku (u Sezimova Ústí), později na Krakovci (u Rakovníka)
    \item[$-$] díly: Knížky o svatokupectví, Dcera (jak mají dívky žít v souladu s bohem), Postila, O církvi (hlavou církve je Kristus, nikoliv papež), Výklad Viery
    \item[$-$] zavádí \textit{nabodeníčka}
    \item[1414 -- 18] \textsc{Kostnický koncil}, musel prokázat svoji nevinu, byl však odsouzen k smrti
    \item[6.7.1415] upálen, popel sypán do Rýnu, aby neměl hrob, kam mohou stoupenci chodit
    \item[$-$] jako reakce na jeho upálení \textit{Stížný list české šlechty}, kde proti tomu protestují
    \item[$\rightarrow$] rozpolcená společnost: husité = kališníci = ultrakvisté (symbol kalich -- přijímání pod obojí) vs. katolíci
\end{itemize}


\section*{Husité}
\begin{itemize}
    \vspace{-0.5em}
    \setlength\itemsep{0.15em}
    \item[$-$] poutě na hory, kde stoupenci Husa poslouchali radikální kněze, kteří varovali před blízkým koncem nespravedlivého světa
    \item[$-$] \textit{chiliasmus} = radikální část Husitů, říkají, že zanikne nespravedlivý svět a vznikne spravedlivá tisíciletá společnost
    \item[$-$] \textit{adamité} = radikální část pikartů, chodili nazí, považují se za potomky prvního člověka Adama
    \item[$-$] \textit{pikartové} = popírají přítomnost Krista v svátostech,  \uv{bratři a sestry svobodného ducha}
    \item[$-$] radikální Husity vede \textbf{Jan Želivský}
    \item[$-$] \textit{valdenští} = mírumilovní, odmítají násilí, jako jedinou platnou věc považují bibli
    \item[30.7.1419] konšelé v Praze vězní kališníky, jsou proti nim $\rightarrow$ první pražská defenestrace, odstartovala Husitské hnutí
    \item[$-$] centrální města husitů: Praha, Tábor (4 hejtmani, Jan Žižka z Trocnova)
    \item[$-$] \textsc{bitva u Sudoměře}, protože Jan Žižka a spol. cestují z Plzně do Tábora a potkali tam křižáky, husité vyhráli
\end{itemize}


\section*{Křížové výpravy}
\begin{itemize}
    \vspace{-0.5em}
    \setlength\itemsep{0.15em}
    \item[$-$] husité všechny vyhráli
\end{itemize}

\subsection*{První křížová výprava, 1420}

\begin{itemize}
    \vspace{-0.5em}
    \setlength\itemsep{0.15em}
    \item[$-$] Zikmund se domníval, že když zlomí husity, dostane se do čela Českého státu, proto se taky postavil do čela této výpravy
    \item[$-$] před boji táborité a pražené vytvořili svůj program, \textit{Čtyři artikuly pražské}: chtějí svobodu kázání slova božího, přijímání pod obojí, odstranění světské vlády církve, trestání hříchů bez rozdílu stavů
    \item[(14.7.)] \textsc{bitva na Vítvkově}, v čele husitů Žižka, porazil Zikmunda
    \item[28.7.] korunovace Zikmunda, ale za krále uznán nebyl
    \item[(1.11.)] \textsc{bitva u Vyšehradu}, vyhráli husité
    \item[(3. -- 7.6.1421)] \textsc{Čáslavský sněm}, sesadili Zikmunda z českého trůnu, Artikuly pražské prohlášeny za zemský zákon, vytvořená prozatimní vláda 20 členů
\end{itemize}

\subsection*{Druhá křížová výprava, 1421 -- 1422}
\begin{itemize}
    \vspace{-0.5em}
    \setlength\itemsep{0.15em}
    \item[$-$] začíná v srpnu, v čele křižáků opět Zikmund, husité opět vyhráli
    \item[$-$] \textsc{bitva u Německého Brodu} (dnešní Havlíčkův Brod), \textsc{Kutná Hora}
    \item[$-$] Žižka už v druhé výpravě nevidomý
    \item[březen 1422] Jan Žlezivský a jeho stoupenci byli vylákání na staroměstskou radnici, kde byli uvězneni a popraveni $\rightarrow$ konec éry, kdy v čele Prahy jsou radikální husité, poté už umírnění
    \item[$-$] \textbf{Zikmund Korybutovič}, litevský vévoda z rodu Jagellonců, zemským správcem, husité s ním počítají jako s budoucím českým králem, ale ukázalo se, že vyjednává se Zikmundem $\rightarrow$ vyhnán
    \item[1423] Jan Žižka odchází do Menšího (Nového) Tábora, protože mu Táborité vyčítali krutost vůči těm, kdo neměli stejný názor a vůči náboženským sektám v Táboře, nebyl tam spokojený
    \item[(7.6.) 1424] \textsc{bitva u Malečova} mezi umírněnými a radikálními husity, Jan Žižka vítězí
    \item[(11.10.) 1424] \textsc{tažení na Přibyslav}, vede Žižka, kde zemřel na otravu krve, dále na Moravu už vede \textbf{Prokop Holý}
    \item[$-$] \textit{panská jednota} = šlechta a umírnění husité
\end{itemize}


\subsection*{Třetí křížová výprava}
\begin{itemize}
    \vspace{-0.5em}
    \setlength\itemsep{0.15em}
    \item[1426] \textsc{bitva u Ústí nad Labem}, hodně mrtvých, vyhráli husité
\end{itemize}


\subsection*{Čtvrtá křížová výprava}
\begin{itemize}
    \vspace{-0.5em}
    \setlength\itemsep{0.15em}
    \item[1427] \textsc{\uv{bitva} u Tachova}, křižáci prchnou
\end{itemize}

\begin{itemize}
    \vspace{-0.5em}
    \setlength\itemsep{0.15em}
    \item[$-$] \textit{spanilé jízdy} = \textit{rejzy} = tažení do sousedních zemí, účel: kořist, šíření huistských myšlenek, \textbf{Jan Čapek ze Sán} došel až k Baltu
\end{itemize}

\subsection*{Pátá křížová výprava}
\begin{itemize}
    \vspace{-0.5em}
    \setlength\itemsep{0.15em}
    \item[1431] \textsc{\uv{bitva} u Domažlic}, křižáci též prchli, křižácký kardinál \textbf{Giuliano Cesarini}
\end{itemize}

\begin{itemize}
    \vspace{-0.5em}
    \setlength\itemsep{0.15em}
    \item[1431 -- 1445] \textsc{Basilejský koncil}: když nemůžeme porazit křižáky silou, musíme s nimi jednat -- tehdejší papež Martin V. svolal tento koncil
    \item[1432] \textit{Soudce chebský} = domluvili se, že jednání budou probíhat na základě biblických pravidel
    \item[$-$] k dohodě však nedošlo, příliš vysoké nároky
    \item[(30.5.) 1434] \textsc{bitva u Lipan} umírnění husité (vede Diviš Bořek z Miletína) proti radikálním (vede Prokop Holý), vyhráli umírnění, radikální popálení $\rightarrow$ otevření dveří pro dohodu s katolickou církví
    \item[$\rightarrow$] \textit{Basilejská kompaktáta} = dohoda s katolickou církví; to, co získali husité za válek si nechají, Češi husité mohou přijímat podobojí
    \item[1436] Zikmund se dostává do čela Českého státu, o rok později umírá
\end{itemize}

\end{document}
