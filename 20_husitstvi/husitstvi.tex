\documentclass{article}
\usepackage{fullpage}
\usepackage[czech]{babel}
\usepackage{amsfonts}

\title{\vspace{-2cm}Husitství\vspace{-1.7cm}}
\date{}
\author{}

\begin{document}
\maketitle

\section*{Příčiny}
\begin{itemize}
    \vspace{-0.5em}
    \setlength\itemsep{0.15em}
    \item[$-$] círekv silná, světská moc ji neovládá, pronikla do každé vesničky
    \item[$-$] kněží nemají požadovanou úroveň ani morální ani vzdělání
    \item[od 1139] \textit{celibát}, ten je však *často* porušován
    \item[$-$] nárůst majetků církve -- za Lucemburků vlastnila třetinu veškeré půdy
    \item[$-$] \textit{desátky}, \textit{odpustky} -- vyhlášeny byzantským papežem, který potřeboval finanční prostředky na boj s druhým papežem
    \item[$-$] kupčení s církevními úřady, kumulace funkcí
    \item[$-$] objevují se kritici, reformátoři: \uv{církev se má vrátit do čistoty, má pečovat jen o duchovní stránku věřících}
    \item[$-$] hnutí na UK: nová zbožnost, \uv{devotio moderna}: církev má jen napomáhat věřícím, ti si mají najít sami cestu k bohu, za Karla IV. příchod \textbf{Konrada Waldhausera} (řeholník, augustinián), kritizoval faráře, že mají ze své práce byznys, vystupuje proti svatokupectví
    \item[$-$] spousta následovníků, třeba \textbf{Jan milíč z Kroměříže}: značnou část majetku rozdal chudým, ze zbytku vybudoval Novopražskou kapli, vychovává nové kazatele
    \item[$-$] ostatní popuzeni, stěžují si u byzantského papeže, musí odjet na obhajobu do Avignonu, umírá      
\end{itemize}


\end{document}
