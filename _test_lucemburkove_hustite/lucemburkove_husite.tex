\documentclass{article}
\usepackage{fullpage}
\usepackage[czech]{babel}
\usepackage{amsfonts}

\title{\vspace{-2cm}Lucemburkové, husité\vspace{-1.7cm}}
\date{}
\author{}

\begin{document}
\maketitle

\section*{Lucemburkové}


\begin{itemize}
    \vspace{-0.5em}
    \setlength\itemsep{0.15em}
    \item[1310] počátek, nastupuje \textbf{Jan Lucemburský}
    \item[$-$] \textbf{Karel IV.}, \textbf{Václav IV.}
    \item[1437] konec za \textbf{Zmikunda Lucemburského}
\end{itemize}

\subsection*{Jindřich VII. Lucemburský}
\begin{itemize}
    \vspace{-0.5em}
    \setlength\itemsep{0.15em}
    \item[$-$] otec Jana Lucemburského
    \item[$-$] zvolen: nastupuje do čela SŘŘ po smrti syna Rudolfa Habsburského, Albrechta
    \item[$-$] podporovatel \textbf{Petr z Aspeltu}, hlasoval pro něj i při volbě, později jeden z poradců
    \item[$-$] tehdejší český vládce: \textbf{Jindřich Korutanský}, který má slabou vládu, období nestability $\rightarrow$ česká šlechta se vydala za Jindřichem Lucemburským, aby v Česku vládl jeho syn po sňatku s \textbf{Eliškou Přemyslovnou}
\end{itemize}

\subsection*{Jan Lucemburský}
\begin{itemize}
    \vspace{-0.5em}
    \setlength\itemsep{0.15em}
    \item[1310] \textsc{dobytí Prahy}, vyhnání Jindřicha z Korutan
    \item[$-$] skvělý válečník, diplomat, \textit{král cizinec}
    \item[$-$] \textbf{Jindřich z Lipé}, vůdce české člechty, žena \textbf{Eliška Rejčka}, žili v Brně
    \item[1310] šlechta si vynutila ústupky $\rightarrow$ \textit{Inaugurační diplom}: šlechta nemusí podnikat zahraniční výboje, nemusí platit daně, do významných úřadů nebudou jmenování cizinci, není plněno
    \item[1318] hrozba občanské války $\rightarrow$ \textsc{vtrhnul do Brna}
    \item[1318] vmísí se do toho císař SŘŘ \textbf{Ludvík Bavorský}, který zprostředkuje domluvu mezi šlechtou a králem: \textit{Úmluvy domažlické}, král rezignuje na správu Čech, zaměřuje se jen na říšskou politiku
    \item[(1319)] krize v manželství, sebral jí děti a byla vykázána na Mělník, následník trůnu Václav do Francie, tam právě vládne \textbf{Karel IV.}, což je jeho strýc
    \item[$-$] územní zisky: Chebsko, Horní Lužice (J od Lužice), část Slezska, Lucca (S Itálie)
    \item[$-$] současníkem \textbf{Kazimír III. Veliký}, společná dohoda: Kazimír se vzdal části území a Jan si na oplátku nenárokoval polskou korunu
    \item[$-$] \textit{dukáty}, mincovna přesunuta do Prahy
    \item[1333] Václav při biřmování přijímá jméno Karel
    \item[1334] kralevic Karel markrabětem moravským
    \item[1337] Karel spoluvladařem
    \item[$-$] položeny základy Staroměstské radnice
    \item[$-$] stavba katedrály Sv. Víta: Matyáš z Arrasu, Petr Parléř
    \item[$-$] vzal si další ženušku \textbf{Beatrix Bourbonskou}
    \item[(1340)] uvědomuje si že je starý a slepý, sepíše \textit{Janovu závěť}: Karel dědí Čechy, Lužici a Slezsko, jeho bratr \textbf{Jan Jindřich} je markrabě moravský a \textbf{Václav Český} (dítě s Beatrix) získává Lucembursko
    \item[(1341)] Karel je \textit{rex junior} = mladší král, počítá se s ním jako s budoucím králem
    \item[(1344)] společná výprava s Karlem za avignonským papežem \textbf{Klementem VI.}, původním jménem Pierre Roger, což byl Karlův bývalý vychovatel ve Francii a díky jejich nadstandardním vztahům je Pražské biskupství povýšeno na arcibiskupství, první arcibiskup \textbf{Arnošt z Pardubic}; též zřízeno nové biskupství v Litomyšli $\rightarrow$ vymanění české církve z nadřízenosti Mohučského arcibiskupství
    \item[1346] v SŘŘ vládne Ludvík Bavor, který má špatné vztahy s Klimentem a kritizuje papeže, takže avignonský papež iniciuje zvolení protikandidáta: 5 hlasů pro něj $\rightarrow$ Karel \textbf{král římský}
    \item[(26.8.) 1346] \textsc{bitva u Kresčaku}, Jan Lucemburský zahynul $\rightarrow$ Karel králem českým, první maželka Blanka z Valois, nechal vytvořit Svatováclavskou korunu
\end{itemize}

\subsection*{Karel IV., otec vlasti (1346 -- 1378)}
\begin{itemize}
    \vspace{-0.5em}
    \setlength\itemsep{0.15em}
    \item[1346] král římský, český
    \item[1355] lombardská koruna
    \item[1355] manželka \textbf{Anna Svídnická}, císař SŘŘ
    \item[$-$] Montecarlo = pevnost v Toskánsku
    \item[$-$] Praha sídelním městěm císaře SŘŘ, Čechy centerm SŘŘ
    \item[$-$] kult Sv. Václava: Svatováclavská koruna, Svatováclavská kaple, Sv. Václav na pečetidle KU
    \item[$-$] Země koruny české: České království a vedlejší země: markrabství moravské, Slezsko, Lužice (Horní i Dolní), Horní Falc, Lucembursko, Braniborsko
    \item[$-$] vysocí úředníci: Arnošt z Pardubic, Jan ze Středy, Francesco Petrarca; všeobecně se opírá o šlechtu
    \item[$-$] Jan Očko z Vlašimi, to jest arcibiskup
    \item[1355] \textit{Maiestas Carolina} = neúspěšný návrh zemského zákoníku, ale pro šlechtu nepřijatelný, údajně shořel
    \item[1356]  \textit{Zlatá bula} = říšský zákoník, zvýhodnil postavení českého krále mezi kurfiřty, pro český trůn platila ženská posloupnost, při volbě krále nemusela platit absolutní shoda, ale jen větší polovina hlasů, platila až do 1806
    \item[$-$] druhé, neoficiální sídlo Karla je Norimberk
    \item[$-$] hospodářství: víno, ovocnářství, PIVO, rybníkářství, bohaté stříbrné doly
    \item[(1348)] Moravské zemské desky = Moravské cúdy (soudy) = dvakrát do roka se konaly zemské soudy a jejich výsledky se píší do těchto desk
    \item[$-$] zakladatelská činnost: Karlova univerzita (první univerzita francouzského typu u nás: 4 fakulty -- artistická, právnická, lékařská, teologická), Nové město pražské, úprava Pražského hradu, Svatovítská katedrála, kamenný most, Staroměstská mostecká věž, Karlštejn, klášter Emauzy, Hladová zeď (součást městského opevnění), chrám Panny Marie Sněžné, Staroměstská radnice
    \item[$-$] manželky: Blanka z Valois, Anna Falcká, Anna Svídnická (syn Václav IV.), Alžběta Pomořanská (synové Zikmund, Jan Zhořelecký a dcera Anna Česká)
    \item[$-$] celkem 11 dětí
    \item[$-$] umírá na zápal plic
\end{itemize}

\subsection*{Kultura}
\begin{itemize}
    \vspace{-0.5em}
    \setlength\itemsep{0.15em}
    \item[$-$] Mistr Theodorik -- malíř, autor nástěnných obrazů v kapli Sv. Kříže
    \item[$-$] Mistr Třeboňského, Vyšebrodského, Rajhradského oltáře
    \item[$-$] \textit{iluminace} = ilustrace knih, např. ve Velislavově bibli
    \item[$-$] socha Sv. Jiří na nádvoří Pražského hradu
    \item[$-$] rozvoj školství: univerzita, klášterní, farní, partikulární ve městech
    \item[$-$] legendy o sv. Kateřině, o sv. Prokopu
    \item[$-$] \textit{postily}, Trojánská kronika, Závišova píseň, \textit{Vita caroli}, Klaretův slovník
    \item[$-$] první překlad bible do češtiny
\end{itemize}

\subsection*{Václav IV.}
\begin{itemize}
    \vspace{-0.5em}
    \setlength\itemsep{0.15em}
    \item[(1363)] už jako dvouletý korunován českým králem
    \item[(1376)] králem SŘŘ
    \item[(1378)] smrt otce, dostává se k moci
\end{itemize}


\subsection*{Zikmund Lucemburský}
\begin{itemize}
    \vspace{-0.5em}
    \setlength\itemsep{0.15em}
    \item[$-$] král Uherský, spolu s bratrem Janem Zhořeleckým drží Braniborsko
\end{itemize}

\subsection*{Jan Zhořelecký}
\begin{itemize}
    \vspace{-0.5em}
    \setlength\itemsep{0.15em}
    \item[$-$] Zhořelecko = Horní Lužice, s bratrem se dělí o Braniborsko
\end{itemize}


\subsection*{Jan Jindřich}
\begin{itemize}
    \vspace{-0.5em}
    \setlength\itemsep{0.15em}
    \item[$-$] bratr Karla IV.
    \item[$-$] synové: Jošt, Prokop, Jan Soběslav, markrabata moravská, ale nakonec je to Jošt
\end{itemize}

\subsection*{Václav IV. (1378 -- 1419)}
\begin{itemize}
    \vspace{-0.5em}
    \setlength\itemsep{0.15em}
    \item[$-$] matka \textbf{Anna Svídnická}, narozen v Norimberku, vzdělaný
    \item[$-$] král SŘŘ díky titulu jeho otce
    \item[$-$] manželky: \textbf{Johana Bavorská}, \textbf{Žofie Bavorská}, žádné děti
    \item[$-$] vládne v době stoleté války
    \item[1378--1417] \textit{papežské schizma} = dvojpapežství, papeži v Římě a Avignonu
    \item[1409] svolal \textsc{Pisánský koncil}, aby zvolili nového papeže, ale oba dosavadní papežové se odmítají vzdát své funkce $\rightarrow$ tři papežové
    \item[$-$] bojí se vysoké šlechty, opírá se o měšťany a drobnou šlechtu, z nich si vybírá dvořany a rádce, to se nelíbí vysoké šlechtě, která se postupně spojuje do tzv. \textit{panské jednoty}
    \item[$-$] arcibiskup pražský: \textbf{Jan z Jenštejna} chce, aby církev podléhala jemu a aby byla nezávislá králi, to se nelíbí Václavovi $\rightarrow$ nepřátelé
    \item[1393] vygradování konfliktu, václavovi lidé vtrhli do arcibiskupova sídla a unesli \textbf{Jana z Pomuku}, podle legendy z Karlova mostu shozen do Vltavy, následně prohlášen za svatého
    \item[$-$] Jan z Jenštejna si stěžuje u arcibiskupa, ale ten potřebuje mít nakloněné České země kvůli možné válce, takže mu nevyhoví
    \item[$-$] do čela šlechty se postavil \textbf{Jošt Lucemburský}
    \item[1394] \textsc{šlechta v čele s Joštem Lucemburským zajala Václava IV.}, protože byli nespokojení s jeho činností, jeho propuštění dojedná Joštův bratr Jan Zhořelecký
    \item[$-$] všichni arcibiskupové kritizují Václava IV., protože neřeší papežské schizma $\rightarrow$
    \item[1400] \textsc{zbaven říšské koruny}, nahradí ho \textbf{Ruprecht III. Falcký}
    \item[$-$] nespokojený s jeho politikou je i Zikmund, král uherský $\rightarrow$
    \item[1402] \textsc{vpád Zikmunda do Čech}, chce na sebe strhnout moc, ale neúspěšně, všichni se za něj postavili
    \item[1409] \textit{Pisánský koncil}, kde byl zvolen další papež, ti předchozí nechtějí odstoupit, nový papež opět zvolil Václava říšským králem
    \item[1410] \textbf{Ruprecht} (král SŘŘ) umírá, noví kandidáti: Zikmund Lucemburský, Jošt Moravský, Václav IV., vítězí Jošt
    \item[1411] \textsc{Jošt umírá} za záhadných okolností $\rightarrow$ ostatní Lucemburkové se dohodli, že si Václav nechá titul, ale fakticky vládne Zikmund
    \item[1417] \textsc{konec trojpapežství} díky \textsc{Koncilu v Kostnici}, který svolal Zikmund, dále řešeny otázky nápravy církve: měl se prot nim vymezit, \textbf{Mistr Jan Hus} se snažil vysvětlit své učení, avšak 1415 \textsc{upálen}, přijel ho obhajovat kamarád \textbf{Jeroným Pražský}, 1416 \textsc{upálen}
    \item[$\rightarrow$] polarizace společnosti, kritika Václava IV. a Zikmunda
    \item[$-$] \textit{Stížný list české šlechty} -- šlechta si stěžuje na šlechtu a Kostnický koncil
    \item[30.7.1419] vyvrcholení: \textsc{První pražská defenestrace}, smrt několika konšelů
    \item[16.8.1419] \textsc{umírá} Václav IV. (mrtvice nebo epilepsie)
\end{itemize}

\subsection*{Zikmund Lucemburský \#2}
\begin{itemize}
    \vspace{-0.5em}
    \setlength\itemsep{0.15em}
    \item[$\rightarrow$] problém nástupnictví: \textbf{Zikmund Lucemburský}, král uherský a SŘŘ (Češi ho nechtějí) $\rightarrow$ využil křížových výprav (celkem 5)
    \item[1420] první: mezi boji se nechal korunovat na českého krále, nikdo ho neuznává
    \item[1421] \textsc{Čáslavský sněm Zikmunda sesadil}, on si ale za titulem stojí
    \item[1431] druhá: \textsc{husité opět vítězí} $\rightarrow$ svolán \textsc{koncil v Bazileji}, kde se řeší, jak se jich zbavit, husité je však sami zbaví trápení
    \item[1434] \textsc{bitva u Lipan} radikální vs. umírnění, vyhrávají umírnění $\rightarrow$ otevření cesty k vyjednávání s katolíky, papežem, uzavřena \textit{Bazilejská kompaktáta} -- Češi mohou přijímat pod obojí
    \item[$\rightarrow$ 1436] Zikmund českým králem
    \item[1437] Zikmund ve Znojmě umírá
    \item[1433] císař SŘŘ -- nejvýznamnější sourozenec Václava IV.
\end{itemize}


\subsection*{Problém nástupnictví}
\begin{itemize}
    \vspace{-0.5em}
    \setlength\itemsep{0.15em}
    \item[$-$] po Zikmundově smrti opět problém -- Václav IV. nemá potomky, Zikmund pouze dceru \textbf{Alžbětu Lucemburskou}, hledají ženicha: \textbf{Albrecht Habsburský}, sňatek
    \item[1437] Albrecht českým králem, ale o dva roky později umírá na úplavici
    \item[$-$] Alžběta je těhotná -- syn \textbf{Ladislav Pohrobek}, vládne 1453 -- 1457, ponechává si \textbf{Jiřího z Kunštátu} a \textbf{z Poděbrad} jaho zemského správce
    \item[$-$] dříve, než se stihne oženit, zemře $\rightarrow$ králem \textbf{Jiří z Poděbrad}



\end{itemize}

\section*{Husité}
\subsection*{Příčiny}
\begin{itemize}
    \vspace{-0.5em}
    \setlength\itemsep{0.15em}
    \item[$-$] círekv silná, světská moc ji neovládá, pronikla do každé vesničky
    \item[$-$] kněží nemají požadovanou úroveň ani morální ani vzdělání
    \item[od 1139] \textit{celibát}, ten je však *často* porušován
    \item[$-$] nárůst majetků církve -- za Lucemburků vlastnila třetinu veškeré půdy
    \item[$-$] \textit{desátky}, \textit{odpustky} -- vyhlášeny byzantským papežem, který potřeboval finanční prostředky na boj s druhým papežem
    \item[$-$] kupčení s církevními úřady, kumulace funkcí
    \item[$-$] objevují se kritici, reformátoři: \uv{církev se má vrátit do čistoty, má pečovat jen o duchovní stránku věřících}
    \item[$-$] hnutí na UK: nová zbožnost, \uv{devotio moderna}: církev má jen napomáhat věřícím, ti si mají najít sami cestu k bohu, za Karla IV. příchod \textbf{Konrada Waldhausera} (řeholník, augustinián), kritizoval faráře, že mají ze své práce byznys, vystupuje proti svatokupectví
    \item[$-$] spousta následovníků, třeba \textbf{Jan Milíč z Kroměříže}: značnou část majetku rozdal chudým, ze zbytku vybudoval Novopražskou kapli, vychovává nové kazatele
    \item[$-$] ostatní popuzeni, stěžují si u byzantského papeže, musí odjet na obhajobu do Avignonu, umírá
    \item[$-$] další následovníci: \textbf{Matěj z Janova}, významný teolog, mistr UK, působil i na Sorboně, kritizoval poměry v církvi, požaduje svobodu slova kazatelů
    \item[1414] \textbf{Jakoubek ze Stříbra}, univerzitní mistr, jako kazatel zavádí přijímání pod obojí, ostatní se potom přidávají
    \item[$-$] laik \textbf{Tomáš Štítný ze Štítného}, nemá vzdělání, ale kritizoval církev a usiloval o mravní nápravu
    \item[$-$] \textbf{Jeroným Pražský}, vypravil se do Kostnice hájit Jana Husa, neúspěšně, protože rok po něm byl též upálen; mistr na 4 univerzitách, studoval na Oxfordu
    \item[$-$] ovlivňování \textbf{Johnem Wycliffem}, profesor Oxfordské univerzity, přeložil bibli do angličtiny, sepsal nejrůznější spisy, kterými se inspirovali ostatní, \uv{náprava je možná pouze světskou mocí} = s pomocí panovníka
\end{itemize}


\subsection*{Dekret kutnohorský}
\begin{itemize}
    \vspace{-0.5em}
    \setlength\itemsep{0.15em}
    \item[$-$] na Karlově univerzitě studovali zahraniční studenti a působili zahraniční profesoři
    \item[$-$] 4 základní národy: český, bavorský, polský, saský; český v menšině (jen $2/5$)
    \item[$-$] Hus využil situace, že Václav IV. potřeboval jeho pomoc, který inicioval svolání Pisánského koncilu -- zástupci z KU tam měli jít
    \item[1409] vydání dekretu Václavem IV., došlo k počeštění univerzity, Češi mají tři hlasy a ostatní národnosti jen jeden
    \item[$-$] cizinci však odchází (většinou do nově vznikající univerzity v Lipsku), univerzita ztrácí prestiž
\end{itemize}


\subsection*{Mistr Jan Hus}
\begin{itemize}
    \vspace{-0.5em}
    \setlength\itemsep{0.15em}
    \item[$-$] vyučoval na Karlově univerzitě, 1401--2 děkanem artistické fakulty
    \item[1402] kázá v Betlémské kapli
    \item[1410] papež vydává nařízení, že se může kázat jen ve farních zařízeních, to nesplňuje Betlémská kaple, tehdejší pražský arcibiskup (Zbyněk Zajíc z Hazmbarka) tedy zakazuje kázání v této kapli, též přikázal spálit Wycleffovy spisy
    \item[1412] Pisánský papež \textbf{Jan XXIII.} vydává bulu o nařízení prodávání odpustků, proti ní Hus neprotestuje, ostatní si stěžují a zesměšňují prodej odpustků
    \item[$\rightarrow$] Václav IV. pro výstrahu tři účastníky protestů popravil, otřes v kritizující komunitě, ale Hus dále pokračuje v kázání
    \item[$-$] později však Hus odpustky kritizuje, je dán do \textit{klatby}, nad Prahou vyhlášen \textit{interdikt} = zákaz provádění církevních obřadů, proto odchází na venkov
    \item[$-$] pobývá na Kozím hrádku (u Sezimova Ústí), později na Krakovci (u Rakovníka)
    \item[$-$] díly: Knížky o svatokupectví, Dcera (jak mají dívky žít v souladu s bohem), Postila, O církvi (hlavou církve je Kristus, nikoliv papež), Výklad Viery
    \item[$-$] zavádí \textit{nabodeníčka}
    \item[1414 -- 18] \textsc{Kostnický koncil}, musel prokázat svoji nevinu, byl však odsouzen k smrti
    \item[6.7.1415] upálen, popel sypán do Rýnu, aby neměl hrob, kam mohou stoupenci chodit
    \item[$-$] jako reakce na jeho upálení \textit{Stížný list české šlechty}, kde proti tomu protestují
    \item[$\rightarrow$] rozpolcená společnost: husité = kališníci = ultrakvisté (symbol kalich -- přijímání pod obojí) vs. katolíci
\end{itemize}


\begin{itemize}
    \vspace{-0.5em}
    \setlength\itemsep{0.15em}
    \item[$-$] poutě na hory, kde stoupenci Husa poslouchali radikální kněze, kteří varovali před blízkým koncem nespravedlivého světa
    \item[$-$] \textit{chiliasmus} = radikální část Husitů, říkají, že zanikne nespravedlivý svět a vznikne spravedlivá tisíciletá společnost
    \item[$-$] \textit{adamité} = radikální část pikartů, chodili nazí, považují se za potomky prvního člověka Adama
    \item[$-$] \textit{pikartové} = popírají přítomnost Krista v svátostech,  \uv{bratři a sestry svobodného ducha}
    \item[$-$] radikální Husity vede \textbf{Jan Želivský}
    \item[$-$] \textit{valdenští} = mírumilovní, odmítají násilí, jako jedinou platnou věc považují bibli
    \item[30.7.1419] konšelé v Praze vězní kališníky, jsou proti nim $\rightarrow$ první pražská defenestrace, odstartovala Husitské hnutí
    \item[$-$] centrální města husitů: Praha, Tábor (4 hejtmani, Jan Žižka z Trocnova)
    \item[$-$] \textsc{bitva u Sudoměře}, protože Jan Žižka a spol. cestují z Plzně do Tábora a potkali tam křižáky, husité vyhráli
\end{itemize}


\subsection*{Křížové výpravy}
\begin{itemize}
    \vspace{-0.5em}
    \setlength\itemsep{0.15em}
    \item[$-$] husité všechny vyhráli
\end{itemize}

\subsubsection*{První křížová výprava, 1420}

\begin{itemize}
    \vspace{-0.5em}
    \setlength\itemsep{0.15em}
    \item[$-$] Zikmund se domníval, že když zlomí husity, dostane se do čela Českého státu, proto se taky postavil do čela této výpravy
    \item[$-$] před boji táborité a pražené vytvořili svůj program, \textit{Čtyři artikuly pražské}: chtějí svobodu kázání slova božího, přijímání pod obojí, odstranění světské vlády církve, trestání hříchů bez rozdílu stavů
    \item[(14.7.)] \textsc{bitva na Vítvkově}, v čele husitů Žižka, porazil Zikmunda
    \item[28.7.] korunovace Zikmunda, ale za krále uznán nebyl
    \item[(1.11.)] \textsc{bitva u Vyšehradu}, vyhráli husité
    \item[(3. -- 7.6.1421)] \textsc{Čáslavský sněm}, sesadili Zikmunda z českého trůnu, Artikuly pražské prohlášeny za zemský zákon, vytvořená prozatimní vláda 20 členů
\end{itemize}

\subsubsection*{Druhá křížová výprava, 1421 -- 1422}
\begin{itemize}
    \vspace{-0.5em}
    \setlength\itemsep{0.15em}
    \item[$-$] začíná v srpnu, v čele křižáků opět Zikmund, husité opět vyhráli
    \item[$-$] \textsc{bitva u Německého Brodu} (dnešní Havlíčkův Brod), \textsc{Kutná Hora}
    \item[$-$] Žižka už v druhé výpravě nevidomý
    \item[březen 1422] Jan Žlezivský a jeho stoupenci byli vylákání na staroměstskou radnici, kde byli uvězneni a popraveni $\rightarrow$ konec éry, kdy v čele Prahy jsou radikální husité, poté už umírnění
    \item[$-$] \textbf{Zikmund Korybutovič}, litevský vévoda z rodu Jagellonců, zemským správcem, husité s ním počítají jako s budoucím českým králem, ale ukázalo se, že vyjednává se Zikmundem $\rightarrow$ vyhnán
    \item[1423] Jan Žižka odchází do Menšího (Nového) Tábora, protože mu Táborité vyčítali krutost vůči těm, kdo neměli stejný názor a vůči náboženským sektám v Táboře, nebyl tam spokojený
    \item[(7.6.) 1424] \textsc{bitva u Malečova} mezi umírněnými a radikálními husity, Jan Žižka vítězí
    \item[(11.10.) 1424] \textsc{tažení na Přibyslav}, vede Žižka, kde zemřel na otravu krve, dále na Moravu už vede \textbf{Prokop Holý}
    \item[$-$] \textit{panská jednota} = šlechta a umírnění husité
\end{itemize}


\subsubsection*{Třetí křížová výprava}
\begin{itemize}
    \vspace{-0.5em}
    \setlength\itemsep{0.15em}
    \item[1426] \textsc{bitva u Ústí nad Labem}, hodně mrtvých, vyhráli husité
\end{itemize}


\subsubsection*{Čtvrtá křížová výprava}
\begin{itemize}
    \vspace{-0.5em}
    \setlength\itemsep{0.15em}
    \item[1427] \textsc{\uv{bitva} u Tachova}, křižáci prchnou
\end{itemize}

\begin{itemize}
    \vspace{-0.5em}
    \setlength\itemsep{0.15em}
    \item[$-$] \textit{spanilé jízdy} = \textit{rejzy} = tažení do sousedních zemí, účel: kořist, šíření huistských myšlenek, \textbf{Jan Čapek ze Sán} došel až k Baltu
\end{itemize}

\subsubsection*{Pátá křížová výprava}
\begin{itemize}
    \vspace{-0.5em}
    \setlength\itemsep{0.15em}
    \item[1431] \textsc{\uv{bitva} u Domažlic}, křižáci též prchli, křižácký kardinál \textbf{Giuliano Cesarini}
\end{itemize}

\section*{Závěr husitství}


\begin{itemize}
    \vspace{-0.5em}
    \setlength\itemsep{0.15em}
    \item[1431 -- 1445] \textsc{Basilejský koncil}: když nemůžeme porazit křižáky silou, musíme s nimi jednat -- tehdejší papež Martin V. svolal tento koncil
    \item[1432] \textit{Soudce chebský} = domluvili se, že jednání budou probíhat na základě biblických pravidel
    \item[$-$] k dohodě však nedošlo, příliš vysoké nároky
    \item[(30.5.) 1434] \textsc{bitva u Lipan} umírnění husité (vede Diviš Bořek z Miletína) proti radikálním (vede Prokop Holý), vyhráli umírnění, radikální popálení $\rightarrow$ otevření dveří pro dohodu s katolickou církví
    \item[$\rightarrow$] \textit{Basilejská kompaktáta} = dohoda s katolickou církví; to, co získali husité za válek si nechají, Češi husité mohou přijímat podobojí
    \item[1435] \textbf{Jan Rokycana} (spolupracovník Jana Žižky) zvolen arcibiskupem pražským, církev ani papež neuznává
    \item[1436] Zikmund se dostává do čela Českého státu, o rok později umírá
    \item[$-$] opevnění na hradu Sion
    \item[$-$] \textbf{Petr Chelčický}: myslitel, kritizuje stávající společnost, dílo O trojím lidu -- panstvo, kněží, poddaní; soubor kázání Postila, Sieť viery pravé; kritizuje války
    \item[$-$] z jeho učení vzniká \textbf{Jednota bratrská}
    \item[$-$] bratřické hnutí (bratříci) -- na území Uher hájí zájmy Ladislava Pohrobka (Zikmundův vnuk), jejich konec 1458 zvolením \textbf{Matyáše Korvína} uherským králem, zlikvidoval zbytky tohoto hnutí

\end{itemize}

\section*{Důsledky}
\begin{itemize}
    \vspace{-0.5em}
    \setlength\itemsep{0.15em}
    \item[$-$] položeny základy stavovské monarchie (panský stav, vyšší šlechta, měšťané), oslabení královské moci
    \item[$-$] izolace českých zemí, znehodnocení měny, menší ekonomický růst
    \item[$-$] vznik nové církve: \textbf{Jednota bratrská}
    \item[$-$] zvýšení povědomí o češtině
    \item[$-$] pokles významu Karlovy univerzity
    \item[$-$] umění stagnuje (husité útočí na kostely, které byly nenávratně zničeny)
\end{itemize}


\end{document}
